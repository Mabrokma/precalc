\documentclass[10pt,twoside]{report}
%\documentclass[10pt,twoside,draft]{report}
% Last edited: Jordan, March 1st, 2013
%====================================
%   BEGIN PACKAGES
%====================================
\usepackage[left=6cm,right=1cm,showframe=false,				    % page settings (showframe is useful!)
    top=1.5cm,bottom=1.5cm,asymmetric=true,bindingoffset=1cm]{geometry}               
\usepackage{parskip}
\usepackage{lipsum}
\usepackage{amssymb}                   		    % mathematical content
\usepackage{mathtools}                          % math tools such as \mathrlap, \mathllap, \mathclap
\usepackage[amsmath,standard,thmmarks]{ntheorem} % custom enumerations *Needs to stay here, otherwise cross references can go wrong*
\usepackage{etoolbox}                           % for conditionals, and lots more
\usepackage{environ}                            % for help with environments
\usepackage{answers}                        	% solutions to problems done *beautifully*
%\usepackage[nosolutionfiles]{answers}
\usepackage[framemethod=tikz]{mdframed}				                % needed for putting boxes round theorems, examples	
\usepackage[explicit]{titlesec}
\usepackage{multicol}                          	% multicols
\usepackage{changepage}                       	% needed for widepage (for exercises)
\usepackage{fancyhdr}                           % headers & footers
\usepackage[small,bf]{caption}                 	% figures/tables in margins, and not floating
\usepackage{subcaption}                         % subfigures, subtables (replaces subfig)
\usepackage{xstring}                            % needed to determine if problems are odd or even
\usepackage{makeidx}                            % make an index
\usepackage{enumitem}                           % enumerate & itemize
\usepackage{placeins}                           % FloatBarrier
\usepackage{varioref}                           % informative references (such as, 'on the next page')
\usepackage{multirow}                           % multirow in tables
\usepackage{minitoc}                            % mini-table of contents
\usepackage{marginfix}                          % fixes margin notes (otherwise they can cut through pages)
\usepackage{textcomp}			                      % gives symbols like \textcent and \textdegree
\usepackage{nicefrac}				  									% \nicefrac{a}{b}
\usepackage{siunitx}														% \si{\meter\per\second}, \SI{1}{\kilo\gram}, much more
	\sisetup{quotient-mode = fraction,per-mode=fraction,fraction-function = \nicefrac}
\usepackage[super]{nth}													% \nth{17} for 17th
\usepackage{mleftright}													% for \mleft and \mright, which fix spacing issue in \left and \right
\usepackage{needspace}                          % needed to keep problems looking good (stops ugly pagebreaks)
\usepackage{xargs}                              % helps with optional arguments for environments
\usepackage{array}          % used with tables (allows \newcolumntype)
\usepackage{booktabs}       % used with tables (allows \toprulewidth etc)
\usepackage{flafter}	      % ensures that floats don't appear until after their placement in the text
\usepackage{pgfplots}       % graphs and graphics (loads tikz and xcolor)
\usetikzlibrary{backgrounds}
\usetikzlibrary{shadows}
\usetikzlibrary{mindmap}
\usetikzlibrary{backgrounds}
\usetikzlibrary{fit}
\usetikzlibrary{positioning}
\usepackage[charter]{mathdesign}                     % changes font
\usepackage[expansion=false,kerning=true]{microtype} % better kerning
\usepackage[T1]{fontenc}														% More reliable copy-and-pasting
\input{glyphtounicode} \pdfgentounicode=1						% More reliable copy-and-pasting
\usepackage{adjustbox}      % used to provide better vertical alignment within solutions

% load the hyperref (nearly) last
\usepackage{hyperref}           % to allow hyper refs in the final pdf document
                                % hyperref NEEDS TO STAY AT THE END!
\usepackage[all]{hypcap}        % needed to help hyperlinks direct correctly;
                                % note that this needs to be loaded AFTER
                                % the hyperref package; the option [all]
                                % makes hyperlinks to figures & tables go 
                                % to the top
\usepackage{cleveref}       % needs to be loaded after hyperref

% hyperref settings- it seemed to work better here rather than
% as options to the \usepackage call above
%
% breaklinks=true is really important! otherwise the varioref makes
% them go over the paragraph boundary!
\hypersetup{colorlinks=true,
linkcolor=blue%,breaklinks=true
}

% really useful during hyperlink debugging
%\usepackage[left]{showlabels}
%\showlabels{hypertarget}
%\showlabels{hyperlink}

\makeindex

%\listfiles
%\show\section % really useful command!

%====================================
%   END PACKAGES
%====================================

% ====================================
% BEGIN GRAPH SETTINGS
% ====================================
%\usetikzlibrary{external}
%\tikzexternalize[mode=list and make]


% cycle list- truly awesome; see section 4.6.7, pg 129 of pgfplots
\pgfplotscreateplotcyclelist{pccstylelist}{%
color=red,mark=none,line width=1pt,<->\\%
color=blue,mark=none,line width=1pt,<->\\%
color=gray,mark=none,line width=1pt,<->\\%
}

% axis style, ticks, etc
\pgfplotsset{every axis/.append style={
					axis x line=middle,    % put the x axis in the middle
					axis y line=middle,    % put the y axis in the middle
                    axis line style={<->}, % arrows on the axis
                    xlabel={$x$},          % default put x on x-axis
                    ylabel={$y$},          % default put y on y-axis
                    scale only axis,       % otherwise width won't be as intended: http://tex.stackexchange.com/questions/36297/pgfplots-how-can-i-scale-to-text-width
                    cycle list name=pccstylelist,
                    tick label style={font=\tiny},
                    label style={font=\small},
                    legend cell align=left,
					legend style={font=\tiny},
                    width=\textwidth,
                    every node near coord/.append style={
                            font=\small},
                    }}
\tikzset{axisnode/.style={font=\tiny,text=black}}

% line style
\pgfplotsset{pccplot/.style={color=red,mark=none,line width=1pt,<->}} % this is pretty redundant in most cases now that cycle list is implemented
\pgfplotsset{asymptote/.style={color=red,mark=none,line width=1pt,dashed}}
\pgfplotsset{soldot/.style={color=red,only marks,mark=*}}
\pgfplotsset{holdot/.style={color=red,fill=white,only marks,mark=*}}

% bar chart style
\pgfplotsset{pccbar/.style={ybar,draw=blue,fill=blue!20}}

% arrow style
\tikzset{>=stealth}

% framing the graphs
\pgfplotsset{framed/.style={axis background/.style ={draw=gray}}}
% next line is a bit more colourful
%\pgfplotsset{framed/.style={axis background/.style ={draw=gray,fill=yellow!20,rounded corners=3ex}}}

% grid style
\pgfplotsset{grid style={dashed,gray}}

% if draft mode is specified then just draw a rectangle 
% http://tex.stackexchange.com/questions/60434/draft-mode-for-pgfplots
\makeatletter
\@tempswafalse
\def\@tempa{draft}
\@for\next:=\@classoptionslist\do
  {\ifx\next\@tempa\@tempswatrue\fi}
\if@tempswa % draft option is active
  \let\tikz@@tikzpicture\tikzpicture
  \let\tikz@@endtikzpicture\endtikzpicture
  \patchcmd\tikz@opt{\tikzpicture}{\tikz@@tikzpicture}{}{}
  \patchcmd\tikz@collectnormalsemicolon{\endtikzpicture}{\tikz@@endtikzpicture}{}{}
  \chardef\@tempa=\catcode`\;
  \catcode`\;=\active
  \patchcmd\tikz@collectactivesemicolon{\endtikzpicture}{\tikz@@endtikzpicture}{}{}
  \catcode`\;=\@tempa
  \let\tikzpicture\relax
  \let\endtikzpicture\relax
      \NewEnviron{tikzpicture}{%
        \begin{pgfpicture}
        \pgfpathrectanglecorners{\pgfpointorigin}{\pgfpoint{3cm}{3cm}}%
        \pgfusepath{stroke}
        \end{pgfpicture}%
      }
    \fi
\makeatother

%Without the following, corves that are plotted using coordinates have arrow tips at the end that point in the wrong direction
\makeatletter
\def\pgf@plot@curveto@handler@finish{%
  \ifpgf@plot@started%
\pgfpathcurvebetweentimecontinue{0}{0.995}{\pgf@plot@curveto@first}{\pgf@plot@curveto@first@support}{\pgf@plot@curveto@second}{\pgf@plot@curveto@second}%
  \fi%
}
\makeatother

% ====================================
% END GRAPH SETTINGS
% ====================================

% figure width
\newlength{\figurewidth}

% set solution figure width
\newlength{\solutionfigurewidth}
\setlength{\solutionfigurewidth}{5cm}


% Figure 1.1, 1.2,... The figure, and table numbering inherits the section numbering
%\numberwithin{figure}{section}
%\numberwithin{table}{section}

% caption setup
\captionsetup[table]{skip=0pt}
\captionsetup[subfigure]{labelformat=parens,labelfont=normal}
\captionsetup[subtable]{labelformat=parens,labelfont=normal}

% itemize and enumerate settings
\setlist{itemsep=0.05em,topsep=0.01em}
\setlist[enumerate]{label*=(\alph*)}

% paragraph settings
%\setlength{\parskip}{3.0mm}
%\setlength{\parindent}{0.0mm}

% use left margin
\reversemarginpar
\setlength{\marginparwidth}{5cm}

% define own color
\definecolor{silver}{rgb}{0.95,0.95,0.95}

% wide page for side by side figures, tables, etc
\newlength{\offsetpage}
\setlength{\offsetpage}{4.0cm}
\newenvironment{widepage}{\begin{adjustwidth}{-\offsetpage}{}%
                            \addtolength{\textwidth}{\offsetpage}}%
                         {\end{adjustwidth}}

% Define fix command
% 	- it puts a comment in the margin
% 	- it writes to a file with a list of things that need fixing
\newcommand{\fixthis}[1]
{%
 \marginpar{\huge \color{red} \framebox{FIX}}%
\typeout{FIXTHIS: p\thepage : #1^^J}%
}

% Define pccname command
% 	- it writes to the log file with a detail of the name-
%     this is useful for tracking names for diversity purposes
\newcommand{\pccname}[1]
{%
#1%
\typeout{PCCNAME: p\thepage : #1}%
}

% steps environment
\newlist{steps}{enumerate}{3}
\setlist[steps]{label*=($S_\arabic*$)}
% we might want to change the steps as we proceed through 
% the different topics- for example, in polynomial functions
% we may want P1, P2, ..., and in rational functions we 
% might want R1, R2, ..., etc
% 
% The following command should ease the process
\newcommand{\reformatstepslist}[1]{\setlist[steps]{label*=(${#1}_\arabic*$)}}

% properties (of exponentials, logarithms, etc) list
\newlist{props}{enumerate}{3}
\setlist[props]{label*=($L_\arabic*$)}
\newcommand{\reformatpropslist}[1]{\setlist[props]{label*=(${#1}_\arabic*$)}}

%====================================
%   BEGIN CLEVEREF
%====================================

% each of the following has two versions
%   \crefname{environmentname}{singular}{plural}, to be used mid-sentence
%   \Crefname{environmentname}{singular}{plural}, to be used at the beginning of a sentence

% standard environments
\crefname{table}{Table}{Tables}
\Crefname{table}{Table}{Tables}
\crefname{figure}{Figure}{Figures}
\Crefname{figure}{Figure}{Figures}
\crefname{section}{Section}{Sections}
\Crefname{section}{Section}{Sections}
\crefname{equation}{Equation}{Equations}
\Crefname{equation}{Equation}{Equations}
% custom environments
\crefname{problem}{Problem}{Problems}
\Crefname{problem}{Problem}{Problems}
\crefname{definition}{Definition}{Definitions}
\Crefname{definition}{Definition}{Definitions}
\crefname{example}{Example}{Examples}
\Crefname{example}{Example}{Examples}
\crefname{worksheet}{Worksheet}{Worksheets}
\Crefname{worksheet}{Worksheet}{Worksheets}
\crefname{subprobenumi}{Problem}{Problems}
\Crefname{subprobenumi}{Problem}{Problems}
\crefname{stepsi}{step}{steps}
\Crefname{stepsi}{Step}{Steps}
\crefname{propsi}{property}{properties}
\Crefname{propsi}{Property}{Properties}
% conjunction between range; e.g Figures 1.2-1.3
\newcommand{\crefrangeconjunction}{--}

% Examples:
%   \crefrange{exp:tab:findformula1}{exp:tab:findformula4}
%       gives Tables 1-4
%   \cref{exp:tab:findformula1,exp:tab:findformula4}
%       gives Tables 1 and 4
%====================================
%   END CLEVEREF
%====================================

%====================================
%   BEGIN MARGIN ENVIRONMENTS (shamelessly copied from tufte documentclass)
%====================================
\makeatletter
% Margin float environment
\newsavebox{\@tufte@margin@floatbox}
\newenvironment{@tufte@margin@float}[2][-1.2ex]%
  {\FloatBarrier% process all floats before this point so the figure/table numbers stay in order.
  \begin{lrbox}{\@tufte@margin@floatbox}%
  \begin{minipage}{\marginparwidth}%
    \def\@captype{#2}%
    \hbox{}\vspace*{#1}%
    \noindent%
  }
  {\end{minipage}%
  \end{lrbox}%
  \marginpar{\usebox{\@tufte@margin@floatbox}}%
  }
    
% Margin figure environment
\newenvironment{marginfigure}[1][-1.2ex]%
  {\begin{@tufte@margin@float}[#1]{figure}%
   \capstart\endgraf}% <-- Hyperlink jumps to start of 'marginfigure'
  {\end{@tufte@margin@float}}
    
% Margin table environment
\newenvironment{margintable}[1][-1.2ex]%
  {\begin{@tufte@margin@float}[#1]{table}%
   \capstart\endgraf}% <-- Hyperlink jumps to start of 'margintable'
  {\end{@tufte@margin@float}}
    
\makeatother
%====================================
%   END MARGIN ENVIRONMENTS
%====================================

%====================================
%   BEGIN SOLUTIONS TO PROBLEMS
%====================================
% open the answer files (short, long, and hints)
\Opensolutionfile{shortsolutions}
\Newassociation{shortsolution}{shortSoln}{shortsolutions}

\Opensolutionfile{longsolutions}
\Newassociation{longsolution}{longSoln}{longsolutions}

%\Opensolutionfile{hints}
%\Newassociation{hint}{hintSoln}{hints}
\newenvironment{hint}{\expandafter\comment}{\expandafter\endcomment}

% put titles in the solution files
\begin{Filesave}{shortsolutions}
% SHORT SOLUTIONS FILE
%
% This file is created automatically every time 111,112document.tex
% is compiled
%
% THERE IS NO POINT EDITING THIS FILE, AS IT CHANGES
% EVERY TIME YOU COMPILE!
%
% Begin by (re-)setting the chapter and section counters to 0
% which are used for hyperlinks from solutions to questions
% 
% This allows us to put Problem 1, 2, 3, etc in every chapter
% as the hyperlinks have the form 
%
% \hypertarget{soln:\theproblem:\thechapter:\thesection}{}
% \hypertarget{soln:\theproblem.\thesubproblem:\thechapter:\thesection}{}
%
\section*{Answers}
\setcounter{chapter}{0}
\setcounter{section}{0}

\end{Filesave}

\begin{Filesave}{longsolutions}
% LONG SOLUTIONS FILE
% Begin by (re-)setting the chapter counter to 0
% which is used for hyperlinks from solutions to questions
% 
% This allows us to put Problem 1, 2, 3, etc in every chapter
% as the hyperlinks have the form \hypertarget{soln:#1\thechapter}{}
\section*{Answers (long)}
\setcounter{chapter}{0}
\setcounter{section}{0}

\end{Filesave}
%\Writetofile{hints}{\protect\section*{Hints}}
%====================================
%   END SOLUTIONS TO PROBLEMS
%====================================

%====================================
%   BEGIN php stuff
%====================================
\renewcommand{\solutionextension}{php}
\Opensolutionfile{crossrefsWEB}
\Writetofile{crossrefsWEB}{<?php}

\newcommand{\phpreference}[2]{%
                    \Writetofile{crossrefsWEB}{// page \thepage}%
                    \Writetofile{crossrefsWEB}{$#1=array("#2","\thepage");}%
                    %\marginpar{%
                    %        \IfFileExists{geogebra-logo.eps}%
                    %        {%
                    %            \href{http://spot.pcc.edu/~chughes/}{\resizebox{2cm}{!}{\includegraphics{geogebra-logo.eps}}}%
                    %        }%
                    %        {%
                    %        \href{http://spot.pcc.edu/~chughes/}{\fbox{\huge{www}}}%
                    %        }%
                    %}%
                    }
%====================================
%   END php stuff
%====================================

%====================================
%   BEGIN CUSTOM THEOREMS (for problems, subproblems)
%====================================
% margin theorem
\makeatletter
\newtheoremstyle{marginpcc}%
{\item[\theorem@headerfont \llap{##1 ##2}]}%
{\item[\theorem@headerfont \llap{##1 ##2} -- ##3\theorem@separator\hskip\labelsep]}%
%{\item[\theorem@headerfont \llap{##1 ##2} -- ##3\IfEndWith{##3}{!}{}{\theorem@separator}\hskip\labelsep]}%
\newtheoremstyle{marginpccsoln}%
{\item[\theorem@headerfont \llap{##1}]}%
{\item[\theorem@headerfont \llap{##1} (##3): ]}%
\makeatother

% example (that uses an associated 'solution' )
% \newcounter{example}[section]
\theoremstyle{marginpcc}
\theorembodyfont{}
\theoremsymbol{\rlap{$\blacksquare$}}
\theoremprework{}
\theorempostwork{}
\theoremseparator{:}
\theorempreskip{\topskip}
\renewtheorem{example}{Example}%[chapter]


% associated solution for the example
\theoremstyle{marginpccsoln}
\theoremheaderfont{\itshape}
\theorembodyfont{}
\theorempostwork{}
\theoremseparator{}
\newtheorem{pccsolution}{Solution}

% make a nobreak \item, which won't allow a pagebreak
% to be put in. This is just a copy of \item and \@item 
% from ltlists.dtx (texdoc source2e), with the \addpenalty\@itempenalty
% changed to \addpenalty\@M
%
% This is based on a question, answer, and comment discussion I had
% on tex exchange: http://tex.stackexchange.com/questions/47204/definitive-guide-to-trivlists/47586#47586
\makeatletter
\def\nobreakitem{%
  \@inmatherr\nobreakitem
  \@ifnextchar [\@nobreakitem{\@noitemargtrue \@nobreakitem[\@itemlabel]}}
\def\@nobreakitem[#1]{%
  \if@noparitem
    \@donoparitem
  \else
    \if@inlabel
      \indent \par
    \fi
    \ifhmode
      \unskip\unskip \par
    \fi
    \if@newlist
      \if@nobreak
        \@nbitem
      \else
        \addpenalty\@beginparpenalty
        \addvspace\@topsep
        \addvspace{-\parskip}%
      \fi
    \else
      \addpenalty\@M
      \addvspace\itemsep
    \fi
    \global\@inlabeltrue
  \fi
  \everypar{%
    \@minipagefalse
    \global\@newlistfalse
    \if@inlabel
      \global\@inlabelfalse
      {\setbox\z@\lastbox
       \ifvoid\z@
         \kern-\itemindent
       \fi}%
      \box\@labels
      \penalty\z@
    \fi
    \if@nobreak
      \@nobreakfalse
      \clubpenalty \@M
    \else
      \clubpenalty \@clubpenalty
      \everypar{}%
    \fi}%
  \if@noitemarg
    \@noitemargfalse
    \if@nmbrlist
      \refstepcounter\@listctr
    \fi
  \fi
  \sbox\@tempboxa{\makelabel{#1}}%
  \global\setbox\@labels\hbox{%
    \unhbox\@labels
    \hskip \itemindent
    \hskip -\labelwidth 
    \hskip -\labelsep
    \ifdim \wd\@tempboxa >\labelwidth
      \box\@tempboxa
    \else
      \hbox to\labelwidth {\unhbox\@tempboxa}%
    \fi
    \hskip \labelsep}%
  \ignorespaces}
\makeatother

% set up showoddsolns and showEVEN
\newbool{showoddsolns}
\setbool{showoddsolns}{true}
\newbool{showevensolns}
\setbool{showevensolns}{true}

% this boolean is used in the problem and subproblem definition (don't change)
\newbool{showSolution}
\setbool{showSolution}{true}

% boolean for core problem or not
\newbool{coreproblemYesNo}
\setbool{coreproblemYesNo}{false}

% boolean to detect if a solution has been written
\newbool{solutionpresent}
\setbool{solutionpresent}{false}

% PROBLEM environment
% \begin{problem}[1][2]
%   ...
% \end{problem}
% 
%  [1]: description of the problem
%  [2]: core
%
% Both arguments are optional; the second argument
% can be left off, empty, or core. If it is non-empty
% and anything other than core (case-sensitive), it 
% will be ignored.
\newcounter{problem}[section]
\newcommand{\coresymbol}{$\bigstar$ }
\newlength{\coresymbollength}
\settowidth{\coresymbollength}{\coresymbol}
\newenvironmentx{problem}[2][1={},2={},usedefault]{%
    \refstepcounter{problem}%
    \Writetofile{shortsolutions}{\protect\hypertarget{soln:\theproblem:\thechapter:\thesection}{}}%
    \global\setbool{solutionpresent}{false}% default: assume no solution
    \needspace{\baselineskip}%
    \ifnumodd{\theproblem}% 
        {%
            \ifbool{showoddsolns}%
            {}% true (by default)
            {\setbool{showSolution}{false}}%
        }%
        {%
            \ifbool{showevensolns}%
            {}% true (by default)
            {\setbool{showSolution}{false}}%
        }%
    % set up the Problem environment as a list
    % the first argument of the list environment is the label, 
    % which for this environment is empty
    \begin{list}{}{%    
            \setlength{\leftmargin}{0mm}%
            \setlength{\rightmargin}{0mm}%
            \setlength{\topsep}{0mm}%
            \setlength{\partopsep}{0mm}%
            \parsep\parskip%
            \setlength{\itemsep}{-\parsep}%
            }\item% Problem X [with possible description]
    \ifstrempty{#2}{%
        %NOT core problem
        % if no second argument (core or not)
        \ifstrempty{#1}{% 
            % Problem withOUT description
            \ifbool{showSolution}%
            {%
                {\hypertarget{prob:\theproblem:\thechapter:\thesection}{}%
                 {{\bfseries Problem \hypersetup{linkcolor=black}\hyperlink{soln:\theproblem:\thechapter:\thesection}{\theproblem}}}}%
            }%
            {%
                 {\bfseries Problem \theproblem}%
            }%
        }%
        {% Problem with description
            \ifbool{showSolution}%
            {%
                {\hypertarget{prob:\theproblem:\thechapter:\thesection}{}%
                 {{\bfseries Problem \hypersetup{linkcolor=black}\hyperlink{soln:\theproblem:\thechapter:\thesection}{\theproblem} \,(#1)}}}%
            }%
            {%
                 {\bfseries Problem \theproblem\, (#1) }%
            }%
        }%
    }%
    {%  core problem
        \ifstrequal{#2}{core}{%
            \setbool{coreproblemYesNo}{true}%
            \Writetofile{shortsolutions}{\protect\setbool{coreproblemYesNo}{true}}%
            % write problem number to corefile
            \addcontentsline{prb}{}{\theproblem}%
            \ifstrempty{#1}{%  
                % core problem with no description
                \ifbool{showSolution}%
                {% 
                    {\hypertarget{prob:\theproblem:\thechapter:\thesection}{}%
                     {\hspace{-\coresymbollength}\coresymbol{\bfseries Problem \hypersetup{linkcolor=black}\hyperlink{soln:\theproblem:\thechapter:\thesection}{\theproblem}}}}%
                }%
                {% 
                    \hspace{-\coresymbollength}\coresymbol{\bfseries Problem \theproblem}%
                }%
            }%
            {% core problem with description
                \ifbool{showSolution}%
                {% with links
                    {\hypertarget{prob:\theproblem:\thechapter:\thesection}{}%
                     {\hspace{-\coresymbollength}\coresymbol{\bfseries Problem \hypersetup{linkcolor=black}\hyperlink{soln:\theproblem:\thechapter:\thesection}{\theproblem} \,(#1)}}}%
                }%
                {%  no links
                     \hspace{-\coresymbollength}\coresymbol{\bfseries Problem \theproblem\, (#1) }%
                }%
            }%
        }{ \textbf{Problem} \theproblem \, (#1) {\bf\huge\color{red}core?}}%
    }%
    \nobreakitem% body of the environment
}%
{%
    % check to see if a solution has been written
    \ifbool{solutionpresent}%
    {}% do nothing if a solution exitst
    {\fixthis{Problem \theproblem:  NEED SOLUTION}}% otherwise we need to write one
    \end{list}%
    % if this was a core problem, switch back the default
    % in the solution file
    \ifbool{coreproblemYesNo}%
    {%
        \Writetofile{shortsolutions}{\protect\setbool{coreproblemYesNo}{false}}%
    }%
    {}%
}%

% if there is a shortsolution, then toggle the boolean
\AfterEndEnvironment{shortsolution}{\global\setbool{solutionpresent}{true}}

% subproblem environment
\newcounter{subproblem}[problem]
\newlength{\subproblabel}
\newlist{subprobenum}{enumerate}{3}
\setlist[subprobenum]{leftmargin=0mm}%
\newenvironment{subproblem}[1][]{%
    \refstepcounter{subproblem}%
    \global\setbool{solutionpresent}{false}% default: assume no solution
        \ifnumodd{\theproblem}% 
        {%
            \ifbool{showoddsolns}%
            {% true (by default), set hypertarget in solution file
                \Writetofile{shortsolutions}{\protect\hypertarget{soln:{\theproblem.\thesubproblem}:\thechapter:\thesection}{}}%
            }%
            {\setbool{showSolution}{false}}%
        }%
        {%
            \ifbool{showevensolns}%
            {% true (by default), set hypertarget in solution file
                \Writetofile{shortsolutions}{\protect\hypertarget{soln:{\theproblem.\thesubproblem}:\thechapter:\thesection}{}}%
            }%
            {\setbool{showSolution}{false}}%
        }%
        \ifstrequal{#1}{core}{%
            \setbool{coreproblemYesNo}{true}%
        }%
        {%
            \setbool{coreproblemYesNo}{false}%
        }%
        \settowidth{\subproblabel}{\bfseries{\theproblem.\thesubproblem}}%
        \ifbool{coreproblemYesNo}%
        {% core problem
            % set boolean in the solutions file
            \Writetofile{shortsolutions}{\protect\setbool{coreproblemYesNo}{true}}%
            % add to list of problems per section
            \addcontentsline{prb}{}{\theproblem.\thesubproblem}%
            \ifbool{showSolution}%
            {%  hyperlink to solution
                \hypertarget{prob:{\theproblem.\thesubproblem}:\thechapter:\thesection}{}%
                \hypersetup{linkcolor=black}%
                \begin{subprobenum}[labelwidth=\subproblabel,label=\llap{$\bigstar$ }\bfseries\protect\hyperlink{soln:{\theproblem.\thesubproblem}:\thechapter:\thesection}{\theproblem.\thesubproblem},ref=\theproblem.\thesubproblem]
                \item% body of the environment
            }%
            {%  no hyperlink
                \begin{subprobenum}[labelwidth=\subproblabel,label=\llap{$\bigstar$ }\bfseries\theproblem.\thesubproblem,ref=\theproblem.\thesubproblem]
                \item% body of the environment
            }%
        }%
        {% NOT core problem
            \ifbool{showSolution}%
            {%  hyperlink to solution
                \hypertarget{prob:{\theproblem.\thesubproblem}:\thechapter:\thesection}{}%
                \hypersetup{linkcolor=black}%
                \begin{subprobenum}[labelwidth=\subproblabel,label=\bfseries\protect\hyperlink{soln:{\theproblem.\thesubproblem}:\thechapter:\thesection}{\theproblem.\thesubproblem},ref=\theproblem.\thesubproblem]
                \item% body of the environment
            }%
            {%  no hyperlink
                \begin{subprobenum}[labelwidth=\subproblabel,label=\bfseries\theproblem.\thesubproblem,ref=\theproblem.\thesubproblem]
                \item% body of the environment
            }%
        }%
        \hypersetup{linkcolor=blue}%
        }%
        {%
            % check to see if a solution has been written
            \ifbool{solutionpresent}%
            {}% do nothing if a solution exitst
            {\fixthis{Problem \theproblem.\thesubproblem:  NEED SOLUTION}}% otherwise we need to write one
            % if this was a core problem, switch back the default
            % in the solution file
            \ifbool{coreproblemYesNo}%
            {%
                \Writetofile{shortsolutions}{\protect\setbool{coreproblemYesNo}{false}}%
            }%
            {}%
        \end{subprobenum}}


% another type of theorem-break environment
% this one is very similar to the break environment, except
% it removes the () from the statement
\makeatletter
\newtheoremstyle{pccbreak}%
{\item[\rlap{\vbox{\hbox{\hskip\labelsep \theorem@headerfont%
##1\ ##2\theorem@separator}\hbox{\strut}}}]}%
{\item[\rlap{\vbox{\hbox{\hskip\labelsep \theorem@headerfont%
##3\theorem@separator}\hbox{\strut}}}]}
\makeatother

% special comment
\theoremstyle{pccbreak}
\theoremheaderfont{\bfseries}
\theoremsymbol{}
\theoremseparator{}
\theoremprework{}
%\setlength{\theorempreskipamount}{0pt}
%\setlength{\theorempostskipamount}{0pt}
\theorempostwork{}
\makeatletter
\if@tempswa % draft option is active
    % make the specialcomment environment without colour, no frame
    \newtheorem{pccspecialcomment}{}%
\else%
    \newmdtheoremenv[outerlinewidth=3,
        innerlinewidth=2,linecolor=gray,
        backgroundcolor=blue!20,%
        innerlinecolor=blue!50,outerlinecolor=red!50,innertopmargin=0pt,%
        splittopskip=\topskip,skipbelow=0pt,%
        ]{pccspecialcomment}{}%[chapter]
\fi
\makeatother

% definition
\theoremstyle{break}
\theoremsymbol{}
\theoremheaderfont{\bfseries}
\theoremprework{}
\theorempostwork{}
\theoremseparator{}
\makeatletter
\if@tempswa % draft option is active
    % make the Definition environment without colour, no frame
    \renewtheorem{definition}{Definition}%
\else%
    \csundef{definition}
    \csundef{definition*}
    \newmdtheoremenv[outerlinewidth=3,
    innerlinewidth=2,linecolor=gray,leftmargin=60,%
    rightmargin=40,
    backgroundcolor=yellow!40,%
    innerlinecolor=blue!50,outerlinecolor=red!50,innertopmargin=0pt,%
    splittopskip=\topskip,%skipbelow=\baselineskip,%
    skipabove=\baselineskip,ntheorem]{definition}{Definition}%[chapter]
\fi
\makeatother
%====================================
%   END CUSTOM THEOREMS (for examples, problems)
%====================================

%====================================
%   BEGIN CORE PROBLEMS
%====================================
\newcounter{probSectCounter}
\setcounter{probSectCounter}{-1}
\newcounter{probCounter}
\setcounter{probCounter}{0}%
\newcounter{echo}
\newread\File

\makeatletter

% need a command to bring the \contentsline
% into the problemlist environment
\let\pcc@contentslist\contentsline%
\newenvironment{problemlist}%
{%
    % renew the \contentsline command so that it just gives 
    % the problem number as a hyperlink (and no page number)
    % 
    % if we renewed the command outside of the environment, it would
    % affect \tableofcontents, \minitoc, and perhaps more- bad!
    \let\contentsline\pcc@contentslist%
    \let\Contentsline\contentsline%
    \renewcommand\contentsline[4]{%
    \StrCount{##2}{heading}[\heading]%
    \ifnumcomp{\heading}{>}{0}%
    {%
        \StrCount{##2}{heading\theprobSectCounter}[\heading]%
        \ifnumcomp{\heading}{>}{0}%
        {%
            \setcounter{echo}{1}%
            \setcounter{probCounter}{0}%
        }%
        {%
            \setcounter{echo}{0}%
        }%
    }%
    {%
        \ifnumequal{\theecho}{1}%
        {%
            \stepcounter{probCounter}%
            \ifnumequal{\theprobCounter}{1}%
            {%
                % if we're at the beginning, no comma
                Core problems in this section (\coresymbol\unskip): \Contentsline{##1}{##2}{}{##4}%
            }%
            {%
                % otherwise put a comma and a space
                \unskip, \Contentsline{##1}{##2}{}{##4}%
            }%
        }%
        {}%
    }%
    }%
}%
{}

\def\listcoreproblems{%
% open the appropriate file do everything in a 
% new environment so that we can renew the 
% \contentsline command locally 
\begin{problemlist}
\par%
\openin\File=\jobname.prb%
\loop\unless\ifeof\File%
\read\File to\fileline%
\fileline%
\repeat%
\closein\File%
\end{problemlist}
}%

% enable the \jobname.prb file
% (hacked from the ntheorem package)
\def\prb@enablelistofproblems{%
\begingroup%
\makeatletter%
\if@filesw%
\expandafter\newwrite\csname tf@prb\endcsname%
\immediate\openout \csname tf@prb\endcsname \jobname.prb\relax%
\fi%
\@nobreakfalse%
\endgroup}%

% enable the \jobname.prb<counter> files at the end
% of the document
\AtEndDocument{\prb@enablelistofproblems}

\makeatother

%====================================
%   END CORE PROBLEMS
%====================================

%====================================
%   BEGIN OUTCOMES
%====================================
\newenvironment{outcomes}%
{%
\itshape%
% define a new environment local to outcomes- the outcomelist
% environment will not work outside of outcomes!
\newenvironment{outcomelist}{\begin{itemize}[topsep=-2mm]}{\end{itemize}}%
%[label=$\triangleright$]
  {Section Themes, Concepts, Issues, Competencies, and Skills:}%
}%
{%
\listcoreproblems%
%\hfill\rule{0.5\textwidth}{1pt}\needspace{\baselineskip}%
%\needspace{\baselineskip}%
%\hfill $\Box$\needspace{\baselineskip}%
%\end{framed}%
}%
%====================================
%   END OUTCOMES
%====================================


%====================================
%   BEGIN TABLE STYLES
%====================================
\renewcommand{\heavyrulewidth}{0.2em}
\renewcommand{\lightrulewidth}{0.1em}
\renewcommand{\cmidrulewidth}{0.1em}
% horizontal lines
\renewcommand{\aboverulesep}{0cm} 
\renewcommand{\belowrulesep}{0cm} 
\newcommand*\heading[1]{\multicolumn{1}{c}{#1}}
\newcommand*\beforeheading{\toprule}
\newcommand*\afterheading{\midrule}
%\newcommand*\normalline{\hline}
\newcommand*\normalline{}
\newcommand*\lastline{\bottomrule}
%====================================
%   END TABLE STYLES
%====================================

%====================================
%   BEGIN essentialskills, tryityourself, exercises
%====================================
\newenvironment{doyouunderstand}%
{\FloatBarrier\setlist[subprobenum]{align=left,leftmargin=!}%
    \begin{list}{}{%    the first argument of the list environment is the label
            \setlength{\leftmargin}{0mm}%
            \setlength{\rightmargin}{0mm}%
            \setlength{\topsep}{0mm}%
            \setlength{\partopsep}{0mm}%
            \parsep\parskip%
            \setlength{\itemsep}{-\parsep}%
            }%
            \needspace{4\baselineskip}\item\llap{\large\textbf{$\bigstar$ try it yourself $\bigstar$} }%
            \vskip-1.8\baselineskip\rule{\textwidth}{2pt}\vskip-16pt\rule{\textwidth}{0.5pt}\vskip-5pt%
            \nobreakitem% body of the environment
}%
{\nobreakitem \hfill \itshape{make sure you try it!}\nobreakitem\vskip-9pt\rule{\textwidth}{0.5pt}\vskip-15pt\rule{\textwidth}{2pt}\end{list}}%

\newenvironment{essentialskills}%
{%
    \begin{list}{}{%    the first argument of the list environment is the label
            \setlength{\leftmargin}{0cm}%
            \setlength{\rightmargin}{0mm}%
            \setlength{\topsep}{0mm}%
            \setlength{\partopsep}{0mm}%
            \parsep\parskip%
            \setlength{\itemsep}{-\parsep}%
            }%
          \item[\llap{\textbf{\large essential skills}  }]\raisebox{3pt}{\rule{\textwidth}{2pt}}%
            \nobreakitem {\itshape{The following problems contain prerequisite skills that are essential for success. Make sure that you 
can complete them before moving on!}}%
            \item% body of the environment
\setlist[subprobenum]{align=left}}%
{\nobreakitem \raisebox{3pt}{\rule{\textwidth}{2pt}}%
\end{list}}%

\newenvironment{exercises}%
{\FloatBarrier\setlist[subprobenum]{align=left,leftmargin=!}%
\begin{adjustwidth}{-4cm}{0cm}%
\needspace{4\baselineskip}\colorbox{gray!80}{\begin{minipage}{\linewidth-6pt}\bfseries\Large Exercises\end{minipage}}}%
{\end{adjustwidth}}

%====================================
%   END essentialskills, doyouunderstand, exercises
%====================================

%====================================
%   BEGIN CUSTOMIZE SECTIONS
%====================================
% useful link: http://tex.stackexchange.com/questions/32495/linking-the-section-text-to-the-toc

% custom chapter
\titleformat{\chapter}[display]
{\normalfont\Large\filcenter\bf}
{\titlerule[1pt]%
    \vspace{1pt}%
    \titlerule
    \typeout{PCCNAME: Chapter \thechapter: #1^^J}% for use with PCCNAME (diversity)
    \LARGE\MakeUppercase{#1} {\LARGE\thechapter}%
}
{1pc}
{\titlerule
\Huge
    % Executed BEFORE the body of the chapter (see titlesec documentation)
    % write to the solutions files to step the chapter counter
    \Writetofile{shortsolutions}{\protect\stepcounter{chapter}\protect\setcounter{section}{0}}%
    \Writetofile{longsolutions}{\protect\stepcounter{chapter}}%
    }

% custom section
\newcommand{\sectionbreak}{\clearpage} % start each section on new page
\titleformat{\section}
{\normalfont\Large\bfseries}
{\llap{\thesection\hskip 9pt}#1}
{0pt}
{%
      % Executed BEFORE the body of the section (see titlesec documentation)
      % redefine the \section command to increment probSectCounter
      % note that this could also be done for \subsection if desired
      \Writetofile{shortsolutions}{\protect\stepcounter{section}\protect\subsection*{Solutions for problems in Section \protect\thesection}}%
      \thispagestyle{plain}%
      \stepcounter{probSectCounter}%
      \addcontentsline{prb}{}{heading\theprobSectCounter}%
}

% section*
\titleformat{name=\section,numberless}
  {\normalfont\Large\bfseries}{}{0pt}{#1}

% custom subsection
\titleformat{\subsection}
{\normalfont\large\bfseries}
{\llap{\thesubsection\hskip 9pt}#1}
{0pt}
{}

% subsection*
\titleformat{name=\subsection,numberless}
  {\normalfont\large\bfseries}{}{0pt}{#1}

% investigations
\titleclass{\investigation}{straight}[\paragraph]
\titleformat{name=\investigation,numberless}[runin]{}{\llap{%
\vbox to -5pt{\hbox{% http://tex.stackexchange.com/questions/23845/is-there-a-vertical-equivalent-for-rlap
\begin{tikzpicture} 
	\draw[rotate=45,yshift=-8ex,xshift=-4ex,left color=brown!20,right color=brown] (0,-0.25) rectangle (.25,.65);
    \node[circle,draw=black,drop shadow={shadow yshift=1ex,shadow xshift=1ex},fill=white] (glass) at (-0.5,-0.25) {\phantom{investig}};
    \node  at (1,0) {\sffamily\bfseries\Large Investigations };
   \end{tikzpicture}}\vss}
}}{0.5em}{}
\titlespacing*{name=\investigation,numberless}{0pt}{3.5ex plus 1ex minus .2ex}{0pt}

%% exercise title
%\titleclass{\exertitle}{straight}[\section]
%\titleformat{name=\exertitle,numberless}{}{\colorbox{silver}{\begin{minipage}{\linewidth-6pt}\bfseries\Large Exercises \end{minipage}}}{0.5em}{}
%\titlespacing*{name=\exertitle,numberless}{0pt}{3.5ex plus 1ex minus .2ex}{0pt}

% From the titlesec package
% \titlespacing{command}{left spacing}{before spacing}{after spacing}[right]
% spacing: how to read {12pt plus 4pt minus 2pt}
%           12pt is what we would like the spacing to be
%           plus 4pt means that TeX can stretch it by at most 4pt
%           minus 2pt means that TeX can shrink it by at most 2pt
%       This is one example of the concept of, 'glue', in TeX

\titlespacing{\chapter}{0pt}{*4}{-0.1cm}
\titlespacing\section{0pt}{12pt plus 4pt minus 2pt}{-5pt plus 2pt minus 2pt}
\titlespacing\subsection{0pt}{12pt plus 4pt minus 2pt}{-6pt plus 2pt minus 2pt}
\titlespacing\subsubsection{0pt}{12pt plus 4pt minus 2pt}{-6pt plus 2pt minus 2pt}

%====================================
%   END CUSTOMIZE SECTIONS
%====================================

%====================================
%   BEGIN CUSTOM SHORTCUTS
%====================================

%Simply to shorten a long command
\newcommand{\dd}{\displaystyle}

%Set-builder notation
\newcommandx\setbuilder[2][2]{%
			\ifstrempty{#2}{%
				\mleft\{#1\mright\}%
				}%
				{%
					\mleft\{#1\;\middle|\;#2\mright\}%
				}%
}%

%We produce this often in the functions chapter
\newcommand\sq[1][]{%
        \operatorname{sqrt}%
        \ifstrequal{#1}{(}{% if we want ()
            (\phantom{x})%
        }%
        {% else
            \ifstrempty{#1}{% if #1 is empty then do nothing
            }%
            {% otherwise put #1 in ()
                (#1)%
                }%
        }%
        }

\sisetup{quotient-mode = fraction,per-mode=fraction,fraction-function = \nicefrac}
	\DeclareSIUnit{\inch}{in}													
	\DeclareSIUnit{\trillion}{trillion}
	\DeclareSIUnit{\Fahrenheit}{\SIUnitSymbolDegree{}F}
	\DeclareSIUnit{\centunit}{\textcent}
	\DeclareSIUnit{\cubiccm}{cc}
	\DeclareSIUnit{\million}{million}
	\DeclareSIUnit{\foot}{ft}
	\DeclareSIUnit{\mile}{mi}
	\DeclareSIUnit{\decade}{decade}
	\DeclareSIUnit{\pound}{lb}
	\DeclareSIUnit{\minute}{min}
	\DeclareSIUnit{\year}{year}

%====================================
%   END CUSTOM SHORTCUTS
%====================================


%====================================
%   BEGIN DOCUMENT
%====================================
\begin{document}

% needed for the mini-tableofcontents
\dominitoc
\faketableofcontents

\fancyhf{} % delete current header and footer
\fancyhead[LE,RO]{\bfseries\thepage}
\fancyhead[LO,RE]{\tiny\rightmark}
\fancyheadoffset[LE,LO]{4cm}

\pagestyle{fancy}
%\include{coverpage}
\typeout{************************************************}
\typeout{Chapter 1 Functions}
\typeout{************************************************}
%
\chapter{Functions}
%
\minitoc
%
\typeout{************************************************}
\typeout{Section 1.1 The Basics of Function Vocabulary}
\typeout{************************************************}
%
\section{The Basics of Function Vocabulary}
%
\begin{outcomes}
\begin{outcomelist}
\item You will understand the definition of a function.%
\item You will be able to use standard notation for functions
	                correctly, and recognize when notation has been used incorrectly.%
\item You will recognize some everyday examples of functions.%
\end{outcomelist}
\end{outcomes}
Most of us are familiar with the $\sqrt{\phantom{x}}$ symbol.
		This symbols is used to turn numbers into their square roots. Sometimes it's
		simple to do this on paper or in our heads, and sometimes it helps a lot to
		have a calculator. We can see some calculations in \cref{fun-tab-squareroots}.
%
\begin{margintable}
\centering
\captionof{table}{Values of $\sqrt{x}$}
\label{fun-tab-squareroots}
\begin{tabular}{r@{}c@{}l}
\beforeheading 
\afterheading 
$\sqrt{\num{9}}$&${}={}$&\num{3}\\\normalline
$\sqrt{\num{1/4}}$&${}={}$&\num{1/2}\\\normalline
$\sqrt{\num{2}}$&${}\approx{}$&\num{1.41}\ldots\\\lastline
\end{tabular}
\end{margintable}
%
\par The $\sqrt{\phantom{x}}$ symbol represents a \emph{process}; it's a way for us to
		turn numbers into other numbers. This idea of finding some numbers based on other numbers is
		fundamental to science and mathematics that use college-level algebra. 
%
\begin{definition}[Function]
A function is a process for turning numbers into (potentially) different numbers. 
			It's important that any input consistently produces the same output.\end{definition}
%
\par This definition is so broad that you probably use functions all the time.
%
\begin{example}
Think about each of these examples. How do they fit the defintion of a function?
%
\begin{itemize}
\item If you use the year a person was born to determine how old they are, you are using a function.%
\item If you look up the Kelly Blue Book value of a Mazda Proteg\'{ e} based on how old it is, 
				you are using a function.%
\item If you use the the amount of money that you have on you to determine how many beers you could buy for 
				your friends at the bar, you are using a function.%
\end{itemize}
%
\end{example}
%
\par The process of using $\sqrt{\phantom{x}}$ to change numbers might feel more ``mathematical''
		than these examples. Let's continue thinking about $\sqrt{\phantom{x}}$ for now, since
		it's a formula-like symbol that we are familiar with. Even though we live in the age of computers,
		this symbol is not found on most
		keyboards. This doesn't stop people from using computers to calculate square roots though. Computer
		technicians write $\sq[(]$ when they want to compute a square root, as we see in \cref{fun-tab-sqrts}.
%
\begin{margintable}
\centering
\captionof{table}{Values of $\sq[x]$}
\label{fun-tab-sqrts}
\begin{tabular}{r@{}c@{}l}
\beforeheading 
\afterheading 
$\sq[\num{9}]$&${}={}$&\num{3}\\\normalline
$\sq\left(\num{1/4}\right)$&${}={}$&\num{1/2}\\\normalline
$\sq[\num{2}]$&${}\approx{}$&\num{1.41}\ldots\\\lastline
\end{tabular}
\end{margintable}
%
\par The parentheses in $\sq[(]$ are very important. To see why, try to put yourself in the
          ``mind'' of a computer, and look closely at $\sq\num{16}$. The computer will recognize $\sq$
		and know that it needs to compute a square root. But computers are very picky with how they interpret input, and 
		they might not see the entire number \num{16}. A computer might read as far as $\sq\num{1}$ and think that it needs to compute this.
		That would leave $1$ with a ``6'' character still hanging. The final result would be \num{16}, and we know we inteded the  
		result to be $4$. And so the purpose of the parentheses in $\sq[\num{16}]$ is 
		to denote exactly what number needs to be operated on.
%
\par This use of $\sq[(]$ serves as a model for the standard notation that is used all over the world to
        	write down most functions. By having a standard notation for communicating about functions,
        	people in China, Venezuela, Senegal, and the United States can all communicate mathematics
        	with each other more easily.
%
\par Functions have their own names. We've seen a function named $\sq$, but any name you can
        	imagine is allowable. In the sciences, it is common to name functions with whole words,
        	like $\operatorname{weight}$ or $\operatorname{health\_index}$. In mathematics, we often
            abbreviate such function names to $w$ or $h$. And of course, since the word ``function''
            starts with ``f'', we will often name a function $f$.
%
\par It's crucial to continue reminding ourselves that functions are \emph{processes} for
        	changing numbers; they are not numbers themselves. And that means that we have a potential
        	for confusion that we need to stay aware of. In some contexts, the symbol $t$ might
        	represent a variablea number that is represented by a letter. For example, it might represent 
		how much time has passed since something started. But in other contexts, $t$
        	might represent a functiona process for changing numbers into other numbers. For example, if you 
		have a year in mind, $t$ might be the function that tells you how many tornadoes there were that year. 
		So $t$ would be a process for turning years into numbers of tornadoes. By
        	staying conscious of the context of an investigation, we avoid confusion.
%
\par Next we need to discuss how we go about using a function's name.
%
\begin{specialnote}[Function notation]
The standard notation for referring to functions involves giving the function itself a name, and then writing
			\begin{displaymath}\begin{array}{cc}
                                \text{name}\\
                                \text{of}\\
                                \text{function}
                        \end{array}
                        \left(
                        \begin{array}{cc}
                                \\
                                \text{input}\\
                                \\
                        \end{array}\right)\end{displaymath}\end{specialnote}
%
\begin{example}
$f(\num{13})$ is pronounced ``f of 13''. The word ``of'' is very important,
        		because it reminds us that $f$ is a process and we are about to apply that
        		process to the input value \num{13}. So $f$ is the function, \num{13} is the
        		input, and $f(\num{13})$ is the output we'd get from using \num{13} as input.
%
\par $f(x)$ is pronounced ``f of x''. This is just like the previous example,
        		except that the input is not some specific number. The value of $x$ could be
        		\num{13} or any other number. Whatever $x$'s value, $f(x)$ means the corresponding
        		output from the function $f$.
%
\par $\operatorname{BudgetDeficit}(2009)$ is pronounced ``BudgetDeficit of 2009''.
        		This is probably a function that takes a year as input, and gives that
        		year's federal budget deficit as output. The process here of changing a year
        		into a dollar amount might not involve any mathematical formula, but rather
        		looking up information from the Congressional Budget Office's website.
%
\par $\operatorname{Celsius}(F)$ is pronounced ``Celsius of F''. This is probably
        		a function that takes a Fahrenheit temperature as input and gives the
        		corresponding Celsius temperature as output. Maybe a formula is used to do this;
        		maybe a chart or some other tool is used to do this. Here, $\operatorname{Celsius}$
        		is the function, $F$ is the input variable, and $\operatorname{Celsius}(F)$ is the output from the function.
%
\end{example}
%
\begin{specialnote}[Function notation (continued)]
While a function has a name like $f$, and the input to that function often
                	has a variable name like $x$, the expression $f(x)$ represents the output of
                	the function. To be clear, $f(x)$ is \emph{not} a function. Rather, $f$ is a
                	function, and $f(x)$ its the output when the number $x$ is used as input.\end{specialnote}
%
\par As mentioned earlier, we need to remain conscious of the context of any symbol we
       		are using. It's possible for $f$ to represent a function (a process), but it's also
       		possible for $f$ to represent a variable (a number). Similarly, parentheses might
       		indicate the input of a function, or they might indicate that two numbers need to
       		be multiplied. It's up to our judgment to interpret mathematical expressions in the
       		right context. Consider the expression $a(b)$. This could easily mean the output of
       		a function $a$ with input $b$. It could also mean that two numbers $a$ and $b$ need
       		to be multiplied. It all depends on the context in which these symbols are being used.
%
\begin{checkpoint}
\begin{problem}
Describe your own example of a function using experience from your life. You will need some 
                    kind of input variable, like ``number of years since 2000'' or ``weight of a bowling ball''. You will 
				need a process for turning that number into a different kind of number. The process does not need to
                		involve a formula; a verbal description would be fine.
%
\par Give your function a name. Write the symbol that you would use to represent
                		input. Write the symbol(s) that you would use to represent output.
%
\begin{longsolution}
%
Answers will vary.
%
\end{longsolution}
%
\begin{shortsolution}
%
Answers will vary.
%
\end{shortsolution}
%
\end{problem}
%
\end{checkpoint}
%
\par Sometimes it's helpful to think of a function as a machine.
%
\par This illustrates how complicated functions are. A number is just a number. But a function has the capacity to 
		take in all kinds of different numbers into it's hopper (feeding tray) and do different things to each of them. So
		functions are complicated.
%
\typeout{************************************************}
\typeout{Subsection 1.1.1 Tables and Graphs}
\typeout{************************************************}
%
\subsection{Tables and Graphs}
%
Since functions are potentially so complicated, we seek out new ways to understand
        		them better. Two basic tools for understanding a function better are tables and graphs.
%
\begin{example}\label{fun-ex-BudgetDeficit}
Consider the function $\operatorname{BudgetDeficit}$, that takes in a year as
        			its input and outputs the US federal budget deficit for that year. For example,
        			the Congressional Budget Office's website tells us that
        			$\operatorname{BudgetDeficit}(2009)$ is $\SI{1.41}[\$]{\trillion}$. If we'd like
        			to understand this function better, we might make a table of all the inputs and
        			outputs we can find. Using the CBO's 
				website\footnote{\href{http://data.bls.gov/timeseries/LNS14000000}{Congressional Budget Office}}, we put together 
				\cref{fun-tab-BudgetDeficit}.
%
\begin{table}[thb]
\centering
\caption{}
\label{fun-tab-BudgetDeficit}
\begin{tabular}{S[table-format=4.0]S[table-format=1.2]}
\beforeheading 
\heading{input}&\heading{output}\\\normalline
\heading{$x$ (year)}&\heading{$\operatorname{BudgetDeficit}(x)$ ($\si{\$\trillion}$)}\\\normalline
\afterheading 
2007&0.16\\\normalline
2008&0.46\\\normalline
2009&1.41\\\normalline
2010&1.29\\\normalline
2011&1.30\\\lastline
\end{tabular}
\end{table}
%
\par How is this table helpful? There are things about the function that we can see now by looking at the numbers in this table.
%
\begin{itemize}
\item We can see that the budget deficit has grown by quite a bit over the entire five-year period.%
\item We can see that there was a particularly large jump in 2008.%
\item We can see that the deficit reduced by a little bit between 2009 and 2010, and then remained stable.%
\end{itemize}
%
\par These observations serve to help us understand the function $\operatorname{BudgetDeficit}$ a little better.
%
\end{example}
%
\begin{example}\label{fun-ex-sqrttable}
Let's return to our example of the function $\sq$. Tabulating some inputs and outputs reveals \cref{fun-tab-sqrtexample}.
%
\begin{table}[thb]
\centering
\caption{}
\label{fun-tab-sqrtexample}
\begin{tabular}{S[table-format=1.0]c}
\beforeheading 
\heading{input}&\heading{output}\\\normalline
\heading{$x$}&\heading{$\sq[x]$}\\\lastline
\afterheading 
0&\num{0}\\\normalline
1&\num{1}\\\normalline
2&$\approx\num{1.41}$\\\normalline
3&$\approx\num{1.73}$\\\normalline
4&\num{2}\\\lastline
\end{tabular}
\end{table}
%
\par How is this table helpful? Here are some observations that we can make now.
%
\begin{itemize}
\item We can see that when input numbers increase, so do output numbers.%
\item We can see even though outputs are increasing, they increase by less and less with each step forward in $x$.%
\end{itemize}
%
\par These observations help us understand $\sq$ a little better. For instance, based on these observations which do you think is larger: 
				the difference between $\sq[23]$ and $\sq[24]$, or the difference between $\sq[24]$ and $\sq[25]$?
%
\end{example}
%
\par Another powerful tool for understanding functions better is a graph.
        		Given a function $f$, one way to make its graph is to take a table of input and
        		output values, and read each row as the coordinates of a point in the $xy$-plane.
%
\begin{example}\label{fun-ex-budgetgraph}
Returning to the function $\operatorname{BudgetDeficit}$ that we studied in \cref{fun-ex-BudgetDeficit}, in order to make a 
				graph of this function we view \cref{fun-ex-BudgetDeficit} as a list of points with $x$ and $y$ coordinates, as in 
				\cref{fun-tab-BudgetDeficitCoords}. We then plot these points on a set of coordinate axes, as in \cref{fun-fig-BudgetDeficit}. 
				The points have been connected with a curve so that we can see the overall pattern given by the progression of points.
        			Since there was not any actual data for inputs in between any two years, the curve is dashed. In other words, this curve is dashed because 
				it just represents someone's best guess as to how to connect the plotted points. Only the plotted points themselves are precise.
%
\begin{table}[thb]
\centering
\caption{}
\label{fun-tab-BudgetDeficitCoords}
\begin{tabular}{*{1}{c}}
\beforeheading 
(input, output)\\\normalline
$(x,\operatorname{BudgetDeficit}(x))$\\\lastline
\afterheading 
$(2007, 0.16)$\\\normalline
$(2008, 0.46)$\\\normalline
$(2009, 1.41)$\\\normalline
$(2010, 1.30)$\\\normalline
$(2011, 1.29)$\\\lastline
\end{tabular}
\end{table}
%
\par How has this graph helped us to understand the function better? All of the observations that we made in \cref{fun-ex-BudgetDeficit} are perhaps even
        			more clear now. For instance, the spike in the deficit between 2008 and 2009 is now visually apparent. Seeking an explanation for this spike, we recall
        			that there was a financial crisis in late 2008. Revenue from income taxes dropped at the same time that federal money was spent to prevent further losses.
%
\end{example}
%
\begin{example}\label{fun-ex-sqrtgraph}
Now let's construct a graph for $\sq$. Tabulating inputs and outputs gives the points in \cref{fun-tab-sqrtCoords}, which in turn gives us the 
				graph in \cref{fun-fig-sqrt}.
%
\begin{table}[thb]
\centering
\caption{}
\label{fun-tab-sqrtCoords}
\begin{tabular}{*{1}{c}}
\beforeheading 
(input, output)\\\normalline
$(x,\sq(x))$\\\lastline
\afterheading 
$(0,0)$\\\normalline
$(1,1)$\\\normalline
$\approx(2,1.41)$\\\normalline
$\approx(3,1.73)$\\\normalline
$(4,2)$\\\lastline
\end{tabular}
\end{table}
%
\par Just as in the previous example, we've plotted points where we have concrete coordinates, and then we have made our best attempt to connect those points
        			with a curve. Unlike the previous example, here we believe that points could continue to be computed and plotted indefinitely to the right, and so we
        			have added an arrowhead to the graph.
%
\par What has this graph done to improve our understanding of $\sq$?  As inputs ($x$-values) increase, the outputs ($y$-values) increase too, 
				although not at the same rate. In fact we can see that our graph is steep on its left, and less steep as we move to the right. This confirms our earlier 
				observation in \cref{fun-ex-sqrttable} that outputs increase by smaller and smaller amounts as the input increases.
%
\end{example}
%
\begin{specialnote}[The graph of a function]
Given a function $f$, when we refer to a \emph{graph of $f$} we are \emph{not} referring to an entire picture, like 
				\cref{fun-fig-sqrt}. A graph of $f$ is only \emph{part} of that picture: the curve and the points that it connects.
				Everything else: axes, tick marks, the grid, labels, and the surrounding white space is just useful decoration, so that we can read the graph more easily.
%
\par It is also common to refer to the graph of $f$ as the graph of the \emph{equation} $y=f(x)$. However we should never refer to ``the graph of $f(x)$''.
				That would indicate a fundamental misunderstanding of our notation. We have decided that $f(x)$ is the output for a certain input $x$. That means that 
				$f(x)$ is just a number; a relatively uninteresting thing compared to $f$ the function, and not worthy of any two-dimensional picture.
%
\end{specialnote}
%
\par While it is important to be able to make a graph of a function $f$, we also need to be capable of looking at a graph and reading it well. 
			A graph of $f$ provides us with helpful specific information about $f$; it tells us what $f$ does to its input values. When we were making graphs, 
			we plotted points of the form \begin{displaymath}(\text{input},\text{output})\text{.}\end{displaymath} Now given a graph of $f$, we interpret coordinates in the same way.
%
\begin{figure}[!htbp]
\centering

                    \begin{tikzpicture}
                        \begin{axis}
                        \addplot{x};
                        \end{axis}
                    \end{tikzpicture}
                  \caption{caption text goes here}
\label{fun-fig-readgraph}
\caption{caption text goes here}
\label{fun-fig-readgraph}
\end{figure}
%
\begin{figure}[!htbp]
\centering

                    \begin{tikzpicture}
                        \begin{axis}
                        \addplot{x};
                        \end{axis}
                    \end{tikzpicture}
                  \caption{caption text goes here}
\caption{caption text goes here}
\end{figure}
%
\par In \cref{fun-fig-readgraph} we have a graph of a function $f$. If we wish to find $f(1)$, we recognize that \num{1} is being used as an input.
			So we would want to find a point of the form $(1,\phantom{y})$. Seeking out $x$-coordinate \num{1} in \cref{fun-fig-readgraph}, we find that 
			the only such point is $(1,2)$. Therefore the output for \num{1} is \num{2}; in other words $f(1)=2$.
%
\begin{checkpoint}
\begin{problem}
Use the graph of $f$ in \cref{fun-fig-readgraph} to find $f(0)$, $f(3)$, and $f(4)$.
%
\begin{longsolution}
%
 To find $f(0)$, locate $0$ on the $x$-axis. Moving up or down from there, the only place you can meet the graph
					of $f$ is at $(0,0.5)$. So $f(0)=0.5$.
%
\par  To find $f(3)$, locate $3$ on the $x$-axis. Moving up or down from there, the only place you can meet the graph
                                        of $f$ is at $(3,3)$. So $f(3)=3$.
%
\par  To find $f(4)$, locate $4$ on the $x$-axis. Moving up or down from there, the only place you can meet the graph
                                        of $f$ is at $(4,2)$. So $f(4)=2$.
%

%
\end{longsolution}
%
\begin{shortsolution}
%
$f(0)=0.5$, $f(3)=3$, and $f(4)=2$
					
%
\end{shortsolution}
%
\end{problem}
%
\end{checkpoint}
%
\cref{test-table}
\begin{figure}[h]
          %
				\begin{minipage}{.50\textwidth}
\centering
                    \begin{tikzpicture}
                        \begin{axis}
                        \addplot{x};
                        \end{axis}
                    \end{tikzpicture}
                  \caption{caption text goes here}
\end{minipage}%
          %
          %
				\begin{minipage}{.50\textwidth}
\centering
                    \begin{tikzpicture}
                        \begin{axis}
                        \addplot{x};
                        \end{axis}
                    \end{tikzpicture}
                  \caption{caption text goes here}
\end{minipage}%
          %
         \end{figure}

\begin{figure}[h]
          %
				\begin{minipage}{.50\textwidth}
\centering
                    \begin{tikzpicture}
                        \begin{axis}
                        \addplot{x};
                        \end{axis}
                    \end{tikzpicture}
                  \caption{caption text goes here}
\end{minipage}%
          %
          %
				\begin{minipage}{.50\textwidth}
\centering
                    \begin{tikzpicture}
                        \begin{axis}
                        \addplot{x};
                        \end{axis}
                    \end{tikzpicture}
                  \caption{caption text goes here}
\end{minipage}%
          %
         \end{figure}
\typeout{************************************************}
\typeout{Section 1.2 foo}
\typeout{************************************************}
%
\section{foo}
%
here is another section
%
\typeout{************************************************}
\typeout{Section 1.3 bar}
\typeout{************************************************}
%
\section{bar}
%
here is another section
%

%%+*** 111,112document.tex
% arara: indent: {overwrite: on, trace: on, localSettings: yes}
%===================================
%
%   Last edited: Jordan
%                3/4/13 (v80)
%
%===================================
\chapter{Exponential Functions}
\minitoc
\section{Introduction}
\begin{outcomes}
	\begin{outcomelist}
		\item Explore increasing and decreasing functions, particularly in the context of concavity;
		\item Determine a function's concavity based on a table of values, a graph, or a description. 
	\end{outcomelist}
\end{outcomes}
In our mathematical adventures so far we have studied linear, quadratic, 
and radical functions. The simplicity of these functions is useful when 
introducing new concepts such as transformations, composition, and 
inverse functions; but it is somewhat restrictive when we wish to 
consider interesting real-world application problems. 

For example, let's say that we were interested in modeling the 
temperature of a hot cup of coffee since it was first poured. We cannot 
write a formula for such a model yet, but perhaps \cref{exp:fig:motivatecool} depicts a reasonable
approximation of the graph of it. Or perhaps
we would like to model the growth in population of the world; again, 
we can not write a formula for such a model at this stage, but you 
might agree that \cref{exp:fig:motivatepop} is a likely candidate 
for the graph of the model.

Clearly the functions depicted in \cref{exp:fig:motivate} belong to 
a different class than those that we have considered so far. In fact, they 
belong to the class known as \emph{exponential} functions, the study of 
which is a fascinating topic that encompasses many applications, and a lot 
of interesting mathematical features. Prepare yourself for a colorful 
and exciting
journey that will take us through the landscape of some of the most useful 
functions that we will every encounter.

\begin{figure}[!htb]
	\centering
	\mbox{}
	\hfill
	\begin{subfigure}{.4\textwidth}
		\begin{tikzpicture}
			\begin{axis}[
					framed,
					xmin=-10,xmax=50,
					ymin=-20,ymax=100,
					xtick={-30}, ytick={-30},
					minor xtick={10,20,...,50},
					minor ytick={20,40,...,100},
					xlabel={$t$},
					grid=minor,
				]
				\addplot+[->]expression[domain=0:50,samples=50]{40*exp(-x/10)+50};
			\end{axis}
		\end{tikzpicture}
		\caption{}
		\label{exp:fig:motivatecool}
	\end{subfigure}%
	\hfill
	\begin{subfigure}{.4\textwidth}
		\begin{tikzpicture}
			\begin{axis}[
					framed,
					xmin=-10,xmax=50,
					ymin=-20,ymax=100,
					xtick={-30}, ytick={-30},
					minor xtick={10,20,...,50},
					minor ytick={20,40,...,100},
					xlabel={$t$},
					grid=minor,
				]
				\addplot+[->]expression[domain=0:45,samples=50]{(1.1)^x+20};
			\end{axis}
		\end{tikzpicture}
		\caption{}
		\label{exp:fig:motivatepop}
	\end{subfigure}
	\hfill
	\mbox{}
	\caption{}
	\label{exp:fig:motivate}
\end{figure}

\begin{pccexample}
	%===================================
	%   Author: Barkin
	%   Date:   April 2011
	%===================================
	Congratulations, you've been offered a job!  Human Resources told you 
	that you would start out making \SI{2}{\centunit} on the first day, and every day you work thereafter your pay will double.  
	Would you take this job?
							
	\begin{table}[!htb]
		\begin{minipage}{.5\textwidth}
			\centering
			\caption{}
			\label{exp:tab:salary}
			\begin{tabular}{S[table-format=2.0]S[table-format=10.0]}
				\beforeheading
				\heading{$d$ (days worked)} & \heading{$p$ (cents)} \\\afterheading
				1                           & 2                     \\\normalline
				2                           & 4                     \\\normalline
				3                           & 8                     \\\normalline
				4                           & 16                    \\\normalline
				5                           & 32                    \\\normalline
				10                          & 1024                  \\\normalline
				30                          & 1073741824            \\\lastline
			\end{tabular}
		\end{minipage}%
		\begin{minipage}{.5\textwidth}
			\centering
			\caption{}
			\label{exp:tab:salaryalt}
			\begin{tabular}{S[table-format=2.0]l}
				\beforeheading
				\heading{$d$ (days worked)} & \heading{$p$ (cents)} \\\afterheading
				1                           & $2$                   \\\normalline
				2                           & $2^2$                 \\\normalline
				3                           & $2^3$                 \\\normalline
				4                           & $2^4$                 \\\normalline
				5                           & $2^5$                 \\\normalline
				10                          & $2^{10}$              \\\normalline
				30                          & $2^{30}$              \\\lastline
			\end{tabular}
		\end{minipage}%
	\end{table}
							
	\Cref{exp:tab:salary} shows how much money you would make per day, 
	in cents, for the first 5 days, and how much you would make on the 10th and 30th days. 
	The amount you make on day 30 is 1073741824 cents, which is
	\[
		\$10\textrm{,}737\textrm{,}418.24   
	\]
	That's over 10 million dollars in a single day! How did this happen? Can we develop 
	a formula to help us understand the mathematics behind this? 
							
	It seems that the dollar amount is multiplied by 2 each day. An alternative way 
	of writing our daily income is shown in \cref{exp:tab:salaryalt}. Can we write 
	a formula that calculates the pay, $p$, in cents, as a function of the number of days
	worked, $d$? 
	According to \cref{exp:tab:salaryalt}, it appears that the day of the month is in the exponent, 
	so let's write
	\[
		p = 2^{d}
	\]
	where $d$ is a positive integer. This is our first example of an exponential function -- exploring 
	these types of functions is our primary goal in this chapter.
\end{pccexample}
\begin{pccdefinition}[Exponential functions]
	An exponential function is a function $f$ that can be described with the formula 
	\[
		f(x)=a\,b^x
	\] 
	where $a$ is a non-zero real number ($a\in\mathbb{R}, a\ne 0$) and $b$ is a positive 
	number other than 1 ($b>0$, $b\neq1$). Notice that the variable is in the exponent 
	and the base is the fixed constant $b$.
							
	Note that in an exponential term the base is fixed and the variable is in the 
	exponent (e.g. $5\cdot 4^x$), whereas in a polynomial term the exponent is fixed the base is the variable (e.g $6x^3$).
\end{pccdefinition}

%===================================
%   Author: Hughes
%   Date:   Feb 2011
%===================================
\begin{pccexample}[Rice on a chessboard]\label{exp:prob:queenschessboard}%
	Many years ago there lived a Queen who loved to play games; so much so, that 
	she had a jester dedicated to devising interesting games for her. The Queen 
	particularly enjoyed mathematical games.
							
	One day the jester brought her a chessboard (see \cref{exp:fig:grainsofrice}) and a bucket filled with rice. 
	\begin{table}[!htb]
		\centering
		\caption{}
		\label{exp:tab:grainsofrice}
		\begin{tabular}[t]{S[table-format=2.0]S[table-format=10.0]}
			\beforeheading
			\heading{square on board} & \heading{grains of rice} \\
			\heading{$x$}             & \heading{$g(x)$}         \\ 
			\afterheading
			1                         & 3                        \\\normalline
			2                         & 9                        \\\normalline
			3                         & 27                       \\\normalline
			4                         & 81                       \\\normalline
			\mbox{\vdots}             & \mbox{\vdots}            \\\normalline
			\mbox{\vdots}             & \mbox{\vdots}            \\\normalline
			20                        & 3486784401               \\\normalline
			\mbox{\vdots}             & \mbox{\vdots}            \\\normalline
			\mbox{\vdots}             & \mbox{\vdots}            \\\normalline
			\mbox{$x$}                & \mbox{$3^x$}             \\\lastline
		\end{tabular}
	\end{table}
							
	\begin{marginfigure}
		\centering
		\begin{tikzpicture}[scale=.25]
			\foreach \x in {0,...,7} \foreach \y in {0,...,7}
			{
				\pgfmathparse{mod(\x+\y,2) ? "black" : "white"}
				\edef\colour{\pgfmathresult}
				\path[fill=\colour] (\x,\y) rectangle ++ (1,1);
			}
			\draw (0,0)--(0,8)--(8,8)--(8,0)--(0,0);
		\end{tikzpicture}
		\captionof{figure}{}
		\label{exp:fig:grainsofrice}
	\end{marginfigure}
	The jester asked the Queen to follow these instructions
	\begin{itemize}
		\item Place 3 grains of rice on the square in the lower left hand corner of the board.
		\item Place 9 grains of rice on the square immediately to the right of the square you were just working with.
		\item Place 27 grains of rice on the square immediately to the right of the square you were just working with.
	\end{itemize}
	The Queen starts to notice a pattern, and records her findings in \cref{exp:tab:grainsofrice}. She also notes
	that as she progresses from square to square, the number of grains appears to be \emph{tripling} each time.
							
	The Queen, being mathematically inclined, decides to try to model 
	the game using a formula. She decides to  let $x$ be the number 
	corresponding to the square on the chessboard, and let $g(x)$ represent 
	the number of grains of rice on that square, and  assumes that she 
	works each row from left to right as she moves up the chessboard.
	\begin{marginfigure}
		\centering
		\begin{tikzpicture}
			\begin{axis}[
					width=.75\textwidth,
					xmin=-5, xmax=5,
					ymin=-1, ymax=10,
				]
				\addplot expression[domain=-4:2.0959]{3^x};
			\end{axis}
		\end{tikzpicture}
		\captionof{figure}{$g$}
		\label{exp:fig:queensgraph}
	\end{marginfigure}
							
	The Queen notices that each of the numbers she writes in \cref{exp:tab:grainsofrice}
	can be written as a power of $3$, and concludes that the formula for the number
	of grains on square $x$ of the chessboard is
	\[
		g(x)=3^x
	\]
	The Queen decides to test her formula by calculating the number of grains 
	on the \nth{17} square. She finds that
	\[
		g(17)=\num{129 140 163}
	\]
	and says that there are (or would be, if they could fit) \num{129 140 163} grains of rice on the \nth{17} square -- wow!
							
	Of course, the Queen knows that this formula only works when $x$ takes the 
	integer values $\{1,2,\ldots,64\}$, but wonders what would happen if 
	she graphed $g$ on her calculator (which she always has with her for just 
	such a situation). The Queen graphs $g$ on her calculator assuming 
	that $g$ can take values outside of the contextual domain, and 
	obtains the graph in \cref{exp:fig:queensgraph}.
							
	The Queen concludes that $g$ is increasing at a faster and faster rate, and 
	cannot imagine ever being able to fit the appropriate amount of rice on 
	each square.
\end{pccexample}
\begin{doyouunderstand}
	\begin{problem}
	Repeat \cref{exp:prob:queenschessboard}, but instead of \emph{tripling} the 
	number of grains of rice on each square, try \emph{quadrupling} them.
	\begin{shortsolution}
		The equivalent of the function $g$ is $h$ and has formula $h(x)=4^x$.
	\end{shortsolution}
	\end{problem}
\end{doyouunderstand}
We have so far seen two exponential functions, $p$ and $g$, both of which increase 
at a faster and faster rate. You may wonder if all exponential functions behave in this way --
the next example demonstrates that they do not.

%===================================
%   Author: Hughes
%   Date:   Feb 2011
%===================================
\begin{pccexample}[Folding paper]
	Have you ever tried to fold a piece of paper in half more than 7 times? No matter the size 
	of the paper, it becomes quite difficult -- the MythBusters tried quite an elaborate experiment 
	along these lines.\footnote{\href{http://www.youtube.com/watch?v=kRAEBbotuIE}{http://www.youtube.com/watch?v=kRAEBbotuIE}} 
							
	We are going to experiment with paper folding and study the mathematics behind the results.
	The area of a `letter' sheet of paper is $\SI{8.5 x 11}{\inch}$, or $\SI{93.5}{\square\inch}$. We will
	use two decimal places in what follows.
							
							
	If we fold a sheet of letter paper in half, the visible surface area is
	\begin{align*}
		\frac{\SI{93.52}{\square\inch}}{2} & =\left(\frac{1}{2}\right)\SI{93.5}{\square\inch} \\
		                                   & =\SI{46.75}{\square\inch}                        
	\end{align*}
	Note that we say, `visible', because the actual surface area of the paper has not changed.
							
	If we fold the sheet in half again, the visible surface area is 
	\begin{align*}
		\frac{\SI{46.75}{\square\inch}}{2} & = \frac{\SI{93.52}{\square\inch}}{4}                \\
		                                   & = \left(\frac{1}{2}\right)^2\SI{93.5}{\square\inch} \\
		                                   & \approx \SI{23.38}{\square\inch}                    
	\end{align*}
	If we fold the sheet in half a third time, the visible surface area is
	\begin{align*}
		\frac{\SI{93.5}{\square\inch}}{8} & = \left(  \frac{1}{2}\right)^3\SI{93.5}{\square\inch} \\
		                                  & \approx \SI{11.69}{\square\inch}                      
	\end{align*}
	Let's try and generalize our results by letting $x$ be the number of 
	paper folds; $x$ will start at 0, and increase in integer values. 
	\Cref{exp:tab:foldingpaper} has two columns, one for the number of folds (up to 7), and one for the visible surface area 
	of the (folded) paper. 
	\begin{table}[!htb]
		\begin{minipage}{.5\textwidth}
			\centering
			\captionof{table}{}
			\label{exp:tab:foldingpaper}
			\begin{tabular}{S[table-format=1.0]S[table-format=2.2]}
				\beforeheading
				\heading{number of folds} & \heading{visible area}           \\
				\heading{$x$}             & \heading{($\si{\inch\squared}$)} \\ 
				\afterheading
				0                         & 93.50                            \\\normalline
				1                         & 46.75                            \\\normalline
				2                         & 23.38                            \\\normalline
				3                         & 11.69                            \\\normalline
				4                         & 5.84                             \\\normalline
				5                         & 2.92                             \\\normalline
				6                         & 1.46                             \\\normalline
				7                         & 0.73                             \\\lastline
			\end{tabular}
		\end{minipage}%
		\begin{minipage}{.5\textwidth}
			\centering
			\begin{tikzpicture}
				\begin{axis}[
						width=.75\textwidth,
						xmin=-5, xmax=5,
						ymin=-50, ymax=400,
						ytick={50,100,...,350},
					]
					\addplot expression[domain=-2:4]{93.5*(1/2)^x};
				\end{axis}
			\end{tikzpicture}
			\captionof{figure}{$y=A(x)$}
			\label{exp:fig:foldingpaper}
		\end{minipage}%
	\end{table}
							
	If we let $A(x)$ represent the visible surface area of the paper after $x$ folds, then 
	a formula for $A$ is
	\[
		A(x) = 93.5\left(\frac{1}{2}\right)^x
	\]
	where $x=0,1,2,\ldots$. If we allow $A$ to take values outside of its contextual domain, then 
	we can graph $y=A(x)$ on a graphing calculator, and obtain \cref{exp:fig:foldingpaper}. Notice 
	in particular that $A$ decreases at a slower and slower rate.
\end{pccexample}

The graphs of exponential functions have certain features that tell us a lot about the quantities they are modeling.  Graphical features like increasing/decreasing and the position of any asymptotes translate to important information about population sizes, bank accounts, and more.
%===================================
%   Author: Hughes
%   Date:   April 2011
%===================================
\begin{pccexample}\label{exp:ex:asymptote}
	Consider the function $f$ in \cref{exp:fig:expmotivation}. 
	There are a number of features that we can note:
	\begin{itemize}
		\item $f$ is increasing;
		\item $f$ is concave up; in particular, $f$ is increasing at a faster and faster rate;
		\item the line $y=3$ is a horizontal asymptote of $f$ as $x\to -\infty$;
		\item $f(x)\to\infty$ as $x\to\infty$;
		\item the range of $f$ is $(3,\infty)$.
	\end{itemize}
	Note that $f$ never touches its horizontal asymptote (see \cref{exp:fig:closeup}).
\end{pccexample}
\begin{figure}[!htb]
	\begin{minipage}[t]{.5\textwidth}
		\centering
		\begin{tikzpicture}
			\begin{axis}[
					framed,
					xmin=-6,xmax=6,
					ymin=-1,ymax=10,
					xtick={-5,-4,...,5},
					minor ytick={1,3,...,9},
					ytick={2,4,...,8},
					grid=both,
				]
				\addplot expression[domain=-6:2.807]{2^x+3};
				\addplot[asymptote]coordinates{(-6,3)(6,3)};
			\end{axis}
		\end{tikzpicture}
		\caption{$f(x)=2^x+3$}
		\label{exp:fig:expmotivation}
	\end{minipage}%
	\begin{minipage}[t]{.5\textwidth}
		\centering
		\begin{tikzpicture}
			\begin{axis}[
					width=.5\textwidth,
					axis x line=box,
					axis y line=box,
					axis line style={-}, 
					xmin=-6,xmax=-3,
					ymin=2,ymax=4,
					xtick={-6,-5,...,-3},
					ytick={2,2.5,...,4},
					grid=both,
				]
				\addplot expression[domain=-6:-3]{2^x+3};
				\addplot[asymptote]coordinates{(-6,3)(6,3)};
			\end{axis}
		\end{tikzpicture}
		\caption{Close up!}
		\label{exp:fig:closeup}
	\end{minipage}%
\end{figure}
%===================================
%   Author: Hughes
%   Date:   February 2012
%===================================
\begin{doyouunderstand}
	\begin{problem}
	Repeat \cref{exp:ex:asymptote} for each of the functions defined 
	by the following formulas.
	\end{problem}
	\begin{multicols}{2}
		\begin{subproblem}[core]
			$g(x)=5-4^x$ 
			\begin{shortsolution}
				The graph of $g$ is shown below. Note that
				\begin{itemize}
					\item $g$ is decreasing
					\item $g$ is concave down; in particular $g$ is decreasing at a faster and faster rate;
					\item the line $y=5$ is a horizontal asymptote of $g$ as $x\rightarrow-\infty$;
					\item $g(x)\rightarrow-\infty$ as $x\rightarrow\infty$;
					\item the range of $g$ is $(-\infty,5)$.
				\end{itemize}
				\begin{tikzpicture}
					\begin{axis}[
							framed,
							xmin=-5,xmax=5,
							ymin=-10,ymax=10,
							xtick={-4,-3,...,4},
							ytick={-8,-6,...,8},
							grid=both,
						]
						\addplot expression[domain=-5:1.95]{5-4^x};
						\addplot[asymptote]coordinates{(-5,5)(5,5)};
					\end{axis}
				\end{tikzpicture}     
			\end{shortsolution}
		\end{subproblem}
		\begin{subproblem}
			$h(x)=\left( \frac{2}{3} \right)^x-4$
			\begin{shortsolution}
				The graph of $h$ is shown below. Note that
				\begin{itemize}
					\item $h$ is decreasing;
					\item $h$ is concave up; in particular $h$ is decreasing at a slower and slower rate;
					\item the line $y=-4$ is a horizontal asymptote of $h$ as $x\rightarrow\infty$;
					\item $h(x)\rightarrow\infty$ as $x\rightarrow-\infty$;
					\item the range of $h$ is $(-4,\infty)$.
				\end{itemize}
				\begin{tikzpicture}
					\begin{axis}[
							framed,
							xmin=-5,xmax=5,
							ymin=-10,ymax=10,
							xtick={-4,-3,...,4},
							ytick={-8,-6,...,8},
							grid=both,
						]
						\addplot expression[domain=-5:5]{(2/3)^x-4};
						\addplot[asymptote]coordinates{(-5,-4)(5,-4)};
					\end{axis}
				\end{tikzpicture}     
			\end{shortsolution}
		\end{subproblem}
	\end{multicols}
\end{doyouunderstand}

Exponential modeling will require familiarity with percentages. This example aims to 
help you (re)acquaint yourself with them.

\begin{pccexample}
	Wild honeybee colonies tend to have around \SI{15}{\percent} drones (males).   If a colony has 4,280 bees, about how many of them are drones?
	\begin{pccsolution}
		The percentage \SI{15}{\percent} should be converted to a decimal in order to do arithmetic with it: 0.15.  
		\begin{align*}
			\textrm{drone count} & = \SI{15}{\percent}\textrm{ of total bee count} \\
			                     & =0.15\cdot4280                                  \\
			                     & = 642                                           
		\end{align*}
		So there are about 642 drones in the colony.
	\end{pccsolution}
\end{pccexample}

\begin{pccdefinition}[Growth factor and growth rate]\label{exp:def:growthfactorrate}
	An exponential function $f$ can be written in (at least) two ways.
	\begin{align*}
		f(t) & =a\,b^t & f(t) & =a(1+r)^t 
	\end{align*}
	\begin{itemize}
		\item The constant $b$ is called the \emph{growth factor}.  When $t$ increases by $1$, the value of $f(t)$ changes by a factor of $b$; that is, every unit of time the value of $f(t)$ is multiplied by $b$.
		\item The constant $r$ is called the \emph{growth rate}.  We usually write $r$ as a decimal and interpret it as a percent.  
		When $t$ increases by $1$, the amount of change in $f(t)$ is $r$. 
		For example, if $r=0.10$, when $t$ increases by $1$ the value of $f(t)$
		increases by \SI{10}{\percent}. Whereas if $r=-0.05$, when $t$ increases by $1$ 
		the value of $f(t)$ decreases by \SI{5}{\percent}.
	\end{itemize}
\end{pccdefinition}

%===================================
%   Author: Hughes
%   Date:   April 2011
%===================================
\begin{pccexample}
	A population is modeled by the formula $P(t) = P_0(1.15)^t$, where $t$ is the amount 
	of time that has passed (in years) since the population 
	was $P_0$. Find each of the following:
	\begin{multicols}{3}
		\begin{enumerate}
			\item 1-year growth factor and 1-year growth rate
			\item 2-year growth factor and 2-year growth rate
			\item 10-year growth factor and 10-year growth rate
		\end{enumerate}
	\end{multicols}
	\begin{pccsolution}
		\begin{enumerate}
			\item The 1-year growth factor is what we multiply $P_0$ by after one year: $(1.15)^1=1.15$.  So the 1-year growth factor is 1.15 and the 1-year growth rate is \SI{15}{\percent}.
			\item The 2-year growth factor is what $P_0$ would be multiplied by after two years: $(1.15)^2\approx 1.32$.  So the 2-year growth factor is about 1.32 and the 2-year growth rate is about \SI{32}{\percent}.
			\item The 10-year growth factor is $(1.15)^{10}\approx 4.05$, and the 10-year growth rate is approximately \SI{305}{\percent}.
		\end{enumerate}
	\end{pccsolution}
\end{pccexample}

%===================================
%   Author: Hughes (idea from Bill Bryson, Short History, .. chapter 20
%   Date:   April 2011
%===================================
\begin{pccexample}
	The bacterium {\em Clostridium perfringens} can reproduce every 9 minutes. 
	Suppose that 
	there are initially 50 bacteria in a jar and that they have access to an adequate supply of nutrients.
	Write a model for this situation and find the following:
	\begin{multicols}{3}
		\begin{enumerate}
			\item 9-minute growth factor and 9-minute growth rate
			\item 1-minute growth factor and 1-minute growth rate
			\item 1-hour growth factor and 1-hour growth rate
		\end{enumerate}
	\end{multicols}
	\begin{pccsolution}
		Let $P(t)$ be the number of bacteria where $t$ is the amount of time that 
		has passed (in minutes) since the population was $50$ bacteria. 
														
		\Cref{exp:tab:bacterium} shows values of $P(t)$ for the first $27$ hours.
		\begin{table}[!htb]
			\centering
			\caption{}
			\label{exp:tab:bacterium}
			\begin{tabular}{S[table-format=2.0]S[table-format=3.0]l}
				\beforeheading
				\heading{$t$} & \heading{$P(t)$} & \heading{Exponential form}                 \\
				\afterheading
				0             & 50               & $50\cdot 2^{\nicefrac{0}{9}}=50\cdot 2$    \\\normalline
				9             & 100              & $50\cdot 2^{\nicefrac{9}{9}}=50\cdot 2^1$  \\\normalline
				18            & 200              & $50\cdot 2^{\nicefrac{18}{9}}=50\cdot 2^2$ \\\normalline
				27            & 400              & $50\cdot 2^{\nicefrac{27}{9}}=50\cdot 2^3$ \\\lastline
			\end{tabular}
		\end{table}
														
		We can deduce from \cref{exp:tab:bacterium} that 
		\[
			P(t) = 50\cdot 2^{\nicefrac{t}{9}}
		\]
		\begin{enumerate}
			\item The 9-minute growth factor is $2^{\nicefrac{9}{9}} = 2.00$, and the 9-minute growth rate is \SI{100}{\percent} (not surprising since 
			we knew the population doubled in 9 minutes).
			\item The 1-minute growth factor is $2^{\nicefrac{1}{9}}\approx 1.08$, and the 1-minute growth rate is approximately \SI{8}{\percent}.
			\item The 1-hr growth factor is $2^{\nicefrac{60}{9}}\approx 101.59$ (or \SI{10159}{\percent}), and the 1-hr growth rate is approximately \SI{10059}{\percent}.
		\end{enumerate}
	\end{pccsolution}
\end{pccexample}

\investigation*{}
%===================================
%   Author: Pettit
%   Date:   April 2011
%===================================
\begin{problem}[The Legend of Payasam]
\Cref{exp:prob:queenschessboard} is a version of the Legend of Payasam. 

According to legend, Lord Krishna once appeared in the form of a sage in the court of a 
king who ruled a region of southern India. Lord Krishna challenged the king to a game of chess. 
The king, being a chess 
enthusiast, gladly accepted the challenge.

The players decided to put a wager on the game; the king let the sage choose the prize.
The sage told the 
king that he was a man of few material needs, and thus all he wished for was a few grains of rice. 

The sage suggested that the amount of rice should be determined 
using the chessboard in the following manner. 
Two grains of rice will be placed on the first square, four grains on the second square, eight 
on the third square, and so on. That is, every square will have double the 
number of grains as its predecessor.

Upon hearing the demand, the king was unhappy since the sage requested only a few grains of 
rice instead of other riches from the kingdom.

We are going to attach monetary value to our calculations. We will assume that
\begin{itemize}
	\item there are approximately 7200 grains of rice in a cup;
	\item there are 3 cups of rice in a 1-lb bag;
	\item a 1-lb bag of rice is worth \$2. 
\end{itemize}
\begin{subproblem}[core]
	Approximate the number of grains of rice that are in a 1-lb bag.
	\begin{shortsolution}
		There are approximately $7200\cdot 3 = 21600$ grains of rice in a 1-lb bag.
	\end{shortsolution}
\end{subproblem}
\begin{subproblem}
	What is the first square on the chessboard that could be used to fill a 1-lb bag (without using rice 
	from the previous squares)?
	\begin{shortsolution}
		$2^{15}=32768>21600$. The 15th square.
	\end{shortsolution}
\end{subproblem}
\begin{subproblem}[core]\label{exp:prob:greedy}
	Before we begin our money calculations, write down how much money you would like 
	to get as a prize from the king.
	\begin{shortsolution}
		Answers will vary.
	\end{shortsolution}
\end{subproblem}
\begin{subproblem}\label{exp:prob:ricevalue}
	If you were to exchange the rice on the 16th square for money, how much would you get?
	\begin{shortsolution}
		The 16th square would give us $2^{16}/21600\approx 3.03$ bags of rice. We would get approximately \$6
		from the rice on the 16th square.
	\end{shortsolution}
\end{subproblem}
\begin{subproblem}
	Using the value you obtained in \cref{exp:prob:ricevalue}, determine the value 
	of the rice on the 17th square.
	\begin{shortsolution}
		Using the previous value of \$6, we would obtain \$12 from the rice on the 17th square.
	\end{shortsolution}
\end{subproblem}
\begin{subproblem}
	Using the value you obtained in \cref{exp:prob:ricevalue}, determine the first square 
	that would give you more than \$1,000,000 worth of rice.
	\begin{shortsolution}
		$6\cdot 2^{18}\approx 1572864>1000000$. We would only have to get to the 34th square on the board in 
		order to get \$1,000,000 worth of rice.
	\end{shortsolution}
\end{subproblem}
\begin{subproblem}
	Using the value you obtained in \cref{exp:prob:ricevalue}, determine the value
	of the rice put on the last square. 
	\begin{shortsolution}
		$\$6\cdot 2^{48}\approx \$ 1.68884986\times 10^{15}$. Quite a lot.
	\end{shortsolution}
\end{subproblem}
\begin{subproblem}
	How does the value of the rice on the last square compare to the amount you wrote down in \cref{exp:prob:greedy}?
	\begin{shortsolution}
		Answers will vary.
	\end{shortsolution}
\end{subproblem}
\end{problem}
%===================================
%   Author: Simonds
%   Date:   Feb 2011
%===================================
\begin{problem}[Changing Rates of Change]
\begin{subproblem} \label{exp:prob:graphsofincreasingfunctions}
	The graphs of several increasing functions are given in \crefrange{exp:fig:functionm}{exp:fig:functiono}.  
	For each function, decide whether the function increases at a constant rate, increases at an increasing 
	rate (concave up), or increases at a slower and slower rate (concave down).
	\begin{shortsolution}
		$m(x)$ in \cref{exp:fig:functionm} is increasing at a faster and faster rate, 
		$n(x)$ in \cref{exp:fig:functionn} is increasing at a faster and faster rate, 
		and $o(x)$ in \cref{exp:fig:functiono} is increasing at a slower and slower rate.
	\end{shortsolution}
\end{subproblem}

\begin{figure}[!htb]
	\mbox{}\hfill
	\begin{minipage}{.25\textwidth}
		\begin{tikzpicture}
			\begin{axis}[
					xmin=-2,xmax=7,
					ymin=-3,ymax=16,
					xtick={-1,...,6},
				]
				\addplot+[->]expression[domain=0:3.87]{x^2+1};
			\end{axis}
		\end{tikzpicture}
		\caption{$y=m(x)$}
		\label{exp:fig:functionm}
	\end{minipage}%
	\hfill
	\begin{minipage}{.25\textwidth}
		\centering
		\begin{tikzpicture}
			\begin{axis}[
					xmin=-2,xmax=7,
					ymin=-3,ymax=16,
					xtick={-1,...,6},
				]
				\addplot expression[domain=-2:3.9]{2^x+1};
			\end{axis}
		\end{tikzpicture}
		\caption{$y=n(x)$}
		\label{exp:fig:functionn}
	\end{minipage}%
	\hfill
	\begin{minipage}{.25\textwidth}
		\centering
		\begin{tikzpicture}
			\begin{axis}[
					xmin=-2,xmax=7,
					ymin=-3,ymax=16,
					xtick={-1,...,6},
				]
				\addplot+[->]expression[domain=0.01:6.5]{(25*x)^(1/2)+1};
			\end{axis}
		\end{tikzpicture}
		\caption{$y=o(x)$}
		\label{exp:fig:functiono}
	\end{minipage}
	\hfill
	\mbox{}
\end{figure}

\begin{subproblem}\label{exp:prob:tablesofincreasingfunctions}
	\Crefrange{exp:tab:functionp}{exp:tab:functionr} show values for $3$ increasing functions.
	For each function, decide whether the function increases at a constant rate, increases at an increasing 
	rate (concave up), or increases at a slower and slower rate (concave down).
	\begin{shortsolution}
		$p(x)$ is increasing at a constant rate, $q(x)$ is increasing at a slower and slower rate, 
		and $r(x)$ is increasing at a faster and faster rate.
	\end{shortsolution}
								
	\begin{table}[!htb]
		\centering
		\hfill
		\begin{minipage}{0.25\textwidth}
			\centering
			\caption{$y=p(x)$}
			\begin{tabular}{S[table-format=2.0]S[table-format=2.0]}
				\beforeheading
				\heading{$x$} & \heading{$y$} \\
				\afterheading
				1             & 3             \\\normalline
				2             & 8             \\\normalline
				4             & 18            \\\normalline
				8             & 38            \\\normalline
				16            & 78            \\\lastline
			\end{tabular}
			\label{exp:tab:functionp}
		\end{minipage}
		\hfill
		\begin{minipage}{0.25\textwidth}
			\centering
			\caption{$y=q(x)$}
			\begin{tabular}{S[table-format=2.0]S[table-format=2.0]}
				\beforeheading
				\heading{$x$} & \heading{$y$} \\
				\afterheading
				1             & 6             \\\normalline
				4             & 7             \\\normalline
				9             & 8             \\\normalline
				16            & 9             \\\normalline
				25            & 10            \\\lastline
			\end{tabular}
			\label{exp:tab:functionq}
		\end{minipage}
		\hfill
		\begin{minipage}{0.25\textwidth}
			\centering
			\caption{$y=r(x)$}
			\begin{tabular}{S[table-format=1.0]S[table-format=2.0]}
				\beforeheading
				\heading{$x$} & \heading{$y$} \\
				\afterheading
				1             & 4             \\\normalline
				2             & 8             \\\normalline
				3             & 16            \\\normalline
				4             & 32            \\\normalline
				5             & 64            \\\lastline
			\end{tabular}
			\label{exp:tab:functionr}
		\end{minipage}
		\hfill
		\mbox{}
	\end{table}
\end{subproblem}


\begin{subproblem}
	Several functions are described below.  For each function, decide whether the function 
	increases at a constant rate, increases at a faster and faster rate (concave up), or increases at a slower and slower rate (concave down).
	\begin{enumerate}
		\item The amount in a bank account where \$5000 is initially invested and the money sits 
		and earns interest at a rate of \SI{6}{\percent} per year.
		\item The distance your car has traveled $t$ seconds after you slammed on the brakes.
		\item The elevation of a typewriter that is falling, $t$ seconds after it is dropped from a plane flying at an 
		elevation of 30,000 feet.
	\end{enumerate}
	\begin{shortsolution}
		In (a.), the investment is increasing at a faster and faster rate.  In (b.), the distance is
		increasing, but at a slower and slower rate.  In (c.), the elevation is decreasing at a faster and faster rate, as
		the speed increases.
	\end{shortsolution}
\end{subproblem}
\begin{subproblem}
	For each function below, decide whether the function increases at a constant rate, increases at an 
	increasing rate (concave up), or increases at a slower and slower rate (concave down).
	\begin{enumerate}
		\item The function $f$, where $f(x)=3+2\sqrt{x}$
		\item The function $g$, where $g(x)=3+2x$
		\item The function $h$, where $h(x)=3+2(4^x)$
	\end{enumerate}
	\begin{shortsolution}
		(a.) is increasing at a slower and slower rate, (b.) is increasing at a constant rate, and (c.) is 
		increasing at a faster and faster rate.
	\end{shortsolution}
\end{subproblem}
\end{problem}

%===================================
%   Author: Vega
%   Date:   March 2011
%===================================
\begin{problem}[Medication]
A medication is injected into your body. The amount of medication in your 
body decays exponentially over time. The original dose you receive is $\SI{4}{\cubiccm}$, and 
the amount in your body decays at a rate of \SI{8.5}{\percent} per hour.
\begin{subproblem}
	Let $Q(t)$ be the amount of medication in your body (in \si{\cubiccm}) at time $t$ in hours since 
	it was injected. Write a formula for $Q(t)$.
	\begin{shortsolution}
		$Q(t)=4(0.915)^t$
	\end{shortsolution}
\end{subproblem}
\begin{subproblem}
	What are the growth rate and growth factor of $Q$?
	\begin{shortsolution}
		The growth factor is $0.915$, and the growth rate is $\SI{-8.5}{\percent}$.  (We could also say that the decay rate is $\SI{8.5}{\percent}$.)
	\end{shortsolution}
\end{subproblem}
\begin{subproblem}
	According to your model, does the medication ever go away completely? Why or why not?
	\begin{shortsolution}
		No; the model is a limited in this way.
	\end{shortsolution}
\end{subproblem}
\end{problem}

%===================================
%   Author: Adams (Jordan)
%   Date:   June 2012
%===================================
\begin{problem}[Melting of Arctic Sea Ice]
Using satellite imagery, scientists now believe that the Arctic sea ice cover is being reduced in area by 8\% every ten years.  In September of 2005 the area of the ice cover was $\SI{5.35}{\million\kilo\meter\squared}$, according to the National Snow and Ice Data Center\footnote{\href{http://nsidc.org/}{http://nsidc.org/}}.  
\begin{subproblem}
	Generate an exponential model for the melting of the Arctic sea ice cover.\label{exp:prob:iceexpmodel}
	\begin{shortsolution}
		$A(t)=5.35(0.92)^t$, where $t$ is the number of decades since 2005, and $A(t)$ is measured in $\si{\million\kilo\meter\squared}$.
	\end{shortsolution}
\end{subproblem}
\begin{subproblem}
	Find the half-life of the ice.  In what year will the Arctic sea ice reach half of its 2005 level?\label{exp:prob:icehalflife}
	\begin{shortsolution}
		A graph indicates that about $8.3$ decades after 2005 (or in 2088), the Arctic sea ice cover will have reached half of its 2005 level. So the half-life of the ice is 8.3 decades.
	\end{shortsolution}
\end{subproblem}
\begin{subproblem}
	According to your model, what was the area in 1995?  \label{exp:prob:1995ice}
	\begin{shortsolution}
		$A(-1)=5.35(0.92)^{-1}\approx5.82$. According to the model, in 1995 the Arctic sea ice level was about $\SI{5.82}{\million\kilo\meter\squared}$
	\end{shortsolution}
\end{subproblem}
\begin{subproblem}
	Use the two data points you determined in \cref{exp:prob:icehalflife,exp:prob:1995ice} to generate a \emph{linear} model for the melting of the Arctic sea ice cover.  Comment on the differences in your two models' predictions for 2010, 2030, and 2050.    \label{exp:prob:icelinmodel}
							
	\begin{shortsolution}
		Using $B(t)$ to represent a linear model for the ice cover, where $t$ is measured in decades since 2005, $B$ has slope of $2.675-5.815/8.3-(-1)$\si{\million\kilo\meter\squared\per\decade}
		or \SI{-0.3377}{\million\kilo\meter\squared\per\decade}. Using the point-slope form of a line equation, $B(t)=-0.3377(t+1)+5.815$.
																
		For the year 2010, $t=0.5$. $A(0.5)\approx5.13$ and $B(0.5)\approx5.31$. So in 2010, the exponential model predicts lower ice cover than the linear model.
																
		For the year 2030, $t=2.5$. $A(2.5)\approx4.34$ and $B(2.5)\approx4.63$. So in 2030, the exponential model still predicts lower ice cover than the linear model, and the difference is larger than it was in 2010.
																
		For the year 2050, $t=4.5$. $A(4.5)\approx3.67$ and $B(4.5)\approx3.95$. So in 2050, the exponential model still predicts lower ice cover than the linear model.
	\end{shortsolution}
\end{subproblem}
\begin{subproblem}
	When do each of your models (from \cref{exp:prob:iceexpmodel,exp:prob:icelinmodel}) predict the Arctic sea ice cover will melt to less than $\SI{100}{\kilo\meter\squared}$?
	\begin{shortsolution}
		Since $A(t)$ and $B(t)$ are measured in $\si{\million\kilo\meter\squared}$, we should work with the value $\SI{0.0001}{\million\square\kilo\meter}$. A graph shows that $B(16.2)\approx0.0001$ and $A(130.5)\approx0.0001$. So the linear model predicts that after 6.2 decades (in 2212) the Arctic sea ice cover will melt to less than $\SI{100}{\square\kilo\meter}$. The exponential model predicts that this will not happen for 130.5 decades, or until the year 3310.
	\end{shortsolution}
\end{subproblem}
\begin{subproblem}
	Graph your exponential model twice.  For the first graph, choose a scale 
	which supports the view that these changes are minimal and nothing to 
	worry about.  For the second graph, choose a scale which supports the view that these changes are drastic and of great concern. When reading graphs produced by someone else which seem to support a particular opinion, what aspects of the graphs are important to consider?
	\begin{shortsolution}
		\begin{tikzpicture}
			\begin{axis}[
					xmin=-0.1,xmax=0.8,
					ymin=-1,ymax=10,
					xtick={-0.1,0,...,0.8},
					ytick={0,...,10},
					xlabel={$t$},
				]
				\addplot+[->]expression[domain=-0.1:0.8]{5.35*0.92^x};
				\legend{$y=A(t)$}
			\end{axis}
		\end{tikzpicture}
																
		\begin{tikzpicture}
			\begin{axis}[
					xmin=-5,xmax=40,
					ymin=-1,ymax=10,
					xtick={-5,0,...,40},
					ytick={0,...,10},
					xlabel={$t$},
				]
				\addplot+[->]expression[domain=-5:40]{5.35*0.92^x};
				\legend{$y=A(t)$}
			\end{axis}
		\end{tikzpicture}
																
		One should consider the scale of a graph and location of the origin when reading graphs.  
	\end{shortsolution}
\end{subproblem}
\begin{subproblem}
	If all the ice melts in the summer, does this mean that the Arctic sea ice cover has permanently disappeared?  What effect does the disappearing Arctic sea ice cover have on the planet?  Are there consequences beyond the Arctic region?
	\begin{shortsolution}
		Responses will vary.
	\end{shortsolution}
\end{subproblem}

\end{problem}


%========================================================================
%
%			Exercises
%
%========================================================================
\begin{exercises}
\begin{problem}[Exponential or not]
Decide if the following formulas correspond to  exponential functions or not.
\begin{multicols}{4}
	\begin{subproblem}
		$f(x)=5^x$
		\begin{shortsolution}
			Exponential.
		\end{shortsolution}
	\end{subproblem}
	\begin{subproblem}
		$g(x)=x^5$
		\begin{shortsolution}
			Not exponential.
		\end{shortsolution}
	\end{subproblem}
	\begin{subproblem}
		$h(x)=3\cdot 2^x$
		\begin{shortsolution}
			Exponential.
		\end{shortsolution}
	\end{subproblem}
	\begin{subproblem}
		$k(x)=-3\cdot 2^x$
		\begin{shortsolution}
			Exponential.
		\end{shortsolution}
	\end{subproblem}
	\begin{subproblem}
		$m(x)=3x^2$
		\begin{shortsolution}
			Not exponential.
		\end{shortsolution}
	\end{subproblem}
	\begin{subproblem}
		$n(x)=\left(\frac{1}{2}\right)^x$
		\begin{shortsolution}
			Exponential.
		\end{shortsolution}
	\end{subproblem}
	\begin{subproblem}
		$p(x)=4$
		\begin{shortsolution}
			Not exponential.
		\end{shortsolution}
	\end{subproblem}
	\begin{subproblem}
		$q(x)=0$
		\begin{shortsolution}
			Not exponential.
		\end{shortsolution}
	\end{subproblem}
	\begin{subproblem}
		$r(x)=\left(\frac{2}{3}\right)^x$
		\begin{shortsolution}
			Exponential.
		\end{shortsolution}
	\end{subproblem}
	\begin{subproblem}
		$s(x)=\left(-\frac{2}{3}\right)^x$
		\begin{shortsolution}
			Not exponential.
		\end{shortsolution}
	\end{subproblem}
	\begin{subproblem}
		$t(x)=5x$
		\begin{shortsolution}
			Not exponential.
		\end{shortsolution}
	\end{subproblem}
	\begin{subproblem}
		$u(x)=\pi$
		\begin{shortsolution}
			Not exponential.
		\end{shortsolution}
	\end{subproblem}
\end{multicols}
\end{problem}
%===================================
%   Author: Hughes
%   Date:   March 2012
%===================================
\begin{problem}[Identify $a$ and $b$]
Each of the following formulas define exponential functions, and have the form $f(x)=a\,b^x$. 
Identify $a$ and $b$ in each case.
\begin{multicols}{4}
	\begin{subproblem}
		$f(x)=2\cdot 3^x$
		\begin{shortsolution}
			$a=2$, $b=3$ 
		\end{shortsolution}
	\end{subproblem}
	\begin{subproblem}
		$g(x)=-4\cdot 5^x$
		\begin{shortsolution}
			$a=-4$, $b=5$ 
		\end{shortsolution}
	\end{subproblem}
	\begin{subproblem}
		$h(t) = \left( \frac{2}{3} \right)^t$
		\begin{shortsolution}
			$a=1$, $b=\frac{2}{3}$ 
		\end{shortsolution}
	\end{subproblem}
	\begin{subproblem}
		$k(y)= -\left( \frac{2}{3} \right)^y$
		\begin{shortsolution}
			$a=-1$, $b=\frac{2}{3}$ 
		\end{shortsolution}
	\end{subproblem}
	\begin{subproblem}
		$F(s)=3^{-s}$
		\begin{shortsolution}
			$a=1$, $b=\frac{1}{3}$ 
		\end{shortsolution}
	\end{subproblem}
	\begin{subproblem}
		$G(r)=\frac{2}{3^r}$
		\begin{shortsolution}
			$a=2$, $b=\frac{1}{3}$ 
		\end{shortsolution}
	\end{subproblem}
	\begin{subproblem}
		$H(w)=-\frac{4^w}{5}$
		\begin{shortsolution}
			$a=-\frac{1}{5}$, $b=4$ 
		\end{shortsolution}
	\end{subproblem}
	\begin{subproblem}
		$K(z)=-10\cdot 5^{-z}$
		\begin{shortsolution}
			$a=-10$, $b=\frac{1}{5}$ 
		\end{shortsolution}
	\end{subproblem}
\end{multicols}
\end{problem}

%===================================
%   Author: Hughes
%   Date:   February 2012
%===================================
\begin{problem}[Exponential function evaluation]\label{exp:prob:fnevaln}%
Evaluate each of the following formulas at $-10$, $-5$, $0$, $5$, and $10$. 
Give the exact answer, and an approximation (where appropriate) using two figures after the decimal.
\begin{multicols}{4}
	\begin{subproblem}
		$f(x)=2^x$ 
		\begin{shortsolution}
			$\begin{aligned}[t]
				f(-10) & =\frac{1}{1024}\approx 0.00 \\
				f(-5)  & =\frac{1}{32}\approx 0.03   \\
				f(0)   & =\frac{1}{1}=1              \\
				f(5)   & =32                         \\
				f(10)  & =1024                       
			\end{aligned}$
		\end{shortsolution}
	\end{subproblem}
	\begin{subproblem}
		$g(x)=\left( \frac{1}{3} \right)^x$
		\begin{shortsolution}
			$\begin{aligned}[t]
				g(-10) & =59049                       \\
				g(-5)  & =243                         \\
				g(0)   & =1                           \\
				g(5)   & =\frac{1}{243}\approx 0.00   \\
				g(10)  & =\frac{1}{59049}\approx 0.00 
			\end{aligned}$
		\end{shortsolution}
	\end{subproblem}
	\begin{subproblem}
		$h(x)=-5^x$ 
		\begin{shortsolution}
			$\begin{aligned}[t]
				h(-10) & =-\frac{1}{9765625}\approx -0.00 \\
				h(-5)  & =-\frac{1}{3125}\approx -0.00    \\
				h(0)   & =-1                              \\
				h(5)   & =-3125                           \\
				h(10)  & =-9765625                        
			\end{aligned}$
		\end{shortsolution}
	\end{subproblem}
	\begin{subproblem}
		$k(x)=-\left( \frac{2}{5} \right)^x$ 
		\begin{shortsolution}
			$\begin{aligned}[t]
				k(-10) & =-\frac{9765625}{1024}\approx -9536.74 \\
				k(-5)  & =-\frac{3125}{32}\approx -97.66        \\
				k(0)   & =-1                                    \\
				k(5)   & =-\frac{32}{3125}\approx -0.01         \\
				k(10)  & =-\frac{1024}{9765625}\approx -0.00    
			\end{aligned}$
		\end{shortsolution}
	\end{subproblem}
\end{multicols}
\end{problem}

%===================================
%   Author: Hughes
%   Date:   February 2012
%===================================
\begin{problem}[Features of exponential functions]
Refer to the functions $f$, $g$, $h$, and $k$ defined in \cref{exp:prob:fnevaln} 
throughout this problem.
\begin{subproblem}
	Decide if each function is always increasing or always decreasing. 
	\begin{shortsolution}
		$f$ is increasing, $g$ is decreasing, $h$ is decreasing, $k$ is increasing. 
	\end{shortsolution}
\end{subproblem}
\begin{subproblem}
	Decide if each function is concave up or concave down.
	\begin{shortsolution}
		$f$ is concave up, $g$ is concave up, $h$ is concave down, and $k$ is concave down.
	\end{shortsolution}
\end{subproblem}
\begin{subproblem}
	Determine the domain and range of each function. 
	\begin{shortsolution}
		\begin{itemize}
			\item $f$ has domain $(-\infty,\infty)$, and range $(0,\infty)$.     
			\item $g$ has domain $(-\infty,\infty)$, and range $(0,\infty)$.
			\item $h$ has domain $(-\infty,\infty)$, and range $(-\infty,0)$.
			\item $k$ has domain $(-\infty,\infty)$, and range $(-\infty),0$.
		\end{itemize}
	\end{shortsolution}
\end{subproblem}
\end{problem}

\begin{problem}[Prerequisite percentage skills]
Answer these questions concerning percentages.
\begin{subproblem}
	Wild honeybee colonies tend to have around \SI{15}{\percent} drones (males).   If a colony has 4,280 bees, about how many of them are drones?
	\begin{longsolution}
		The percentage \SI{15}{\percent} should be converted to a decimal in order to do arithmetic with it: 0.15.  
		\begin{align*}
			\textrm{drone count} & = \SI{15}{\percent}\textrm{ of total bee count} \\
			                     & = 0.15\cdot4280                                 \\
			                     & = 642                                           
		\end{align*}
		So there are about 642 drones in the colony.
	\end{longsolution}
\end{subproblem}
\begin{subproblem}
	The human body is made up of approximately \SI{66}{\percent} water. If a person weighs $\SI{180}{\pound}$, how 
	much water do they contain?
	\begin{shortsolution}
		$0.66\cdot 180=118.80$. The person contains approximately $\SI{118}{\pound}$ of water.
	\end{shortsolution}
\end{subproblem}
\begin{subproblem}
	The air we breathe is roughly \SI{20}{\percent} oxygen. If you are in a room with dimensions $\SI{40 x 4 x 10}{\foot}$, approximately how much oxygen is there in the room?
	\begin{shortsolution}
		$0.20\cdot 16000=3200$. There is approximately $\SI{3200}{\cubic\foot}$ of oxygen in the room.
	\end{shortsolution}
\end{subproblem}
\begin{subproblem}
	You are at a restaurant and receive the check for \$26. You tip \SI{15}{\percent}. How 
	much is the total bill?
	\begin{shortsolution}
		$(1.15)\cdot 26=29.90$. The total bill is \$29.90.
	\end{shortsolution}
\end{subproblem}
\begin{subproblem}
	You are working in a sales job and manage to secure a client worth \$100,000. You get a 
	\SI{1}{\percent} commission, of which your supervisor gets \SI{50}{\percent}. How much do you get?
	\begin{shortsolution}
		$0.01\cdot 100,000 \cdot 0.5 = 500$. You get \$500.
	\end{shortsolution}
\end{subproblem}
\end{problem}

%===================================
%   Author: Vega
%   Date:   March 2011
%===================================
\begin{problem}[Growth factor]
For each of the following, identify the growth factor, the growth rate, and the initial value.
\begin{multicols}{2}
	\begin{subproblem}
		$y=5\cdot 3^x$
		\begin{shortsolution}
			Growth factor is $3$, initial value is 5.  Growth rate is \SI{200}{\percent}.
		\end{shortsolution}
	\end{subproblem}
	\begin{subproblem}
		$y=6\cdot (0.5)^x$
		\begin{shortsolution}
			Decay factor is $0.5$, initial value is 6.  Growth rate is $\SI{-50}{\percent}$.
		\end{shortsolution}
	\end{subproblem}
	\begin{subproblem}
		$\dd y=2\cdot \left(\nicefrac{3}{4}\right)^x$
		\begin{shortsolution}
			Decay factor is $0.25$, initial value is 2.  Growth rate is $\SI{-25}{\percent}$.
		\end{shortsolution}
	\end{subproblem}
	\begin{subproblem}
		$y=500\cdot (2.5)^x$
		\begin{shortsolution}
			Growth factor is $2.5$, initial value is 500.  Growth rate is \SI{150}{\percent}.
		\end{shortsolution}
	\end{subproblem}
\end{multicols}
\end{problem}

%===================================
%   Author: Vega
%   Date:   March 2011
%===================================
\begin{problem}[Growth factor applications]
Use the following descriptions to determine the growth rate and growth factor as decimals.
\begin{subproblem}
	A store has 100 hats and is increasing their stock at a rate of \SI{10}{\percent} per day.
	\begin{shortsolution}
		Growth rate is \SI{10}{\percent} or 0.1, growth factor is 1.1.
	\end{shortsolution}
\end{subproblem}
\begin{subproblem}
	A fire department is losing \SI{6}{\percent} of its annual funding with each passing year, where they had \$10,500 in annual funding when the station opened. 
	\begin{shortsolution}
		Growth rate is $\SI{-6}{\percent}$ or $-0.06$, growth factor is $0.94$.
	\end{shortsolution}
\end{subproblem}
\end{problem}



%===================================
%   Author: Barkin
%   Date:   April 2011
%===================================
\begin{problem}[Increasing exponential functions]\label{exp:prob:increasingexptabs}%
Consider the functions $f$ and $g$ that have formulas $f(x) = 2^x$ and $g(x) = 3^x+4$.  
\begin{table}[!htb]
	\centering
	\begin{minipage}{.3\textwidth}
		\centering
		\caption{}
		\label{exp:tab:increasingexp}
		\begin{tabular}{S[table-format=1.0]rr}
			\beforeheading
			\heading{$x$} & \heading{$f(x)$} & \heading{$g(x)$} \\
			\afterheading
			-3            &                  &                  \\\normalline
			-2            &                  &                  \\\normalline
			-1            &                  &                  \\\normalline
			0             &                  &                  \\\normalline
			1             &                  &                  \\\normalline
			2             &                  &                  \\\lastline
		\end{tabular}
	\end{minipage}%
	\hfill
	\begin{minipage}{.3\textwidth}
		\centering
		\captionof{table}{}
		\begin{tabular}{S[table-format=1.0]rr}
			\beforeheading
			\heading{$x$} & \heading{$m(x)$} & \heading{$n(x)$} \\
			\afterheading
			-3            &                  &                  \\\normalline
			-2            &                  &                  \\\normalline
			-1            &                  &                  \\\normalline
			0             &                  &                  \\\normalline
			1             &                  &                  \\\normalline
			2             &                  &                  \\\lastline
		\end{tabular}
		\label{exp:tab:decreasingexp}
	\end{minipage}%
	\hfill\mbox{}
\end{table}

\begin{subproblem}
	Graph the functions $f$ and $g$ over the interval $-3\le x \le 2$ after first filling in \cref{exp:tab:increasingexp}.  
	Label the functions $f$ and $g$ on your graph.
	\begin{shortsolution}
		\begin{tabular}[t]{S[table-format=1.0]cc}
			\beforeheading
			\heading{$x$} & \heading{$f(x)$} & \heading{$g(x)$} \\
			\afterheading
			-3            & \num{1/8}        & \num{109/27}     \\\normalline
			-2            & \num{1/4}        & \num{37/9}       \\\normalline
			-1            & \num{1/2}        & \num{13/3}       \\\normalline
			0             & \num{1}          & \num{5}          \\\normalline
			1             & \num{2}          & \num{7}          \\\normalline
			2             & \num{4}          & \num{13}         \\\lastline
		\end{tabular}
																
		The functions $f$ and $g$ are graphed below.
																
		\begin{tikzpicture}
			\begin{axis}[
					framed,
					xmin=-3,xmax=2,
					ymin=-2,ymax=10,
					xtick={-2,...,2},
					ytick={5},
					minor ytick={-1,...,10},
					grid=both,
					legend pos=north west,
				]
				\addplot expression[domain=-3:2]{2^x};
				\addlegendentry{$y=2^x$};
				\addplot expression[domain=-3:1.5]{3^x+4};
				\addlegendentry{$y=3^x+4$};
			\end{axis}
		\end{tikzpicture}
	\end{shortsolution}
\end{subproblem}
\begin{subproblem}
	What are the horizontal asymptotes of $f$ and $g$?
	\begin{shortsolution}
		$f$ has asymptote $y=0$, and $g$ has asymptote $y=4$.
	\end{shortsolution}
\end{subproblem}
\begin{subproblem}
	Notice that both $f$ and $g$ are increasing as $x$ increases.  Which function is increasing at the faster rate?
	\begin{shortsolution}
		$g$ is increasing at a faster rate than $f$.
	\end{shortsolution}
\end{subproblem}
\begin{subproblem}
	Can the value of $f(x)$ be zero?  Can it be negative?  Why or why not?
	\begin{shortsolution}
		No. There is no number such that $2^x=0$, and no real number such that $2^x$ is negative.
	\end{shortsolution}
\end{subproblem}
\begin{subproblem}
	What are the domain and range of the functions $f$ and $g$?
	\begin{shortsolution}
		Both have domain $(-\infty,\infty)$; the range of $f$ is $(0,\infty)$ and the range of $g$ is $(4,\infty)$.
	\end{shortsolution}
\end{subproblem}
\end{problem}

%===================================
%   Author: Barkin
%   Date:   April 2011
%===================================
\begin{problem}[Decreasing exponential functions]
Consider the functions $m$ and $n$ that have formulas $m(x) = \left(\nicefrac{1}{2}\right)^x$ and $n(x) = \left(\nicefrac{1}{3}\right)^x-2$.  
\begin{subproblem}
	Graph these functions over the interval $-2\le x \le 3$ after first filling in \cref{exp:tab:increasingexp}.  
	Label the functions $m$ and $n$.
	\begin{shortsolution}
		\begin{tabular}[t]{S[table-format=1.0]cc}
			\beforeheading
			\heading{$x$} & \heading{$m(x)$} & \heading{$n(x)$} \\
			\afterheading
			-3            & \num{8}          & \num{25}         \\\normalline
			-2            & \num{4}          & \num{7}          \\\normalline
			-1            & \num{2}          & \num{1}          \\\normalline
			0             & \num{1}          & \num{-1}         \\\normalline
			1             & \num{1/2}        & \num{-5/3}       \\\normalline
			2             & \num{1/4}        & \num{-17/9}      \\\lastline
		\end{tabular}
																
		The functions $m$ and $n$ are graphed below.
																
		\begin{tikzpicture}
			\begin{axis}[
					framed,
					xmin=-2,xmax=3,
					ymin=-5,ymax=5,
					xtick={-1,...,2},
					ytick={-4,...,4},
					grid=both,
				]
				\addplot expression[domain=-2:3]{(1/2)^x};
				\addlegendentry{$y=\left( \frac{1}{2} \right)^x$};
				\addplot expression[domain=-1.5:3]{(1/3)^x-2};
				\addlegendentry{$y=\left( \frac{1}{3} \right)^x-2$};
			\end{axis}
		\end{tikzpicture}
	\end{shortsolution}
\end{subproblem}
\begin{subproblem}
	What are the horizontal asymptotes of $m$ and $n$?
	\begin{shortsolution}
		$m$ has asymptote $y=0$, and $n$ has asymptote $y=-2$.
	\end{shortsolution}
\end{subproblem}
\begin{subproblem}
	What are the domain and range of the functions $m$ and $n$?
	\begin{shortsolution}
		Both have domain $(-\infty,\infty)$; the range of $m$ is $(0,\infty)$ and the range of $n$ is $(-2,\infty)$.
	\end{shortsolution}
\end{subproblem}
\begin{subproblem}
	Notice that both of these functions are decreasing as $x$ increases.  Why do these functions decrease when 
	the functions in \cref{exp:prob:increasingexptabs} increase?
	\begin{shortsolution}
		Because $b<1$ and $a>0$.
	\end{shortsolution}
\end{subproblem}
\end{problem}

%===================================
%   Author: Hughes
%   Date:   Feb 2011
%===================================
\begin{problem}[Horizontal asymptotes when $b>1$]\label{exp:prob:matchexponentials}%
Consider the functions $f$, $g$, $h$, and $j$ that have formulas
\begin{align*}
	f(x) & =3^x-1, & g(x) & =-4^x-3, & h(x) & =2^x+1, & j(x) & =-5^x+2 
\end{align*}
%++++++++++++++++
\begin{figure}[!htb]
	\begin{widepage}
	\setlength{\figurewidth}{.2\textwidth}
	\centering
	\begin{subfigure}{\figurewidth}
		\begin{tikzpicture}
			\begin{axis}[
					framed,
					xmin=-10,xmax=10,ymin=-10,ymax=10,
					xtick={-8,-4,4,8},
					ytick={-8,-4,4,8},
					grid=major,
				]
				\addplot expression[domain=-9:3]{2^x+1};
			\end{axis}
		\end{tikzpicture}
		\caption{}
		\label{exp:fig:matchexponential1}
	\end{subfigure}
	\hfill
	\begin{subfigure}{\figurewidth}
		\begin{tikzpicture}
			\begin{axis}[
					framed,
					xmin=-10,xmax=10,ymin=-10,ymax=10,
					xtick={-8,-4,4,8},
					ytick={-8,-4,4,8},
					grid=major,
				]
				\addplot expression[domain=-9:2]{3^x-1};
			\end{axis}
		\end{tikzpicture}
		\caption{}
		\label{exp:fig:matchexponential2}
	\end{subfigure}
	\hfill
	\begin{subfigure}{\figurewidth}
		\begin{tikzpicture}
			\begin{axis}[
					framed,
					xmin=-10,xmax=10,ymin=-10,ymax=10,
					xtick={-8,-4,4,8},
					ytick={-8,-4,4,8},
					grid=major,
				]
				\addplot expression[domain=-9:1.5]{(-1)*5^x+2};
			\end{axis}
		\end{tikzpicture}
		\caption{}
		\label{exp:fig:matchexponential3}
	\end{subfigure}
	\hfill
	\begin{subfigure}{\figurewidth}
		\begin{tikzpicture}
			\begin{axis}[
					framed,
					xmin=-10,xmax=10,ymin=-10,ymax=10,
					xtick={-8,-4,4,8},
					ytick={-8,-4,4,8},
					grid=major,
				]
				\addplot expression[domain=-9:1.25]{(-1)*4^x-3};
			\end{axis}
		\end{tikzpicture}
		\caption{}
		\label{exp:fig:matchexponential4}
	\end{subfigure}
	\caption{Graphs for \cref{exp:prob:matchexponentials}.}
	\label{exp:fig:matchexponentials}
	\end{widepage}
\end{figure}
%++++++++++++++++
\begin{subproblem}
	Match each of the functions $f$, $g$, $h$, and $j$ with one of the graphs in \cref{exp:fig:matchexponentials}.
	\begin{shortsolution}
		\Vref{exp:fig:matchexponential2} shows $f(x)=3^x-1$.\\
		\Vref{exp:fig:matchexponential4} shows $f(x)=-4^x-3$.\\
		\Vref{exp:fig:matchexponential1} shows $f(x)=2^x+1$.  \\
		\Vref{exp:fig:matchexponential3} shows $f(x)=-5^x+2$.
	\end{shortsolution}
\end{subproblem}
\begin{subproblem}\label{exp:prob:asympneginf}
	As $x\to-\infty$, $f(x)\to -1$. This tells us two things:
	\begin{itemize}
		\item the horizontal asymptote as $x\to-\infty$ is the line $y=-1$;
		\item since the function is increasing, the range of the function is $(-1,\infty)$.
	\end{itemize}
	For the remaining three functions $g$, $h$, and $j$, deduce the behavior as $x\to-\infty$, the horizontal 
	asymptote as $x\to-\infty$, and the range.
	\begin{shortsolution}
		\Vref{exp:fig:matchexponential4}, $g(x)=-4^x-3$: 
		\begin{itemize}
			\item $g(x)\to -3$ as $x\to-\infty$;
			\item the horizontal asymptote as $x\to-\infty$ is the line $y=-3$;
			\item the range of $g$ is range: $(-\infty,-3)$.
		\end{itemize}
		\Vref{exp:fig:matchexponential1}, $h(x)=2^x+1$: 
		\begin{itemize}
			\item $h(x)\to 1$ as $x\to -\infty$;
			\item the horizontal asymptote as $x\to-\infty$ is the line $y=1$;
			\item the range of $h$ is $(1,\infty)$.
		\end{itemize}
		\Vref{exp:fig:matchexponential3}, $j(x)=-5^x+2$: 
		\begin{itemize}
			\item $j(x)\to 2$ as $x\to-\infty$;
			\item the horizontal asymptote as $x\to-\infty$ is the line $y=2$;
			\item the range of $j$ is $(-\infty,2)$.
		\end{itemize}
	\end{shortsolution}
\end{subproblem}
\begin{subproblem}\label{exp:prob:asymptotes}
	Using the appropriate function in \cref{exp:fig:matchexponentials}, we can observe that
	$f(x)\to \infty$, as $x\to\infty$.
	Make analogous statements about the remaining functions in \cref{exp:fig:matchexponentials}.
	\begin{shortsolution}
		$-4^x-3\to -\infty$ as $x\to\infty$.\\ 
		$2^x+1\to \infty$ as $x\to\infty-$.\\
		$-5^x+2\to -\infty$ as $x\to\infty$.	  	  
	\end{shortsolution}
\end{subproblem}
\begin{subproblem}
	Recall that an equivalent way of writing that $f(x)\to \infty$ as $x\to \infty$ is to use limit notation:
	\[
		\lim_{x\to\infty}f(x)=\infty
	\]
	Similarly, we can say 
	\[
		\lim_{x\to-\infty}f(x)=1.
	\]
	Use limit notation to re-write your answers from \cref{exp:prob:asympneginf,exp:prob:asymptotes}.
	\begin{shortsolution}
		$\dd\lim_{x\to-\infty}(-4^x-3)=3$;     $\dd\lim_{x\to\infty}(-4^x-3)= -\infty$.	\\  	    
		$\dd\lim_{x\to-\infty}(2^x+1)=1$;     $\dd\lim_{x\to\infty}(2^x+1)= \infty$.	\\  	    
		$\dd\lim_{x\to-\infty}(-5^x+2) =2$;     $\dd\lim_{x\to\infty}(-5^x+2)= -\infty$.
	\end{shortsolution}
\end{subproblem}
\end{problem}

%===================================
%   Author: Hughes
%   Date:   Feb 2011
%===================================
\begin{problem}[Horizontal asymptotes when $0<b<1$]\label{exp:prob:matchexpslessthan1}%
Consider the functions $F$, $G$, $H$ and $J$ that have formulas
\begin{align*}
	F(x) & =\left(\frac{1}{3}\right)^x-1, & G(x) & =-\left(\frac{1}{4}\right)^x-3, & H(x) & =-\left(\frac{1}{5}\right)^x+2, & J(x) & =\left(\frac{1}{2}\right)^x+1 
\end{align*}
\begin{subproblem}
	Match each of the functions $F$, $G$, $H$, and $J$ with one of the graphs in \cref{exp:fig:matchexpslessthan1}.
	\begin{shortsolution}
		\Vref{exp:fig:matchexpsbaselt12} shows $\dd F(x)=\left(\frac{1}{3}\right)^x-1$.\\
		\Vref{exp:fig:matchexpsbaselt14} shows $\dd G(x)=-\left(\frac{1}{4}\right)^x-3$.\\
		\Vref{exp:fig:matchexpsbaselt13} shows $\dd H(x)=-\left(\frac{1}{5}\right)^x+2$.\\
		\Vref{exp:fig:matchexpsbaselt11} shows $\dd J(x)=\left(\frac{1}{2}\right)^x+1$.
	\end{shortsolution}
\end{subproblem}
%++++++++++++++++
\begin{figure}[!htb]
	\begin{widepage}
	\setlength{\figurewidth}{.2\textwidth}
	\centering
	\begin{subfigure}{\figurewidth}
		\begin{tikzpicture}
			\begin{axis}[
					framed,
					xmin=-10,xmax=10,ymin=-10,ymax=10,
					xtick={-8,-4,4,8},
					ytick={-8,-4,4,8},
					grid=major,
				]
				\addplot expression[domain=-3:9]{(1/2)^x+1};
			\end{axis}
		\end{tikzpicture}
		\caption{}
		\label{exp:fig:matchexpsbaselt11}
	\end{subfigure}
	\hfill
	\begin{subfigure}{\figurewidth}
		\begin{tikzpicture}
			\begin{axis}[
					framed,
					xmin=-10,xmax=10,ymin=-10,ymax=10,
					xtick={-8,-4,4,8},
					ytick={-8,-4,4,8},
					grid=major,
				]
				\addplot expression[domain=-2:9]{(1/3)^x-1};
			\end{axis}
		\end{tikzpicture}
		\caption{}
		\label{exp:fig:matchexpsbaselt12}
	\end{subfigure}
	\hfill
	\begin{subfigure}{\figurewidth}
		\begin{tikzpicture}
			\begin{axis}[
					framed,
					xmin=-10,xmax=10,ymin=-10,ymax=10,
					xtick={-8,-4,4,8},
					ytick={-8,-4,4,8},
					grid=major,
				]
				\addplot expression[domain=-1.35:9]{-(1/5)^x+2};
			\end{axis}
		\end{tikzpicture}
		\caption{}
		\label{exp:fig:matchexpsbaselt13}
	\end{subfigure}
	\hfill
	\begin{subfigure}{\figurewidth}
		\begin{tikzpicture}
			\begin{axis}[
					framed,
					xmin=-10,xmax=10,ymin=-10,ymax=10,
					xtick={-8,-4,4,8},
					ytick={-8,-4,4,8},
					grid=major,
				]
				\addplot expression[domain=-1.35:9]{-(1/4)^x-3};
			\end{axis}
		\end{tikzpicture}
		\caption{}
		\label{exp:fig:matchexpsbaselt14}
	\end{subfigure}
	\caption{Graphs for \cref{exp:prob:matchexpslessthan1}.}
	\label{exp:fig:matchexpslessthan1}
	\end{widepage}
\end{figure}
%++++++++++++++++
\begin{subproblem}\label{exp:prob:asymptoteslt1}
	Using the appropriate function in \cref{exp:fig:matchexpslessthan1},
	we can say that $\left( \frac{1}{2} \right)^x+1\to 1$ as $x\to\infty$.
	Similarly, we can say that $\left( \frac{1}{2} \right)^x+1\to \infty$ as $x\to-\infty$.
	Make analogous statements about the remaining functions in \cref{exp:fig:matchexpslessthan1}.
	\begin{shortsolution}
		$\left(\frac{1}{3}\right)^x-1 \to   -1$ as $x\to\infty$; $\left(\frac{1}{3}\right)^x-1 \to \infty$ as $x\to-\infty$.\\ 
		$-\left(\frac{1}{5}\right)^x+2\to   2$ as $x\to\infty$;  $-\left(\frac{1}{5}\right)^x+2\to -\infty$ as $x\to-\infty$.\\ 
		$-\left(\frac{1}{4}\right)^x-3\to -3$  as $x\to\infty$;  $-\left(\frac{1}{4}\right)^x-3\to -\infty$  as $x\to-\infty$ .
	\end{shortsolution}
\end{subproblem}
\begin{subproblem}
	Use limit notation to re-write your answers from \cref{exp:prob:asymptoteslt1}.
	\begin{shortsolution}
		$\dd\lim_{x\to\infty}\left(\left(\frac{1}{3}\right)^x-1\right) =  -1$;  $\dd\lim_{x\to-\infty}\left(\left(\frac{1}{3}\right)^x-1\right) = \infty$.\\
		$\dd\lim_{x\to\infty}-\left(\left(\frac{1}{5}\right)^x+2\right)=  2$ ;  $\dd\lim_{x\to-\infty}\left(-\left(\frac{1}{5}\right)^x+2\right)= -\infty$.\\
		$\dd\lim_{x\to\infty}\left(-\left(\frac{1}{4}\right)^x-3\right)=-3$  ;  $\dd\lim_{x\to-\infty}\left(-\left(\frac{1}{4}\right)^x-3\right)= -\infty$.
	\end{shortsolution}
\end{subproblem}
\end{problem}
%===================================
%   Author: Hughes
%   Date:   Feb 2011
%===================================
\begin{problem}[Increasing and decreasing functions]
Let $f$ be an exponential function that has formula
\[
	f(x) = a\,b^x
\]
where $b>0$. In each of the following cases, give values of $a$ and $b$ that satisfy the given requirements.

\begin{subproblem}
	$f$ is always increasing, and $b>1$
	\begin{shortsolution}
		$f(x)=2^x$. There are infinitely many other choices available.
	\end{shortsolution}
\end{subproblem}
\begin{subproblem}
	$f$ is always increasing, and $0<b<1$
	\begin{shortsolution}
		$f(x)=-\left(\frac{1}{3}\right)^x$. There are infinitely many other choices available.
	\end{shortsolution}
\end{subproblem}
\begin{subproblem}
	$f$ is always decreasing, and $0<b<1$
	\begin{shortsolution}
		$f(x)=\left(\frac{1}{4}\right)^x$. There are infinitely many other choices available.
	\end{shortsolution}
\end{subproblem}
\begin{subproblem}
	$f$ is always decreasing, and $b>1$
	\begin{shortsolution}
		$f(x) = -5^x$. There are infinitely many other choices available.
	\end{shortsolution}
\end{subproblem}
\end{problem}
%===================================
%   Author: Simonds
%   Date:   Feb 2011
%===================================
\begin{problem}[Successive ratios]
In this activity, we are going to focus on functions that grow at faster and faster rates (concave up).  In 
casual conversation, all such functions are sometimes said to be ``growing exponentially''.  
While that's just fine while you are muttering about the price of gas, in 
the sciences the phrase ``exponential growth'' has a much more precise meaning.

One thing implied by the definition of an exponential function is that if $f(x)=a\,b^x$ (where $b>0$, $b\neq1$), 
then for all values of $x$, $\displaystyle\frac{f(x+1)}{f(x)}=b$.

\begin{subproblem}
	Verify that last assertion by simplifying the expression $\frac{ab^{x+1}}{ab^x}$.
	\begin{shortsolution}
		$\displaystyle\frac{f(x+1)}{f(x)}=\frac{ab^{x+1}}{ab^x}=\frac{b^{x+1}}{b^x}=b$
	\end{shortsolution}
\end{subproblem}
\begin{subproblem}\label{exp:prob:exponentialgraphs}
	For each of the functions graphed in \crefrange{exp:fig:expfunctionm}{exp:fig:expfunctionr}, 
	determine whether or not the function is exponential.  If it is, determine the base $b$ of the function. Your first step might be to choose 
	3 ordered pairs that have successive $x$-values (e.g 1, 2, 3), and compare the ratio of the $y$-values.
	\begin{shortsolution}
		The function graphed in \cref{exp:fig:expfunctionn} is exponential, $n(x)=2^x$.  
		The function graphed in \cref{exp:fig:expfunctionr} is exponential, and $r(x)=\left(\frac13\right)^x$.
		The function graphed in \cref{exp:fig:expfunctionm} is not exponential.  
	\end{shortsolution}
\end{subproblem}

\begin{figure}[!htb]
	\begin{widepage}
	\begin{minipage}{.25\textwidth}
		\centering
		\begin{tikzpicture}
			\begin{axis}[
					framed,
					xmin=-2,xmax=4,
					ymin=-3,ymax=11,
					xtick={-1,1,2,3},
					ytick={4,8},
					minor ytick={2,4,...,8},
					grid=both,
				]
				\addplot expression[domain=-2:3.2]{2^x};
				\addplot[soldot]coordinates{(0,1)(1,2)(2,4)(3,8)};
			\end{axis}
		\end{tikzpicture}
		\caption{$y=n(x)$}
		\label{exp:fig:expfunctionn}
	\end{minipage}
	\hfill
	\begin{minipage}{.25\textwidth}
		\centering
		\begin{tikzpicture}
			\begin{axis}[
					framed,
					xmin=-3,xmax=4,
					ymin=-3,ymax=11,
					xtick={-2,...,3},
					ytick={4,8},
					minor ytick={2,4,...,8},
					grid=both,
				]
				\addplot expression[domain=-2.19:4]{(1/3)^x};
				\addplot[soldot]coordinates{(0,1)(-1,3)(-2,9)};
			\end{axis}
		\end{tikzpicture}
		\caption{$y=r(x)$}
		\label{exp:fig:expfunctionr}
	\end{minipage}
	\hfill
	\centering
	\begin{minipage}{.25\textwidth}
		\centering
		\begin{tikzpicture}
			\begin{axis}[
					framed,
					xmin=-3,xmax=4,
					ymin=-3,ymax=11,
					xtick={-2,...,3},
					ytick={4,8},
					minor ytick={2,4,...,8},
					grid=both,
				]
				\addplot+[->]expression[domain=0:3.17]{x^2+1};
				\addplot[soldot]coordinates{(0,1)(1,2)(2,5)(3,10)};
			\end{axis}
		\end{tikzpicture}
		\caption{$y=m(x)$}
		\label{exp:fig:expfunctionm}
	\end{minipage}
	\end{widepage}
\end{figure}

\begin{subproblem}\label{exp:prob:exponentialtables}
	For each of the functions implied by the data in \crefrange{exp:tab:expfunctiong}{exp:tab:expfunctionk}, 
	determine whether or not the function is exponential.  If it is, determine the base $b$ of the function. 
	\begin{shortsolution}
		The function tabulated in \cref{exp:tab:expfunctiong} is exponential, and $g(x)=3^x$.  
		The function tabulated in \cref{exp:tab:expfunctionh} is exponential, and 
		$h(x)=\sqrt[3]{25}\cdot(\sqrt[3]{5})^x$.  
		The function tabulated in \cref{exp:tab:expfunctionk} is not exponential.
	\end{shortsolution}
\end{subproblem}
\begin{table}[!htb]
	\begin{widepage}
	\centering
	\hfill
	\begin{minipage}{.25\textwidth}
		\centering
		\caption{$y=g(x)$}
		\begin{tabular}{S[table-format=1.0]S[table-format=3.0]}
			\beforeheading
			\heading{$x$} & \heading{$y$} \\
			\afterheading
			1             & 3             \\\normalline
			2             & 9             \\\normalline
			3             & 27            \\\normalline
			4             & 81            \\\normalline
			5             & 241           \\\lastline
		\end{tabular}
		\label{exp:tab:expfunctiong}
	\end{minipage}
	\hfill
	\begin{minipage}{.25\textwidth}
		\centering
		\caption{$y=h(x)$}
		\begin{tabular}{S[table-format=2.0]S[table-format=4.0]}
			\beforeheading
			\heading{$x$} & \heading{$y$} \\
			\afterheading
			1             & 5             \\\normalline
			4             & 25            \\\normalline
			7             & 125           \\\normalline
			10            & 625           \\\normalline
			13            & 3125          \\\lastline
		\end{tabular}
		\label{exp:tab:expfunctionh}
	\end{minipage}
	\hfill
	\begin{minipage}{.25\textwidth}
		\centering
		\caption{$y=k(x)$}
		\begin{tabular}{S[table-format=2.0]S[table-format=4.0]}
			\beforeheading
			\heading{$x$} & \heading{$y$} \\
			\afterheading
			1             & 4             \\\normalline
			2             & 16            \\\normalline
			4             & 64            \\\normalline
			8             & 256           \\\normalline
			16            & 1024          \\\lastline
		\end{tabular}
		\label{exp:tab:expfunctionk}
	\end{minipage}
	\hfill
	\end{widepage}
\end{table}

\end{problem}
%===================================
%   Author: Hughes
%   Date:   Feb 2011
%===================================
\begin{problem}[True or false?]
Consider the function $f$ that has formula
\[
	f(x)=a\,b^x+c
\]
where $a$ and $c$ are any real numbers and $b$ is a positive real number not equal to $1$. Label each 
of the following statements as true (T) or false (F); if you believe a statement is false, 
provide an example that supports your answer.

\begin{subproblem}
	If $b>1$, then $f$ is an increasing function for any value of $a$.
	\begin{shortsolution}
		False. Consider $f(x)=-2^x$.
	\end{shortsolution}
\end{subproblem}
\begin{subproblem}
	If $0<b<1$, then $f$ is a decreasing function for any value of $a$.
	\begin{shortsolution}
		False. Consider $f(x)=-\left(\frac{1}{9}^{x}\right)$.
	\end{shortsolution}
\end{subproblem}
\begin{subproblem}
	If $b>1$, then $f(x)\to c$ as $x\to -\infty$.
	\begin{shortsolution}
		True.
	\end{shortsolution}
\end{subproblem}
\begin{subproblem}
	If $0<b<1$, then $f(x)\to c$ as $x\to\infty$.
	\begin{shortsolution}
		True.
	\end{shortsolution}
\end{subproblem}
\begin{subproblem}
	If $b>1$, then $f(x)\to \infty$ as $x\to \infty$.
	\begin{shortsolution}
		False. Consider $f(x)=-8^{x}$.
	\end{shortsolution}
\end{subproblem}
\begin{subproblem}
	If $0<b<1$, then $f(x)\to \infty$ as $x\to -\infty$.
	\begin{shortsolution}
		False. Consider $f(x)=-\left(\frac{1}{8}\right)^{x}$.
	\end{shortsolution}
\end{subproblem}
\begin{subproblem}
	If $0<b<1$, then $\dd\lim_{x\to\infty}f(x)=0$.
	\begin{shortsolution}
		True.
	\end{shortsolution}
\end{subproblem}
\begin{subproblem}
	If $b>1$, then $\dd\lim_{x\to-\infty}f(x)=\infty$.
	\begin{shortsolution}
		False. Consider $f(x)=8^{x}$.
	\end{shortsolution}
\end{subproblem}
\begin{subproblem}
	If $b>1$, then $\dd\lim_{x\to\infty}f(x)=\infty$ or $\dd\lim_{x\to\infty}f(x)=-\infty$.
	\begin{shortsolution}
		True.
	\end{shortsolution}
\end{subproblem}
\begin{subproblem}
	If $b>1$, then the line defined by $y=0$ is an asymptote of $f$.
	\begin{shortsolution}
		True.
	\end{shortsolution}
\end{subproblem}
\begin{subproblem}
	If $0<b<1$, then the line defined by $y=0$ is an asymptote of $f$.
	\begin{shortsolution}
		True.
	\end{shortsolution}
\end{subproblem}
\begin{subproblem}
	If $0<b<1$, then the line defined by $x=0$ is an asymptote of $f$.
	\begin{shortsolution}
		False.  Consider any exponential function.  There are no vertical asymptotes.
	\end{shortsolution}
\end{subproblem}
\end{problem}
\end{exercises}

\section{Solving exponential equations}
As we have seen in the past, we are often interested in finding what value or values of a variable will cause certain conditions to be met.  
We've solved for the unknown variable $x$ in equations like $2x+4=5$ and $x^2+8x-6=0$.  Now we will solve for $x$ in equations where $x$ is in the exponent.
Before we begin though, we will refresh ourselves on the rules of exponents.

\reformatpropslist{E}
\begin{pccspecialcomment}[Properties of exponents]
	Recall some of the basic rules for exponents, assuming that $a$, $b$, $x$,
	and $y$ are real numbers:
	\begin{props}
		\item\label{exp:prop:pow} $a^xb^x=(ab)^x$ 
		\item\label{exp:prop:add} $a^xa^y =a^{x+y}$
		\item\label{exp:prop:mult} $a^{(xy)} = (a^x)^y$
	\end{props}
\end{pccspecialcomment}
We can often make use of these rules to simplify exponential expressions.
\begin{essentialskills}
	%===================================
	%   Author: Hughes
	%   Date:   February 2012
	%===================================
	\begin{problem}
	Evaluate each of the following without the use of a calculator. 
	\begin{multicols}{4}
		\begin{subproblem}
			$2^2$ 
			\begin{shortsolution}
				$4$ 
			\end{shortsolution}
		\end{subproblem}
		\begin{subproblem}
			$-2^3$ 
			\begin{shortsolution}
				$-8$ 
			\end{shortsolution}
		\end{subproblem}
		\begin{subproblem}
			$-2^4$ 
			\begin{shortsolution}
				$-16$ 
			\end{shortsolution}
		\end{subproblem}
		\begin{subproblem}
			$(-3)^2$     
			\begin{shortsolution}
				$9$
			\end{shortsolution}
		\end{subproblem}
		\begin{subproblem}
			$\left( \frac{2}{3} \right)^2$
			\begin{shortsolution}
				$\frac{4}{9}$ 
			\end{shortsolution}
		\end{subproblem}
		\begin{subproblem}
			$ \frac{4}{9^2}$ 
			\begin{shortsolution}
				$\frac{4}{81}$ 
			\end{shortsolution}
		\end{subproblem}
		\begin{subproblem}
			$-\frac{5^2}{6}$ 
			\begin{shortsolution}
				$-\frac{25}{6}$ 
			\end{shortsolution}
		\end{subproblem}
		\begin{subproblem}
			$-\left( \frac{7}{10} \right)^2$ 
			\begin{shortsolution}
				$-\frac{49}{100}$ 
			\end{shortsolution}
		\end{subproblem}
		\begin{subproblem}
			$\left(- \frac{7}{10} \right)^2$ 
			\begin{shortsolution}
				$\frac{49}{100}$ 
			\end{shortsolution}
		\end{subproblem}
		\begin{subproblem}
			$4^0$ 
			\begin{shortsolution}
				$1$ 
			\end{shortsolution}
		\end{subproblem}
		\begin{subproblem}
			$-58^0$ 
			\begin{shortsolution}
				$-1$ 
			\end{shortsolution}
		\end{subproblem}
		\begin{subproblem}
			$-3\left( -2^{768} \right)^0$ 
			\begin{shortsolution}
				$-3$ 
			\end{shortsolution}
		\end{subproblem}
	\end{multicols}
	\end{problem}
\end{essentialskills}
%===================================
%   Author: Hughes, Jordan
%   Date:   Aug 2011
%===================================
\begin{pccexample}
	Use properties of exponents to write the following formulas in the 
	form $f(t)=a\,b^t$. Leave your answer in exact form. 
	\begin{multicols}{3}
		\begin{enumerate}
			\item $f(t)  = 2\left(3^t\right)\left(3^{3t}\right) $
			\item $g(t)  = 7(1.08)^t(3.2)^t $
			\item $h(t)  = 8^t\left(2^{\frac{t+1}{2}}\right)$
		\end{enumerate}
	\end{multicols}
	\begin{pccsolution}
		We use the properties of exponents, and demonstrate the simplification one step per-line
		\begin{enumerate}
			\item
			$\begin{aligned}[t]
				f(t) & = 2\left(3^t\right)\left(3^{3t}\right) &                             \\
				     & = 2\left(3^{t+3t}\right)               & \text{\cref{exp:prop:add}}  \\
				     & = 2\left(3^{4t}\right)                 &                             \\
				     & = 2(3^4)^t                             & \text{\cref{exp:prop:mult}} \\
				     & = 2(81)^t                              &                             
			\end{aligned}$
			\item
			$\begin{aligned}[t]
				g(t) & = 7(1.08)^t(3.2)^t  &                            \\
				     & = 7(1.08\cdot3.2)^t & \text{\cref{exp:prop:pow}} \\
				     & = 7(3.456)^t        &                            
			\end{aligned}$
			\item
			$\begin{aligned}[t]
				h(t) & = 8^t\left(2^{\frac{t+1}{2}}\right)                             &                            \\
				     & = (2^3)^t\left(2^{\frac{t+1}{2}}\right)                         &                            \\
				     & = 2^{3t}\left(2^{\frac{t+1}{2}}\right)                          &                            \\
				     & = 2^{3t+\frac{t+1}{2}}                                          & \text{\cref{exp:prop:add}} \\
				     & = 2^{\frac{7t+1}{2}}                                            &                            \\
				     & = \left(2^{\nicefrac{1}{2}}\right)^{7t+1}                       & \text{\cref{exp:prop:pow}} \\
				     & = \left(\sqrt{2}\right)\left(\sqrt{2}\right)^{7t}               & \text{\cref{exp:prop:add}} \\
				     & = \left(\sqrt{2}\right)\left(\left(\sqrt{2}\right)^{7}\right)^t & \text{\cref{exp:prop:pow}} \\
				& = \left(\sqrt{2}\right)\left(\sqrt{128}\right)^t
			\end{aligned}$
		\end{enumerate}
		\mbox{}
	\end{pccsolution}
\end{pccexample}

\begin{doyouunderstand}
	%===================================
	%   Author: Kouzes, Hughes
	%   Date:   Apr,Aug 2011
	%===================================
	\begin{problem}
	Use properties of exponents to write the following formulas in the 
	form $f(t)=a\,b^t$. Leave your answer in exact form. 
	\[
		f(t)=2^{t+1}\cdot 2^{3t}
	\]
	\begin{shortsolution}
		$\begin{aligned}[t]
			f(t) & = 2^{t+1}\cdot 2^{3t}    \\
			     & = 2^t\cdot 2\cdot 2^{3t} \\
			     & = 2\cdot 2^{4t}          \\
			     & = 2\cdot 16^t            
		\end{aligned}$
	\end{shortsolution}
	\end{problem}
\end{doyouunderstand}

\begin{pccexample}
	Use your knowledge of exponents to solve the following 
	equations:
	\begin{multicols}{2}
		\begin{enumerate}
			\item $3^{x}=27$
			\item $2^{x^2}=16$
		\end{enumerate}
	\end{multicols}
	\begin{pccsolution}
		\begin{enumerate}
			\item We know that $3^3=27$, so $x=3$.
			\item We know that $2^4=16$, which means that 
			\[
				x^2=4
			\]
			The two solutions are $2$ and $-2$.
			It is left an exercise to check that both of these solutions satisfy the 
			equation $2^{x^2}=16$.
		\end{enumerate}
	\end{pccsolution}
\end{pccexample}
\begin{doyouunderstand}
	%===================================
	%   Author: Hughes
	%   Date:   April 2011
	%===================================
	\begin{problem}
	Solve the following exponential equation. Check your solution. 
	\[
		2^x=4
	\]
	\begin{shortsolution}
		$x=2$ (The solution is $2$.)
	\end{shortsolution}
	\end{problem}
\end{doyouunderstand}

%===================================
%   Author: Hughes
%   Date:   August 2011
%===================================
\begin{pccexample}\label{exp:ex:factor}
	Solve the following equation by factoring. Hint: put $t=3^x$ as 
	your first step.
	\begin{equation}\label{exp:eq:solvefact}
		3^{2x}-10\cdot 3^x+9=0
	\end{equation}
	\begin{pccsolution}
		Following the hint, we substitute $t=3^x$ into \cref{exp:eq:solvefact}, which gives
		\[
			t^2 - 10t +9 = 0
		\]
		This equation can be factored and written as $(t-9)(t-1)=0$. We therefore 
		have to solve the equations
		\[
			3^x = 9, \qquad 3^x = 1
		\]
		Using our knowledge of exponents, we see that $x=2$ or $x=0$. It is left 
		an exercise to check that both of these solutions satisfy \cref{exp:eq:solvefact}.
	\end{pccsolution}
\end{pccexample}

\begin{doyouunderstand}
	\begin{problem}
	Solve the following equation.
	Try putting $t=4^x$ as your first step. Check your solutions.
	\[
		% factor: (4^x-1)*(4^x-4)
		4^{2x}-5\cdot 4^x-4=0
	\]
	\begin{shortsolution}
		$x=0,2$ (The solutions are $0$, and $2$)
	\end{shortsolution}
	\end{problem}
\end{doyouunderstand}

Not all solutions are integers; in fact in applied problems solutions are almost never integers.
We don't yet have the tools to find exact solutions to most exponential equations, so we
are going to explore techniques for approximating solutions to equations.

%===================================
%   Author: Hughes
%   Date:   July 2011
%===================================
\begin{pccexample}
	Use a table of values or a graph to solve the following exponential 
	equation
	\[
		2^x = 13
	\]
	\begin{pccsolution}
		\Cref{exp:tab:solvenoninteger} 
		shows the solution is on the interval $[3,4]$; given that 13 is closer to 16 than 
		it is to 8, we expect our solution to be closer to $4$ than it is to $3$.
														
		\Cref{exp:fig:solvenoninteger} shows $y=2^x$ and $y=13$. The solution 
		to the equation $2^x=13$ is the $x$-coordinate of the point of intersection; using a calculator, we find
		the $x$-coordinate is approximately $3.70$. 
	\end{pccsolution}
\end{pccexample}

\begin{table}[!htb]
	\mbox{}\hfill
	\begin{minipage}{.3\textwidth}
		\centering
		\captionof{table}{}
		\begin{tabular}{S[table-format=1.0]S[table-format=2.0]}
			\beforeheading
			\heading{$x$} & \heading{$2^x$} \\
			\afterheading
			1             & 2               \\\normalline
			2             & 4               \\\normalline
			3             & 8               \\\normalline
			4             & 16              \\\normalline
			\lastline
		\end{tabular}
		\label{exp:tab:solvenoninteger}
	\end{minipage}%
	\hfill
	\begin{minipage}{.4\textwidth}
		\centering
		\begin{tikzpicture}
			\begin{axis}[
					framed,
					xmin=-1,xmax=5,
					ymin=-3,ymax=18,
					xtick={1,...,4},
					grid=both,
				]
				\addplot expression[domain=-1:4]{2^x};
				\addplot expression[domain=-1:5]{13};
			\end{axis}
		\end{tikzpicture}     
		\captionof{figure}{$y=2^x$}
		\label{exp:fig:solvenoninteger}
	\end{minipage}%
	\hfill\mbox{}
\end{table}

\begin{doyouunderstand}
	%===================================
	%   Author: Hughes
	%   Date:   July 2011
	%===================================
	\begin{problem}
	Solve the following exponential equation using either a graph 
	or a table of values; state your solution to 2 decimal places, and check your answer. 
	\[
		2^x=10
	\]
	\begin{shortsolution}
		$x\approx 3.32$ (The solution is approximately $3.32$.)
	\end{shortsolution}
	\end{problem}
\end{doyouunderstand}

\investigation*{}
%===================================
%   Author: Hughes/Simonds
%   Date:   September 2011
%===================================
\begin{problem}[To the Moon]
Let $x$ represent the horizontal
distance (in inches) that you are from your front door. Imagine that
for every inch that you move away from your front door,
your elevation above the ground $y$ (in inches) magically increases
according to the rule $y=2^x$. 
\begin{subproblem}
	What is your elevation above the ground when you are at a horizontal distance of \SI{1}{\foot}
	from your front door? State your answers in \si{\foot}, using the fact that there are \SI{12}{\inch}
	in \SI{1}{\foot}.
	\begin{shortsolution}
		$\frac{2^{12}}{12}\approx 341.33$; our elevation is approximately \SI{341}{\foot} above the ground when 
		we are \SI{1}{\foot} from our front door.
	\end{shortsolution}
\end{subproblem}
\begin{subproblem}
	What is your elevation above the ground when you are at a horizontal distance of \SI{2}{\foot}
	from your front door? State your answers in \si{\mile}, using the fact that there are \SI{63360}{\inch}
	in \SI{1}{\mile}.
	\begin{shortsolution}
		$\frac{2^{24}}{63360}\approx 264.79$; our elevation is approximately \SI{264.79}{\mile} above the ground when
		we are \SI{2}{\foot} from our front door.
	\end{shortsolution}
\end{subproblem}
\begin{subproblem}
	Given that the distance between
	the Earth and the Moon is approximately $\SI{1.595 e10}{\inch}$,
	what is your horizontal distance from your front door when you reach the Moon?
	Use a graphing calculator to help you.
	\footnote{The distance from the Earth to the Moon (in miles) is \SI{227037}{\mile}.}
	\begin{shortsolution}
		We need to solve the equation $2^x=1.595\times 10^{10}$. Using a graphing
		calculator, we obtain $x\approx 33.89$; we conclude that when we reach the Moon,
		our horizontal distance from our front door is approximately $\SI{34}{\inch}$.
	\end{shortsolution}
\end{subproblem}
\end{problem}


\begin{exercises}
\begin{problem}[Algebraic manipulation]	
Use properties of exponents to write each of the following formulas in the 
form $f(t)=a\,b^t$. Leave your answers in exact form. 
\begin{multicols}{2}
	\begin{subproblem}
		$f(t)=4^{t+1}\cdot 2^{3t}$
		\begin{shortsolution}
			$\begin{aligned}[t]
				f(t) & = 4^{t+1}\cdot 2^{3t}                  \\
				     & = \left( 2^2 \right)^{t+1}\cdot 2^{3t} \\
				     & = 2^{2t}\cdot 2^2\cdot 2^{3t}          \\
				     & = 4\cdot 2^{5t}                        \\
				     & = 4\cdot 32^t                          
			\end{aligned}$
		\end{shortsolution}
	\end{subproblem}
	\begin{subproblem}
		$f(t)=4^\frac{t+1}{2}\cdot 3^{2t}$. 
		\begin{shortsolution}
			$\begin{aligned}[t]
				f(t) & =  4^\frac{t+1}{2} \cdot 3^{2t}          \\
				     & =  4^{\frac{1}{2}(t+1)} \cdot 3^{2t}     \\
				     & =  (4^{\frac{1}{2}})^{t+1} \cdot (3^2)^t \\
				     & =  2^{t+1} \cdot 9^t                     \\
				     & =  2^1 \cdot 2^t \cdot 9^t               \\
				     & =  2 \cdot (2 \cdot 9)^t                 \\
				     & =  2 \cdot 18^t                          
			\end{aligned}$
		\end{shortsolution}
	\end{subproblem}
\end{multicols}
\end{problem}	

\begin{problem}[Integer solutions]
Solve each of the following exponential equations. Check your solutions. 
\phpreference{exponentialIntSolns}{\theproblem}
\begin{multicols}{3}
	\begin{subproblem}
		$2^{-x}=\frac{1}{8}$
		\begin{shortsolution}
			$x=3$ (The solution is $3$.)
		\end{shortsolution}
	\end{subproblem}
	\begin{subproblem}
		$4^x=64$
		\begin{shortsolution}
			$x=3$ (The solution is $3$.)
		\end{shortsolution}
	\end{subproblem}
	\begin{subproblem}
		$3\cdot 5^x=75$
		\begin{shortsolution}
			$x=2$ (The solution is $2$.)
		\end{shortsolution}
	\end{subproblem}
	\begin{subproblem}
		$2^{x^2}-16=0$
		\begin{shortsolution}
			$x=\pm 2$ (The solutions are $\pm 2$.)
		\end{shortsolution}
	\end{subproblem}
	\begin{subproblem}
		$2^{x^2-5x}=2^{-6}$
		\begin{shortsolution}
			$x=2,3$ (The solutions are $2$ and $3$.)
		\end{shortsolution}
	\end{subproblem}
	\begin{subproblem}
		$7^{x^2}=49^{x}$
		\begin{shortsolution}
			$x=2,0$ (The solutions are $2$ and $0$.)
		\end{shortsolution}
	\end{subproblem}
\end{multicols}
\end{problem}

\begin{problem}[Non-integer solutions]
Solve each of the following exponential equations using either a graph 
or a table of values; state your solutions to 2 decimal places, and check your answers. 
\begin{multicols}{4}
	\begin{subproblem}
		$3^x=11$
		\begin{shortsolution}
			$x\approx 2.18$ (The solution is approximately $2.18$.)
		\end{shortsolution}
	\end{subproblem}
	\begin{subproblem}
		$4^x=29$
		\begin{shortsolution}
			$x\approx 2.43$ (The solution is approximately $2.43$.)
		\end{shortsolution}
	\end{subproblem}
	\begin{subproblem}
		$5^x=61$
		\begin{shortsolution}
			$x\approx 2.55$ (The solution is approximately $2.55$.)
		\end{shortsolution}
	\end{subproblem}
	\begin{subproblem}
		$7^x=-1$
		\begin{shortsolution}
			No solution.
		\end{shortsolution}
	\end{subproblem}
	\begin{subproblem}
		$-6^x = -31$
		\begin{shortsolution}
			$x\approx 1.92$ (The solution is approximately $1.92$.)
		\end{shortsolution}
	\end{subproblem}
	\begin{subproblem}
		$3\cdot 5^x = 7$
		\begin{shortsolution}
			$x\approx 0.53$ (The solution is approximately $0.53$.)
		\end{shortsolution}
	\end{subproblem}
	\begin{subproblem}
		$2^x+5 = 10$
		\begin{shortsolution}
			$x\approx 2.32$ (The solution is approximately $2.32$.)
		\end{shortsolution}
	\end{subproblem}
	\begin{subproblem}
		$7^x=50$
		\begin{shortsolution}
			$x\approx 2.01$ (The solution is approximately $2.01$.)
		\end{shortsolution}
	\end{subproblem}
\end{multicols}
\end{problem}

\begin{problem}[Factoring]
Use the technique demonstrated in \cref{exp:ex:factor} to help you solve the following 
equations.
For \cref{exp:prob:needfactor} try putting $t=4^x$ as your first step. Check your solutions.
\begin{multicols}{3}
	\begin{subproblem}\label{exp:prob:needfactor}
		% factor: (4^x-1)*(4^x+3)
		$4^{2x}+2\cdot 4^x=3$
		\begin{shortsolution}
			$x=0$ (The solution is $0$.) Note that there are no solutions to $4^x+3=0$.
		\end{shortsolution}
	\end{subproblem}
	% factor : (5^x-1)(5^x+1)
	\begin{subproblem}
		$5^{2x}-1=0$
		\begin{shortsolution}
			$x=1$ (The solution is $1$.) Note that there are no solutions to $5^x=-1$.
		\end{shortsolution}
	\end{subproblem}
	\begin{subproblem}
		% factor: 5^x*(5^x-25)*(5^x-1)
		$125^x-26\cdot 25^x+25\cdot 5^x=0$
		\begin{shortsolution}
			$x=0,2$ (The solutions are $0$ and $2$.) Note that there are no solutions to $5^x=0$.
		\end{shortsolution}
	\end{subproblem}
\end{multicols}
\end{problem}
%===================================
%   Author: Hughes
%   Date:   February 2012
%===================================
\begin{problem}[Matching graphs to equations]\label{exp:prob:matcheqns}%
\pccname{Jake} and \pccname{Marisa} are solving the following exponential equations. To guide them, 
they have graphed the functions involved in \cref{exp:fig:matcheqns}. Match the equations
with the appropriate graph.
\[
	4^{-x^2}=\frac{1}{4}, \qquad 
	-\left( \frac{1}{3} \right)^{x}=-3, \qquad
	5^x=5, \qquad 
	3^{x^2}=3 \qquad 
\]
\begin{shortsolution}
	\begin{itemize}
		\item \Cref{exp:fig:matcheqns1} represents $3^{x^2}=3$
		\item \Cref{exp:fig:matcheqns2} represents $4^{-x^2}=\frac{1}{4}$
		\item \Cref{exp:fig:matcheqns3} represents $5^x=5$
		\item \Cref{exp:fig:matcheqns4} represents $-\left( \frac{1}{3} \right)^{x}=-3$
	\end{itemize}
\end{shortsolution}
\end{problem}
\begin{figure}[!htb]
	\begin{widepage}
	\centering
	\begin{subfigure}{.2\textwidth}
		\begin{tikzpicture}
			\begin{axis}[
					framed,
					xmin=-3,xmax=3,
					ymin=-5,ymax=5,
					ytick={-2,2},
					grid=both,
				]
				\addplot expression[domain=-1.2:1.2]{3^(x^2)};
				\addplot expression[domain=-3:3]{3};
			\end{axis}
		\end{tikzpicture}     
		\caption{}
		\label{exp:fig:matcheqns1}
	\end{subfigure}%
	\hfill
	\begin{subfigure}{.2\textwidth}
		\begin{tikzpicture}
			\begin{axis}[
					framed,
					xmin=-3,xmax=3,
					ymin=-1,ymax=2,
					ytick={1},
					grid=both,
				]
				\addplot expression[domain=-1.7:1.7]{4^(-x^2)};
				\addplot expression[domain=-3:3]{1/4};
			\end{axis}
		\end{tikzpicture}     
		\caption{}
		\label{exp:fig:matcheqns2}
	\end{subfigure}%
	\hfill
	\begin{subfigure}{.2\textwidth}
		\begin{tikzpicture}
			\begin{axis}[
					framed,
					xmin=-3,xmax=3,
					ymin=-5,ymax=6,
					ytick={-2,2},
					grid=both,
				]
				\addplot expression[domain=-1.2:1.1]{5^(x)};
				\addplot expression[domain=-3:3]{5};
			\end{axis}
		\end{tikzpicture}
		\caption{}
		\label{exp:fig:matcheqns3}
	\end{subfigure}%
	\hfill
	\begin{subfigure}{.2\textwidth}
		\begin{tikzpicture}
			\begin{axis}[
					framed,
					xmin=-3,xmax=3,
					ymin=-5,ymax=6,
					ytick={-2,2},
					grid=both,
				]
				\addplot expression[domain=-1.4:1.1]{-(1/3)^x};
				\addplot expression[domain=-3:3]{-3};
			\end{axis}
		\end{tikzpicture}
		\caption{}
		\label{exp:fig:matcheqns4}
	\end{subfigure}%
	\caption{Graphs for \cref{exp:prob:matcheqns}}
	\label{exp:fig:matcheqns}
	\end{widepage}
\end{figure}

%===================================
%   Author: Hughes
%   Date:   February 2012
%===================================
\begin{problem}[Zeros]
Consider the functions $f$, $g$, $h$, and $k$ that have formulas
\[
	f(x)=4^{-x^2}-\frac{1}{4}, \qquad 
	g(x)=3-\left( \frac{1}{3} \right)^{x}, \qquad
	h(x)=5^x-5, \qquad 
	k(x)=3^{x^2}-3
\]
Use your work from \cref{exp:prob:matcheqns} to help you decide how 
many zeros each function has.
\begin{shortsolution}
	\begin{itemize}
		\item $f$ has $2$ zeros
		\item $g$ has $1$ zero
		\item $h$ has $1$ zero
		\item $k$ has $2$ zeros
	\end{itemize}
\end{shortsolution}
\end{problem}

%===================================
%   Author: Hughes
%   Date:   February 2012
%===================================
\begin{problem}[True or false?]
\pccname{Myron} and \pccname{Win} have studied all of the exponential equations so far, and 
are trying to generalize their results. They begin with the equation
\[
	2^x=c
\]
where $c$ can be any real number. Help them decide if the following statements
are true of false, if they are false, provide an example that supports your 
answer.
\begin{subproblem}
	If $c>0$ then the equation has a solution.
	\begin{shortsolution}
		True. 
	\end{shortsolution}
\end{subproblem}
\begin{subproblem}
	If $c<0$ then the equation has a solution. 
	\begin{shortsolution}
		False; consider $2^x=-3$. 
	\end{shortsolution}
\end{subproblem}
\begin{subproblem}
	If $c=0$ then the equation has a solution. 
	\begin{shortsolution}
		False; there are no values of $x$ that satisfy $2^x=0$.
	\end{shortsolution}
\end{subproblem}
\end{problem}


%===================================
%   Author: Hughes
%   Date:   February 2012
%===================================
\begin{problem}[Beyond exponential equations]
The function $f$ that has formula $f(x)=\left( \frac{1}{2} \right)^x+\frac{1}{2}$ is shown in \cref{exp:fig:extension}, 
along with a mystery function $g$.
\end{problem}
\begin{figure}[!htb]
	\begin{widepage}
	\begin{tikzpicture}
		\begin{axis}[
				framed,
				height=.25\textwidth,
				xmin=-1,xmax=10,
				ymin=-3,ymax=3,
				xtick={1,...,9},
				grid=both,
				legend pos=south east,
			]
			\addplot expression[domain=-1:10]{(1/2)^x+1/2};
			\addplot expression[domain=-1:10,samples=200]{sin(deg(pi*x))+1/2};
			\legend{$f$,$g$}
		\end{axis}
	\end{tikzpicture}     
	\caption{}
	\label{exp:fig:extension}
	\end{widepage}
\end{figure}
\begin{subproblem}
	Use \cref{exp:fig:extension} to decide how many solutions there are to the equation $f(x)=g(x)$
	on the interval $[0,9]$.
	\begin{shortsolution}
		By counting the number of points of intersection, there are $10$ solutions on 
		the interval $[0,9]$.
	\end{shortsolution}
\end{subproblem}
\begin{subproblem}
	Hence determine how many zeros the function $h(x)=f(x)-g(x)$
	has on the interval $[0,9)$.
	\begin{shortsolution}
		The function $h$ has $10$ zeros on the interval $[0,9]$.
	\end{shortsolution}
\end{subproblem}
\end{exercises}
			
\section{Finding an exponential function given two points}\label{exp:sec:findformula}
\begin{outcomes}
	\begin{outcomelist}
		\item You will be able to find a formula for an exponential curve that connects two points in the plane. 
		\item You will be able to predict the population of the United States and other countries in the year 2050.
		\item You will be able to model the amount of $\mathrm{CO}_2$ emitted in the United States over time.
	\end{outcomelist}
\end{outcomes}
%===================================
%   Author: Jordan
%   Date:   Nov 2011
%===================================
Consider a puzzle where two points are given on the plane, and you are asked to connect them with a curve.  
There are infinitely many ways to do this.  In the past, you've connected two points with a straight 
line (as in \cref{exp:fig:pointsline}) and found the equation that represents that line.  Of course, you 
could also connect the points with a random curve of your liking (as in \cref{exp:fig:pointscurve}).  
In this section we will try to connect the points with an \emph{exponential} 
curve (as in \cref{exp:fig:pointsexp}) and simultaneously find that curve's equation. 
			
\begin{figure}[!htb]
	\begin{subfigure}[t]{0.25\textwidth}
		\begin{tikzpicture}
			\begin{axis}[
					xmin=-2, xmax=8,
					ymin=-3, ymax=10,
					xtick={-1,...,7},
					ytick={3,6,9},
				]
				\addplot expression[domain=-2:8]{3/6*(x-1)+2};
				\addplot[soldot]coordinates{(1,2)(7,5)};
			\end{axis}
		\end{tikzpicture}
		\caption{Points connected with a line}
		\label{exp:fig:pointsline}
	\end{subfigure}%
	\hfill
	\begin{subfigure}[t]{0.25\textwidth}
		\begin{tikzpicture}
			\begin{axis}[
					xmin=-2, xmax=8,
					ymin=-3, ymax=10,
					xtick={-1,...,7},
					ytick={3,6,9},
				]
				\addplot expression[domain=-2:8,samples=100]{(x-1)/2+2+2*sin(deg(2*3.1415926/6*(x-1)))};
				\addplot[soldot]coordinates{(1,2)(7,5)};
			\end{axis}
		\end{tikzpicture}
		\caption{Points connected with an arbitrary curve}
		\label{exp:fig:pointscurve}
	\end{subfigure}%
	\hfill
	\begin{subfigure}[t]{0.25\textwidth}
		\begin{tikzpicture}
			\begin{axis}[
					xmin=-2, xmax=8,
					ymin=-3, ymax=10,
					xtick={-1,...,7},
					ytick={3,6,9},
				]
				\addplot expression[domain=-2:8]{2*(5/2)^((x-1)/6)};
				\addplot[soldot]coordinates{(1,2)(7,5)};
			\end{axis}
		\end{tikzpicture}
		\caption{Points connected with an exponential curve}
		\label{exp:fig:pointsexp}
	\end{subfigure}%
	\caption{Connecting two points with various curves}
\end{figure}
			
\begin{essentialskills}
	\begin{problem} Simplify the given expressions.
	\begin{multicols}{4}
		\begin{subproblem}
			$\dfrac{b}{b^{-2}}$
			\begin{shortsolution}
				$b^{3}$
			\end{shortsolution}
		\end{subproblem}
		\begin{subproblem}
			$\dfrac{b^{-4}}{b^{-10}}$
			\begin{shortsolution}
				$b^6$
			\end{shortsolution}
		\end{subproblem}
		\begin{subproblem}
			$\dfrac{b^{-3}}{b^3}$
			\begin{shortsolution}
				$\dfrac{1}{b^6}$ or $b^{-6}$
			\end{shortsolution}
		\end{subproblem}
		\begin{subproblem}
			$\left(\dfrac{b^{-1}}{b^{-5}}\right)^{-2}$
			\begin{shortsolution}
				$\dfrac{1}{b^8}$ or $b^{-8}$
			\end{shortsolution}
		\end{subproblem}
	\end{multicols}
	\end{problem}
											
	\begin{problem} Simplify the given expressions.
	\begin{multicols}{4}
		\begin{subproblem}
			$8^{\nicefrac{1}3}$
			\begin{shortsolution}
				$2$
			\end{shortsolution}
		\end{subproblem}
		\begin{subproblem}
			$8^{-\nicefrac{2}3}$
			\begin{shortsolution}
				$\frac{1}{4}$
			\end{shortsolution}
		\end{subproblem}	
		\begin{subproblem}
			$\left(8^5\right)^{\nicefrac{1}{3}}$
			\begin{shortsolution}
				$32$
			\end{shortsolution}
		\end{subproblem}	
		\begin{subproblem}
			$\sqrt[3]{27^4}$
			\begin{shortsolution}
				$81$
			\end{shortsolution}
		\end{subproblem}	
	\end{multicols}
	\end{problem}
											
	\begin{problem}
	Solve the equations for $x$.  Give both exact values and decimal approximations.
	\begin{multicols}{4}
		\begin{subproblem}
			$x^4=81$
			\begin{shortsolution}
				$x= 3$ or $x=-3$
			\end{shortsolution}
		\end{subproblem}
		\begin{subproblem}
			$x^5=-32$
			\begin{shortsolution}
				$x= -2$
			\end{shortsolution}
		\end{subproblem}
		\begin{subproblem}
			$x^6=-64$
			\begin{shortsolution}
				There are no real solutions for $x$.
			\end{shortsolution}
		\end{subproblem}
		\begin{subproblem}
			$x^4=19$
			\begin{shortsolution}
				$x= \sqrt[4]{19}=19^{\nicefrac14}\approx 2.0878$ or $x= -\sqrt[4]{19}=-19^{\nicefrac14}\approx -2.0878$
			\end{shortsolution}
		\end{subproblem}
	\end{multicols}
	\end{problem}
	%	\fixthis{Appendix?}
\end{essentialskills}
			
In many application problems we will encounter situations where we wish to find an exponential function that 
goes through the two points $(x_1,y_1)$ and $(x_2,y_2)$. Let's practice this skill before moving on to the applied problems. 
%===================================
%   Author: Hughes
%   Date:   April 2011
%===================================
\begin{pccexample} \label{exp:ex:findab}
	Find an exponential function $f$ given by $f(t)=a\,b^t$ whose graph goes through the points $\left(-2,\frac{3}{4}\right)$ and $(2,12)$
	as shown in \cref{exp:fig:expmodellingwithcurve}.
	\begin{figure}
		\centering
		\begin{tikzpicture}
			\begin{axis}[
					framed,
					width=.5\textwidth,
					xmin=-5,xmax=5,
					ymin=-2,ymax=15,
					xtick={-4,-2,...,10},
					minor xtick={-3,-1,...,3}, 
					ytick={2,4,...,14},
					grid=both,
				]
				\addplot expression[domain=-5:2.322]{3*2^x};
				\addplot[soldot]coordinates{(-2,0.75)(2,12)};
			\end{axis}
		\end{tikzpicture}
		\caption{}
		\label{exp:fig:expmodellingwithcurve}
	\end{figure}
										
	\begin{pccsolution}
		We will need to find $a$ and $b$ in the equation $f(t)=a\,b^t$. We begin by using the given ordered pairs to write a system 
		of equations
		\begin{align*}
			\nicefrac{3}{4} & =  a\,b^{-2} \\
			12              & =  a\,b^2    
		\end{align*}
		We can eliminate $a$ by equating the quotients formed by the two sides of the equations. One 
		result is 
		\begin{equation*}
			\frac{12}{\nicefrac{3}{4}} = \frac{a\,b^{2}}{a\,b^{-2}}\qquad \Longrightarrow\qquad 16 = b^4
		\end{equation*}
																	
		This implies that either $b=2$ or $b=-2$. Since the base of an exponential function must be 
		positive ($b>0$), we can conclude that $b=2$. Substituting this into one of the original equations, we find that
		\begin{equation*}
			12 = a\,(2)^2 \qquad \Longrightarrow\qquad a = 3
		\end{equation*}
		Note that we can substitute the value of $b$ into either of the original equations.
																	
		We conclude that $f(t)=3\cdot 2^t$, which is graphed in \cref{exp:fig:expmodellingwithcurve}.
		The number $2$ is called the growth factor; every time $t$ increases by $1$, the value of $f(t)$ 
		grows by a factor of $2$.
	\end{pccsolution}
\end{pccexample}
			
\begin{doyouunderstand}
	\begin{problem}
	Find an exponential function $f$ of the form $f(t)=a\,b^t$ whose graph goes through the ordered 
	pairs $(1,8)$ and $(3,128)$. Identify the growth factor.
	\begin{shortsolution}
		$f(t)=2\cdot 4^t$; the growth factor is $4$.
	\end{shortsolution}
	\end{problem}
\end{doyouunderstand}
			
When dealing with application problems the values of $a$ and $b$ will rarely evaluate to integers. 
The method for finding an exponential model remains the same though.
			
%===================================
%   Author: Hughes
%   Date:   April 2011
%===================================
\begin{pccexample}
	Find an exponential function $f(x)=a\,b^x$ that goes through the points $(1,7)$ and $(10,53)$.
	\begin{pccsolution}
		We will need to find $a$ and $b$ in the equation $f(x)=a\,b^x$. We begin by using the given ordered pairs to write a system 
		of equations
		\begin{align*}
			7  & =  a\,b      \\
			53 & =  a\,b^{10} 
		\end{align*}
		We can eliminate $a$ by equating the quotients formed by the two sides of the equations.
		\begin{equation*}
			\frac{53}{7} = \frac{a\,b^{10}}{a\,b}
		\end{equation*}
		which simplifies to $b^9=\frac{53}{7}$. This means that 
		\[
			b = \left(\frac{53}{7}\right)^{\nicefrac{1}{9}}
		\]
		If we substitute this 
		value of $b$ into the first equation, we obtain
		\begin{align*}
			a & = \frac{7}{b}                                           \\
			  & = \frac{7}{\left(\frac{53}{7}\right)^{\nicefrac{1}{9}}} \\
			  & = \frac{7^{\nicefrac{10}{9}}}{53^{\nicefrac{1}{9}}}     
		\end{align*}
		We conclude that 
		\begin{align*}
			f(x) & = \frac{7^{\nicefrac{10}{9}}}{53^{\nicefrac{1}{9}}}\left(\frac{53}{7}\right)^{\nicefrac{x}{9}}                \\
			     & = \frac{7^{\nicefrac{10}{9}}}{53^{\nicefrac{1}{9}}}\left(\left(\frac{53}{7}\right)^{\nicefrac{1}{9}}\right)^x \\
			     & \approx 5.59 (1.252237)^x                                                                                     
		\end{align*}
	\end{pccsolution}
\end{pccexample}
			
\begin{doyouunderstand}
	\begin{problem}
	Find an exponential function $f$ of the form $f(x)=a\,b^x$ whose graph goes through the ordered pairs $(2,10)$ and $(11,23)$. 
	Identify the growth factor.
	\begin{shortsolution}	
		$f(x)=\frac{10^{\nicefrac{11}{9}}}{23^{\nicefrac{2}{9}}}\left(\frac{23}{10}\right)^{\nicefrac{x}{9}}\approx 8.31 (1.096963)^x$
	\end{shortsolution}
	\end{problem}
\end{doyouunderstand}
			
%===================================
%   Author: Jordan
%   Date:   Nov 2011
%===================================
Suppose that we have some reason to model a situation using an exponential function.  In real-world applications, we 
might only have two points of data to work with.  If we treat these data points as two points on a plane, we can 
use the skills we have been practicing to explicitly give the exponential model that we would like to use.
\begin{pccexample}
	An outbreak of avian flu occurs in a crowded city.  Doctors immediately identify 25 patients who are infected.  
	One week later, there are 2391 people infected.  If we assume that the number of infected patients can be modeled with 
	exponential growth, what is the rule for the exponential model?
	\begin{pccsolution}
		We choose to model this situation by treating the given data as points on a plane.  Initially (or when $t=0$) there are $25$ patients, 
		so $(0,25)$ is one point.  At time $7$ (measured in days) there are $2391$ patients, so $(7,2391)$ is the other point.  
		These points are sketched in \cref{exp:fig:fludata}.
																	
		\begin{figure}[!htb]
			\centering
			\begin{tikzpicture}
				\begin{axis}[
						width=.3\textwidth,
						xmin=-2,xmax=10,
						ymin=-100,ymax=3000,
						ytick={500,1000,...,2500},
						xlabel=$t$
					]
					\addplot[soldot]coordinates{(0,25)(7,2391)};
				\end{axis}
			\end{tikzpicture}
			\caption{Flu patient data}
			\label{exp:fig:fludata}
		\end{figure}
																	
		Since we are trying to use an exponential model, we are searching for values of $a$ and $b$ such that the curve $y=a\,b^t$ will 
		pass through these points.  We need
		\begin{align*}
			25   & =  a\,b^0   \\
			2391 & =  a\,b^{7} 
		\end{align*}
		The first equation immediately tells us that $a=25$.  Now the second equation reduces to
		\begin{alignat*}{2}
			2391 {}={}25\,b^{7} & \quad \Longrightarrow\quad &   & \frac{2391}{25}  {}={} b^7                         \\
			                    & \quad \Longrightarrow      &   & b {}={} \left(\frac{2391}{25}\right)^{\nicefrac17} 
		\end{alignat*}
		So we can model the number of infected people as a function of the number of days since the outbreak was first noticed using the model 
		\begin{align*}
			f(t) & =25\left(\left(\frac{2391}{25}\right)^{\nicefrac17}\right)^t \\
			     & =25\left(\frac{2391}{25}\right)^{\nicefrac{t}7}              
		\end{align*}
	\end{pccsolution}
										
	Once we have this model we can answer interesting questions with it.  For example, how many people will be infected $3$ days 
	after the initial outbreak?  
	\begin{pccsolution} 
		Since 
		\begin{align*}
			f(3) & = 25\left(\frac{2391}{25}\right)^{\nicefrac{3}7} \\
			     & \approx 177                                      
		\end{align*}
		we can say that there will be about $177$ infected people $3$ days after the initial outbreak.
	\end{pccsolution}
										
	We can also look at a graph of the model (as in \cref{exp:fig:flugraph}) and answer other interesting questions.  For 
	example, when will the number of infected persons reach 20,000?  
	\begin{figure}[!htb]
		\centering
		\begin{tikzpicture}
			\begin{axis}[
					width=.3\textwidth,
					xmin=-2,xmax=15,
					ymin=-20,ymax=300,
					xtick={2,4,...,14},
					xlabel=$t$,
					ytick={50,100,...,250},
					yticklabels={5000,10000,15000,20000,25000}
				]
				\addplot expression[domain=-1:10.7]{.25*(2391/25)^(x/7)};
				\addplot[soldot]coordinates{(0,.25)(7,23.91)};
			\end{axis}
		\end{tikzpicture}
		\caption{Flu patient model}
		\label{exp:fig:flugraph}
	\end{figure}
	\begin{pccsolution} 
		The graph suggests that around the tenth day after the initial outbreak, there will be $20000$ infected persons.
	\end{pccsolution}
\end{pccexample}
			
%===================================
%   Author: Simonds, typeset by Hughes
%   Date:   Dec 2011
%===================================
\begin{pccexample}
	Many buildings in Detroit have been vacant and unattended for several
	years. Suppose that the amount of wood (in $\SI{}{\cubic\foot}$) that remains
	attached to a building $t$ years after it has been abandoned is 
	an exponential function of time. A particular building was abandoned on July 3, 1999. On July 3, 2002 there were 53 cubic yards of wood attached to the building and on July 3, 2009 there were 47 cubic yards of wood attached to the building.
										
	Find a function $w$, given by $w(t)=a\, b^t$, that outputs the volume of remaining wood
	on the structure $t$ years after July 3, 1999. Round the value of $a$ to the nearest tenth 
	and the value of $b$ to the nearest thousandth.
	\begin{pccsolution}
		We are given that $w(3)=53$ and $w(10)=37$ so
		\begin{align}
			a\,b^3    & = 53\label{exp:eq:wood} \\
			a\,b^{10} & =47 \nonumber           
		\end{align}
		Eliminating for $b$ gives $b^7=\frac{47}{53}$ and therefore
		\begin{align*}
			b & = \sqrt[7]{\frac{47}{53}} \\
			  & \approx 0.983             
		\end{align*}
		Using this value of $b$ in \cref{exp:eq:wood} gives
		\begin{align*}
			a & = \frac{53}{\left( \frac{47}{53} \right)^{\nicefrac{3}{7}}} \\
			  & \approx 55.8                                                
		\end{align*}
		We conclude that 
		\[
			w(t)  \approx 55.8(0.983)^t
		\]
	\end{pccsolution}
\end{pccexample}
			
%===================================
%   Author: Kouzes
%   Date:   Apr 2011
%===================================
\investigation*{}
\begin{problem}[US population]\label{exp:prob:uspopngreen}%
The population of the United States has increased roughly exponentially from 76.2 million 
people in 1900 to 309 million people in 2010.\footnote{\href{http://www.wolframalpha.com/input/?i=us+population+1900}{http://www.wolframalpha.com/input/?i=us+population+1900}} 
\begin{subproblem} \label{exp:prob:USApop19002010}
	Find a formula that approximates the number of people in the U.S.A., $p(t)$, in millions, $t$ 
	years after 1900 assuming that the population grows exponentially. 
	\begin{shortsolution}
		$p(t)=76.2\left( \frac{309}{76.2} \right)^{\nicefrac{t}{110}}\approx 76.2(1.012808)^t$
	\end{shortsolution}
\end{subproblem}
\begin{subproblem} \label{exp:prob:USAprediction2000}
	Use your model to approximate the population of the U.S.A.\ in the year 2000.     
	\begin{shortsolution}
		$p(100)\approx 272.072992$; the population of the U.S.A.\ in the year 2000 was approximately 272 million people, according to the model.
	\end{shortsolution}
\end{subproblem}
\begin{subproblem} 
	The actual population of the U.S.A.\ in 2000 was about 282 million people.\footnote{\href{http://www.wolframalpha.com/input/?i=us+population+1900}{http://www.wolframalpha.com/input/?i=us+population+2000}}   Did your approximation from \cref{exp:prob:USAprediction2000} 
	underestimate or overestimate, and by how much?		
	\begin{shortsolution}
		The model underestimated the actual population by about 10 million people.
	\end{shortsolution}
\end{subproblem}
\begin{subproblem}\label{exp:prob:USApop20002010}
	Use the actual populations of the U.S.A.\ in the years 2000 and 2010 to find a formula that approximates the number of people in the U.S.A., $P(t)$, in millions, $t$ years after 2000 assuming that the population grows exponentially.
	\begin{shortsolution}
		$P(t)=282\left( \frac{309}{282} \right)^{\nicefrac{t}{10}}\approx 282(1.009185)^t$
	\end{shortsolution}
\end{subproblem}
\begin{subproblem}
	The population of the U.S.A.\ was 151 million people in 1950\footnote{\href{http://www.wolframalpha.com/input/?i=us+population+1950}{http://www.wolframalpha.com/input/?i=us+population+1950}}. How close of an approximation does your model from \cref{exp:prob:USApop20002010} give? How close of an approximation does your model from \cref{exp:prob:USApop19002010} give? 	Why are both models off?  Which model is better for making a prediction about 1950?
	\begin{shortsolution}
		$P(-50) \approx 178.526320$ and $p(50)\approx 143.985979$.  $P(-50)$ was an overestimate by about 27.5 million people, or \SI{18.2}{\percent}.  $p(50)$ was an underestimate by about 7 million people, or \SI{4.6}{\percent}.  So the prediction based off of the model that used the years 1900 and 2000 is better.  Generally, it is better to interpolate than to extrapolate.
	\end{shortsolution}
\end{subproblem}		
\end{problem}
			
\begin{problem}[Greenhouse gases]\label{exp:prob:greenhouseorig}%
As our cars, homes, businesses and industry become more efficient, each person is responsible for the production of  
less carbon dioxide (CO$_{2}$) a greenhouse gas. The US Energy Information Administration 
provides the estimates for future quantities of CO$_{2}$ that each person will produce. They know 
that in 2010, on average each US citizen was responsible for the production of 18.1 metric tons of CO$_{2}$, and they predict 
that in 2031 the average US citizen will be responsible for 16.3 metric tons of CO$_{2}$.\footnote{\href{http://www.eia.doe.gov/forecasts/aeo/}{http://www.eia.doe.gov/forecasts/aeo/}}
\begin{subproblem}
	Find a formula for $c(t)$, an approximation for the average amount of CO$_{2}$ each US citizen 
	will be responsible for emitting, in metric tons, $t$ years after the year 2010 assuming that the 
	amount of CO$_{2}$ produced decays exponentially.
	\begin{shortsolution}
		$c(t) = 18.1\left( \frac{16.3}{18.1} \right)^{\nicefrac{t}{21}}\approx 18.1(0.995024)^t$
	\end{shortsolution}
\end{subproblem}		
\begin{subproblem}
	Use your model to approximate the average amount of CO$_2$ each US citizen was responsible for in the 
	year 2000.
	\begin{shortsolution}
		$c(-10)\approx 19.025713$.  Each US citizen was responsible for approximately 19.03 tons of CO$_2$ in 2000.
	\end{shortsolution}
\end{subproblem}
\begin{subproblem}
	If you have an answer to \cref{exp:prob:USApop20002010}, use it to find a formula for $C(t)$, an approximation for the \emph{total} amount 
	of CO$_{2}$ that all citizens of the U.S.A.\  
	will be responsible for emitting, in millions of metric tons, $t$ years after the year 2000.
	\begin{shortsolution}
		$C(t)=P(t)c(t)=282\left( \frac{309}{282} \right)^{\nicefrac{t}{10}}18.1\left( \frac{16.3}{18.1} \right)^{\nicefrac{t}{21}}\approx5104.2(1.00416412)^t$
	\end{shortsolution}
\end{subproblem}		
\end{problem}
			
%===================================
%   Author: Kouzes, Hughes (scaffolding)
%   Date:   Apr 2011
%===================================
\begin{problem}[Tapfish app]\label{exp:prob:tapfishtable}%
Tapfish is a free mobile device app where you set up a digital aquarium.  
We can use the application to create approximations of exponential functions. 
We have simplified the nature of the application to make this problem easier.
The app allows you to buy and sell fish once per day.  
The fish grow in size (and therefore value) after owning them for just one day.  Our 
goal is to maximize the number of fish in our aquarium over a period of 9 days.  
			
Each fish  costs 10 coins and sells for 15 coins after 1 day (because of its larger size). 
Given that  we start with 20 coins
on day 0, the transactions for the  first 3 days will be as follows:
\begin{itemize}
	\item on day $0$ we buy $2$ fish;
	\item on day $1$ we sell both fish for a total of $30$ coins, and then buy $3$ fish;
	\item on day $2$ we sell our $3$ fish for a total of 45 coins,and then buy $4$ fish;
	we have $5$ coins remaining as we cannot have fractions of fish.
\end{itemize}
			
\begin{margintable}
	\centering
	\captionof{table}{}
	\begin{tabular}{S[table-format=1.0]S[table-format=1.0]l}
		\beforeheading
		\heading{$t$} & \heading{$F(t)$} & \heading{$A(t)$} \\
		\afterheading
		0             & 2                &                  \\\normalline 
		1             & 3                &                  \\\normalline 
		2             & 4                &                  \\\normalline 
		3             &                  &                  \\\normalline
		4             &                  &                  \\\normalline 
		5             &                  &                  \\\normalline
		6             &                  &                  \\\normalline
		7             &                  &                  \\\normalline
		8             &                  &                  \\\lastline
	\end{tabular}
	\label{exp:tab:tapfish}
\end{margintable}
			
\begin{subproblem}
	Let $F(t)$ be the number of fish we have in our aquarium on day $t$. 
	Complete the $F(t)$ column in \cref{exp:tab:tapfish}. Assume 
	that we always buy and sell the maximum number of fish, that we use whatever leftover coins we have from previous transactions, and that we cannot buy fractions of fish.
	\begin{shortsolution}
		\begin{tabular}[t]{S[table-format=1.0]S[table-format=2.0]}
			\beforeheading
			\heading{$t$} & \heading{$F(t)$} \\
			\afterheading
			0             & 2                \\\normalline 
			1             & 3                \\\normalline 
			2             & 4                \\\normalline 
			3             & 6                \\\normalline
			4             & 9                \\\normalline 
			5             & 14               \\\normalline
			6             & 21               \\\normalline
			7             & 31               \\\normalline
			8             & 47               \\\lastline
		\end{tabular}
	\end{shortsolution}
\end{subproblem}
\begin{subproblem}
	Make a graph of $y=F(t)$ using the points in your table.
	\begin{shortsolution}
		\begin{tikzpicture}
			\begin{axis}[
					framed,
					xmin=-2,xmax=9,
					ymin=-8,ymax=55,
					xlabel={$t$},
					xtick={1,...,8},
					ytick={4,8,...,48},
					nodes near coords,
					grid=major,
				]
				\addplot[soldot]coordinates{(0,2)(1,3)(2,4)(3,6)(4,9)(5,14)(6,21)(7,31)(8,47)};
			\end{axis}
		\end{tikzpicture}
	\end{shortsolution}
\end{subproblem}
\begin{subproblem}
	The function $F$ represents the exact number of fish that we have on a given day. 
	We are going to find an approximate formula for $F$ using an exponential function.  To be clear, we will use $A$ to represent the function that \emph{approximates} $F$.
											
	Pick any two of the points from \cref{exp:tab:tapfish} and find a  function of the form, $A(t)=a\,b^t$, that 
	approximates   the number of fish in the tank $t$ days after starting.  (Answers will vary depending on which two points you use.)
	\begin{shortsolution}
		Choosing $(0,2)$ and $(8,47)$, 
		\begin{align*}            
			A(t) & = 2 \left(\left(\frac{47}{2}\right)^{\nicefrac{1}{8}}\right)^t \\
			     & \approx 2(1.483828)^t                                          
		\end{align*}
	\end{shortsolution}
\end{subproblem}
\begin{subproblem}
	Complete the $A(t)$ column in \cref{exp:tab:tapfish} using 6 decimal places, and compare the 
	values of $A(t)$ to the values of $F(t)$.
											
	\begin{shortsolution}
		\begin{tabular}[t]{S[table-format=1.0]S[table-format=2.0]S[table-format=2.6]}
			\beforeheading
			\heading{$t$} & \heading{$F(t)$} & \heading{$A(t)$} \\
			\afterheading
			0             & 2                & 2.000000         \\\normalline 
			1             & 3                & 2.967655         \\\normalline 
			2             & 4                & 4.403489         \\\normalline 
			3             & 6                & 6.503420         \\\normalline
			4             & 9                & 9.695360         \\\normalline 
			5             & 14               & 14.386240        \\\normalline
			6             & 21               & 21.346710        \\\normalline
			7             & 31               & 31.674840        \\\normalline
			8             & 47               & 47.000000        \\\lastline
		\end{tabular}
	\end{shortsolution}
\end{subproblem}
\end{problem}
			
%========================================================================
%
%			Exercises
%
%========================================================================
\begin{exercises}
%===================================
%   Author: Jordan
%   Date:   Nov 2011
%===================================
\begin{problem}[Prerequisite simplification skills]
Simplify the given expressions.
\begin{multicols}{4}
	\begin{subproblem}
		$12^x12^y$
		\begin{shortsolution}
			$12^{x+y}$
		\end{shortsolution}
	\end{subproblem}
	\begin{subproblem}
		$a^{5x}a^{x+3}$
		\begin{shortsolution}
			$a^{6x+3}$
		\end{shortsolution}
	\end{subproblem}
	\begin{subproblem}
		$\left(3x^3y^2\right)^2$
		\begin{shortsolution}
			$9x^6y^4$
		\end{shortsolution}
	\end{subproblem}
	\begin{subproblem}
		$\left(2b^9c^4\right)\left(5bc^3\right)$
		\begin{shortsolution}
			$10b^{10}c^7$
		\end{shortsolution}
	\end{subproblem}
	\begin{subproblem}
		$\dfrac{14x^5}{12x^2}$
		\begin{shortsolution}
			$\dfrac{7}{6}x^3$
		\end{shortsolution}
	\end{subproblem}
	\begin{subproblem}
		$\dfrac{6x^7}{27x^{-3}}$
		\begin{shortsolution}
			$\dfrac{2}{9}x^{10}$
		\end{shortsolution}
	\end{subproblem}
	\begin{subproblem}
		$\left(\dfrac{3x^{-4}}{4x^2}\right)^{-3}$
		\begin{shortsolution}
			$\dfrac{64}{27}x^{18}$
		\end{shortsolution}
	\end{subproblem}
	\begin{subproblem}
		$8^{\nicefrac{x}3}$
		\begin{shortsolution}
			$2^x$
		\end{shortsolution}
	\end{subproblem}
	\begin{subproblem}
		$100^{\nicefrac{x}2}$
		\begin{shortsolution}
			$10^x$
		\end{shortsolution}
	\end{subproblem}	
	\begin{subproblem}
		$\left(3b^{\nicefrac43}\right)^{3x}$
		\begin{shortsolution}
			$27^xb^{4x}$
		\end{shortsolution}
	\end{subproblem}	
	\begin{subproblem}
		$16^t(64)^{\nicefrac{t}{3}}$
		\begin{shortsolution}
			$64^t$
		\end{shortsolution}
	\end{subproblem}	
	\begin{subproblem}
		$\left( 27t^3 \right)^{\nicefrac{4}{3}}$
		\begin{shortsolution}
			$81t^4$
		\end{shortsolution}
	\end{subproblem}
\end{multicols}
\end{problem}
			
%===================================
%   Author: Hughes
%   Date:   Dec 2011
%===================================
\begin{problem}[Prerequisite solving skills]
Solve each of the following equations. Give both the exact value 
and the decimal approximation where appropriate.
\begin{multicols}{3}
	\begin{subproblem}
		$2x^2=32$
		\begin{shortsolution}
			$x=4$ or $x=-4$
		\end{shortsolution}
	\end{subproblem}
	\begin{subproblem}
		$y^3=\dfrac{27}{8}$
		\begin{shortsolution}
			$y=\nicefrac{3}{2}$
		\end{shortsolution}
	\end{subproblem}
	\begin{subproblem}
		$\dfrac{1}{x^2}=\dfrac{1}{25}$
		\begin{shortsolution}
			$x= 5$ or $x=-5$
		\end{shortsolution}
	\end{subproblem}
	\begin{subproblem}
		$\dfrac{1}{x^5+1}=1$
		\begin{shortsolution}
			$x=0$ 
		\end{shortsolution}
	\end{subproblem}
	\begin{subproblem}
		$x^4-9=0$
		\begin{shortsolution}
			$x=\sqrt{3}$ or $x=-\sqrt{3}$
		\end{shortsolution}
	\end{subproblem}
	\begin{subproblem}
		$x^6-2x^3-35=0$    % factor $(x^3-7)(x^3+5)=0$
		\begin{shortsolution}
			$x=\sqrt[3]{7}$ or $x=-\sqrt[3]{5}$
		\end{shortsolution}
	\end{subproblem}
\end{multicols}
\end{problem}
			
%===================================
%   Author: Hughes
%   Date:   July 2011
%===================================
\begin{problem}[Find a formula, $a$ and $b$ rational]
In each of the following problems, find an exponential function of the form $f(t)=a\,b^t$ that goes through the given ordered pairs. Identify the growth factor in each problem.
\begin{multicols}{2}
	\begin{subproblem}
		$(0,-3)$, $(4,-1875)$
		\begin{shortsolution}
			$f(t)=-3\cdot 5^t$; the growth factor is $5$.
		\end{shortsolution}
	\end{subproblem}
	\begin{subproblem}
		$(2,36)$, $(5,7776)$
		\begin{shortsolution}
			$f(t)=6^t$; the growth factor is $6$.
		\end{shortsolution}
	\end{subproblem}
	\begin{subproblem}
		$\left(2,\frac{2}{9}\right)$, $\left(4,\frac{2}{81}\right)$
		\begin{shortsolution}
			$f(t)=2\left(\frac{1}{3}\right)^t$; the growth factor is $\nicefrac{1}{3}$.
		\end{shortsolution}
	\end{subproblem}
	\begin{subproblem}
		$\left(-3,\frac{2125}{8}\right)$, $\left(3,\frac{136}{125}\right)$
		\begin{shortsolution}
			$f(t)=17\left(\frac{2}{5}\right)^t$; the growth factor is $\nicefrac{2}{5}$.
		\end{shortsolution}
	\end{subproblem}
	\begin{subproblem}
		$(-1,279)$, $(1,124)$
		\begin{shortsolution}
			$f(t)=186\left( \nicefrac{2}{3} \right)^t$; the growth factor is $\nicefrac{2}{3}$.
		\end{shortsolution}
	\end{subproblem}
	\begin{subproblem}
		$(2,64)$, $(-1,125)$
		\begin{shortsolution}
			$f(t)=100\left( \nicefrac{4}{5} \right)^t$; the growth factor is $\nicefrac{4}{5}$.
		\end{shortsolution}
	\end{subproblem}
\end{multicols}
\end{problem}
			
			
			
%===================================
%   Author: Hughes
%   Date:   July 2011
%===================================
\begin{problem}[Find a formula, $a$ and $b$ irrational]
In each of the following problems, find an exponential function of the form $f(x)=a\,b^x$ that 
goes through the given ordered pairs. Be sure to give both the exact form and the approximate form of $f$.
\begin{multicols}{2}
	\begin{subproblem}
		$(-5,10)$, $(3,4)$
		\begin{shortsolution}
			$f(x)=4\left(\frac{5}{2}\right)^{\nicefrac{3}{8}}\left(\frac{2}{5}\right)^{\nicefrac{x}{8}}\approx 5.64(0.891780)^x$
		\end{shortsolution}
	\end{subproblem}
	\begin{subproblem}
		$(4,7)$, $(8,2)$
		\begin{shortsolution}
			$f(x)=\frac{49}{2}\left(\frac{2}{7}\right)^{\nicefrac{x}{4}}\approx 24.5(0.731110)^x$
		\end{shortsolution}
	\end{subproblem}
	\begin{subproblem}
		$(3,6)$, $(13,29)$
		\begin{shortsolution}
			$f(x)=\frac{6^{\nicefrac{13}{10}}}{29^{\nicefrac{3}{10}}}\left(\frac{29}{6}\right)^{\nicefrac{x}{10}}\approx 3.74 (1.170644)^x$
		\end{shortsolution}
	\end{subproblem}
	\begin{subproblem}
		$(-5,6)$, $(7,20)$
		\begin{shortsolution}
			$f(x)=6\left(\frac{10}{3}\right)^{\nicefrac{5}{12}}\left(\frac{10}{3}\right)^{\nicefrac{x}{12}}\approx 9.91 (1.105537)^x$
		\end{shortsolution}
	\end{subproblem}
\end{multicols}
\end{problem}
			
			
%===================================
%   Author: Hughes
%   Date:   Dec 2011
%===================================
\begin{problem}[Find a formula from a graph]
For each of the functions in \cref{exp:fig:findab}, assume that 
the vertical intercept is an integer value. Use this, together with 
the information in each caption, to find a formula for each function.
\begin{shortsolution}
	$f(x)=2\cdot 3^x$, $g(x)=2\left( \frac{1}{2} \right)^x$, $h(x)=-5^x$, $k(x)=-4\left( \frac{1}{3} \right)^x$  
\end{shortsolution}
\end{problem}
			
\begin{figure}[!htb]
	\begin{widepage}
	\centering
	\setlength{\figurewidth}{.2\textwidth}
	\begin{subfigure}{\figurewidth}
		\begin{tikzpicture}
			\begin{axis}[
					framed,
					xmin=-5,xmax=2,
					ymin=-2,ymax=8,
					xtick={-4,-2},
					ytick={2,4,6},
					width=\figurewidth,
					grid=major,
				]
				\addplot expression[domain=-4.5:1.26186]{2*3^x};
			\end{axis}
		\end{tikzpicture}
		\caption{$f(1)=6$}
	\end{subfigure}
	\hfill
	\begin{subfigure}{\figurewidth}
		\begin{tikzpicture}
			\begin{axis}[
					framed,
					xmin=-5,xmax=5,
					ymin=-2,ymax=8,
					xtick={-4,-2,...,4},
					ytick={2,4,6},
					width=\figurewidth,
					grid=major,
				]
				\addplot expression[domain=-2:3.2]{2*(1/2)^x};
			\end{axis}
		\end{tikzpicture}
		\caption{$g(2)=\dfrac{1}{2}$}
	\end{subfigure}
	\hfill
	\begin{subfigure}{\figurewidth}
		\begin{tikzpicture}
			\begin{axis}[
					framed,
					xmin=-5,xmax=2,
					ymin=-8,ymax=2,
					width=\figurewidth,
					xtick={-4,-2},
					ytick={-6,-4,-2},
					grid=major,
				]
				\addplot expression[domain=-4.5:1.2920]{-1*(5^x)};
			\end{axis}
		\end{tikzpicture}
		\caption{$h(-4)=-\dfrac{1}{625}$}
	\end{subfigure}
	\hfill
	\begin{subfigure}{\figurewidth}
		\begin{tikzpicture}
			\begin{axis}[
					framed,
					xmin=-3,xmax=6,
					ymin=-6,ymax=2,
					width=\figurewidth,
					xtick={-2,2,4},
					ytick={-4,-2},
					grid=major,
				]
				\addplot expression[domain=-0.3691:5]{-4*(1/3)^x};
			\end{axis}
		\end{tikzpicture}
		\caption{$k(3)=-\dfrac{4}{27}$}
	\end{subfigure}
	\caption{}
	\label{exp:fig:findab}
	\end{widepage}
\end{figure}
			
%===================================
%   Author: Hughes
%   Date:   Dec 2011
%===================================
\begin{problem}[Concavity]
For the functions in \cref{exp:fig:findab}, which graphs are concave up? Which are concave down? 
\begin{shortsolution}
	The graphs of $f$ and $g$ are concave up; the graphs of $h$ and $k$ are concave down.
\end{shortsolution}
\end{problem}
			
%===================================
%   Author: Hughes
%   Date:   Dec 2011
%===================================
\begin{problem}[Find a formula from a table]
Find a formula for each of the functions implied by the data values in \crefrange{exp:tab:findab10}{exp:tab:findabfifth}.
Assume that each function has the form $f(x)=a\, b^x$.
\begin{shortsolution}
	\Cref{exp:tab:findab10}: $f(x)=10^x$; \cref{exp:tab:findab7}: $f(x)=-9\cdot 7^x$; \cref{exp:tab:findabthird}: $f(x)=-6\left( \frac{1}{3} \right)^x$; \cref{exp:tab:findabfifth}: $f(x)=3\left( \frac{1}{5} \right)^x$.  
\end{shortsolution}
\end{problem}
% manual table stretch, str. by factor of 1.25
\renewcommand{\arraystretch}{1.25}
\begin{table}[!htb]
	\begin{widepage}
	\begin{minipage}{.2\textwidth}
		\centering
		\caption{} \label{exp:tab:findab10}
		\begin{tabular}{S[table-format=1.0]S[table-format=6.0]}
			\beforeheading
			\heading{$x$} & \heading{$y$} \\\afterheading
			1             & 10            \\\normalline
			2             & 100           \\\normalline
			3             & 1000          \\\normalline
			4             & 10000         \\\normalline
			5             & 100000        \\\lastline
		\end{tabular}
	\end{minipage}
	\hfill
	\begin{minipage}{.2\textwidth}
		\centering
		\caption{} \label{exp:tab:findab7}
		\begin{tabular}{S[table-format=1.0]c}
			\beforeheading
			\heading{$x$} & \heading{$y$} \\\afterheading
			-5            & \num{9/16897} \\\normalline
			-4            & \num{9/2401}  \\\normalline
			-3            & \num{9/343}   \\\normalline
			-2            & \num{9/49}    \\\normalline
			-1            & \num{9/7}     \\\lastline
		\end{tabular}
	\end{minipage}
	\hfill
	\begin{minipage}{.2\textwidth}
		\centering
		\caption{}\label{exp:tab:findabthird}
		\begin{tabular}{S[table-format=2.0]c}
			\beforeheading
			\heading{$x$} & \heading{$y$}  \\
			\afterheading
			2             & \num{-2/3}     \\\normalline
			4             & \num{-2/27}    \\\normalline
			6             & \num{-2/243}   \\\normalline
			8             & \num{-2/2187}  \\\normalline
			10            & \num{-2/19683} \\\lastline
		\end{tabular}
	\end{minipage}
	\hfill
	\begin{minipage}{.2\textwidth}
		\centering
		\caption{}\label{exp:tab:findabfifth}
		\begin{tabular}{S[table-format=2.0]S[table-format=9.0]}
			\beforeheading
			\heading{$x$} & \heading{$y$} \\
			\afterheading
			-12           & 732421875     \\\normalline
			-9            & 5859375       \\\normalline
			-6            & 46875         \\\normalline
			-3            & 375           \\\normalline
			0             & 3             \\\lastline
		\end{tabular}
	\end{minipage}
	\end{widepage}
\end{table}
			
%===================================
%   Author: Hughes
%   Date:   Dec 2011
%===================================
\begin{problem}[Increasing and decreasing functions]
Refer to the functions in \cref{exp:fig:findab} and \crefrange{exp:tab:findab10}{exp:tab:findabfifth} to help you decide 
if the following statements are true or false. If the answer is false, provide an 
example that supports your answer.
\begin{subproblem}
	Exponential functions always increase.
	\begin{shortsolution}
		False; consider $g$ and $h$ in \cref{exp:fig:findab}.
	\end{shortsolution}
\end{subproblem}
\begin{subproblem}
	Exponential functions always decrease.
	\begin{shortsolution}
		False; consider $f$ and $k$ in \cref{exp:fig:findab}.
	\end{shortsolution}
\end{subproblem}
\begin{subproblem}
	Some exponential functions increase and some decrease.
	\begin{shortsolution}
		True.
	\end{shortsolution}
\end{subproblem}
\begin{subproblem}
	If $x$ increases by a constant amount, then $y$ increases by a constant amount.
	\begin{shortsolution}
		False; consider \crefrange{exp:tab:findab10}{exp:tab:findabfifth}.
	\end{shortsolution}
\end{subproblem}
\begin{subproblem}
	If $x$ increases by a constant amount, then the successive ratios of $y$ are the same.
	\begin{shortsolution}
		True.
	\end{shortsolution}
\end{subproblem}
\begin{subproblem}
	If $x$ increases by a constant amount, then the successive ratios of $y$ increase by the same amount.
	\begin{shortsolution}
		False; consider \crefrange{exp:tab:findab10}{exp:tab:findabfifth}.
	\end{shortsolution}
\end{subproblem}
\end{problem}
%===================================
%   Author: Hughes
%   Date:   May 2011
%===================================
\begin{problem}[Solar capacity]
The global solar photovoltaic power capacity (in \si{\mega\watt}) grew from 2000-2007\footnote{\href{http://www.energyandcapital.com/articles/solar-stock-outlook/750}{http://www.energyandcapital.com/articles/solar-stock-outlook/750}} and 
is shown in \cref{exp:fig:solarcapacity}. 
			
\begin{figure}[!htb]
	\begin{widepage}
	\centering
	\begin{tikzpicture}
		\begin{axis}[
				x axis line style={->},
				y axis line style={->},
				xlabel={$t$},
				ylabel={Solar PV Power Capacity by Year (\si{\mega\watt})},
				width=.95\textwidth,
				height=.5\textwidth,
				xmin=-1,xmax=9,
				xtick={1,2,...,9},
				xticklabels={2000,2001,...,2007},
				ymin=-500,ymax=9000,
				ytick={1000,2000,...,8000},
				yticklabels={1000,2000,...,8000},
				nodes near coords,
				every node near coord/.append style={anchor=north},
			]
			\addplot[pccbar,bar width=30] plot coordinates
			{
				(1, 877)
				(2, 1161)
				(3, 1543)
				(4, 2090)
				(5, 2919)
				(6, 4556)
				(7, 5990)
				(8, 8325)
			};
		\end{axis}
	\end{tikzpicture}
	\caption{Solar capacity.}
	\label{exp:fig:solarcapacity}
	\end{widepage}
\end{figure}
			
\begin{subproblem}
	Let $S(t)$ represent the Solar PV Power Capacity (in \si{\mega\watt}) at time $t$ in years 
	since 2000. Use the first and last data points in \cref{exp:fig:solarcapacity} to 
	write down two ordered pairs that lie on the graph of $S$.
	\begin{shortsolution}
		$(0,877)$, $(7,8325)$.
	\end{shortsolution}
\end{subproblem}
\begin{subproblem}
	Use your ordered pairs to find a formula for $S$ in the form $S(t)=a\,b^t$. State $b$ to 
	three decimal places.
	\begin{shortsolution}
		$S(t)=877\left( \frac{8325}{877} \right)^{\nicefrac{t}{7}}\approx 877(1.38)^t$
	\end{shortsolution}
\end{subproblem}
\begin{subproblem}
	What are the growth rate and growth factor for your function $S$?
	\begin{shortsolution}  
		The growth factor is $\left( \frac{8325}{877} \right)^{\nicefrac{1}{7}}\approx 1.379$. 
		The growth rate is $\left( \frac{8325}{877} \right)^{\nicefrac{1}{7}}-1 \approx 0.379$, or about $\SI{37.9}{\percent}$ per year.
	\end{shortsolution}
\end{subproblem}
\begin{subproblem}
	According to the article, the data implies `a compounded annual growth rate (CAGR) of \SI{37.9}{\percent}.'     
	Is your model consistent with this statement?
	\begin{shortsolution}
		Yes; we found $b\approx 1.379$, which means the annual growth rate is approximately \SI{37.9}{\percent}.  
	\end{shortsolution}
\end{subproblem}
\begin{subproblem}
	According to your model, what is the Solar PV Power Capacity in 2010? Compare this 
	with the actual value of  $\SI{13729}{\mega\watt}$, and give reasons for the difference.
	\begin{shortsolution}
		$S(10)\approx \SI{21840.54}{\mega\watt}$. 
		This is much larger than $\SI{13729 }{\mega\watt}$, which indicates that the 
		growth rate did not continue at \SI{37.9}{\percent} after 2007.
	\end{shortsolution}
\end{subproblem}
\end{problem}
\end{exercises}
			
\section{Exponential modeling}\label{exp:sec:populationmodels}
\begin{outcomes}
	\begin{outcomelist}
		\item stuff
	\end{outcomelist}
\end{outcomes}
In this section we explore some common applications of exponential functions. This will 
require a working knowledge of percentages.
\begin{essentialskills}
	%===================================
	%   Author: Hughes
	%   Date:   February 2012
	%===================================
	\begin{problem}
	Perform the given percentage calculations. Give your answer correct 
	to two decimal places when an approximation is appropriate.
	\begin{multicols}{2}
		\begin{subproblem}
			Find $\SI{10}{\percent}$ of 100. 
			\begin{shortsolution}
				$0.1\cdot 100 = 10$ 
			\end{shortsolution}
		\end{subproblem}
		\begin{subproblem}
			Find $\SI{20}{\percent}$ of 10. 
			\begin{shortsolution}
				$0.2\cdot 10=2$ 
			\end{shortsolution}
		\end{subproblem}
		\begin{subproblem}
			Find $\SI{13}{\percent}$ of 28. 
			\begin{shortsolution}
				$0.13\cdot 28=3.64$ 
			\end{shortsolution}
		\end{subproblem}
		\begin{subproblem}
			Find $\SI{81}{\percent}$ of 3. 
			\begin{shortsolution}
				$0.81\cdot 3=2.43$ 
			\end{shortsolution}
		\end{subproblem}
		\begin{subproblem}
			Increase $17$ by $\SI{28}{\percent}$. 
			\begin{shortsolution}
				$17\cdot 1.28=21.76$ 
			\end{shortsolution}
		\end{subproblem}
		\begin{subproblem}
			Increase $42$ by $\SI{67}{\percent}$. 
			\begin{shortsolution}
				$42\cdot (1.67=70.14$ 
			\end{shortsolution}
		\end{subproblem}
		\begin{subproblem}
			Decrease $107$ by $\SI{10}{\percent}$. 
			\begin{shortsolution}
				$107\cdot 0.9=96.3$ 
			\end{shortsolution}
		\end{subproblem}
		\begin{subproblem}
			Decrease $243$ by $\SI{76}{\percent}$. 
			\begin{shortsolution}
				$243\cdot 0.24=58.32$ 
			\end{shortsolution}
		\end{subproblem}
	\end{multicols}
	\end{problem}
\end{essentialskills}
			
\subsection*{Simple population models}
Exponential functions are often used to model the size of a population as time passes.  A population 
could be growing or declining at an exponential rate.
			
\begin{pccdefinition}[Simple population model]
	If an initial population, $P_0$, changes by the same percentage each year, then 
	the population at time $t$ in years, $P(t)$, is given by the formula
	\begin{equation*}
		P(t)=P_0(1+r)^t, 
	\end{equation*}
	where $r$ is the decimal form of the percentage change.
										
	When dealing with human population models, we generally restrict the domain of $P$ to integer values of $t$. This 
	means that we are only modeling population values on one specific day of the year.
										
	Note that $r$ can be positive (for a growing population) or negative (for a declining population).  
	Also, note that we don't have to measure time in years.  Using years makes sense for modeling human populations, 
	but a smaller unit like hours might make sense for a bacteria population. 
\end{pccdefinition}
			
%===================================
%   Author: Kissick
%   Date:   April 2011
%===================================
\begin{pccexample}\label{exp:ex:innercity}
	Since 2001 the population of an inner city has been decreasing as people 
	move to the suburbs. It is decreasing at a rate of \SI{1}{\percent} per year; find a model 
	for this situation.
	\begin{pccsolution}
		Let $P(t)$ be the population of the inner city at time $t$ in years since 2001. 
		Let $P_0$ be the population of the inner city in 2001.
		Then, since $r=-0.01$,
		\[
			P(t) = P_0(0.99)^t
		\]
	\end{pccsolution}
\end{pccexample}
			
%===================================
%   Author: Hughes
%   Date:   March 2012
%===================================
\begin{doyouunderstand}
	\begin{problem}
	Repeat \cref{exp:ex:innercity} for a population that decreases at a rate of \SI{8}{\percent} per year.
	\begin{shortsolution}
		$P(t)=P_0(0.92)^t$ 
	\end{shortsolution}
	\end{problem}
\end{doyouunderstand}
			
%===================================
%   Author: Kouzes
%   Date:   Apr 2011
%===================================
\begin{pccexample}	
	At the beginning of the year 2000, the population of One Horse was 10,000 people. By the beginning
	of 2005 the population had decreased to 9700. Assuming exponential decay, determine the population 
	at the beginning of 2012. Round your answer to the nearest person.
	\begin{pccsolution}
		Over 5 years, the population decreased by 300 people.  As $\frac{300}{10000}=0.03$, the population 
		decreased by $\SI{3}{\percent}$ over the five-year period.  The growth factor for this five-year period is $0.97$ and 
		therefore the growth factor for one year is $(0.97)^{1/5}$.  Using this, the population $P(t)$ at 
		time $t$ in years after 2000 can be represented by
		\[
			P(t) = 10000\left(0.97\right)^{t/5}
		\]  
		To estimate the population in 2012, we evaluate $P(12)$:
		\begin{align*}
			P(12) & = 10000\left(0.97\right)^{12/5} \\
			      & \approx 9295                    
		\end{align*}
		In 2012, the population of One Horse is expected to be approximately 9,295 people.
	\end{pccsolution}
\end{pccexample}
%===================================
%   Author: Cary
%   Date:   September 2011
%===================================
\begin{doyouunderstand}
	\begin{problem}[Oregonians]
	In 2000 there were $3.83$  million people living in Oregon.  By 2010, the population had increased 
	by $\SI{12.0}{\percent}$.\footnote{2000 census}  Assuming the population continues to grow at this rate, find a 
	model representing the size of the population $t$ years after 2000.
	\begin{shortsolution}
		Let $P(t)$ be the population of Oregon (in millions of people) at time $t$ in years since 2000. 
		Since the initial population is 3.83 million people and $r=0.120$,
		\[
			P(t) = 3.83(1.12)^t
		\]
	\end{shortsolution}
	\end{problem}
\end{doyouunderstand}
			
\subsection*{Radioactive decay}
%===================================
%   Author: Simonds
%   Date:   Feb 2011
%===================================
One common application of decaying exponential functions involves radioactivity.  If you have a substance 
where some of the atoms are radioactive, the radioactive atoms will decay into other atoms in a very predictable way.  
In fact, each radioactive element has an 
associated half-life, which is the time it takes for \SI{50}{\percent} of the radioactive atoms to decay into something else.
\begin{pccexample}\label{exp:prob:radioactive}%
	Suppose that we have 8000 atoms of Carbon-12 (which is non-radioactive).  Without an intervening act, 1,000,000 years from 
	now that sample of carbon will still have 8000 atoms of Carbon-12.  However, if we start with a sample 
	of 8000 
	radioactive C-14 atoms, over time fewer and fewer of those atoms remain radioactive.
										
										
	If we define $R(t)$ to be the number of radioactive atoms that remain in the sample $t$ years from today, then we can 
	model the function $R$ using the template $R(t)=a\,b^t$ for unknown 
	constants $a$ and $b$.  The number of remaining radioactive atoms $t$ years from today is 
	shown in \cref{exp:tab:C14decay}.  
										
	\begin{table}[!htb]
		\centering
		\caption{Decaying Radioactive Carbon-14}\label{exp:tab:C14decay}
		\begin{tabular}{S[table-format=5.0]S[table-format=4.0]}
			\beforeheading
			\heading{Number of years} & \heading{Number of remaining} \\
			\heading{from today}      & \heading{radioactive atoms}   \\
			\afterheading
			0                         & 8000                          \\\normalline
			5730                      & 4000                          \\\normalline
			11460                     & 2000                          \\\normalline
			17190                     & 1000                          \\\normalline
			22920                     & 500                           \\\lastline
		\end{tabular}
	\end{table}
										
	Using the data in \cref{exp:tab:C14decay} we can determine the formula for $R(t)$, 
	the number of atoms that remain radioactive after $t$ years; choosing 
	two ordered pairs and assuming that $R(t)=a\ b^t$
	\begin{align*}
		8000 & = ab^0     \\
		4000 & =ab^{5730} 
	\end{align*}
	This clearly means that $a=8000$, and 
	\begin{align*}
		\frac{1}{2} & = b^{5730} \Longrightarrow b = \left( \frac{1}{2} \right)^{\frac{1}{5730}} 
	\end{align*}
	We can therefore say that
	\[
		R(t)=8000\left(\frac12\right)^{\frac1{5730}t}
	\]
	We can now determine how many atoms remain radioactive after 100 years
	\[
		R(100)\approx 7904
	\] 
	Approximately 7904 atoms remain radioactive after 100 years.
										
	How long it would take until there are only 6000 radioactive atoms remaining?
	We need to solve the equation 
	\[
		6000=8000\left( \frac{1}{2} \right)^{\frac{t}{5730}}
	\]
	Using a graphing calculator, we obtain 
	$t\approx 2378$. We conclude that 6000 radioactive atoms remain after approximately $2378$ years.
\end{pccexample}
			
\subsection*{Simple interest}
%===================================
%   Author: Kissick
%   Date:   April 2011
%===================================
Let's assume that we invest an amount $P$ into an account that pays 
an interest rate $r$ per year.
			
At the end of year 1 we have
\begin{align*}
	A & = \begin{array}[b]{c} 
	$\small beginning$ \\
	$\small balance$\\
	\downarrow\\
	P
	\end{array}
	+  
	\begin{array}[b]{c}
	$\small interest$ \\
	$\small added$\\
	\downarrow \\
	P\cdot r \\
	\end{array}\\
	  & = P(1+r)              
\end{align*}
\fixthis{pick one, or something else}
At the end of year 1 we have
\begin{align*}
	A & = \overbrace{P}^{\substack{\text{beginning} \\\text{balance}}} +  
	\overbrace{P\cdot r}^{\substack{interest\\added}}\\
	  & = P(1+r)                                    
\end{align*}
After 2 years we have
\begin{align*}
	A & = \begin{array}[b]{c}          \\
	$\small year 1$ \\
	$\small balance$\\
	\downarrow\\
	P(1+r)
	\end{array}
	+  
	\begin{array}[b]{c}
	$\small interest$ \\
	$\small added$\\
	\downarrow \\
	P(1+r)\cdot r 
	\end{array}\\
	  & = P(1+r)\cdot 1 +P(1+r)\cdot r \\
	  & = P(1+r)(1+r)                  \\    
	  & = P(1+r)^2                     
\end{align*}
After 3 years we have
\begin{align*}
	A & = \begin{array}[b]{c} 
	$\small year 2$ \\
	$\small balance$\\
	\downarrow\\
	P(1+r)^2
	\end{array}
	+  
	\begin{array}[b]{c}
	$\small interest$ \\
	$\small added$\\
	\downarrow \\
	P(1+r)^2\cdot r 
	\end{array}\\
	  & = P(1+r)^2(1+r)       \\    
	  & = P(1+r)^3            
\end{align*}
After $t$ years we have
\begin{align*}
	A & = \begin{array}[b]{c} 
	$\small year $ t-1 \\
	$\small balance$\\
	\downarrow\\
	P(1+r)^{t-1}
	\end{array}
	+  
	\begin{array}[b]{c}
	$\small interest$ \\
	$\small added$\\
	\downarrow \\
	P(1+r)^{t-1}\cdot r \\
	\end{array}\\
	  & = P(1+r)^{t-1}(1+r)   \\    
	  & = P(1+r)^t            
\end{align*}
			
\begin{pccdefinition}[Simple interest]
	If we invest an amount $P$ in an account with interest rate $r$ (per year), then 
	the amount in the account is
	\begin{equation}\label{exp:eq:simpleinterest}
		A(t)=P(1+r)^t
	\end{equation}
	where $t$ is the amount of time that has passed (in years) since the initial 
	investment.
\end{pccdefinition}
			
%===================================
%   Author: Kissick
%   Date:   April 2011
%===================================
\begin{pccexample}
	You invest \$5000 in an account that pays \SI{3}{\percent} simple interest. What is 
	the balance after four years? How long will it take your investment to double in value?
										
	\begin{pccsolution}
		We use \cref{exp:eq:simpleinterest}, with $P=5000$, $r=0.03$, and $t=4$.
		\begin{align*}
			A(4) & =  5000(1+0.03)^4 \\
			     & \approx  5627.54  
		\end{align*}
		There is approximately \$5627.54 in the account after 4 years.
																	
		To find when the investment will double in value, our first thought might 
		be to solve the equation
		\[
			10000 = 5000(1.03)^t
		\]
		for $t$. This is problematic in that the solution to this equation is almost 
		certainly non-integer and the domain of our function is restricted to the 
		integers. A more straight forward approach might be to simply find the first 
		year the balance is at least \$10,000.
		\begin{margintable}
			\centering
			\captionof{table}{Simple interest}
			\begin{tabular}{S[table-format=2.0]S[table-format=5.2]}
				\beforeheading
				\heading{$t$} & \heading{$A(t)$} \\ 
				\afterheading
				0             & 5000.00          \\\normalline
				1             & 5150.00          \\\normalline
				\mbox{\vdots} & \mbox{\vdots}    \\\normalline
				23            & 9867.93          \\\normalline
				24            & 10163.97         \\\lastline
			\end{tabular}
			\label{exp:tab:simpleinterest}
		\end{margintable}
																	
		Using \cref{exp:tab:simpleinterest}, we conclude that the 
		investment will double in about 24 years. 
	\end{pccsolution}
\end{pccexample}
			
			
\subsection*{Compound interest}
We have just studied simple interest. Most banks, however, 
use a slightly more complicated system.
			
%===================================
%   Author: Kouzes
%   Date:   Apr 2011
%===================================
\begin{pccexample}[An offer you can't refuse]
	Two brothers, Michael and Fredo, each deposited \$100 into brand new bank accounts.
										
	Michael's bank account gave him a \SI{12}{\percent} annual interest rate. 
	How much money did he have in his account after one year if he earned 
	\SI{12}{\percent} in simple annual interest?
										
	We need to calculate $100(1.12) = 112$. Michael had \$112 after one year.
										
	Fredo's bank account gave him a 12\% annual rate compounded monthly, 
	which means that each month he received one-twelfth of the \SI{12}{\percent} (which is 
	\SI{1}{\percent} interest) on his account:
	\begin{itemize}
		\item after one month, he had $\$100(1.01)=\$101.00$ in his account;
		\item after two months he had $\$100(1.01)^2=\$102.01$ in his account.
	\end{itemize}
	We complete \cref{exp:tab:fredo} using 
	2 decimal places, where $t$ indicates the number of months
	since first investing and $A(t)$ is the amount of money that was in his account (in 
	dollars) at time $t$. 
										
	\begin{margintable}
		\centering
		\captionof{table}{Fredo's account}
		\begin{tabular}{S[table-format=2.0]S[table-format=3.2]}
			\beforeheading
			\heading{$t$} & \heading{$A(t)$} \\
			\afterheading
			0             & 100.00           \\\normalline
			1             & 101.00           \\\normalline
			2             & 102.01           \\\normalline
			3             & 103.03           \\\normalline
			4             & 104.06           \\\normalline
			5             & 105.10           \\\normalline
			6             & 106.15           \\\normalline
			7             & 107.21           \\\normalline
			8             & 108.29           \\\normalline
			9             & 109.37           \\\normalline
			10            & 110.46           \\\normalline
			11            & 111.57           \\\normalline
			12            & 112.68           \\\lastline
		\end{tabular}
		\label{exp:tab:fredo}
	\end{margintable}
										
	If we look closely at what happened in \cref{exp:tab:fredo}, we notice that 
	to find $A(2)$ we calculate
	\begin{align*}
		A(2) & =  100(1.01)(1.01) \\
		     & =  100(1.01)^2     
	\end{align*}
	Similarly, $A(12)=100(1.01)^{12}$.
										
	Fredo didn't actually earn \SI{12}{\percent} interest after one year on his initial investment, 
	so the \SI{12}{\percent} is in name only, and is called the {\em nominal} 
	interest rate.
										
	After one year, what percent did Fredo actually earn?  We calculate the \emph{effective rate}: 
	\[
		\left( 1+\frac{0.12}{12} \right)^{12}-1\approx 0.126825 
	\] 
	The effective rate is approximately \SI{12.68}{\percent}.
										
	Note that although both Michael and Fredo both had accounts that were nominally at \SI{12}{\percent} annual interest, they ended up with different effective annual interest rates due to the different compounding periods.  
										
	A third brother, Sonny, invested \$100 into another account. His account earned 
	\SI{12}{\percent} nominal rate compounded daily. This means that he earned $\frac{\SI{12}{\percent}}{365}$, which is approximately 
	$\SI{0.0328767}{\percent}$ each day. 
	Find how much money Sonny had in his account after 1 year. 
										
	We calculate
	\[
		100\left(1+\frac{0.12}{365}  \right)^{365}\approx 112.75
	\]
	and therefore Sonny had $\$112.75$ after one year. 
										
	We can calculate the effective interest rate on Sonny's account using
	\[
		\left( 1+\frac{0.12}{365} \right)^{365}-1\approx  0.127475
	\]
	The effective rate was approximately $\SI{12.75}{\percent}$.
\end{pccexample}
			
\begin{margintable}
	\centering
	\captionof{table}{}
	\begin{tabular}{lS[table-format=3.0]}
		\beforeheading
		\heading{Words} & \heading{$n$} \\
		\afterheading
		yearly          & 1             \\\normalline
		annually        & 1             \\\normalline
		semi-annually   & 2             \\\normalline
		quarterly       & 4             \\\normalline
		monthly         & 12            \\\normalline
		biweekly        & 26            \\\normalline
		weekly          & 52            \\\normalline
		daily           & 365           \\\normalline
		non-stop        & \mbox{??}     \\\lastline
	\end{tabular}
	\label{exp:tab:compoundingfreq}
\end{margintable}
			
\begin{pccdefinition}[Compound interest]\label{exp:def:compoundint}
	The compound interest formula is
	\begin{align}
		A(t) = P_0\left(1+\frac{r}{n}\right)^{nt}\label{exp:form:compoundinterest} 
	\end{align}
	where:
	\begin{itemize}
		\item $P_0$ is the initial amount invested;
		\item $r$ is the nominal interest rate;
		\item $n$ is the compounding frequency (see \cref{exp:tab:compoundingfreq});
		\item $t$ is the amount of time that has passed since the initial investment (in years);
		\item $A(t)$ is the account balance (in dollars).
	\end{itemize}
\end{pccdefinition}
			
			
%===================================
%   Author: Hughes
%   Date:   April 2011
%===================================
\begin{pccexample}\label{exp:ex:compoundinterest}
	We have \$7000 to invest into an account that earns interest at a \SI{3}{\percent} nominal rate. Find the amount that we 
	will have after four years assuming each of the following compounding frequencies:
	\begin{multicols}{4}
		\begin{enumerate}
			\item yearly
			\item monthly
			\item weekly
			\item daily
		\end{enumerate}
	\end{multicols}
	\begin{pccsolution} Apply the compound interest formula (\cref{exp:form:compoundinterest}) to each case.
		\begin{multicols}{2}
			\begin{enumerate}
				\vspace{\fill}
				\item yearly:
				\begin{align*}
					A(4) & =  7000\left(1+\frac{0.03}{1}\right)^{1\cdot 4} \\
					     & \approx  7878.56                                
				\end{align*}
				\item monthly:
				\begin{align*}
					A(4) & =  7000\left(1+\frac{0.03}{12}\right)^{12\cdot 4} \\
					     & \approx  7891.30                                  
				\end{align*}
				\vspace{\fill}
				\columnbreak
				\vspace{\fill}
				\item weekly:
				\begin{align*}
					A(4) & =  7000\left(1+\frac{0.03}{52}\right)^{52\cdot 4} \\
					     & \approx  7892.20                                  
				\end{align*}
				\item daily:
				\begin{align*}
					A(4) & =  7000\left(1+\frac{0.03}{365}\right)^{365\cdot 4} \\
					     & \approx  7892.44                                    
				\end{align*}
				\vspace*{\fill}
			\end{enumerate}
		\end{multicols}
		After 4 years, assuming compounding frequencies of yearly, monthly, weekly, daily, the balance 
		will be, respectively, \$7878.56, \$7891.30, \$7892.20, and \$7892.44.
	\end{pccsolution}
\end{pccexample}
			
We have covered a lot of rates including growth rates and decay rates. In the context of finance we've talked 
about nominal interest rates and effective interest rates. Recall that the nominal interest rate 
is the stated rate before any compounding has been applied.
Now we will formally define effective interest rate.
			
\begin{pccdefinition}[Effective rate]
	Given a nominal annual interest rate, $r$, and a compounding frequency, $n$, the effective rate 
	is the interest rate that is \emph{actually} earned over the course of one year. 
	The effective rate is calculated using the expression
	\[
		\left( 1+\frac{r}{n} \right)^n - 1
	\]
										
	The quantity $\left( 1+\frac{r}{n} \right)^n$ represents the annual growth factor defined in \cref{exp:def:growthfactorrate}.
	This comes from the compound interest formula (\cref{exp:def:compoundint}) and the equality
	\[
		\left( 1+\frac{r}{n} \right)^{nt}= \left(\left( 1+\frac{r}{n} \right)^n\right)^t 
	\]
	\mbox{}
\end{pccdefinition}
			
%===================================
%   Author: Hughes
%   Date:   April 2011
%===================================
\begin{pccexample}\label{exp:ex:effectiverate}
	Find the effective rate of interest for each of the problems in \cref{exp:ex:compoundinterest}.
										
	\begin{pccsolution}
		We will keep many decimal places in our solutions so that we can see the differences between 
		the results.
		\begin{enumerate}
			\item $\left( 1+\frac{0.03}{1} \right)-1=0.03$ (not surprising)
			\item $\left( 1+\frac{0.03}{12} \right)^{12}-1\approx 0.030415957$
			\item $\left( 1+\frac{0.03}{52} \right)^{52}-1\approx 0.030445620$
			\item $\left(1+\frac{0.03}{365}\right)^{365}-1\approx 0.030453264$
		\end{enumerate}
		These have been tabulated in \cref{exp:tab:effectiverate}.
	\end{pccsolution}
\end{pccexample}
\begin{margintable}
	\centering
	\captionof{table}{}
	\begin{tabular}{S[table-format=3.0]S[table-format=1.0]S[table-format=1.7]}
		\beforeheading
		\heading{$n$} & \heading{nominal}   & \heading{effective} \\
		              & \heading{rate (\%)} & \heading{rate} (\%) \\
		\afterheading
		1             & 3                   & 3                   \\ \normalline
		12            & 3                   & 3.0415957           \\ \normalline 
		52            & 3                   & 3.0445620           \\ \normalline
		365           & 3                   & 3.0453264           \\ \lastline
	\end{tabular}
	\label{exp:tab:effectiverate}
\end{margintable}
			
\Cref{exp:ex:effectiverate} demonstrates that the nominal interest rate is usually different from the effective interest rate.  
In practice, an effective interest rate is almost always a bit more than the nominal interest rate.  
			
\investigation*{}
%===================================
%   Author: Kouzes, Jordan
%   Date:   Apr 2011
%===================================
\begin{problem}[Atmospheric C-14]
During the Cold War, many above-ground thermonuclear tests were done in the South 
Pacific region, making many islands completely uninhabitable for decades (if not centuries) due to lingering 
radiation. In fact, the thermonuclear testing drastically increased the amount of atmospheric C-14 above normal levels worldwide. In this problem, we are going to investigate the long-term effects of worldwide 
nuclear testing half 
a world away from the South Pacific.
			
			
\begin{figure}[!htb]
	\centering
	\begin{tikzpicture}[trim axis left]
		\begin{axis}[
				framed,
				xmin=1945,xmax=2000,
				ymin=-20,ymax=120,
				xlabel={$t$},
				width=.6\textwidth,
				height=.3\textwidth,
				grid=both,
				axis line style=->,
				xtick={1955,1965,...,1995},
				xticklabels={1955,1965,...,1995},
				ytick={20,40,...,100},
				axis x discontinuity=crunch,
				nodes near coords,
			]
			\addplot[soldot]coordinates{ (1963,95) (1967,70) (1970,52) (1976,40) (1980,30) (1983,22)};
		\end{axis}
	\end{tikzpicture}
	\caption{Atmospheric C-14 in Austria (\si{\percent} above pre-nuclear age levels)}%
	\label{exp:fig:coldwar}
\end{figure}
			
\Cref{exp:fig:coldwar} charts the percent of atmospheric C-14 above the natural pre-nuclear test level in the air above Austria\footnote{\href{http://en.wikipedia.org/wiki/Carbon-14}{http://en.wikipedia.org/wiki/Carbon-14}}.  The decay that you see has to do with C-14 being flushed out of the atmosphere and into the ground and ocean, not radioactive decay.
\begin{subproblem}
	Write a sentence that contextually interprets the data point $(1963,95)$.
	\begin{shortsolution}
		In $1963$ the atmospheric concentration of C-14 in Austria was \SI{95}{\percent} greater than
		the pre-nuclear age concentration.
	\end{shortsolution}
\end{subproblem}
\begin{subproblem}\label{exp:prob:c14plot}
	Using the first and last data point determine a formula that gives an approximation of C-14 in the atmosphere in 
	Austria as a function of the number of years since 1963. 
	Round your growth factor to the second digit after the decimal point. 
	Use your calculator to check the relative accuracy of your graph. 
	\begin{shortsolution}
		Let $f(t)$ represent the percent of C-14 in the atmosphere compared to normal $t$ years after 1963. 
		Using the points $(0,95)$ and $(20,22)$, $f(t) \approx 95(0.93)^t$.  
	\end{shortsolution}
\end{subproblem}
\begin{subproblem}
	Repeat \cref{exp:prob:c14plot} using the data points $(1970,52)$ and $(1976,40)$.
	\begin{shortsolution}
		Using $(7,52)$ and $(13,40)$ we obtain $f(t)=\frac{52^{\nicefrac{13}{6}}}{40^{\nicefrac{7}{6}}}\left( \frac{40}{52} \right)^{\nicefrac{t}{6}}\approx 70.62(0.96)^t$.
	\end{shortsolution}
\end{subproblem}
\begin{subproblem}
	Which model is the better fit? \footnote{In fact in statistics you learn techniques called regression that 
	allows you to take all of the data points into consideration and develop the most accurate model.}
	\begin{shortsolution}
		Based on visual inspection, the model in \cref{exp:prob:c14plot} most closely 
		fits each of the data points.
	\end{shortsolution}
\end{subproblem}
\end{problem}
			
%===================================
%   Author: Jordan
%   Date:   August 2011
%===================================
\begin{problem}[CSI]
Almost every compound in your body is replaced frequently by the food, water, and air that you take in.  
One of the few exceptions is the enamel in your teeth.  Enamel is formed in early childhood, and is never replaced.   
Whatever C-14 level was in the atmosphere at that time becomes permanently locked in your teeth.  Because the 
half-life of C-14 is so long, the level will remain steady throughout your life.
			
Forensic scientists have found a fascinating application for this information.  After the devastating tsunami in 
Indonesia in 2004, there were many unidentified bodies.  To help identify bodies with the names on lists of 
missing persons, researchers examined the C-14 level in the teeth of the tsunami victims.  Through comparison with 
known C-14 levels, the forensic scientists were able to determine the birth year of victims to within 18 months.  
			
Suppose that skeletal human remains are found in a forest in Austria.  A first-molar from the remains is found to 
have \SI{145}{\percent} of pre-nuclear age C-14.  First-molars develop while a person is about 6 or 7 years of age.  In what 
year was the person born?  Use the model determined in 
\cref{exp:prob:c14plot} while working this problem.
			
\begin{shortsolution}
	We will use our answer from \cref{exp:prob:c14plot}:  $f(t) \approx 95(0.936)^t$ where $t$ is years since 1963. \\
	\begin{tikzpicture}[trim axis left]
		\begin{axis}[
				framed,
				xmin=1945,xmax=2000,
				ymin=-20,ymax=120,
				xlabel={$t$},
				grid=both,
				axis line style=->,
				xtick={1955,1965,...,1995},
				ytick={20,40,...,160},
				axis x discontinuity=crunch,
			]
			\addplot+[->]expression[domain=1962:2000]{(0.93562711)^(x -1963)*95};
			\addplot[soldot,nodes near coords]coordinates{ (1963,95) (1967,70) (1970,52) (1976,40) (1980,30) (1983,22)};
			\draw (axis cs:1945,45)--(axis cs:1974.2298,45)--(axis cs:1974.2298,0);
		\end{axis}
	\end{tikzpicture}
											
	So this person was probably 6 or 7 years old in about 1974.  The person was probably born in 1967 or 1968.
\end{shortsolution}
\end{problem}
			
%===================================
%   Author: Vega
%   Date:   March 2011
%===================================
\begin{problem}[Resources]
In 2005 a town has $1000$ barrels of oil in a well. Some of the oil is easy to reach and some 
of it is hard to reach. As more of the easily accessible oil is drilled, the amount that they can drill 
in a given year is reduced.  In fact, limitations of drilling equipment lead to a \SI{20}{\percent} reduction in drilling capacity each year. In 2005, $200$ barrels of oil were drilled. 
Let $Q(t)$ represent the amount of oil (in barrels) that is drilled where $t$ is the number of years after 2005. Assume that in each year the town drills to its maximum capacity. 
\begin{margintable}
	\centering
	\captionof{table}{}
	\begin{tabular}{S[table-format=1.0]l}
		\beforeheading
		\heading{$t$ (years)} & \heading{$Q$ (barrels)} \\
		\afterheading
		0                     &                         \\\normalline 
		1                     &                         \\\normalline 
		2                     &                         \\\normalline 
		3                     &                         \\\normalline
		4                     &                         \\\normalline 
		5                     &                         \\\lastline
	\end{tabular}
	\label{exp:tab:modelresource}
\end{margintable}
			
\begin{subproblem}
	Complete \cref{exp:tab:modelresource} to two decimal places.
	\begin{shortsolution}
		\begin{tabular}[t]{S[table-format=1.0]S[table-format=4.2]}
			\beforeheading
			\heading{$t$ (years)} & \heading{$Q$ (barrels)} \\
			\afterheading
			0                     & 1000.00                 \\\normalline 
			1                     & 800.00                  \\\normalline 
			2                     & 640.00                  \\\normalline 
			3                     & 512.00                  \\\normalline
			4                     & 409.60                  \\\normalline 
			5                     & 327.68                  \\\lastline
		\end{tabular}
	\end{shortsolution}
											
\end{subproblem}
\begin{subproblem}
	Find a formula for $Q(t)$.
	\begin{shortsolution}
		$Q(t)=1000(0.8)^t$
	\end{shortsolution}
\end{subproblem}
\begin{subproblem}
	What is the first year that the amount of oil drilled will be less than 100 barrels?
	\begin{shortsolution}
		We need to solve the equation $1000(0.8)^t=100$. Using a graph, we find that $t\approx 10.31$. We conclude that
		the first year that the amount of oil drilled will be less than 100 barrels is 2015.
	\end{shortsolution}
\end{subproblem}
\begin{subproblem}
	How long should it take until no more oil can be drilled?  Make sure to justify your answer.
	\begin{shortsolution}
		Never. $Q(t)$ never crosses the horizontal axis if the model holds.  However the amount that can be drilled 	eventually becomes small.
	\end{shortsolution}
\end{subproblem}
\end{problem}
\begin{problem}[Revisiting greenhouse gases]
Recall in \vref{exp:prob:uspopngreen} we modeled the US population and 
in \vref{exp:prob:greenhouseorig} we modeled the average amount of CO$_2$ emissions
for which each US citizen is responsible. The functions we came up with were, 
respectively,
\[
	P(t)=282\left( \frac{309}{282} \right)^{\nicefrac{t}{10}} \quad \textrm{and}\quad
	C(t)=18.1\left( \frac{16.3}{18.1} \right)^{\nicefrac{-10}{21}}\left( \frac{16.3}{18.1} \right)^{\nicefrac{t}{21}}
\]
			
\begin{subproblem}\label{exp:prob:USACO2}
	The total amount of CO$_{2}$ that the U.S.A.\ emits can be found by multiplying the amount that each person 
	emits with the total U.S.\ population. Find and simplify a formula for $T(t)$, the total amount of CO$_{2}$ 
	released by the U.S.A., in millions of metric tons, $t$ years since 2000.
	\begin{shortsolution}
		$\begin{aligned}[t]
			T(t) & =76.2\left( \frac{309}{76.2} \right)^{\nicefrac{100}{110}}18.1\left( \frac{16.3}{18.1} \right)^{\nicefrac{-10}{21}}\left( \frac{309}{76.2} \right)^{\nicefrac{t}{110}}\left( \frac{16.3}{18.1} \right)^{\nicefrac{t}{21}} \\
			     & \approx 5176.382674(1.007769)^t                                                                                                                                                                                           
		\end{aligned}$
	\end{shortsolution}
\end{subproblem}		
\begin{subproblem}
	Based on your answer to \cref{exp:prob:USACO2}, is the total amount of CO$_{2}$ produced in the U.S.A.\ each year increasing 
	or decreasing, and by what percent?
	\begin{shortsolution}
		The total amount of CO$_{2}$ produced in the US each year is increasing by approximately \SI{0.7769}{\percent}.
	\end{shortsolution}
\end{subproblem}
\begin{subproblem}
	The US\ Energy Information Administration estimates that the U.S.A.\ will produce 5,679 million metric tons 
	of CO$_{2}$ in the year 2015. Use your model from \cref{exp:prob:USACO2} to estimate the average CO$_2$ emissions in the year 2015.  How close is your approximation to the USEIA estimate?
	\begin{shortsolution}
		$T(15) \approx 5813$.  The approximation is only about 134 million metric tons of CO$_{2}$ off, which may seem large but is really only about \SI{2}{\percent} off of the actual estimate.
	\end{shortsolution}
\end{subproblem}
\end{problem}
			
\begin{exercises}
%===================================
%   Author: Hughes
%   Date:   March 2012
%===================================
\begin{problem}[Given description, write formula]
In each of the following, assume that the population of a town 
changes at the given rate.
Write a formula for $P(t)$, the population at time $t$,
measured in years since $2012$.
\begin{multicols}{2}
	\begin{subproblem}
		$P_0=500$, increasing at $\SI{6}{\percent}$ per year.
		\begin{shortsolution}
			$P(t)=500(1.06)^t$ 
		\end{shortsolution}
	\end{subproblem}
	\begin{subproblem}
		$P_0=1500$, increasing at $\SI{12}{\percent}$ per year.
		\begin{shortsolution}
			$P(t)=1500(1.12)^t$
		\end{shortsolution}
	\end{subproblem}
	\begin{subproblem}
		$P_0=2700$, increasing at $\SI{23}{\percent}$ per year.
		\begin{shortsolution}
			$P(t)=2700(1.23)^t$ 
		\end{shortsolution}
	\end{subproblem}
	\begin{subproblem}
		$P_0=3600$, increasing at $\SI{52}{\percent}$ per year.
		\begin{shortsolution}
			$P(t)=3600(1.52)^t$ 
		\end{shortsolution}
	\end{subproblem}
	\begin{subproblem}
		$P_0=700$, decreasing at $\SI{6}{\percent}$ per year.
		\begin{shortsolution}
			$P(t)=700(0.94)^t$ 
		\end{shortsolution}
	\end{subproblem}
	\begin{subproblem}
		$P_0=2405$, decreasing at $\SI{12}{\percent}$ per year.
		\begin{shortsolution}
			$P(t)=2405(0.88)^t$
		\end{shortsolution}
	\end{subproblem}
	\begin{subproblem}
		$P_0=4302$, decreasing at $\SI{23}{\percent}$ per year.
		\begin{shortsolution}
			$P(t)=4302(0.77)^t$ 
		\end{shortsolution}
	\end{subproblem}
	\begin{subproblem}
		$P_0=7300$, decreasing at $\SI{52}{\percent}$ per year.
		\begin{shortsolution}
			$P(t)=7300(0.48)^t$ 
		\end{shortsolution}
	\end{subproblem}
\end{multicols}
\end{problem}
%===================================
%   Author: Hughes
%   Date:   March 2012
%===================================
\begin{problem}[Given formula, write interpretaion]
Each of the following formulas model the population of a town 
at time $t$ (in years) since 1998. Determine the initial population, $P_0$,
and the percentage change, $r$, and give a sentence that describes the model.
\begin{multicols}{2}
	\begin{subproblem}
		$P(t)=1000(1.1)^t$ 
		\begin{shortsolution}
			$P_0=1000$, $r=\SI{10}{\percent}$; the population is initially $1000$ people, and 
			increases at $\SI{10}{\percent}$ per year.
		\end{shortsolution}
	\end{subproblem}
	\begin{subproblem}
		$P(t)=1800(1.07)^t$ 
		\begin{shortsolution}
			$P_0=1800$, $r=\SI{7}{\percent}$; the population is initially $1800$ people, and increases
			at $\SI{7}{\percent}$ per year.
		\end{shortsolution}
	\end{subproblem}
	\begin{subproblem}
		$P(t)=200(0.87)^t$ 
		\begin{shortsolution}
			$P_0=200$, $r=\SI{-13}{\percent}$; the population is initially $200$ people, and decreases
			at $\SI{13}{\percent}$ per year.
		\end{shortsolution}
	\end{subproblem}
	\begin{subproblem}
		$P(t)=907(0.76)^t$ 
		\begin{shortsolution}
			$P_0=907$, $r=\SI{-24}{\percent}$; the population is initially $907$ people, and decreases
			at $\SI{24}{\percent}$ per year.
		\end{shortsolution}
	\end{subproblem}
\end{multicols}
\end{problem}
%===================================
%   Author: Vega
%   Date:   March 2011
%===================================
\begin{problem}[US population]
In the year 2000, the population of the U.S.A.\ was around 281 million people with an estimated 
percentage growth rate of \SI{0.7}{\percent} per year\footnote{2000 census}.
			
\begin{subproblem}
	Let $P(t)$ represent the number of people in the U.S.A., in millions of people, at time $t$ 
	in years since the year 2000. Write a formula for $P(t)$.
	\begin{shortsolution}
		$P(t)=281(1.007)^t$
	\end{shortsolution}
\end{subproblem}
\begin{subproblem}
	According to your model, how many people lived in the U.S.A.\ in 2010? How many will live in the 
	U.S.A.\ in 2035? 
	Round your answers to the nearest million.
	\begin{shortsolution}
		$P(10)\approx 301$; in 2010 there were approximately $301$ million people. $P(35)\approx 359$; in 2035 there will be  
		approximately $359$ million people.
	\end{shortsolution}
\end{subproblem}
\begin{subproblem}
	Graph $P$ using the grid in \cref{exp:fig:uspopulation}. When graphing the population, we graph it
	continuously to see the trend, even though we are only modeling it for one specific day of the year. 
	\begin{shortsolution}
		\begin{tikzpicture}
			\begin{axis}[
					framed,
					xmin=-40,xmax=150,
					ymin=-2,ymax=11,
					xtick={20,40,...,140},
					ytick={1,...,10},
					yticklabels={100,200,...,1000},
					xlabel={$t$},
					grid=both,
				]
				\addplot+[->]expression[domain=0:145]{2.81*(1.007)^x};
				\addplot[soldot]coordinates{(0,2.81)};
			\end{axis}
		\end{tikzpicture}
	\end{shortsolution}
\end{subproblem}
			
			
\begin{subproblem}\label{exp:prob:uspopStrWeak}
	What are the strengths and weaknesses of an exponential model like this?
	Are there any other factors or information you might want to know before making 
	a more informed decision on whether this percentage growth rate changes or stays the same? What 
	other information might help solidify or change your model?
	\begin{shortsolution}
		Answers will vary.
		Possible strengths: \% increases are realistic in some sense -- a family reproduces, then 
		their children reproduce, and so on. 
																			
		Possible weaknesses: doesn't allow for immigration or emigration. Does it account for death?
	\end{shortsolution}
\end{subproblem}
\end{problem}
			
\begin{figure}[!htb]
	\begin{widepage}
	\setlength{\figurewidth}{8cm}
	\begin{minipage}{\figurewidth}
		\begin{tikzpicture}
			\begin{axis}[
					framed,
					xmin=-20,xmax=150,
					ymin=-1,ymax=11,
					xtick={20,40,...,140},
					ytick={1,...,10},
					yticklabels={100,200,...,1000},
					xlabel={$t$},
					grid=both,
				]
			\end{axis}
		\end{tikzpicture}
		\captionof{figure}{US population}
		\label{exp:fig:uspopulation}
	\end{minipage}
	\hfill
	\begin{minipage}{\figurewidth}
		\begin{tikzpicture}
			\begin{axis}[
					framed,
					xmin=-40,xmax=220,
					ymin=-0.15,ymax=1.4,
					xtick={20,40,...,200},
					ytick={0.1,0.2,...,1.3},
					xlabel={$t$},
					grid=both,
				]
			\end{axis}
		\end{tikzpicture}
		\captionof{figure}{China population}
		\label{exp:fig:chinapopulation}
	\end{minipage}
	\end{widepage}
\end{figure}
%===================================
%   Author: Vega
%   Date:   March 2011
%===================================
\begin{problem}[China population]
In the year 2009 China had a population around 1.3 billion people and its government had a goal of 
reducing the population to 700 million by 2050. It aimed to do this by 
limiting the number of children people had. Assume that with natural death rates and the restricted birth rate
the population is decreasing by \SI{0.5}{\percent} each year.
\begin{subproblem}
	Let $P(t)$ represent the number of people in China, in billions of people, at 
	time $t$ in years since 2009. Write a formula for $P$.
	\begin{shortsolution}
		$P(t)=1.3(0.995)^t$.
	\end{shortsolution}
\end{subproblem}
\begin{subproblem}
	According to your model, how many people will live in China in 2025? 2050? 
	\begin{shortsolution}
		In 2025: $P(16)\approx 1.20$; the population of China in 2025 will be approximately $1.20$ billion people.
		In 2050: $P(41)\approx 1.06$; the population of China in 2050 will be approximately $1.06$ billion people.
	\end{shortsolution}
\end{subproblem}
\begin{subproblem}
	Graph $P$ using the grid in \cref{exp:fig:chinapopulation}. When graphing the population, we graph it
	continuously to see the trend, even though we are only modeling it for one specific day of the year. 
	\begin{shortsolution}
		\begin{tikzpicture}
			\begin{axis}[
					framed,
					xmin=-40,xmax=220,
					ymin=-0.15,ymax=1.4,
					xtick={20,40,...,200},
					ytick={0.1,0.2,...,1.3},
					xlabel={$t$},
					grid=both,
				]
				\addplot+[->]expression[domain=0:200]{1.3*(0.995)^x};
				\addplot[soldot]coordinates{(0,1.3)};
			\end{axis}
		\end{tikzpicture}
	\end{shortsolution}
\end{subproblem}
\begin{subproblem}
	According to your model, in what year will China reach its goal of having a population of 700 million people?
	\begin{shortsolution}
		Using a graph or table of values, the population will be 700 million people in approximately the year 2132.
	\end{shortsolution}
\end{subproblem}
\begin{subproblem}
	Do you think the decay rate is likely to stay at \SI{0.5}{\percent} per year? If not, how do 
	you think it might change?
	\begin{shortsolution}
		Answers will vary.
	\end{shortsolution}
\end{subproblem}
\begin{subproblem}\label{exp:prob:chinapopStrWeak}
	What are the strengths and weaknesses of an exponential model like this?
	Are there any other factors or information you might want to know before making 
	a more informed decision on whether this population rate changes or stays the same? What 
	other information might help solidify or change your model?
	\begin{shortsolution}
		Answers will vary.
		Possible strengths: \% decreases are realistic in some sense -- a family has fewer children, then 
		the next generation has fewer people who also have fewer children, and so on.
																			
		Possible weaknesses: doesn't allow for immigration or emigration. Does it account for death?
	\end{shortsolution}
\end{subproblem}
\end{problem}
			
\begin{problem}[Textiles]
When textiles are made from a certain plant that grows alongside the Nile river, each square inch contains 142 billion atoms of radioactive C-14.  
\begin{subproblem}
	Using the half-life of 5730 years	for C-14, determine a formula for the number of radioactive atoms (in billions), $N(t)$,
	that remain in a one-square-inch sample of this textile after $t$ years.
	\begin{shortsolution}
		$N(t)=142\left(\frac12\right)^{\frac1{5730}t}$.
	\end{shortsolution}
\end{subproblem}
\begin{subproblem}
	How many atoms in the sample would remain radioactive after 3000 years?
	\begin{shortsolution}
		$N(3000)\approx 98.8$; about 98.8 billion atoms remain radioactive after 3000 years.
	\end{shortsolution}
\end{subproblem}
\begin{subproblem}
	Archaeologists found clothing preserved in a tomb made from this textile, and the clothing contained 93 billion atoms of 
	radioactive C-14 per square inch.  Roughly how old is the clothing?
	\begin{shortsolution}
		We need to solve the equation $93=142\left( \frac{1}{2} \right)^{\frac{t}{5730}}$. Using a graphing calculator, we obtain 
		$t\approx 3499$. We conclude that the clothing is approximately 3500 years old.
	\end{shortsolution}
\end{subproblem}
\end{problem}
			
%===================================
%   Author: Hughes
%   Date:   July 2011
%===================================
\begin{problem}[Simple interest]
Use \cref{exp:eq:simpleinterest} to find the given unknowns 
in each of the following problems. Give your answers to 2 decimal places.
\begin{multicols}{2}
	\begin{subproblem}
		$P=1000$, $r=\SI{4}{\percent}$; find $A(3)$
		\begin{shortsolution}
			$A(3)=1000(1+0.04)^3\approx 1124.86$
		\end{shortsolution}
	\end{subproblem}
	\begin{subproblem}
		$P=2500$, $r=\SI{7}{\percent}$; find $A(4)$
		\begin{shortsolution}
			$A(4)=2500(1+0.07)^4\approx 3276.99$
		\end{shortsolution}
	\end{subproblem}
	\begin{subproblem}
		$A(4)=2500$, $r=\SI{7}{\percent}$; find $P$
		\begin{shortsolution}
			$P=\frac{2500}{(1+0.07)^4}\approx 1907.24$
		\end{shortsolution}
	\end{subproblem}
	\begin{subproblem}
		$A(6)=3600$, $r=\SI{10}{\percent}$; find $P$
		\begin{shortsolution}
			$P=\frac{3600}{(1+0.10)^6}\approx 2032.11$
		\end{shortsolution}
	\end{subproblem}
	\begin{subproblem}
		$A(6)=3600$, $P=1200$; find $r$
		\begin{shortsolution}
			$r= \left( \frac{3600}{1200} \right)^{\nicefrac{1}{6}}-1\approx 0.20$
		\end{shortsolution}
	\end{subproblem}
	\begin{subproblem}
		$A(20)=5000$, $P=600$; find $r$
		\begin{shortsolution}
			$r= \left( \frac{5000}{600} \right)^{\nicefrac{1}{20}}-1\approx 0.11$
		\end{shortsolution}
	\end{subproblem}
\end{multicols}
\end{problem}
%===================================
%   Author: Vega
%   Date:   March 2011
%===================================
\begin{problem}[Investment]
You invest \$15,000 in the year 2010 into an investment account earning 
5\% simple interest annually.
\begin{subproblem}
	Find a formula for the amount of money you have in total, $A(t)$, at time 
	$t$ in years since 2010.
	\begin{shortsolution}
		$A(t)=15000(1.05)^t$
	\end{shortsolution}
\end{subproblem}
\begin{subproblem}
	How much will this investment be worth in 2020? 2030? 2040? Give your answers 
	to two decimal places.
	\begin{shortsolution}
		In 2020: $A(10)\approx 24433.42$; In 2030: $A(20)\approx 39799.47$; In 2040: $A(30)\approx 64829.14$
	\end{shortsolution}
\end{subproblem}
\begin{subproblem}
	How long will it be until you have at least \$20,000 in the account?
	\begin{shortsolution}
		Using a graph or table of values, approximately 6 years.
	\end{shortsolution}
\end{subproblem}
\end{problem}
			
			
%===================================
%   Author: Hughes
%   Date:   April 2011
%===================================
\begin{problem}[Compound interest]
Assume that you invest \$8000 into an account that earns interest at a \SI{5}{\percent} nominal rate. 
\begin{subproblem}
	How much will you have after four years if the interest is compounded yearly?
	\begin{shortsolution}
		$8000\left(1+\frac{0.05}{1}\right)^{4\cdot 1}\approx 9724.05$. There is  $\$9724.05$ in the account.
	\end{shortsolution}
\end{subproblem}
\begin{subproblem}
	How much will you have after four years if the interest is compounded weekly?
	\begin{shortsolution}
		$8000\left(1+\frac{0.05}{52}\right)^{4\cdot 52}\approx 9770.28$.  There is  $\$9770.28$ in the account.
	\end{shortsolution}
\end{subproblem}
\begin{subproblem}
	How much will you have after four years if the interest is compounded daily?
	\begin{shortsolution}
		$8000\left(1+\frac{0.05}{365}\right)^{4\cdot 365}\approx 9771.09$. There is  $\$9771.09$ in the account.
	\end{shortsolution}
\end{subproblem}
\begin{subproblem}
	If the interest is compounded daily, how long will it take your 
	investment to double? Triple?
	Use two decimal places in your answers.
	\begin{shortsolution}
		A table of values says that the function $8000\left(1+\frac{0.05}{365}\right)^{365t}$ reaches 16000 
		when $t$ is between 13 and 14 years. The function reaches 24000 between 21 and 22 years.
	\end{shortsolution}
\end{subproblem}
\begin{subproblem}
	The bank manager gives you the option to invest in a mystery account. You are 
	told that the interest is compounded daily, and that if you invest your \$8000 
	for 3 years you will have \$10479.40. What is the interest rate in this 
	mystery account?
	\begin{shortsolution}
		Trial and error gives $r=\SI{9}{\percent}$.
	\end{shortsolution}
\end{subproblem}
\end{problem}
			
%===================================
%   Author: Hughes
%   Date:   April 2011
%===================================
\begin{problem}[Exploring effective rate]
We have an amount $P_0$ to invest in an account that earns interest at a \SI{6}{\percent} nominal rate. 
Answer each of the following through the eighth digit after the decimal point.
\begin{subproblem}
	Find the effective rate if the interest is compounded annually.
	\begin{shortsolution}
		$\left( 1+\frac{0.06}{1} \right)^1-1 = 0.06000000$. The effective rate is
		\SI{6}{\percent}.
	\end{shortsolution}
\end{subproblem}
\begin{subproblem}
	Find the effective rate if the interest is compounded quarterly.
	\begin{shortsolution}
		$\left( 1+\frac{0.06}{4} \right)^4-1 \approx 0.06136355$. The effective rate 
		is approximately \SI{6.136355}{\percent}.
	\end{shortsolution}
\end{subproblem}
\begin{subproblem}
	Find the effective rate if the interest is compounded daily.
	\begin{shortsolution}
		$\left( 1+\frac{0.06}{365} \right)^{365}-1 \approx 0.06183131$. The effective rate
		is approximately \SI{6.183131}{\percent}.
	\end{shortsolution}
\end{subproblem}
\begin{subproblem}
	Find the effective rate if the interest is compounded every second.
	\begin{shortsolution}
		There are $365\cdot 24\cdot 60\cdot 60 = 31536000$ seconds in a year.
		\[
			\left( 1+\frac{0.06}{31536000} \right)^{31536000}-1 \approx 0.06183696
		\]
		The effective rate is approximately \SI{6.183696}{\percent}.
	\end{shortsolution}
\end{subproblem}
\begin{subproblem}\label{exp:prob:compoundedhalfsecond}
	Do you think there would be much difference if the interest was compounded 
	every half second compared to compounding it every second?
	\begin{shortsolution}
		No practical difference.
	\end{shortsolution}
	\begin{longsolution}
		The compounding frequency would be $31536000\cdot 2 = 63072000$
		\[
			\left( 1+\frac{0.06}{63072000} \right)^{63072000}-1 \approx 0.06183696
		\]
		which is as before. No practical difference.
	\end{longsolution}
\end{subproblem}
\end{problem}
			
%===================================
%   Author: Cary
%   Date:   September 2011
%===================================
\begin{problem}[Credit card]
A credit card company advertises an annual rate of 24\%.  What is the effective annual interest rate if your bill is compiled monthly?
\begin{shortsolution}
	$\left(1+\frac{0.24}{12}\right)^{12}-1\approx 0.26824179$; the effective annual interest rate is about $\SI{26.82}{\percent}$.
\end{shortsolution}
\end{problem}
			
%===================================
%   Author: Cary
%   Date:   September 2011
%===================================
\begin{problem}[Payday loan]
A payday loan company makes this offer to a customer: they will receive $\$375$ today and when they get paid in two weeks, they owe the payday loan company $\$450$.  
\begin{subproblem}\label{exp:prob:paydayloan}
	What percent interest is charged over the two-week period?
	\begin{shortsolution}
		The interest charged on the $\$375$ loan was $\$75$. As $\frac{75}{375}=0.2$, the 2-week percentage rate is $\SI{20}{\percent}$.
	\end{shortsolution}	
\end{subproblem}
\begin{subproblem}
	Using your answer in \cref{exp:prob:paydayloan}, compute the effective annual rate.
	\begin{shortsolution}
		$1.2^{26} - 1 \approx 113.475$; the effective annual rate is approximately \SI{11347.5}{\percent}.
	\end{shortsolution}
\end{subproblem}
\end{problem}
			
%===================================
%   Author: Kouzes
%   Date:   April 2011
%===================================
\begin{problem}[Find the rate]
The amount of money in an account grows by a total of \SI{45}{\percent} over a period of 10 years. Find the nominal 
interest rate (to 5 decimal places) if the account was compounded:
\begin{multicols}{2}
	\begin{subproblem}
		annually
		\begin{shortsolution}
			We need to solve $(1.45)P_0=P_0(1+r)^{10}$. This gives $r\approx \SI{3.7855}{\percent}$.
		\end{shortsolution}
	\end{subproblem}
	\begin{subproblem}
		monthly
		\begin{shortsolution}
			We need to solve $(1.45)P_0=P_0\left( 1+\frac{r}{12} \right)^{120}$. This gives $r\approx \SI{3.7214}{\percent}$. 
		\end{shortsolution}
	\end{subproblem}
\end{multicols}
\end{problem}
\end{exercises}
			
\section{The number $e$}
\begin{outcomes}
	\begin{outcomelist}
		\item learn about $e$
	\end{outcomelist}
\end{outcomes}
\marginpar{\centering\resizebox{\marginparwidth/2}{!}{$e$}}
			
%===================================
%   Author: Hughes
%   Date:   April 2011
%===================================
\subsection*{Continuously compounded interest and the number $e$}
Congratulations, you've just won the grand prize of one dollar in the 
giga-millions Genuine Wheel of Fortune lottery game.  
One dollar doesn't sound like much of a grand prize, but here's the 
thing -- you can deposit that dollar into a savings account that earns 
interest at a rate of \SI{100}{\percent} per year!  So after one year your prize 
will have grown to \$2, after two years it will have grown to \$4, wait 
a minute \ldots  even after 10 years it has only grown to \$1024.  But 
you know how these increasing exponential functions work; once they 
get going, they really take off.  If you can manage to hold off your 
withdrawal for twenty years you'll reap a little over a million dollars and 
wait just five more years after that and you'll take home over 
thirty-three million dollars!
			
All of those figures are based upon the assumption that the 
interest is compounded only once a year -- that's where the wheel 
comes in.  You spin the wheel to see the way in which the interest 
will be compounded.  \Cref{exp:tab:wheelfortune} shows the amount that will 
be in your account at the end of year one if the interest is compounded 
yearly, monthly, weekly, daily, every hour, every minute, and every second.  
Make sure that you see the connection between the expression $\left( 1+\frac{1}{n} \right)^n$
and the compound interest formula as it applies to this application.
			
\begin{margintable}
	\centering
	\caption{}
	\begin{tabular}{S[table-format=8.0]S[table-format=1.8]}
		\beforeheading
		\heading{$n$} & \heading{$\dd\left( 1+\frac{1}{n} \right)^n$} \\\afterheading
		1             & 2                                             \\\normalline
		12            & 2.61303529                                    \\\normalline  
		52            & 2.69259695                                    \\\normalline
		365           & 2.71456748                                    \\\normalline
		8760          & 2.71812669                                    \\\normalline
		525600        & 2.71827922                                    \\\normalline
		31536000      & 2.71828247                                    \\\lastline
	\end{tabular}
	\label{exp:tab:wheelfortune}
\end{margintable}
			
We can see that through the fifth digit after the decimal point the 
amount in your account will be the same whether the compounding is 
done by the minute or by the second.  If we were to continue increasing 
the number of times we compound, we would find that the digits start to 
get fixed farther and farther to the right of the decimal point.  
However, no matter how many times we compound, if we look far enough to 
the right of the decimal point the digits will be different if 
the interest is compounded just one additional time. 
That is to say
\[
	\lim_{n\to\infty}\left( 1+\frac{1}{n} \right)^{n}
\]
is an irrational number (a number that in decimal form never terminates 
and never forms a forever-repeating pattern).  
			
It turns out that this very same irrational number pops up in a wide 
variety of applications, and as such it is worthy of a name.  The 
number is called Euler's number and is symbolized by the letter $e$.  
If we let the number of times the interest is compounded increase without 
bound, we say that interest is being compounded continuously.  In our 
lottery game, if the interest is compounded continuously, then the amount 
that would be in your giga-millions account at the end of year one 
would be $e$ dollars where $e\approx 2.71828$.
			
\begin{pccdefinition}[The number $e$]\label{exp:def:e}
	The number $e$ is called Euler's number named after the famous
	Swiss mathematician Leonhard Euler (1707-1783). It is also called 
	the natural base of an exponential function.
	\begin{align*}
		e & = \lim_{n\to\infty}\left( 1+\frac{1}{n} \right)^n \\
		  & \approx 2.718281828                               
	\end{align*}
	Another definition of $e$ is hinted at in \cref{exp:prob:factorials}.
\end{pccdefinition}
			
\subsection*{Continuous growth and decay}
We can use the natural base, $e$, to model real life situations that involve 
\emph{continuous} growth and decay. The following definition will guide us 
in what follows.
\begin{pccdefinition}[Continuous growth and decay]\label{exp:def:contgrowthdecay}
	If we are modeling a situation that involves continuous growth or decay, then 
	the model will have the following form:
	\[
		Q(t)=Q_0\,e^{kt}
	\]
	\begin{itemize}
		\item If $k>0$, then the function $Q$ is increasing, and $k$ is called the \emph{continuous growth rate}.
		\item If $k<0$, then the function $Q$ is decreasing, and $|k|$ is called the \emph{continuous decay rate}.
	\end{itemize}
	In the context of a continuously compounded interest problems, $k$ is the nominal interest rate and $e^k-1$ is the effective interest rate.
\end{pccdefinition}
			
			
%===================================
%   Author: Hughes
%   Date:   April 2011
%===================================
\begin{pccexample}\label{exp:ex:compcont}
	We have \$7000 to invest in an account that has a nominal interest rate of \SI{3}{\percent} compounded 
	{\em continuously}. 
	\begin{enumerate}
		\item Find a model for this situation.
		\item Find the amount in the account after 4 years.
		\item Compare the answer with the results of \vref{exp:ex:compoundinterest}.
		\item Find the effective annual rate of interest.
	\end{enumerate}
	\begin{pccsolution}
		\begin{enumerate}
			\item Using \cref{exp:def:contgrowthdecay}, 
			\[
				Q(t) = 7000\,e^{0.03t}
			\]
			\item We evaluate the function $Q$ when $t=4$
			\begin{align*}
				Q(4) & =  7000\,e^{0.03(4)} \\
				     & \approx  7892.48     
			\end{align*}
			The amount in the account after four years is $\$7892.48$.
			\item This is larger than any of the values we found in \cref{exp:ex:compoundinterest}.
			\item We calculate the effective annual rate of interest using \cref{exp:def:contgrowthdecay}
			\[
				e^{0.03} -1 \approx 0.030454534
			\]
			The effective annual rate of interest is approximately $\SI{3.0454534}{\percent}$. Note that this is greater than any of the values we found in \vref{exp:ex:effectiverate}.
		\end{enumerate}
	\end{pccsolution}
\end{pccexample}
\begin{doyouunderstand}
	\begin{problem}
	Repeat \cref{exp:ex:compcont} using an investment of \$4000, and a 
	nominal rate of \SI{2}{\percent}.
	\begin{shortsolution}
		\begin{enumerate}
			\item Using \cref{exp:def:contgrowthdecay}, 
			\[
				Q(t) = 4000\,e^{0.02t}
			\]
			\item We evaluate the function $Q$ when $t=4$
			\begin{align*}
				Q(4) & = 4000\,e^{0.02(4)} \\
				     & \approx 4333.15     
			\end{align*}
			The amount in the account after four years is $\$4333.15$.
			\item Not relevant to this problem.
			\item We calculate the effective annual rate of interest using \cref{exp:def:contgrowthdecay}
			\[
				e^{0.02}-1\approx 0.0202
			\]
			The effective annual rate of interest is approximately $\SI{2.02}{\percent}$. 
		\end{enumerate}
	\end{shortsolution}
	\end{problem}
\end{doyouunderstand}
			
%===================================
%   Author: Hughes
%   Date:   July 2011
%===================================
\begin{pccexample}\label{exp:prob:compound5all}%
	You have \$2000 to invest in an interest-bearing account that has a nominal interest rate of $\SI{5}{\percent}$.
	\begin{enumerate}
		\item Calculate the effective annual growth rates if the interest is compounded daily and if the interest is compounded continuously.  
		\item State the annual growth factor for each account.
		\item Calculate the account balance after $10$ years if the interest is compounded annually, daily, and continuously. 
	\end{enumerate}
	\begin{pccsolution}
		\begin{enumerate}
			\item If the interest is compounded daily, then we calculate
			\[
				\left(1+\frac{0.05}{365}\right)^{365\cdot 1} -1 \approx 0.051267496
			\]
			The effective annual rate is approximately $\SI{5.1267496}{\percent}$.
																									
			If the interest is compounded \emph{continuously} then we calculate
			\[
				e^{0.05}-1\approx 0.05127110
			\]
			The effective annual rate is approximately $\SI{5.127110}{\percent}$.
			\item If the interest is compounded daily, then the annual 
			growth factor is calculated using
			\[
				\left(1+\frac{0.05}{365}\right)^{365\cdot 1} \approx 1.051267496
			\]
			The growth factor is approximately $1.051267496$.
																									
			If the interest is compounded continuously, then the annual 
			growth factor is calculated using
			\[
				e^{0.05} \approx 1.05127110
			\]
			The growth factor is approximately $1.05127110$.
			\item The account balance after 10 years if the interest is compounded annually, daily, and continuously 
			is calculated using (respectively)
			\begin{align*}
				2000(1.05)^{10}                                   & \approx 3257.79 \\
				2000\left(1+\frac{0.05}{365}\right)^{10\cdot 365} & \approx 3297.33 \\
				2000e^{0.05\cdot 10}                              & \approx 3297.44 
			\end{align*}
			We conclude that 
			\begin{itemize}
				\item when the interest is compounded annually the balance will be $\$3257.79$;
				\item when the interest is compounded daily the balance will be $\$3297.33$;
				\item when the interest is compounded continuously the balance will be $\$3297.44$.
			\end{itemize}
		\end{enumerate}
	\end{pccsolution}
\end{pccexample}
			
\begin{doyouunderstand}
	\begin{problem}
	Repeat \cref{exp:prob:compound5all} using an initial investment of \$15,000 at a nominal interest rate of $\SI{8}{\percent}$.
	\begin{shortsolution}
		Growth rates:
		\begin{itemize}
			\item Daily: $\left(1+\frac{0.08}{365}\right)^{365\cdot 1} -1 \approx 0.083277572$; the effective annual rate is approximately $\SI{8.3277572}{\percent}$.
			\item Continuous : $e^{0.08}-1\approx 0.08328707$; the effective annual rate is approximately $\SI{8.328707}{\percent}$.
		\end{itemize}
		Growth factors:
		\begin{itemize}
			\item Daily: $\left(1+\frac{0.08}{365}\right)^{365\cdot 1} \approx 8.3277572$; the annual growth factor is approximately $8.3277572\%$.
			\item Continuous: $e^{0.08} \approx 1.08328707$; the growth factor is approximately $1.08328707$.
		\end{itemize}
		Balance after $10$ years:
		\begin{itemize}
			\item Annually: $15000(1.08)^{10} \approx 32,383.87$; the balance will be $\$32,383.87$.
			\item Daily: $15000\left(1+\frac{0.08}{365}\right)^{10\cdot 365} \approx 33380.19$; the balance will be $\$33380.19$.
			\item Continuously: $15000e^{0.08\cdot 10} \approx 33383.11$; the balance will be $\$33383.11$.
		\end{itemize}
	\end{shortsolution}
	\end{problem}
\end{doyouunderstand}
			
			
\begin{pccexample}
	We have \$600 to invest in an account that compounds interest continuously.  
	\begin{enumerate}
		\item If the nominal interest rate is \SI{8}{\percent}, what will the effective annual growth rate be?
		\item If the effective annual interest rate is \SI{8}{\percent}, what will the nominal interest rate be?
	\end{enumerate}
	\begin{pccsolution}
		\begin{enumerate}
			\item The amount (in dollars) in the account will be $600\,e^{0.08t}$ $t$ years after the money is invested.  We need to find the value of $r$ in the equation $$A(t)=600(1+r)^t.$$  Well, 
			\begin{align*}
				A(t) & = 600\,e^{0.08t}                     \\
				     & = 600\left(e^{0.08}\right)^t         \\
				     & \approx 600\left(1.08328707\right)^t 
			\end{align*}
			So the growth rate is $e^{0.08}-1$, or about $\SI{8.328707}{\percent}$.
			\item This time the account will have $600(1.08)^t$, $t$ years after the money  is invested.  We need to find the value of $r$ in the equation $$A(t)=600\,e^{rt}.$$  
			Since the two expressions for $A(t)$ must be equal,
			\begin{align*}
				600\,e^{rt}  = 600(1.08)^t & \Longrightarrow e^{rt}  = (1.08)^t \\
				                           & \Longrightarrow e^r  = 1.08        
			\end{align*}
			Later we will learn how to solve for $r$ exactly using logarithms.  For now, we can find an approximate solution for $r$ using a graphing calculator.  The value of $r$ is about 0.077, so the nominal interest rate is about \SI{7.7}{\percent}.
		\end{enumerate} 
	\end{pccsolution}
\end{pccexample}
			
\subsection*{Radioactive decay}
%===================================
%   Author: Hughes
%   Date:   July 2011
%===================================
You may remember that we studied radioactive decay in \vref{exp:prob:radioactive}. In 
fact it is often more appropriate to use the natural base, $e$, in such applications.
			
\begin{pccexample}
	The number of radioactive atoms in a sample of Carbon-14 decays according to the model
	\[
		Q(t)= Q_0\,e^{-0.000120968t},
	\]
	where $Q_0$ is the initial mass of the radioactive atoms and $Q(t)$ is the mass of radioactive atoms $t$ years after the sample was established.
										
	Assuming that the radioactive atoms have an initial mass of $\SI{10}{\milli\gram}$  ($Q_0=10$),
	what is the mass after 5730 years?
	\begin{pccsolution}
		We evaluate
		\[
			Q(5730)\approx 5
		\]
		The radioactive atoms have a mass of approximately $\SI{5}{\milli\gram}$  after 5730 years; in other 
		words, the sample has decayed by half.
	\end{pccsolution}
\end{pccexample}
			
			
\subsection*{Another occurrence of $e$}
By definition, a linear function is a function with constant slope.  
Amongst other things, this definition implies that non-linear 
functions do not have constant slope.  That begs the question, 
just what do we mean when we talk about the slope of a non-linear 
function?  
			
The slope of non-linear functions is dealt with in calculus, but it boils 
down to finding the slope of the line that best mimics the direction of 
motion along the function at the point of interest.  These lines are 
called tangent lines, and the tangent lines at the point $(0,1)$ are shown for three 
different exponential functions in \crefrange{exp:fig:eusingslopes2}{exp:fig:eusingslopes3}.
			
\begin{figure}[!htb]
	\begin{widepage}
	\setlength{\figurewidth}{0.3\textwidth}
	\begin{minipage}{\figurewidth}
		\begin{tikzpicture}
			\begin{axis}[
					framed,
					xmin=-2,xmax=1,
					ymin=-1,ymax=2,
					xtick={-1},
					ytick={1},
					grid=both,
				]
				\addplot expression[domain=-2:1]{2^x};
				\addplot[asymptote,<->] expression[domain=-2:1]{0.693147181*x+1}node[axisnode,inner sep=.5cm,anchor=north,pos=0.8,]{$m\approx 0.69$};
			\end{axis}
		\end{tikzpicture}
		\captionof{figure}{$f(x)=2^x$}
		\label{exp:fig:eusingslopes2}
	\end{minipage}%
	\hfill
	\begin{minipage}{\figurewidth}
		\begin{tikzpicture}
			\begin{axis}[
					framed,
					xmin=-2,xmax=1,
					ymin=-1,ymax=2,
					xtick={-1},
					ytick={1},
					grid=both,
				]
				\addplot expression[domain=-2:.69]{exp(x)};
				\addplot[asymptote,<->] expression[domain=-2:1]{x+1}node[axisnode,inner sep=.5cm,anchor=north,pos=0.8,]{$m=1$};
			\end{axis}
		\end{tikzpicture}
		\captionof{figure}{$g(x)=e^x$}
		\label{exp:fig:eusingslopese}
	\end{minipage}%
	\hfill
	\begin{minipage}{\figurewidth}
		\begin{tikzpicture}
			\begin{axis}[
					framed,
					xmin=-2,xmax=1,
					ymin=-1,ymax=2,
					xtick={-1},
					ytick={1},
					grid=both,
				]
				\addplot expression[domain=-2:.65]{3^x};
				\addplot[asymptote,<->] expression[domain=-1.8:0.9]{1.09861229*x+1}node[axisnode,inner sep=.5cm,anchor=north,pos=0.8,]{$m\approx 1.10$};
			\end{axis}
		\end{tikzpicture}
		\captionof{figure}{$h(x)=3^x$}
		\label{exp:fig:eusingslopes3}
	\end{minipage}
	\end{widepage}
\end{figure}
			
It can be proven (using calculus) that at any given point the slope of a function of the form
$y=b^x$ is directly proportional to the $y$-coordinate of the point; the 
proportionality constant is whatever the slope is at the point $(0,1)$.
That is, for functions of the form $y=b^x$, at any given point $(x,y)$  
the slope of the function is $ky$ where $k$ is the slope of the curve at $(0,1)$.
For example, the function of the form $y=b^x$ that has a slope of $4$ at the point
$(0,1)$ has a slope of $12$ at the point where the $y$-coordinate is $3$ and a slope of $80$ 
at the point where $y$-coordinate is $20$.
			
One implication of the proportionality between the slope of the function 
and the $y$-coordinate of the function is that the function of the 
form $y=b^x$ that has a slope of one at $(0,1)$ has a very unique 
property -- at any given point the slope of the function is exactly 
equal to the $y$-coordinate of the point. As suggested in \cref{exp:fig:eusingslopese}, 
it turns out that the base that creates this unique situation is that 
same number that showed up in the continuously compounded interest 
application, the number $e$. So on the graph of $y=e^x$, at any given 
point the slope of the curve is exactly equal to the $y$-coordinate of 
the point.
			
\investigation*{}
%===================================
%   Author: Hughes
%   Date:   February 2012
%===================================
\begin{problem}[Trusting the definition of $e$]
\Cref{exp:def:e} says that 
\[
	e = \lim_{n\rightarrow\infty}\left( 1+\frac{1}{n} \right)^n   
\]
Let's see if we can verify this by doing some numerical calculations. Let $f$ 
be the function that has formula
\[
	f(x) = \left( 1+\frac{1}{x} \right)^x   
\]
\begin{subproblem}\label{exp:prob:exploree}
	Evaluate $f(10)$, $f(100)$, $f(1000)$, $f(10000)$, and $f(100000)$. Give 
	each answer correct to 5 decimal places
	\begin{shortsolution}
		$f(10)\approx 2.59374$, $f(100)\approx 2.70481$, $f(1000)\approx 2.71692$, $f(10000)\approx 2.71814$, $f(100000)\approx 2.71827$ 
	\end{shortsolution}
\end{subproblem}
\begin{subproblem}
	Now use your calculator to evaluate $e$ correct to 5 decimal places, and compare your answer
	to those you obtained in \cref{exp:prob:exploree}.
	\begin{shortsolution}
		$e\approx 2.71828$; note that $f(100000)$ is closest to this.
	\end{shortsolution}
\end{subproblem}
\begin{subproblem}
	Why can't we just put $f(\infty)$ into our calculator? 
	\begin{shortsolution}
		Because $\infty$ is not a number -- it is a concept that we use to represent that a variable is growing 
		without bound.
	\end{shortsolution}
\end{subproblem}
\end{problem}
%===================================
%   Author: Kouzes
%   Date:   April 2011
%===================================
\begin{problem}[Compounding continuously]
Imagine that you deposit \$100 into a bank account which  accrues interest at a nominal 
rate of \SI{5}{\percent} compounded continuously. The amount $Q(t)$ in the account  $t$ years 
after opening the account is given by $Q(t)=100\,e^{0.05t}$.
			
\begin{subproblem}
	Find $Q(1)$ correct to two decimal places and interpret the result.
	\begin{shortsolution}
		$Q(1)=100e^{0.05}\approx 105.13$. After 1 year there is approximately \$105.13 in the account.
	\end{shortsolution}
\end{subproblem}
\begin{subproblem}
	What is the effective annual growth rate? State your answer to five decimal places.
	\begin{shortsolution}
		The effective annual growth rate is $e^{0.05}-1\approx 0.05127 $ or approximately $\SI{5.1271}{\percent}$.
	\end{shortsolution}
\end{subproblem}
\begin{subproblem}
	Use your calculator to graph
	\[
		Q(t)=100\,e^{0.05t} \qquad \textrm{and} \qquad P(t)=100(1.05127)^t
	\]
	on the same axes. What do you notice?
	\begin{shortsolution}
		We know that $e^{0.05}\approx 1.05127$, so we are not surprised to see that $Q$ and $P$ 
		are approximately the same function. The function $Q$ is plotted below as a solid line, 
		and the function $P$ is plotted as dots.
																			
		\begin{tikzpicture}
			\begin{axis}[
					framed,
					xmin=-1,xmax=10,
					ymin=80,ymax=170,
					axis line style=->,
					xtick={0,2,...,8},
					xlabel={$t$},
					axis y discontinuity=crunch,
				]
				\addplot+[->]expression[domain=0:10]{100*(1.05127)^x};
				\addplot+[-,width=3pt]expression[domain=0:10]{100*exp(0.05*x)};
			\end{axis}
		\end{tikzpicture}
	\end{shortsolution}
\end{subproblem}
\end{problem}
			
\begin{exercises}
%===================================
%   Author: Hughes
%   Date:   March 2012
%===================================
\begin{problem}[Given description, write formula]
In each of the following, assume that the population of a town 
changes at the given rate.
Write a formula for $P(t)$, the population at time $t$,
measured in years since $2012$.
\begin{multicols}{2}
	\begin{subproblem}
		$P_0=600$, increasing at $\SI{7}{\percent}$ per year.
		\begin{shortsolution}
			$P(t)=600(1.07)^t$ 
		\end{shortsolution}
	\end{subproblem}
	\begin{subproblem}
		$P_0=1500$, increasing at $\SI{7}{\percent}$ continuously per year.
		\begin{shortsolution}
			$P(t)=1500e^{0.07t}$
		\end{shortsolution}
	\end{subproblem}
	\begin{subproblem}
		$P_0=2300$, increasing at $\SI{27}{\percent}$ per year.
		\begin{shortsolution}
			$P(t)=2300(1.27)^t$ 
		\end{shortsolution}
	\end{subproblem}
	\begin{subproblem}
		$P_0=3600$, increasing at $\SI{52}{\percent}$ continously per year.
		\begin{shortsolution}
			$P(t)=3600e^{0.52t}$
		\end{shortsolution}
	\end{subproblem}
	\begin{subproblem}
		$P_0=450$, decreasing at $\SI{6}{\percent}$ per year.
		\begin{shortsolution}
			$P(t)=450(0.94)^t$ 
		\end{shortsolution}
	\end{subproblem}
	\begin{subproblem}
		$P_0=2405$, decreasing at $\SI{12}{\percent}$ continuously per year.
		\begin{shortsolution}
			$P(t)=2405e^{-0.12t}$
		\end{shortsolution}
	\end{subproblem}
	\begin{subproblem}
		$P_0=4402$, decreasing at $\SI{19}{\percent}$ per year.
		\begin{shortsolution}
			$P(t)=4402(0.81)^t$ 
		\end{shortsolution}
	\end{subproblem}
	\begin{subproblem}
		$P_0=7203$, decreasing at $\SI{31}{\percent}$ continuously per year.
		\begin{shortsolution}
			$P(t)=7203e^{-0.31t}$
		\end{shortsolution}
	\end{subproblem}
\end{multicols}
\end{problem}
%===================================
%   Author: Hughes
%   Date:   March 2012
%===================================
\begin{problem}[Given formula, write interpretaion]
Each of the following formulas model the population of a town 
at time $t$ (in years) since 1998. Determine the initial population, $P_0$,
and the percentage change, $r$, and give a sentence that describes the model.
\begin{multicols}{2}
	\begin{subproblem}
		$P(t)=1000(1.1)^t$ 
		\begin{shortsolution}
			$P_0=1000$, $r=\SI{10}{\percent}$; the population is initially $1000$ people, and 
			increases at $\SI{10}{\percent}$ per year.
		\end{shortsolution}
	\end{subproblem}
	\begin{subproblem}
		$P(t)=1000e^{0.11t}$
																	 
		\begin{shortsolution}
			$P_0=1000$, $r=\SI[exponent-base=e]{e0.11}{\percent}\approx\SI{1.11627807}{\percent}$; the population is 
			initially $1800$ people, and increases at approximately $\SI{11.627807}{\percent}$ per year.
		\end{shortsolution}
	\end{subproblem}
	\begin{subproblem}
		$P(t)=300(0.83)^t$ 
		\begin{shortsolution}
			$P_0=300$, $r=\SI{-17}{\percent}$; the population is initially $200$ people, and decreases
			at $\SI{17}{\percent}$ per year.
		\end{shortsolution}
	\end{subproblem}
	\begin{subproblem}
		$P(t)=907e^{-0.08t}$
		\begin{shortsolution}
			$P_0=907$, $r=\SI[exponent-base=e]{e-0.08}{\percent}\approx\SI{0.923116346}{\percent}$; the population is 
			initially $907$ people, and decreases
			at approximately $\SI{7.69}{\percent}$ per year.
		\end{shortsolution}
	\end{subproblem}
\end{multicols}
\end{problem}
%===================================
%   Author: Hughes
%   Date:   July 2011
%===================================
\begin{problem}[The number $e$]
Put the following numbers in ascending order:
\[
	e, \quad \nicefrac{1}{3},\quad 9,\quad e^{-1},\quad 1,\quad \nicefrac{1}{4},\quad e,\quad 3,\quad \nicefrac{1}{e^2}, \quad 2, \quad e^2
\]
\begin{shortsolution}
	In ascending order: $\nicefrac{1}{e^2}<\nicefrac{1}{4}<\nicefrac{1}{3}<\nicefrac{1}{e}<2<e<3<e^2<9$
\end{shortsolution}
\end{problem}
%===================================
%   Author: Hughes
%   Date:   February 2012
%===================================
\begin{problem}[Investing in an account]
You have \$2000 to invest in an account that accrues interest at a nominal rate of $\SI{3.75}{\percent}$. 
Write a formula for $A(t)$, the amount of money in the account $t$ years 
after opening the account, assuming that the interest is compounded 
in each of the following ways. Calculate the effective annual rate in each case.
\begin{multicols}{4}
	\begin{subproblem}
		Annually. 
		\begin{shortsolution}
			$A(t)=2000(1.0375)^t$; the effective annual rate is \SI{3.75}{\percent}. 
		\end{shortsolution}
	\end{subproblem}
	\begin{subproblem}
		Monthly 
		\begin{shortsolution}
			$A(t)=2000\left( 1+\frac{0.0375}{12} \right)^{12t}$; the effective annual rate is 
			calculated using $\left( 1+\frac{0.0375}{12} \right)^{12}\approx 1.03815129$. The 
			effective annual rate is approximately \SI{1.03815129}{\percent}.
		\end{shortsolution}
	\end{subproblem}
	\begin{subproblem}
		Daily 
		\begin{shortsolution}
			$A(t)=2000\left( 1+\frac{0.0375}{365} \right)^{365t}$; the effective annual rate is 
			calculated using $\left( 1+\frac{0.0375}{365} \right)^{365}\approx 1.0382100$. The 
			effective annual rate is approximately \SI{1.0382100}{\percent}.
		\end{shortsolution}
	\end{subproblem}
	\begin{subproblem}
		Continuously 
		\begin{shortsolution}
			$A(t)=2000e^{0.0375t}$; the effective annual rate is calculated using $e^{0.0375}\approx 1.0382120$.
			The effective annual rate is approximately \SI{1.0382120}{\percent}.
		\end{shortsolution}
	\end{subproblem}
\end{multicols}
\end{problem}
%===================================
%   Author: Hughes
%   Date:   February 2012
%===================================
\begin{problem}[Solving an equation involving $e$]
You saved $\$7,000$ toward the purchase of a car costing $\$9,000$. 
How long would the $\$7,000$ have to be invested in an account that 
earns \SI{8}{\percent} compounded continuously to grow to $\$9,000$?
\begin{shortsolution}
	We have to solve the equation $9000=7000e^{0.08t}$ for $t$. This can 
	be simplified to $\nicefrac{9}{7}=e^{0.08t}$.  We can solve this 
	using our graphing calculator as shown below -- this gives that $t\approx 3.14$. 
	So it would take just over $3$ years for our account to reach $\$9,000$.
											
	\begin{tikzpicture}
		\begin{axis}[
				framed,
				xmin=0,xmax=4,
				ymin=0,ymax=2,
				xtick={1,2,3},
				ytick={0.5,1,1.5},
				xlabel={$t$},
				axis line style=->,
				grid=both,
			]
			\addplot+[->]expression[domain=0:4]{exp(.08*x)};
			\addplot+[-]expression[domain=0:4]{9/7};
		\end{axis}
	\end{tikzpicture}
\end{shortsolution}
\end{problem}
%===================================
%   Author: Kouzes
%   Date:   April 2011
%===================================
\begin{problem}[Annual effective rate vs continuous growth rate]
Imagine that you have a bank account with a principal of \$2500 earning an 
annual \emph{effective} rate of \SI{10}{\percent}.
\begin{subproblem}
	Write a formula for $A(t)$, the amount of money in the account $t$ years 
	after opening the account.
	\begin{shortsolution}
		$A(t)=2500(1.1)^t$.
	\end{shortsolution}
\end{subproblem}
\begin{subproblem}
	If the interest is compounded continuously, use your calculator to find the 
	continuous growth rate to 5 decimal places.
	\begin{shortsolution}
		We need to solve the equation $e^r=1.1$; using a table of values, we 
		find $r\approx 0.09531$, and the continuous growth rate is 
		approximately $\SI{9.531}{\percent}$.
	\end{shortsolution}
\end{subproblem}
\end{problem}
%===================================
%   Author: Hughes
%   Date:   February 2012
%===================================
\begin{problem}[Newton's Law of Cooling]
You may have noticed that when you leave a cup of hot coffee in a room, 
the coffee's temperature will decrease to room temperature. Provided that 
the difference between the temperature of the liquid and its surrounding 
is not too great, then \emph{Newton's Law of Cooling} applies.
			
One day you buy a cup of coffee that starts at \SI{90}{\degreeCelsius}, 
and you are stood outside where the temperature is \SI{0}{\degreeCelsius}. Let
$T(t)$ represent the temperature of your coffee at time $t$ (in minutes) since you bought it.
\begin{subproblem}
	What is $T(0)$?
	\begin{shortsolution}
		$T(0)=90$. 
	\end{shortsolution}
\end{subproblem}
\begin{subproblem}
	Do you expect $T(t)$ to tend toward a value as $t\rightarrow\infty$?
	\begin{shortsolution}
		$T(t)\rightarrow 0$ as $t\rightarrow\infty$. 
	\end{shortsolution}
\end{subproblem}
\begin{subproblem}
	The temperature of the coffee \emph{decreases} at a continuous rate of \SI{7}{\percent} per minute. Write a
	formula for $T(t)$.
	\begin{shortsolution}
		$T(t)=90e^{-0.07t}$ 
	\end{shortsolution}
\end{subproblem}
\begin{subproblem}
	According to your model, what is the temperature of the coffee \SI{10}{\minute} after you bought it?
	\begin{shortsolution}
		$T(10)=90e^{-0.07\cdot 10}\approx 45$; the temperature of the coffee
		is approximately \SI{45}{\degreeCelsius}.
	\end{shortsolution}
\end{subproblem}
\begin{subproblem}
	According to your model, how long after you bought your coffee is 
	the temperature of the coffee \SI{5}{\degreeCelsius}?
	\begin{shortsolution}
		We need to solve the equation $5=90e^{-0.07t}$ for $t$. Using a graphing calculator, 
		we find $t\approx 41.29$. The coffee will be \SI{5}{\degreeCelsius} approximately \SI{41}{\minute} 
		after it was bought.
	\end{shortsolution}
\end{subproblem}
\begin{subproblem}
	According to your model, is the temperature of the coffee ever \SI{0}{\degreeCelsius}? 
	\begin{shortsolution}
		No -- the model is limited in this way. 
	\end{shortsolution}
\end{subproblem}
\end{problem}
			
%===================================
%   Author: Neft (Hughes)
%   Date:   August 2012
%===================================
\begin{problem}[The RC circuit]
A \emph{capacitor} is a device that stores electrical energy in the form of charged particles.
The \emph{voltage} on the capacitor is a result of the electric field created by the particles and
is proportional to the amount of charge stored. A \emph{resistor} is a device that dissipates
electrical energy. If a capacitor is charged up and then connected across a resistor, the
capacitor discharges and the voltage drops.
			
The voltage (in \si{\volt}), on the capacitor as it is being discharged is modeled by the
function $V$ that has formula 
\[
	V(t)=V_0e^{-\frac{t}{RC}}
\]
where $V_0$ is the initial capacitor voltage, $R$ is the value of the
resistor (in \si{\ohm}), $C$ is the value of the capacitor (in \si{\farad}) and $t$ is time (in \si{\second}).
\begin{subproblem}
	Suppose that a $2\times 10^{-6}$-farad capacitor, initially charged up 
	to $\SI{10}{\volt}$, is connected across a 50,000-ohm resistor. 
	Write the formula for $V(t)$.
	\begin{shortsolution}
		$V(t)=10e^{-\frac{t}{0.1}}=10e^{-10t}$
	\end{shortsolution}
\end{subproblem}
\begin{subproblem}\label{exp:prob:circuittab}
	Construct a table of values of $V(t)$ using $t=0,0.1,...,0.5$.
	\begin{shortsolution}
		\begin{tabular}[t]{S[table-format=1.1]S[table-format=2.3]} 
			\beforeheading
			\heading{$t$}            & \heading{$V(t)$}       \\
			\heading{(\si{\second})} & \heading{(\si{\volt})} \\
			\afterheading
			0.0                      & 10.000                 \\  \normalline
			0.1                      & 3.369                  \\  \normalline
			0.2                      & 1.353                  \\  \normalline
			0.3                      & 0.498                  \\  \normalline
			0.4                      & 0.183                  \\  \normalline
			0.5                      & 0.067                  \\  \lastline
		\end{tabular}
	\end{shortsolution}
\end{subproblem}
\begin{subproblem}
	Based on your answer to \cref{exp:prob:circuittab}, do you think the 
	graph of $y=V(t)$ will be concave up or concave down? Why?
	\begin{shortsolution}
		Concave up, since the slopes between successive entries 
		are increasing by becoming less negative.
	\end{shortsolution}
\end{subproblem}
\begin{subproblem}
	Graph the function $V$.
	\begin{shortsolution}
		$y=V(t)$ is shown below.
																			
		\begin{tikzpicture}
			\begin{axis}[
					framed,
					xmin=-0.2,xmax=1,
					ymin=-1,ymax=11,
					xtick={0.2,0.4,...,0.8},
					minor xtick={0.1,0.3,...,0.9},
					ytick={2,4,...,10},
					minor ytick={1,3,...,9},
					xlabel={$t$},
					grid=both,
				]
				\addplot+[->]expression[domain=0:1,samples=50]{10*exp(-10*x)};
			\end{axis}
		\end{tikzpicture}
	\end{shortsolution}
\end{subproblem}
\begin{subproblem}
	By what percentage has the voltage decreased on the capacitor after $\SI{0.1}{\second}$? 
	After $\SI{0.2}{\second}$?
	\begin{shortsolution}
		After $\SI{0.1}{\second}$, the voltage on the capacitor has decreased by about
		$\SI{63}{\percent}$. After $\SI{0.2}{\second}$, the voltage on the 
		capacitor has decreased by about $\SI{86}{\percent}$.
	\end{shortsolution}
\end{subproblem}
\end{problem}
			
%===================================
%   Author: Neft (Hughes)
%   Date:   August 2012
%===================================
\begin{problem}[Charging a capacitor]
If a voltage source, such as a battery, is connected to a resistor and a uncharged
capacitor, the capacitor will charge up. The voltage (in \si{\volt}), 
on the capacitor as it is being charged is modeled by the function $V$ 
that has formula
\[
	V(t)=V_S\left( 1-e^{-\frac{t}{RC}} \right)
\]
where $V_S$ represents the voltage of the source (in \si{\volt}), $R$
is the value of the resistor (in \si{\ohm}), $C$ is the
value of the capacitor (in \si{\farad}) and $t$ is time (in \si{\second}).
\begin{subproblem}
	Suppose that a 10-volt battery is connected to a $2\times 10^{-6}$-farad capacitor 
	and a 50,000-ohm resistor.  Write the formula for $V(t)$.
	\begin{shortsolution}
		$V(t)=10( 1-e^{-0.1t} )$ 
	\end{shortsolution}
\end{subproblem}
\begin{subproblem}\label{exp:prob:chargecaptab}
	Construct a table of values of $V(t)$ using $t=0,0.1,...,0.5$.
	\begin{shortsolution}
		\begin{tabular}[t]{S[table-format=1.1]S[table-format=1.3]} 
			\beforeheading
			\heading{$t$}            & \heading{$V(t)$}       \\
			\heading{(\si{\second})} & \heading{(\si{\volt})} \\
			\afterheading
			0.0                      & 0.000                  \\  \normalline
			0.1                      & 6.321                  \\  \normalline
			0.2                      & 8.647                  \\  \normalline
			0.3                      & 9.502                  \\  \normalline
			0.4                      & 9.817                  \\  \normalline
			0.5                      & 9.933                  \\  \lastline
		\end{tabular}
	\end{shortsolution}
\end{subproblem}
\begin{subproblem}
	Based on your answer to \cref{exp:prob:chargecaptab}, do you think the 
	graph of $y=V(t)$ will be concave up or concave down? Why?
	\begin{shortsolution}
		The graph will be concave down since the slopes between successive
		entries are decreasing by becoming less positive.
	\end{shortsolution}
\end{subproblem}
\begin{subproblem}
	Graph the function $V$.
	\begin{shortsolution}
		$y=V(t)$ is shown below.
																			
		\begin{tikzpicture}
			\begin{axis}[
					framed,
					xmin=-0.2,xmax=1,
					ymin=-1,ymax=11,
					xtick={0.2,0.4,...,0.8},
					minor xtick={0.1,0.3,...,0.9},
					ytick={2,4,...,10},
					minor ytick={1,3,...,9},
					xlabel={$t$},
					grid=both,
				]
				\addplot+[->]expression[domain=0:1,samples=50]{10*(1-exp(-10*x))};
			\end{axis}
		\end{tikzpicture}
	\end{shortsolution}
\end{subproblem}
\begin{subproblem}
	After $\SI{0.1}{\second}$, the voltage on the capicitor is 
	what percentage of the battery voltage? After $\SI{0.2}{\second}$?
	\begin{shortsolution}
		After $\SI{0.1}{\second}$, the capacitor has charged up to about
		$\SI{63}{\percent}$ of the battery voltage. After $\SI{0.2}{\second}$,
		the capacitor has charged up to about $\SI{86}{\percent}$ of the 
		battery voltage.
	\end{shortsolution}
\end{subproblem}
\begin{subproblem}
	Notice that even though the charge on the capacitor increasing, the formula modeling
	the voltage contains a decaying exponential. Explain why, mathematically, the function $V$ is
	increasing even though it contains a decaying exponential.
	\begin{shortsolution}
		The value of the function can increase even though the exponential term is decaying
		because it is being subtracted from $1$. As the exponential term decreases, the difference
		between its value and $1$ increases, creating an increasing function.
	\end{shortsolution}
\end{subproblem}
\end{problem}
			
%===================================
%   Author: Hughes
%   Date:   July 2011
%===================================
\begin{problem}[Half-life exploration]
In each of the following problems, assume that 
\[
	Q(t) = Q_0\,e^{kt}
\]
models the mass of radioactive atoms in a substance (in \si{\milli\gram}) $t$ years after the sample was established. Use a graphing calculator to approximate the half-life for the given values of $Q_0$ and $k$. State 
your answers correctly to 2 decimal places.
\begin{multicols}{2}
	\begin{subproblem}
		$Q_0=4$, $k=-0.0001$
		\begin{shortsolution}
			The half-life is approximately 6931.47 years.
		\end{shortsolution}
	\end{subproblem}
	\begin{subproblem}\label{exp:prob:q0k}
		$Q_0=15$, $k=-0.05$
		\begin{shortsolution}
			The half-life is approximately 13.86 years.
		\end{shortsolution}
	\end{subproblem}
\end{multicols}
\begin{subproblem}
	Does the half-life depend upon $Q_0$? Does it depend upon $k$?
	\begin{shortsolution}
		The half-life does not depend upon $Q_0$. It does depend upon $k$.
	\end{shortsolution}
\end{subproblem}
\end{problem}
\begin{problem}[Continuous growth rate]
For each of the exponential functions defined by the formulas below, find the continuous growth rate and the growth rate.
\begin{multicols}{3}
	\begin{subproblem}
		$f(t) = e^t$
		\begin{shortsolution}
			$e^t\approx 2.7182818^t$, so $f$ has continuous growth rate 1 and growth rate about 1.7182818.
		\end{shortsolution}
	\end{subproblem}
	\begin{subproblem}
		$g(t)  = e^{0.2t}$
		\begin{shortsolution}
			$e^{0.2t}\approx 1.2214028^t$,  so $f$ has continuous growth rate $\SI{20}{\percent}$ and growth rate about $\SI{22.14028}{\percent}$.
		\end{shortsolution}
	\end{subproblem}
	\begin{subproblem}
		$h(t)  = e^{-0.1t}$
		\begin{shortsolution}
			$e^{-0.1t}\approx 0.9048374^t$,  so $f$ has continuous decay rate $\SI{10}{\percent}$ and decay rate about $\SI{9.51626}{\percent}$.
		\end{shortsolution}
	\end{subproblem}
\end{multicols}
\end{problem}
%===================================
%   Author: Hughes
%   Date:   March 2012
%===================================
\begin{problem}[The function $f(x)=e^x$]
Let $f$ be the function that has formula $f(x)=e^x$. 
\begin{subproblem}
	Use your calculator to help you construct a table of values for $f(x)$
	when $x$ takes all integer values between $-3$ and $3$. Use $5$ decimal 
	places for each value of $f(x)$.
	\begin{shortsolution}
		\begin{tabular}[t]{S[table-format=1.0]S[table-format=1.5]}
			\beforeheading
			\heading{$x$} & \heading{$f(x)$} \\
			\afterheading
			-3            & 0.04979          \\\normalline
			-2            & 0.13534          \\\normalline
			-1            & 0.36788          \\\normalline
			0             & 1.00000          \\\normalline
			1             & 2.71828          \\\normalline
			2             & 7.38906          \\\normalline
			3             & 20.0855          \\\lastline
		\end{tabular}
	\end{shortsolution}
\end{subproblem}
\begin{subproblem}
	What is the domain of $f$? What is the range of $f$? Is $f$ concave up or concave down?
	\begin{shortsolution}
		$f$ has domain $(-\infty,\infty)$, and range $(0,\infty)$; $f$ is concave up.
	\end{shortsolution}
\end{subproblem}
\begin{subproblem}
	Now graph the functions $g$ and $h$ that have formulas $g(x)=e^x+4$ and $h(x)=-e^x$.
	What are the domain and range of $g$ and $h$?
	\begin{shortsolution}
		\begin{itemize}
			\item $g$ has domain $(-\infty,\infty)$ and range $(4,\infty)$ 
			\item $h$ has domain $(-\infty,\infty)$ and range $(-\infty,0)$ 
		\end{itemize}
	\end{shortsolution}
\end{subproblem}
\begin{subproblem}\label{exp:prob:matche23}
	\Cref{exp:fig:matche23} shows $y=2^x$, $y=e^x$, and $y=3^x$. Match each 
	curve to the appropriate formula.
	\begin{shortsolution}
		\begin{tikzpicture}
			\begin{axis}[
					framed,
					xmin=-5,xmax=5,
					ymin=-5,ymax=50,
					ytick={10,20,...,40},
					grid=both,
					legend pos=north west,
				]
				\addplot+[dashed]expression[domain=-10:5,samples=50]{2^x};
				\addlegendentry{$y=2^x$};
				\addplot expression[domain=-10:3.91,samples=50]{exp(x)};
				\addlegendentry{$y=e^x$};
				\addplot expression[domain=-10:3.56,samples=50]{3^x};
				\addlegendentry{$y=3^x$};
			\end{axis}
		\end{tikzpicture}
	\end{shortsolution}
\end{subproblem}
\begin{subproblem}\label{exp:prob:matchehalfthird}
	\Cref{exp:fig:matchehalfthird} shows $y=\left( \frac{1}{2} \right)^x$, $y=e^{-x}$, and $y=\left( \frac{1}{3} \right)^x$. Match each 
	curve to the appropriate formula.
	\begin{shortsolution}
		\begin{tikzpicture}
			\begin{axis}[
					framed,
					xmin=-5,xmax=5,
					ymin=-5,ymax=50,
					ytick={10,20,...,40},
					grid=both,
					legend pos=north east,
				]
				\addplot expression[domain=-5:10,samples=50]{(1/2)^x};
				\addlegendentry{$y=\left( \frac{1}{2} \right)^x$};
				\addplot+[dashed]expression[domain=-3.91:10,samples=50]{exp(-x)};
				\addlegendentry{$y=e^{-x}$};
				\addplot expression[domain=-3.56:10,samples=50]{(1/3)^x};
				\addlegendentry{$y=\left( \frac{1}{3} \right)^x$};
			\end{axis}
		\end{tikzpicture}
	\end{shortsolution}
\end{subproblem}
\begin{figure}[!htb]
	\begin{widepage}
	\begin{minipage}{.45\textwidth}
		\centering
		\begin{tikzpicture}
			\begin{axis}[
					framed,
					xmin=-5,xmax=5,
					ymin=-5,ymax=50,
					ytick={10,20,...,40},
					grid=both,
				]
				\addplot+[dashed]expression[domain=-10:5,samples=50]{2^x};
				\addplot expression[domain=-10:3.91,samples=50]{exp(x)};
				\addplot expression[domain=-10:3.56,samples=50]{3^x};
			\end{axis}
		\end{tikzpicture}
		\caption{Graph for \cref{exp:prob:matche23}}
		\label{exp:fig:matche23}
	\end{minipage}%
	\hfill
	\begin{minipage}{.45\textwidth}
		\centering
		\begin{tikzpicture}
			\begin{axis}[
					framed,
					xmin=-5,xmax=5,
					ymin=-5,ymax=50,
					ytick={10,20,...,40},
					grid=both,
				]
				\addplot expression[domain=-5:10,samples=50]{(1/2)^x};
				\addplot+[dashed]expression[domain=-3.91:10,samples=50]{exp(-x)};
				\addplot expression[domain=-3.56:10,samples=50]{(1/3)^x};
			\end{axis}
		\end{tikzpicture}
		\caption{Graph for \cref{exp:prob:matchehalfthird}}
		\label{exp:fig:matchehalfthird}
	\end{minipage}%
	\end{widepage}
\end{figure}
\end{problem}
			
\begin{problem}[Slopes]
In this problem you are going to explore the proportionality relationship between the slope and $y$-coordinate of an exponential function. 
\begin{subproblem}
	In \cref{exp:fig:eusingslopes2} we see that the graph of $y=2^x$ has a slope of about
	$0.69$ at the point $(0,1)$. That means the slope of the curve at the point 
	$(1,2)$ is about $0.69(2)$ which is roughly $1.4$.
	Lay out your ruler with a slope of 1.4 at the point $(1,2)$  and see 
	that it follows the direction of the curve at that point.
	\begin{shortsolution}
		Verify graphically using \vref{exp:fig:eusingslopese}.
	\end{shortsolution}
\end{subproblem}
\begin{subproblem}
	In \cref{exp:fig:eusingslopes3} we see that the slope at $(0,1)$ is about $1.10$.  What is the slope at the point where the $y$-coordinate is 2?  Verify this slope using your ruler.
	\begin{shortsolution}
		The slope is about $2.20$. Verify graphically using \vref{exp:fig:eusingslopes3}.
	\end{shortsolution}
\end{subproblem}
\end{problem}
\begin{problem}[Factorials]\label{exp:prob:factorials}%
There is a function called the factorial function which is symbolized by 
an exclamation point.  The domain of the function is limited to the 
non-negative integers. 
\Cref{exp:tab:factorial} shows enough values of the function for you to 
hopefully see the way the function works.
			
\begin{table}[!htb]
	\begin{widepage}
	\begin{minipage}{.25\textwidth}
		\centering
		\caption{}
		\label{exp:tab:factorial}
		\begin{tabular}{S[table-format=1.0]l}
			\beforeheading
			\heading{$k$} & \heading{$k!$}                         \\\afterheading
			0             & $1$                                    \\\normalline  
			1             & $1$                                    \\\normalline
			2             & $2\cdot 1$                             \\\normalline
			3             & $3\cdot 2\cdot 1$                      \\\normalline
			4             & $4\cdot 3\cdot 2\cdot 1$               \\\normalline
			5             & $5\cdot 4\cdot 3\cdot 2\cdot 1$        \\\normalline
			6             & $6\cdot 5\cdot 4\cdot 3\cdot 2\cdot 1$ \\\lastline
		\end{tabular}
	\end{minipage}%
	\begin{minipage}{.75\textwidth}
		\centering
		\caption{}
		\label{exp:tab:factoriale}
		\renewcommand{\arraystretch}{2}
		\begin{tabular}{ll}
			\beforeheading
			\heading{Exact value}                                                                                                     & \heading{Decimal value} \\\afterheading
			$\dd\frac{1}{0!}+\frac{1}{1!}+\frac{1}{2!}+\frac{1}{3!}+\frac{1}{4!}+\frac{1}{5!}$                                        &                         \\\normalline  
			$\dd\frac{1}{0!}+\frac{1}{1!}+\frac{1}{2!}+\frac{1}{3!}+\frac{1}{4!}+\frac{1}{5!}+\frac{1}{6!}$                           &                         \\\normalline  
			$\dd\frac{1}{0!}+\frac{1}{1!}+\frac{1}{2!}+\frac{1}{3!}+\frac{1}{4!}+\frac{1}{5!}+\frac{1}{6!}+\frac{1}{7!}$              &                         \\\normalline  
			$\dd\frac{1}{0!}+\frac{1}{1!}+\frac{1}{2!}+\frac{1}{3!}+\frac{1}{4!}+\frac{1}{5!}+\frac{1}{6!}+\frac{1}{7!}+\frac{1}{8!}$ &                         \\\lastline  
		\end{tabular}
	\end{minipage}%
	\end{widepage}
\end{table}
			
Find, correct to 5 decimal places,
the decimal values of each of the expressions in \cref{exp:tab:factoriale}.  
What do you observe?
\begin{shortsolution}
	\begin{tabular}[t]{ll}
		\beforeheading
		\heading{Exact value}                                                                                                                                         & \heading{Decimal value (to 5 d.p)} \\\afterheading
		$\dd\nicefrac{1}{0!}+\nicefrac{1}{1!}+\nicefrac{1}{2!}+\nicefrac{1}{3!}+\nicefrac{1}{4!}+\nicefrac{1}{5!}$                                                    & $2.71667$                          \\\normalline  
		$\dd\nicefrac{1}{0!}+\nicefrac{1}{1!}+\nicefrac{1}{2!}+\nicefrac{1}{3!}+\nicefrac{1}{4!}+\nicefrac{1}{5!}+\nicefrac{1}{6!}$                                   & $2.71806$                          \\\normalline  
		$\dd\nicefrac{1}{0!}+\nicefrac{1}{1!}+\nicefrac{1}{2!}+\nicefrac{1}{3!}+\nicefrac{1}{4!}+\nicefrac{1}{5!}+\nicefrac{1}{6!}+\nicefrac{1}{7!}$                  & $2.71825$                          \\\normalline  
		$\dd\nicefrac{1}{0!}+\nicefrac{1}{1!}+\nicefrac{1}{2!}+\nicefrac{1}{3!}+\nicefrac{1}{4!}+\nicefrac{1}{5!}+\nicefrac{1}{6!}+\nicefrac{1}{7!}+\nicefrac{1}{8!}$ & $2.71828$                          \\\lastline  
	\end{tabular}
											
	We observe that as we add more terms, the decimal value appears to approach $e$.
\end{shortsolution}
\end{problem}
\end{exercises}
			
\section{Comparing linear and exponential functions}
\begin{outcomes}
	\begin{outcomelist}
		\item Determine if real-world data establish a linear pattern or an exponential pattern (or neither).
		\item Find a reasonable formula for modeling social data over time.
		\item Understand the long-term differences between investing your money linearly versus exponentially.
	\end{outcomelist}
\end{outcomes}
In professions where people work with a lot of data, it is frequently necessary to determine the 
formula that best fits the data at hand.  In statistics, you learn techniques called regressions 
that allow you to determine the best model for a given set of data.  You also learn how to 
quantify the validity of the model and where and when it is appropriate to apply the model. 
			
\begin{essentialskills}
	\begin{problem}[Exponential or  linear]
	Decide if the functions defined by the following formulas are linear or exponential.
	\begin{multicols}{2}
		\begin{subproblem}
			$f(x)=2^x$
			\begin{shortsolution}
				Exponential.
			\end{shortsolution}
		\end{subproblem}
		\begin{subproblem}
			$g(x)=2x-10$
			\begin{shortsolution}
				Linear.
			\end{shortsolution}
		\end{subproblem}
		\begin{subproblem}
			$k(x)=10-5x$
			\begin{shortsolution}
				Linear.
			\end{shortsolution}
		\end{subproblem}
		\begin{subproblem}
			$l(x)=-13\cdot 6^x$
			\begin{shortsolution}
				Exponential.
			\end{shortsolution}
		\end{subproblem}
	\end{multicols}
	\end{problem}
											
	\begin{problem}[Sketching linear functions]
	Sketch a graph of each of the linear functions defined by the following 
	formulas.
	\begin{multicols}{3}
		\begin{subproblem}
			$f(x)=\frac{1}{2}x+10$
			\begin{shortsolution}
				The graph of $y=\frac{1}{2}x+10$ is shown below.
																																			
				\begin{tikzpicture}
					\begin{axis}[
							framed,
							xmin=-2,xmax=10,
							ymin=-3,ymax=20,
							xtick={1,2,...,9},
							ytick={2,4,...,18},
							grid=major,
						]
						\addplot expression[domain=-2:10]{0.5*x+10};
						\addplot[soldot]coordinates{(0,10)};
					\end{axis}
				\end{tikzpicture}
			\end{shortsolution}
		\end{subproblem}
		\begin{subproblem}
			$f(x)=15-0.8x$
			\begin{shortsolution}
				The graph of $y=15-0.8x$ is shown below.
																																			
				\begin{tikzpicture}
					\begin{axis}[
							framed,
							xmin=-2,xmax=10,
							ymin=-3,ymax=20,
							xtick={1,2,...,9},
							ytick={2,4,...,18},
							grid=major,
						]
						\addplot expression[domain=-2:10]{15-0.8*x};
						\addplot[soldot]coordinates{(0,15)};
					\end{axis}
				\end{tikzpicture}
			\end{shortsolution}
		\end{subproblem}
		\begin{subproblem}
			$f(x)=\frac{1}{3}(x-7)+4$
			\begin{shortsolution}
				The graph of $y=\frac{1}{3}(x-7)+4$ is shown below.
																																			
				\begin{tikzpicture}
					\begin{axis}[
							framed,
							xmin=-2,xmax=10,
							ymin=-2,ymax=10,
							xtick={1,2,...,9},
							ytick={1,2,...,9},
							grid=major,
						]
						\addplot expression[domain=-2:10]{1/3*(x-7)+4};
						\addplot[soldot]coordinates{(7,4)};
					\end{axis}
				\end{tikzpicture}
			\end{shortsolution}
		\end{subproblem}
	\end{multicols}
	\end{problem}
											
	\begin{problem}[Find the slope]
	Find the slope between each pair of points.
	\begin{multicols}{2}
		\begin{subproblem}
			$(12,30)$, $(5,28)$
			\begin{shortsolution}
				$m=\frac{28-30}{5-12}=\frac{2}{7}$
			\end{shortsolution}
		\end{subproblem}
		\begin{subproblem}
			$(2001,18.4)$, $(2012,15.3)$
			\begin{shortsolution}
				$m=\frac{15.3-18.4}{2012-2001}=\frac{-3.1}{11}=-\frac{31}{110}$
			\end{shortsolution}
		\end{subproblem}
		\begin{subproblem}
			$(t,118)$, $(17,y)$
			\begin{shortsolution}
				$m=\frac{y-118}{17-t}$
			\end{shortsolution}
		\end{subproblem}
	\end{multicols}
	\end{problem}
											
	\begin{problem}
	Determine if the data in \crefrange{exp:tab:lineardata}{exp:tab:neitherdata} could 
	reasonably be modeled with a linear function. 
	\begin{shortsolution}
		\Cref{exp:tab:lineardata}: this is linear data since $y$ values increase by a constant $6$ when $x$ increases by $1$.
		\Cref{exp:tab:expdata}: this data is reasonably close to linear since $y$ values increase by \emph{almost} a constant $0.2$ when $x$ increases by $1$.
		\Cref{exp:tab:neitherdata}: this data is not linear.
	\end{shortsolution}
	\end{problem}
											
	\begin{table}[!htb]
		\centering
		\null \hfill
		\begin{minipage}{0.25\textwidth}
			\centering
			\caption{} \label{exp:tab:lineardata}
			\begin{tabular}{S[table-format=4.0]S[table-format=2.0]}
				\beforeheading
				\heading{$x$} & \heading{$y$} \\\afterheading
				2005          & 40            \\\normalline
				2006          & 46            \\\normalline
				2007          & 52            \\\normalline
				2008          & 58            \\\normalline
				2009          & 64            \\\normalline
				2010          & 70            \\\lastline
			\end{tabular}
		\end{minipage}
		\hfill
		\begin{minipage}{0.25\textwidth}
			\centering
			\caption{} \label{exp:tab:expdata}
			\begin{tabular}{S[table-format=2.0]S[table-format=1.2]}
				\beforeheading
				\heading{$x$} & \heading{$y$} \\\afterheading
				42            & 1.7           \\\normalline
				43            & 1.92          \\\normalline
				44            & 2.11          \\\normalline
				45            & 2.31          \\\normalline
				46            & 2.53          \\\normalline
				47            & 2.72          \\\lastline
			\end{tabular}
		\end{minipage}
		\hfill
		\begin{minipage}{0.25\textwidth}
			\centering
			\caption{}\label{exp:tab:neitherdata}
			\begin{tabular}{S[table-format=2.0]S[table-format=3.0]}
				\beforeheading
				\heading{$x$} & \heading{$y$} \\
				\afterheading
				0             & 312           \\\normalline
				3             & 330           \\\normalline
				6             & 338           \\\normalline
				9             & 341           \\\normalline
				12            & 340           \\\normalline
				15            & 330           \\\lastline
			\end{tabular}
		\end{minipage}
		\hfill
	\end{table}
\end{essentialskills}
			
There are formal techniques for deciding the proper type of function to use to model data. It can be useful to first think about the type of function that might work best.  For example, you might be presented with data where you need to decide whether the data is best modeled by a linear function, an exponential function, or something else.  How might you make such a decision?
\begin{pccexample}
	Decide if the function that is described is more like an exponential function, a linear function, or neither.
										
	Every hour, the number of yeast cells in a vat of fermenting wine increases.  
	When there are more cells present, more cells are able to split and reproduce. The number of cells present in the vat
	is a function of the number of hours the wine has been fermenting.
	\begin{pccsolution}
		After the first hour, the cells will reproduce and the number of cells will be larger.  After the second hour, there will be \emph{even more} new cells added, since we will have more cells in the first place capable of reproducing.  This pattern continues.  The population is increasing at a faster and faster rate, so an exponential model may be appropriate.
	\end{pccsolution}
\end{pccexample}
			
\begin{pccexample}
	Decide if the function that is described is more like an exponential function, a linear function, or neither.
										
	Your house is being repainted, and every hour the workers paint an additional $\SI{400}{\square\foot}$.  
	The amount that has been painted is a function of the number of hours that have been worked.
										
	\begin{pccsolution}
		After the first hour, the workers paint $\SI{400}{\square\foot}$.  After the second hour they have painted an additional 
		$\SI{400}{\square\foot}$.  The amount of space that has been painted is increasing at a constant rate, so a linear model is appropriate.
	\end{pccsolution}
\end{pccexample}
			
Suppose that you have some actual data and that you want to determine whether it would be 
best to model the data using a linear model or an exponential model.  How might you make 
that decision? If the data is truly linear, then when the input values change at a 
steady pace, the output values also change at a steady pace.  This is illustrated 
in \cref{exp:tab:dataexample1} where the values of $x$ increase by $4$ from row to row 
and the values of $y$ decrease by $9$ from row to row.
			
\begin{table}[!htb]
	\centering
	\begin{minipage}{0.2\textwidth}
		\centering
		\caption{} \label{exp:tab:dataexample1}
		\begin{tabular}{S[table-format=2.0]S[table-format=2.0]}
			\beforeheading
			\heading{$x$} & \heading{$y$} \\\afterheading
			-7            & 30            \\\normalline
			-3            & 21            \\\normalline
			1             & 12            \\\normalline
			5             & 3             \\\normalline
			9             & -6            \\\normalline
			13            & -15           \\\lastline
		\end{tabular}
	\end{minipage}
	\hfill
	\begin{minipage}{0.2\textwidth}
		\centering
		\caption{} \label{exp:tab:dataexample2}
		\begin{tabular}{S[table-format=2.0]S[table-format=3.0]}
			\beforeheading
			\heading{$x$} & \heading{$y$} \\\afterheading
			22            & 3.5           \\\normalline
			28            & 7             \\\normalline
			34            & 14            \\\normalline
			40            & 28            \\\normalline
			46            & 56            \\\normalline
			52            & 112           \\\lastline
		\end{tabular}
	\end{minipage}
	\hfill
	\begin{minipage}{0.2\textwidth}
		\centering
		\caption{}\label{exp:tab:dataexample3}
		\begin{tabular}{S[table-format=1.0]S[table-format=1.7]}
			\beforeheading
			\heading{$x$} & \heading{$y$} \\
			\afterheading
			-8            & 72            \\\normalline
			-6            & 18            \\\normalline
			-4            & 4.5           \\\normalline
			-2            & 1.125         \\\normalline
			0             & 0.28125       \\\normalline
			2             & 0.0703125     \\\lastline
		\end{tabular}
	\end{minipage}
	\hfill 
	\begin{minipage}{0.2\textwidth}
		\centering
		\caption{} \label{exp:tab:dataexample4}
		\begin{tabular}{S[table-format=2.0]S[table-format=2.0]}
			\beforeheading
			\heading{$x$} & \heading{$y$} \\
			\afterheading
			-15           & 7             \\\normalline
			-10           & 2             \\\normalline
			-5            & -3            \\\normalline
			0             & -8            \\\normalline
			5             & -13           \\\normalline
			10            & -18           \\\lastline
		\end{tabular}
	\end{minipage}
\end{table}
If the data is truly exponential, then when the input values change at a steady pace, the 
values of the output change at a constant ratio.  This is illustrated in \cref{exp:tab:dataexample2} 
where the value of $x$ increases by 6 from row to row and the ratio of the successive values of $y$ is always 2.
			
\begin{pccexample}
	For each of the data sets in \cref{exp:tab:dataexample3,exp:tab:dataexample4}, determine if the data is 
	linear, exponential, or neither.  If the data is either linear or exponential, find a formula that models the data.
										
	In \cref{exp:tab:dataexample3}, every time the value of $x$ increases by $2$, the ratio of the successive $y$-values is $\frac{1}{4}$.  
	This is indicative of an exponential function.  
										
	If $f(x)=a\,b^x$, then from the data points $(-8,72)$ and $(-6,18)$ we get $\frac{a\,b^{-6}}{a b^{-8}}=\frac{18}{72}$.  
	So $b^2=\frac{1}{4}$ which means that the base of the function $b$ must be $\frac{1}{2}$ (since the base cannot be negative).  
	Using the data point $(0,0.28125)$ along with our newly discovered base we find $a(0.5)^0=0.28125$, so $a=0.28125$.  
										
	In conclusion, the data in \cref{exp:tab:dataexample3} is modeled by the exponential function $f$ where $f(x)=0.28125\left(\frac12\right)^x$.
										
	In \cref{exp:tab:dataexample4}, every time the value of $x$ increase by 5, the value of $y$ decreases by 5.  This is 
	the behavior of a linear function with a slope of $-1$.  Using either the slope-intercept form of a linear 
	equation or the point-slope form of a linear equation, we can deduce that the data is modeled by the linear 
	function $g$ where $g(x)=-x-8$.
\end{pccexample}
			
In nature, data sets are almost never exactly linear nor exactly exponential.  When dealing with data one 
might need to decide which model \emph{best} fits the data, linear or exponential.
			
\begin{pccexample}
	Is the data in \cref{exp:tab:dataexample5} better modeled with a linear function or an exponential function?
	\begin{table}[!htb]
		\centering
		\captionof{table}{} \label{exp:tab:dataexample5}
		\begin{tabular}{S[table-format=2.0]S[table-format=2.2]}
			\beforeheading
			\heading{$x$} & \heading{$f(x)$} \\\afterheading
			10            & 8                \\\normalline
			11            & 12.01            \\\normalline
			12            & 16.02            \\\normalline
			13            & 20               \\\normalline
			14            & 24               \\\lastline
		\end{tabular}
	\end{table}
	\begin{pccsolution}
		Let's compute the successive differences and the successive ratios:
		\begin{align*}
			12.01-8     & =4.01 & 12.01/8     & \approx1.50 \\
			16.02-12.01 & =4.01 & 16.02/12.01 & \approx1.33 \\
			20-16.01    & =3.98 & 10/16.01    & \approx1.24 \\
			24-20       & =4    & 24/20       & =1.2        
		\end{align*}
		The differences are not constant, but they are all fairly close to having the constant value of 4.  
		The ratios however are not close to being constant.  It might be appropriate to model this 
		data with a linear function, but it would not be appropriate to model it with an exponential function.
	\end{pccsolution}
\end{pccexample}
			
\begin{doyouunderstand}
	\begin{problem}
	Determine if each of the data sets in \crefrange{exp:tab:dataexample55}{exp:tab:dataexample8} 
	suggest an exponential relationship, a linear relationship, or neither. 
	\begin{shortsolution}
		\Cref{exp:tab:dataexample55}: since successive differences are all equal to $2$, a linear model would be appropriate. 
		Successive ratios decrease steadily, so an exponential model would not be appropriate.
		\Cref{exp:tab:dataexample6}: since successive ratios are all equal to $3$, an exponential model would be appropriate.  
		Successive differences increase, so a linear model would not be appropriate.
		\Cref{exp:tab:dataexample7}: since successive differences are all quite close to $2.3$, a linear model might be appropriate.  
		Successive ratios drop steadily from about $1.13$ to $1.09$, so an exponential model would not be appropriate.
		\Cref{exp:tab:dataexample8}: since successive ratios are all quite close to $0.92$, an exponential model might be appropriate.  
		Successive differences rise steadily from $-44$ to $-31$, so a linear model would not be appropriate.
	\end{shortsolution}
	\end{problem}
											
	\begin{table}[!htb]
		\centering
		\null\hfill
		\begin{minipage}{0.20\textwidth}
			\centering
			\caption{}\label{exp:tab:dataexample55}
			\begin{tabular}{S[table-format=1.0]S[table-format=2.0]}
				\beforeheading
				\heading{$x$} & \heading{$y$} \\\afterheading
				1             & 2             \\\normalline
				2             & 4             \\\normalline
				3             & 6             \\\normalline
				4             & 8             \\\normalline
				5             & 10            \\\normalline
				6             & 12            \\\lastline
			\end{tabular}
		\end{minipage}
		\hfill
		\centering
		\begin{minipage}{0.20\textwidth}
			\centering
			\caption{} \label{exp:tab:dataexample6}
			\begin{tabular}{S[table-format=4.0]S[table-format=3.0]}
				\beforeheading
				\heading{$x$} & \heading{$y$} \\\afterheading
				2005          & 1             \\\normalline
				2006          & 3             \\\normalline
				2007          & 9             \\\normalline
				2008          & 27            \\\normalline
				2009          & 81            \\\normalline
				2010          & 243           \\\lastline
			\end{tabular}
		\end{minipage}
		\hfill
		\begin{minipage}{0.20\textwidth}
			\centering
			\caption{} \label{exp:tab:dataexample7}
			\begin{tabular}{S[table-format=4.0]S[table-format=2.3]}
				\beforeheading
				\heading{$x$} & \heading{$y$} \\\afterheading
				1950          & 17.3          \\\normalline
				1960          & 19.6          \\\normalline
				1970          & 21.9          \\\normalline
				1980          & 24.1          \\\normalline
				1990          & 26.4          \\\normalline
				2000          & 28.7          \\\lastline
			\end{tabular}
		\end{minipage}
		\hfill
		\begin{minipage}{0.20\textwidth}
			\centering
			\caption{}\label{exp:tab:dataexample8}
			\begin{tabular}{S[table-format=2.0]S[table-format=3.0]}
				\beforeheading
				\heading{$x$} & \heading{$y$} \\
				\afterheading
				0             & 546           \\\normalline
				4             & 502           \\\normalline
				8             & 462           \\\normalline
				12            & 425           \\\normalline
				16            & 391           \\\normalline
				20            & 360           \\\lastline
			\end{tabular}
		\end{minipage}
		\hfill\null
	\end{table}
\end{doyouunderstand}
%===================================
%   Author: Jordan
%   Date:   Nov 2011
%===================================
Our examples in this section so far have not had any context; they have just been tables of numbers.  As a student of the natural and social sciences, you will encounter data sets that come from interesting sources.  Understanding how to model these data sets can help you understand that subject better.
			
\begin{pccexample}
	A study of reproductive health care found the data in \cref{exp:ex:Csection} concerning the percentage of births in the U.S.A.\ that were delivered via a C\ae{}sarian section.
										
	Can we model this data with a linear or exponential function?
	\begin{pccsolution}
		It is always a good idea to plot data like this.  Some patterns might be quickly evident from a graph that are not so quickly evident numerically.
																	
		\begin{figure}[!htb]
			\hfill
			\begin{minipage}[b]{.3\textwidth}
				\centering
				\captionof{table}{US C\ae{}sarian Section Data}
				\begin{tabular}{S[table-format=4.0]S[table-format=2.1]}
					\beforeheading
					\heading{Year} & \heading{C-sections (\%)} \\
					\afterheading
					1996           & 21.0                      \\\normalline
					1998           & 22.6                      \\\normalline
					2000           & 24.2                      \\\normalline
					2002           & 26.0                      \\\normalline
					2004           & 27.9                      \\\normalline
					2006           & 30.0                      \\\normalline
					2008           & 32.2                      \\\lastline
				\end{tabular}
				\label{exp:ex:Csection}
			\end{minipage}
			\hfill
			\begin{minipage}{.3\textwidth}
				\begin{tikzpicture}[trim axis left]
					\begin{axis}[
							xmin=-2,xmax=14,
							ymin=-5,ymax=40,
							ytick={5,10,...,35},
							xtick={6,12},
							xlabel={$t$},
							xticklabels={2002,2008},
						]
						\addplot[soldot]coordinates{(0,21)(2,22.6)(4,24.2) (6,26.0) (8,27.9) (10,30) (12,32.2)};
					\end{axis}
				\end{tikzpicture}
				\caption{US C\ae{}sarian section data}
				\label{exp:fig:Csections}
			\end{minipage}%
			\hfill\mbox{}
		\end{figure}
																	
		\Cref{exp:fig:Csections} shows us that there is a clear upward trend in C\ae{}sarian section deliveries.  Is the trend linear, exponential, or neither?  
																	
		You might be able to see a slight upward bend to the plot, suggesting an exponential growth pattern.  Then again, your eyes might see these points as lying in a relatively straight line, suggesting a linear growth pattern.   Let's look at successive ratios and differences in \cref{exp:ex:CsectionSuccessive}.
																	
		In \cref{exp:ex:CsectionSuccessive}, we see successive differences that become larger and larger.  This tells us that a linear model would not be appropriate for the C\ae{}sarian section data.  On the other hand, the successive ratios all bounce around close to $1.074$.  This suggests that an exponential model would be appropriate.
		\begin{table}[!htb]
			\centering
			\captionof{table}{US C\ae{}sarian Section Data}
			\label{exp:ex:CsectionSuccessive}
			\begin{tabular}{S[table-format=1.1]l}
				\beforeheading
				\heading{Differences} & \heading{Ratios} \\
				\afterheading
				1.6                   & $1.076\ldots$    \\\normalline
				1.6                   & $1.071\ldots$    \\\normalline
				1.8                   & $1.074\ldots$    \\\normalline
				1.9                   & $1.073\ldots$    \\\normalline
				2.1                   & $1.075\ldots$    \\\normalline
				2.2                   & $1.073\ldots$    \\\lastline
			\end{tabular}
		\end{table}
	\end{pccsolution}
										
	If we have decided that an exponential model is appropriate, can we explicitly write down a model?
										
	\begin{pccsolution}
		We have many data points, but we only require two in order to determine a formula for an exponential function.  And it's likely that different choices of points will lead to slightly different models.  In an intermediate statistics course, students learn how to deal with this issue.  For now, we will use the following rule of thumb: use data points that are a little inward from the edges.  In the current example, we will use the data from 1998 and 2006.
																	
		Since we typically wish to associate $t=0$ to a year where the data was relevant, we will identify $t=0$ with the year 2000. Now if $f(t)=a\,b^t$, then the data from 1998 and 2006 tell us that 
		\[
			\begin{cases}
				22.6   =  a\,b^{-2} 
				30.0  =  a\,b^6     
			\end{cases}
		\]
		We can eliminate $a$ by equating the quotients formed by the two sides of the equations. 
		\begin{equation*}
			\frac{30.0}{22.6} = \frac{a\,b^{6}}{a\,b^{-2}}\qquad \Longrightarrow\qquad b=\left(\frac{30.0}{22.6}\right)^{\nicefrac18}\approx1.036
		\end{equation*}
		The approximation of $b$ by $1.036$ is particularly valid in an application such as this, where we know that other choices of data points would have given different values of $b$ anyway.
																	
		Solving for $a$ in the equation for 2006:
		\begin{align*}
			30.0 = a\left( \frac{30.0}{22.6} \right)^{\nicefrac{6}{8}}\qquad \Longrightarrow\qquad a=\frac{30}{\left( \dfrac{30.0}{22.6} \right)^{\nicefrac{3}{4}}}\approx 24.26 
		\end{align*}
		Therefore a model for the C\ae{}sarian section data is 
		\begin{align*}
			f(t) & = 30^{\nicefrac{1}{4}}22.6^{\nicefrac{3}{4}}\left( \frac{30.0}{22.6} \right)^{\nicefrac{t}{8}}                  \\
			     & = 30^{\nicefrac{1}{4}}22.6^{\nicefrac{3}{4}}\left(  \left( \frac{30.0}{22.6} \right)^{\nicefrac{1}{8}}\right)^t \\
			     & \approx 24.26(1.036)^t                                                                                          
		\end{align*}
		and it is acceptable to say that the percentage of C-section births is growing exponentially in the U.S.A.  For completeness, we can examine a graph of this model overlaying the data in \cref{exp:fig:CsectionsModel}.
	\end{pccsolution}
\end{pccexample}
			
\begin{figure}[!htb]
	\centering
	\begin{tikzpicture}
		\begin{axis}[
				xmin=-2,xmax=14,
				ymin=-5,ymax=40,
				ytick={5,10,...,35},
				xtick={6,12},
				xlabel={$t$},
				xticklabels={2002,2008},
				width=.3\textwidth,
			]
			\addplot expression[domain=-2:14]{24.26*(1.036^(x-4))};  
			\addplot[soldot]coordinates{(0,21)(2,22.6)(4,24.2) (6,26.0) (8,27.9) (10,30) (12,32.2)};
		\end{axis}
	\end{tikzpicture}
	\caption{US C\ae{}sarian section model}
	\label{exp:fig:CsectionsModel}
\end{figure}
			
We have so far modeled data using exponential and linear functions; this
has helped us determine how the two classes of function behave differently 
in the short term. You may wonder how the two classes of function behave differently in the long term.
			
%===================================
%   Author: Hughes
%   Date:   Dec 2011
%===================================
\begin{pccexample}\label{exp:ex:complinexp}
	Consider the functions $f$ and $g$ that have formulas
	\[
		f(x)=4x+1, \qquad g(x)=4^x
	\]
	Describe the behavior of both functions as $x\to\infty$ and as $x\to -\infty$.
	\begin{pccsolution}
		The functions $f$ and $g$ are graphed in \cref{exp:fig:comparelinexp}.
		Note that both functions grow without bound as $x\to\infty$. Another 
		way to express this is to say
		\[
			f(x)\to\infty \qquad\mathrm{and}\qquad g(x)\to\infty
		\]
		as $x\to\infty$.
																	
		However, even though both functions grow without bound, the exponential 
		function $g$ does so at a \emph{much faster rate}.
																	
		The behavior of the functions as $x\to-\infty$ is quite different.
		Note that $g$ has a horizontal asymptote ($y=0$) and that $f$ does not; in 
		fact $f(x)\to-\infty$ as $x\to-\infty$.
	\end{pccsolution}
\end{pccexample}
			
\begin{figure}[!htb]
	\centering
	\hfill
	\begin{minipage}{0.3\textwidth}
		\begin{tikzpicture}
			\begin{axis}[
					framed,
					xmin=-5,xmax=5,
					ymin=-40,ymax=100,
					xtick={-4,...,4},
					ytick={-20,0,...,80},
					minor ytick={-30,-20,...,90},
					grid=both,
					legend pos=north west,
				]
				\addplot expression[domain=-5:5]{4*x+1};
				\addplot expression[domain=-5:3.3219]{4^x};
				\legend{$f$, $g$};
			\end{axis}
		\end{tikzpicture}
		\caption{}
		\label{exp:fig:comparelinexp}
	\end{minipage}
	\hfill
	\begin{minipage}{0.3\textwidth}
		\begin{tikzpicture}
			\begin{axis}[
					framed,
					xmin=-5,xmax=5,
					ymin=-100,ymax=40,
					xtick={-4,...,4},
					ytick={-80,-60,...,20},
					minor ytick={-90,-70,...,30},
					grid=both,
					legend pos=south west,
				]
				\addplot expression[domain=-5:2]{-10^x};
				\addplot expression[domain=-5:5]{-6*x};
				\legend{$m$,$n$};
			\end{axis}
		\end{tikzpicture}
		\caption{}
		\label{exp:fig:comparelinexpmn}
	\end{minipage}
	\hfill\null
\end{figure}
			
%===================================
%   Author: Hughes
%   Date:   Dec 2011
%===================================
\begin{pccexample}
	Now consider the functions $m$ and $n$ that have formulas
	\[
		m(x) = -10^x, \qquad n(x)=-6x
	\]
	Describe the behavior of both functions as $x\to\infty$ and as $x\to-\infty$.
	\begin{pccsolution}
		The functions $m$ and $n$ are both graphed in \cref{exp:fig:comparelinexpmn}. 
		Note that both functions decrease without bound as $x\to\infty$. However, $m$ does so 
		at \emph{a much faster rate}.
																	
		As in \cref{exp:ex:complinexp}, the behavior of $m$ and $n$ as $x\to-\infty$
		is different. Notice that $m$ has a horizontal asymptote ($y=0$) and that 
		$n$ does not.  In this case $n(x)\to \infty$ as $x\to-\infty$.
	\end{pccsolution}
\end{pccexample}
			
\investigation*{}
%===================================
%   Author: Vega
%   Date:   March 2011
%===================================
\begin{problem}[Cell phones]
A cell phone company is 
studying the number of cell phone subscriptions in the U.S.A. In 2003, there were 
158 million cell phone users. In 2006, there were 233 million cell phone users.\footnote{CTIA - The Wireless Association}
\begin{subproblem}
	Write two ordered pairs suggested by the information above.  Then find the slope between the two points and {\em interpret} its meaning including units.  (Write a complete sentence that explains the meaning of the slope that you have calculated, without using the word `slope'.)
	\begin{shortsolution}
		$(3,158)$ and $(6,233)$.
		\begin{align*}
			m & =  \frac{233-158}{6-3} \\
			  & =  \frac{75}{3}        \\
			  & = 25                   
		\end{align*}
		The units in the numerator are millions and the units in the denominator are years.  So the slope of 25 has meaning as a rate of $\SI{25}{\million\per\year}$.  This means there are 25 million new cell phone subscriptions each year since 2000.
	\end{shortsolution}
\end{subproblem}
\begin{subproblem} \label{exp:prob:phonelinear}
	Assuming that the relationship between years and cell phone users is linear, find 
	a formula that determines the number of cell phone users as a function of the number of years since 2000.
	\begin{shortsolution}
		Let $f(t)=mt+b$, be the number of cell phone users in millions at time $t$ in years since 2000. 
		We have already found $m$; we need to find $b$. 
		\begin{align*}
			158 & =  25(3)+b \\
			b   & =  83      
		\end{align*}
		Therefore, $f(t)=25t+83$.
	\end{shortsolution}
\end{subproblem}
\begin{subproblem} \label{exp:prob:phoneexp}
	Assuming that the relationship between years and cell phone users is exponential, find 
	a formula that determines the number of cell phone users as a function of the number of years since 2000.
	\begin{shortsolution}
		$f(t)=\frac{(158)^2}{233}\left(\frac{233}{158}\right)^{\nicefrac{t}{3}}\approx107.14(1.13824)^t$.
	\end{shortsolution}
\end{subproblem}
\begin{subproblem}
	According to your models in \cref{exp:prob:phonelinear,exp:prob:phoneexp}, how many cell phone users were there in 2008?
	In 2008 there were approximately 263 million cell phone users in the United States. Which of the two models 
	does a better job of predicting this number?
	\begin{shortsolution}
		The linear model predicts that there were about 283 million cell phone users in 2008.  
		The exponential model predicts that there were about 302 million cell phone users in 2008. The linear model 
		does a better job of predicting the number of cell phone users in 2008.
	\end{shortsolution}
\end{subproblem}
\end{problem}
			
%===================================
%   Author: Hughes
%   Date:   May 2011
%===================================
\begin{problem}[Wind power]
The wind energy production capacity (in \si{\mega\watt}) for the world\footnote{\href{http://www.thewindpower.net/statistics\_world.php}{http://www.thewindpower.net/statistics\_world.php}} is shown in 
\vref{exp:fig:windpower}. 
			
\begin{figure}[!htb]
	\centering
	\begin{tikzpicture}
		\begin{axis}[
				x axis line style={->},
				y axis line style={->},
				xlabel={$t$},
				ylabel={Wind Power Capacity By Year (MW)},
				width=.95\textwidth,
				height=.5\textwidth,
				xmin=-1,xmax=17,
				xtick={0,2,...,18},
				xticklabels={1994,1996,...,2010},
				ymin=-10,ymax=220,
				ytick={0,20,...,200},
				yticklabels={0,20000,40000,60000,80000,100000,120000,140000,160000,180000,200000},
			]
			\addplot[pccbar,bar width=15] plot coordinates
			{
				(1,4.800)
				(2,6.100)
				(3,7.480)
				(4,9.667)
				(5,13.701)
				(6,18.040)
				(7,24.319)
				(8,31.181)
				(9,41.343)
				(10,49.463)
				(11,59.137)
				(12,74.178)
				(13,93.952)
				(14,121.328)
				(15,158.008)
			(16,194.154)};
		\end{axis}
	\end{tikzpicture}
	\caption{Wind power.}
	\label{exp:fig:windpower}
\end{figure}
			
\begin{subproblem}
	Which would be more appropriate to model this data, an exponential function or a linear function?
	\begin{shortsolution}
		An exponential function seems to be more appropriate.  First of all the graph of the data has the basic shape of an exponential function.  More by the numbers, we can examine the approximate successive ratios.  It's difficult to read the chart on its left side with any relative precision.  Starting from the right and using rough approximate readings from the chart:
		\setlength\jot{10pt}	%manual vertical spacing
		\begin{align*}
			\frac{195000}{160000} & \approx 1.22 \\
			\frac{160000}{125000} & \approx 1.28 \\
			\frac{125000}{95000}  & \approx 1.32 \\
			\frac{95000}{75000}   & \approx 1.27 \\
			\frac{75000}{60000}   & \approx 1.25 \\
			\frac{60000}{50000}   & \approx 1.20 \\
		\end{align*}
		The successive ratios are all fairly close to each other, so an exponential model would be appropriate.    
	\end{shortsolution}
\end{subproblem}
\begin{subproblem}
	Let $W(t)$ represent the wind energy production (in \si{\mega\watt}) at time $t$ in 
	years since 1995. 
	In 1995 the wind power capacity was \SI{4800}{\mega\watt} and in 2010 the wind power capacity was \SI{194154}{\mega\watt}.
	Using two ordered pairs, $(0,4800)$ and $(15,194154)$, and assuming that an exponential model is appropriate, we can 
	show that a formula that approximates $W$ is
	\begin{align*}
		W(t) & =      4800\left( \frac{194154}{4800} \right)^{\nicefrac{t}{15}}                  \\
		     & =      4800\left( \left( \frac{194154}{4800} \right)^{\nicefrac{1}{15}} \right)^t \\
		     & \approx 4800(1.279756)^t                                                          
	\end{align*}
	Using a table of values or a graph, find when the world's wind power capacity will be \SI{250000}{\mega\watt}.
	\begin{shortsolution}
		Using a table of values, we find that $W(16)\approx 248470$, and $W(17)\approx 317980$. We conclude that, 
		according to the model, the world's wind power capacity will be \SI{250000}{\mega\watt} in early 2012. 
	\end{shortsolution}
\end{subproblem}
\end{problem}
			
%===================================
%   Author: Jordan
%   Date:   Dec 2011
%===================================
\begin{problem}[Population of Africa]
\Cref{exp:tab:AfricaPopulation} gives the population of Africa for each year form 1999 to 2009.\footnote{\href{http://www.earth-policy.org/books/wote}{http://www.earth-policy.org/books/wote}}
			
\begin{table}[!htb]
	\centering
	\captionof{table}{Human Population of Africa}
	\label{exp:tab:AfricaPopulation}
	\begin{tabular}{S[table-format=4.0]S[table-format=3.1]}
		\beforeheading
		\heading{Year} & \heading{Millions} \\
		\afterheading
		1999           & 800.2              \\\normalline
		2000           & 819.5              \\\normalline
		2001           & 839.0              \\\normalline
		2002           & 858.9              \\\normalline
		2003           & 879.2              \\\normalline
		2004           & 899.9              \\\normalline
		2005           & 921.1              \\\normalline
		2006           & 942.7              \\\normalline
		2007           & 964.7              \\\normalline
		2008           & 987.1              \\\normalline
		2009           & 1009.9             \\\lastline
	\end{tabular}
\end{table}
			
\begin{subproblem}
	Give a good reason why it would be appropriate to model this data with an exponential function.
	\begin{shortsolution}
		Answers will vary.  By the numbers, the successive ratios 
		are all between $1.0231$ and $1.0242$.  Since they are so close to each other, 
		an exponential model is appropriate.  Alternatively, a plot of the data reveals a concave up trend.
	\end{shortsolution}
\end{subproblem}
\begin{subproblem}\label{exp:prob:Africamodel}
	Find an explicit exponential model for the population of Africa as a function of time.  
	That is, find an exponential function $P$ where $P(t)=a\,b^t$ such that $P(t)$ approximates the 
	population of Africa at time $t$.  For convenience, take $t=0$ to mean the year 2000, with time measured in years.
	\begin{shortsolution}
		Answers will vary.  One solution uses the exponential curve that passes through $(2001,839.0)$ and $(2007,964.7)$: $P(t)\approx819.7(1.023530)^t$.  However it is valid to use other pairs of points and find slightly different functions.
																			
		An alternative solution would use the initial population of $819.5$ and the average of the successive ratios: $1.023600$.  This would give $P(t)\approx819.5(1.023600)^t$
	\end{shortsolution}
\end{subproblem}
\begin{subproblem}
	Use the model that you found in \cref{exp:prob:Africamodel} to estimate the population of Africa in the year 2020.
	\begin{shortsolution}
		The model suggests that in 2020 the population of Africa will be about $1310$ million, or $1.31$ billion.
	\end{shortsolution}
\end{subproblem}
\begin{subproblem}
	Use a graph of your model to estimate when the population of Africa might reach $1.5$ billion people.  ($1.5$ billion is $1500$ million.)
	\begin{shortsolution}
		According to the model, it appears that the population of Africa will reach $1.5$ billion in the year 2026.
																			
		\begin{tikzpicture}
			\begin{axis}[
					xmin=-5,xmax=35,
					ymin=-200,ymax=2000,
					xlabel={$t$},
					xtick={5,10,...,30},
					ytick={500,1000,1500},
					xticklabels={2005,{},{},{},{},{}},
				]
				\addplot expression[domain=-5:35]{819.7*1.0236^x};
				\draw[dashed,->](axis cs:0,1500)--(axis cs:25.98,1500)--(axis cs:25.98,0);
			\end{axis}
		\end{tikzpicture}
	\end{shortsolution}
\end{subproblem}
\begin{subproblem}
	How good is your model?  Use your model to add a third column to  \cref{exp:tab:AfricaPopulation} that displays the model's predicted population.  Discuss the accuracy of the model.
	\begin{shortsolution}
		Answers will vary as models vary.  Using the model $P(t)=819.7(1.023530)^t$:\\
																			
		\begin{tabular}{S[table-format=4.0]S[table-format=4.1]S[table-format=4.1]}
			\beforeheading
			\heading{Year} & \heading{Actual} & \heading{Model} \\
			\afterheading
			1999           & 800.2            & 800.9           \\\normalline
			2000           & 819.5            & 819.7           \\\normalline
			2001           & 839.0            & 839.0           \\\normalline
			2002           & 858.9            & 858.8           \\\normalline
			2003           & 879.2            & 879.0           \\\normalline
			2004           & 899.9            & 900.0           \\\normalline
			2005           & 921.1            & 920.8           \\\normalline
			2006           & 942.7            & 942.5           \\\normalline
			2007           & 964.7            & 964.7           \\\normalline
			2008           & 987.1            & 987.4           \\\normalline
			2009           & 1009.9           & 1010.6          \\\lastline
		\end{tabular}
																			
		The model seems quite accurate.
	\end{shortsolution}
\end{subproblem}
\end{problem}
			
%===================================
%   Author: Hughes
%   Date:   Dec 2011
%===================================
\begin{problem}[Tortoise and the Hare]
Aesop's fable of the Tortoise and the Hare depicts an unlikely race between
the two animals. The Hare, being known as a quick and lively animal, brims
with confidence. The Tortoise, who is known as a slow and more methodical
creature is a little nervous. We are going to model a version of this fable.
			
The Tortoise and the Hare are going to race over \SI{1000}{\meter}. The Hare boasts
that he can run at $\SI{20}{\meter\per\second}$ for as long as he likes. The 
Tortoise doesn't know how fast he can run, but he says that his distance from the 
their starting line obeys the rule $y=2^t$, where $t$ is the time (in seconds) 
since they begin the race; he does ask the Hare if he can have a \SI{1}{\meter} head start. 
The Hare laughs at the Tortoise and says, `Fine by me!'
\begin{subproblem}
	Let $H(t)$ represent the Hare's distance from the starting line (in meters) $t$ seconds 
	after the race begins. Write a formula for $H$.
	\begin{shortsolution}
		$H(t)=20t$
	\end{shortsolution}
\end{subproblem}
\begin{subproblem}
	Let $T(t)$ represent the Tortoise's distance from the starting line (in meters) $t$ seconds
	after the race begins. Write a formula for $T$.
	\begin{shortsolution}
		$T(t)=2^t$
	\end{shortsolution}
\end{subproblem}
\begin{subproblem}
	Graph $H$ and $T$ on the same axis for $t$ in $[0,10]$. Who wins the race?
	\begin{shortsolution}
		The graphs of $H$ and $T$  are shown below.
																			
		\begin{tikzpicture}
			\begin{axis}[
					framed,
					xmin=-3,xmax=10,
					ymin=-200,ymax=1100,
					xtick={-2,-1,...,9},
					ytick={100,200,...,1000},
					xlabel={$t$},
					grid=major,
					legend pos=north east,
				]
				\addplot+[->]expression[domain=0:10]{2^x};
				\addlegendentry{$y=T(x)$};
				\addplot+[->]expression[domain=0:10]{20*x};
				\addlegendentry{$y=H(x)$};
			\end{axis}
		\end{tikzpicture}\\
		The Tortoise wins the race.
	\end{shortsolution}
\end{subproblem}
Unbeknown to both competitors, the Hare's older sister has been watching the race.
She approaches the Tortoise and says that if she had been racing, she would easily have won
because she can run at $\SI{40}{\meter\per\second}$ for as long as she likes. 
\begin{subproblem}
	Let $S(t)$ represent the Hare's sister's distance from the starting line (in meters) $t$ seconds 
	after the race begins. Write a formula for $S$.
	\begin{shortsolution}
		$S(t)=40t$
	\end{shortsolution}
\end{subproblem}
\begin{subproblem}
	Let's assume that all 3 animals race together. Graph $H$, $T$, and $S$ on the same axis for $t$ in $[0,10]$. Who wins the race?
	\begin{shortsolution}
		The graphs of $H$, $T$, and $S$  are shown below.
																			
		\begin{tikzpicture}
			\begin{axis}[
					framed,
					xmin=-3,xmax=10,
					ymin=-200,ymax=1100,
					xtick={-2,-1,...,9},
					ytick={100,200,...,1000},
					xlabel={$t$},
					grid=major,
					legend pos=north east,
				]
				\addplot+[->]expression[domain=0:10]{2^x};
				\addlegendentry{$y=T(x)$};
				\addplot+[->]expression[domain=0:10]{20*x};
				\addlegendentry{$y=H(x)$};
				\addplot+[->]expression[domain=0:10]{40*x};
				\addlegendentry{$y=S(x)$};
			\end{axis}
		\end{tikzpicture}\\
		The Tortoise wins the race.
	\end{shortsolution}
\end{subproblem}
\end{problem}
			
%=================================================================
%
%				Exercises
%
%=================================================================
\begin{exercises}
\begin{problem}[Linear or exponential?]
Decide if the functions defined by the following formulas are linear or exponential.
\begin{multicols}{4}
	\begin{subproblem}
		$B(y)=-10\cdot 17^y$
		\begin{shortsolution}
			Exponential.
		\end{shortsolution}
	\end{subproblem}
	\begin{subproblem}
		$A(t)=4t+10$
		\begin{shortsolution}
			Linear.
		\end{shortsolution}
	\end{subproblem}
	\begin{subproblem}
		$C(\alpha) = 5\alpha-\pi$
		\begin{shortsolution}
			Linear.
		\end{shortsolution}
	\end{subproblem}
	\begin{subproblem}
		$D(z)=\pi^z$
		\begin{shortsolution}
			Exponential.
		\end{shortsolution}
	\end{subproblem}
\end{multicols}
\end{problem}
			
%===================================
%   Author: Kouzes
%   Date:   Apr 2011
%===================================
\begin{problem}[Linear, exponential, or neither?]
Decide if the function that is described is more like an exponential function, a linear function, or neither.
\begin{subproblem}
	The number of plants that germinate is a function of the number of seeds sown.
	For a particular crop of peas, exactly \SI{70}{\percent} of the seeds sown germinate. 
	\begin{shortsolution}
		Linear. 
	\end{shortsolution}
	\begin{longsolution}
		Linear. The function can be modeled by $g(x)=0.7x$, where $g(x)$ is the number 
		of plants that germinate and $x$ is the number of seeds sown. An increase in the $x$ 
		value results in an increase in the $g(x)$ value by \SI{70}{\percent} of the $x$ value. 
	\end{longsolution}
\end{subproblem}
\begin{subproblem}
	The number of tulip bulbs in a garden is a function of the number of years since first planting.
	Every year, each of the tulip bulbs divides into two. 
	\begin{shortsolution}
		Exponential. 
	\end{shortsolution}
	\begin{longsolution}
		Exponential. The function can be modeled by $f(x)=2^x$, where $f(x)$ is the 
		number of bulbs $x$ years since first planting. Every increase in the $x$ value 
		results in the doubling of the value of $f(x)$.
	\end{longsolution}
\end{subproblem}
\begin{subproblem}
	The amount of money in Ross' bank account is a function of the number of years since the account was opened. 	
	Every year the amount of money in Ross' account increases by \$5000. 
	\begin{shortsolution}
		Linear. 
	\end{shortsolution}
	\begin{longsolution}
		Linear. The function can be modeled by $g(x)=5000x$, where $g(x)$ is the amount of money 
		in the bank account and $x$ is the number of years since I opened the account. Every 
		increase in the $x$ value results in an increase in the value $g(x)$ by 5000.
	\end{longsolution}
\end{subproblem}
\begin{subproblem}
	The amount of money in Serena's account is a function of the number 
	of years since she opened the account.
	Every year the amount of money Serena's account increases by \SI{2}{\percent} of the amount that was in the 
	account the year before. 
	\begin{shortsolution}
		Exponential.
	\end{shortsolution}
	\begin{longsolution}
		Exponential. The function can be modeled by $f(x)=1.02^x$, where $f(x)$ is the amount of money 
		in the bank account and $x$ is the number of years since I opened the account. Every increase 
		in the $x$ value results in \SI{102}{\percent} of the value of $f(x)$.
	\end{longsolution}
\end{subproblem}
\begin{subproblem}
	The number of pages you still have to read in a book is a function of the page number you're on.
	\begin{shortsolution}
		Linear.
	\end{shortsolution}
	\begin{longsolution}
		The function can be modeled by $g(x)=b-x$ where $b$ is the total number of pages in the book, 
		$x$ is the page number that you're on and $g(x)$ is the number of pages left to read. For every 
		one page you read, you have one page fewer still to read.
	\end{longsolution}
\end{subproblem}
\end{problem}
			
%===================================
%   Author: Barkin
%   Date:   April 2011
%===================================
\begin{problem}[Linear or exponential?]
Consider the data sets in \crefrange{exp:tab:findformula1}{exp:tab:dataproblem3}.
\begin{subproblem}
	For each data set in \crefrange{exp:tab:findformula1}{exp:tab:findformula4}, state whether a linear or 
	exponential function (or neither) would better model the data.  If a linear or exponential model is appropriate,  
	find an exact formula for the model.
	\begin{shortsolution}
		$f(x)=\frac{3}{2}x-3$, $g(x)=16\left( \frac{1}{4} \right)^x$, $h(x)=3\left( \frac{3}{2} \right)^x$, $k(x)=16-4x$.
	\end{shortsolution}
\end{subproblem}
			
\begin{table}[!htb]
	\begin{widepage}
	\begin{minipage}{0.2\textwidth}
		\centering
		\caption{}
		\label{exp:tab:findformula1}
		\begin{tabular}{S[table-format=1.0]S[table-format=1.1]}
			\beforeheading
			\heading{$x$} & \heading{$f(x)$} \\
			\afterheading
			-1            & -4.5             \\\normalline
			0             & -3.0             \\\normalline
			1             & -1.5             \\\normalline
			2             & 0.0              \\\normalline
			3             & 1.5              \\\lastline
		\end{tabular}
	\end{minipage}
	\hfill
	\begin{minipage}{0.2\textwidth}
		\centering
		\caption{}
		\label{exp:tab:findformula2}
		\begin{tabular}{S[table-format=1.0]S[table-format=2.0]}
			\beforeheading
			\heading{$x$} & \heading{$g(x)$} \\
			\afterheading
			-1            & 64               \\\normalline
			0             & 16               \\\normalline
			1             & 4                \\\normalline
			2             & 1                \\\normalline
			3             & \nicefrac{1}{4}  \\\lastline
		\end{tabular}
	\end{minipage}
	\hfill
	\begin{minipage}{0.2\textwidth}
		\centering
		\caption{}
		\label{exp:tab:findformula3}
		\begin{tabular}{S[table-format=1.0]S[table-format=2.3]}
			\beforeheading
			\heading{x} & \heading{h(x)} \\
			\afterheading
			-1          & 2.000          \\\normalline
			0           & 3.000          \\\normalline
			1           & 4.500          \\\normalline
			2           & 6.750          \\\normalline
			3           & 10.125         \\\lastline
		\end{tabular}
	\end{minipage}
	\hfill
	\begin{minipage}{0.2\textwidth}
		\centering
		\caption{}
		\label{exp:tab:findformula4}
		\begin{tabular}{S[table-format=1.0]S[table-format=2.0]}
			\beforeheading
			\heading{$x$} & \heading{$k(x)$} \\
			\afterheading
			-1            & 20               \\\normalline
			0             & 16               \\\normalline
			1             & 12               \\\normalline
			2             & 8                \\\normalline
			3             & 4                \\\lastline
		\end{tabular}
	\end{minipage}
	\vspace{0.5cm}
	\\
	\null\hfill
	\begin{minipage}{0.25\textwidth}
		\centering
		\caption{}
		\label{exp:tab:dataproblem1}
		\begin{tabular}{S[table-format=1.0]S[table-format=2.2]}
			\beforeheading
			\heading{$x$} & \heading{$y$} \\
			\afterheading
			4             & 7.32          \\\normalline
			5             & 8.05          \\\normalline
			6             & 8.86          \\\normalline
			7             & 9.74          \\\normalline
			8             & 10.72         \\\lastline
		\end{tabular}
	\end{minipage}
	\hfill
	\begin{minipage}{0.25\textwidth}
		\centering
		\caption{}
		\label{exp:tab:dataproblem2}
		\begin{tabular}{S[table-format=1.0]S[table-format=1.2]}
			\beforeheading
			\heading{$x$} & \heading{$y$} \\
			\afterheading
			2             & 4.27          \\\normalline
			3             & 5.4           \\\normalline
			4             & 6.52          \\\normalline
			5             & 7.63          \\\normalline
			6             & 8.75          \\\lastline
		\end{tabular}
	\end{minipage}
	\hfill
	\begin{minipage}{0.25\textwidth}
		\centering
		\caption{}
		\label{exp:tab:dataproblem3}
		\begin{tabular}{S[table-format=1.0]S[table-format=2.0]}
			\beforeheading
			\heading{$x$} & \heading{$y$} \\
			\afterheading
			5             & 20            \\\normalline
			6             & 30            \\\normalline
			7             & 42            \\\normalline
			8             & 56            \\\normalline
			9             & 72            \\\lastline
		\end{tabular}
	\end{minipage}
	\hfill
	\null
	\end{widepage}
\end{table}
			
\begin{subproblem}\label{exp:prob:dataproblem1}
	For each data set in \crefrange{exp:tab:dataproblem1}{exp:tab:dataproblem3}, state whether a 
	linear or exponential function (or neither) would better model the data.  If a linear or exponential 
	model is appropriate,  find an approximate formula for the model.
	\begin{shortsolution}
		The data in \cref{exp:tab:dataproblem1} could be modeled with an exponential function, since all of the ratios are fairly close to 1.1.  The model could be $y=f(x)$, where $f(x)=5(1.1)^x$.
		The data in \cref{exp:tab:dataproblem2} could be modeled with a linear function, since all of the differences are fairly close to 1.12.  The model could be $y=g(x)$, where $g(x)=1.12x+2.03$.
		The data in \cref{exp:tab:dataproblem3} should not be modeled with either a linear function or an exponential function, since the differences are increasing and the ratios are decreasing.
	\end{shortsolution}
\end{subproblem}
\end{problem}
			
			
%===================================
%   Author: Hughes
%   Date:   Dec 2011
%===================================
\begin{problem}[Which is greater?]
Let $f$ and $g$ be functions that have the formulas
\[
	f(x)=1000x+1\times 10^6,\qquad  g(x)=2^x
\]
\begin{subproblem}
	Evaluate $f(1)$ and $g(1)$. Which is greater?
	\begin{shortsolution}
		$f(1)=1\times 10^6$, $g(1)=2$. Clearly $f(1)>g(1)$.
	\end{shortsolution}
\end{subproblem}
\begin{subproblem}
	Evaluate $f(10)$ and $g(10)$. Which is greater?
	\begin{shortsolution}
		$f(10)=1.01\times 10^6$, $g(10)=1024$. Clearly $f(10)>g(10)$.
	\end{shortsolution}
\end{subproblem}
\begin{subproblem}\label{exp:prob:expovertake}
	Do you think that that $g(x)$ will ever be greater than $f(x)$?
	\begin{shortsolution}
		Answers will vary.
	\end{shortsolution}
\end{subproblem}
\begin{subproblem}
	Evaluate $f(20)$ and $g(20)$. Does this change your answer to \cref{exp:prob:expovertake}?
	\begin{shortsolution}
		$f(20)=1.02\times 10^6$, $g(20)\approx 1.05\times 10^6$. 
	\end{shortsolution}
\end{subproblem}
\end{problem}
			
%===================================
%   Author: Hughes
%   Date:   Dec 2011
%===================================
\begin{problem}
Consider the ordered pairs $(3,10)$ and $(7,15)$.
\begin{subproblem}
	Find the formula for the linear function, $f$, that goes through the ordered pairs.
	\begin{shortsolution}
		$f(x)=\frac{5}{4}x+\frac{25}{4}$
	\end{shortsolution}
\end{subproblem}
\begin{subproblem}
	Find the formula for the exponential function, $g$, that goes through the ordered pairs.
	\begin{shortsolution}
		$g(x)=\frac{10}{\left( \nicefrac{3}{2} \right)^{\nicefrac{3}{4}}}\left( \sqrt[4]{\frac{3}{2}} \right)^x\approx 7.38(1.106682)^x$
	\end{shortsolution}
\end{subproblem}
\begin{subproblem}
	What is the first integer value of $x$ that makes $g(x)>f(x)$?
	\begin{shortsolution}
		$x=8$ (perhaps obviously).
	\end{shortsolution}
\end{subproblem}
\end{problem}
			
%===================================
%   Author: Hughes
%   Date:   Dec 2011
%===================================
\begin{problem}[True or false]
Answer the following questions as True or False; if you believe the answer 
to be False, provide justification that supports your answer.
\begin{subproblem}
	Linear functions are concave down.
	\begin{shortsolution}
		False; linear functions are neither concave up nor concave down.
	\end{shortsolution}
\end{subproblem}
\begin{subproblem}
	Linear functions are concave up.
	\begin{shortsolution}
		False; linear functions are neither concave up nor concave down.
	\end{shortsolution}
\end{subproblem}
\begin{subproblem}
	It is possible to write a linear function that has a slope of $2$.
	\begin{shortsolution}
		True.
	\end{shortsolution}
\end{subproblem}
\begin{subproblem}
	It is possible to write an exponential function that has a slope of $2$.
	\begin{shortsolution}
		False; exponential functions do not have a constant slope.
	\end{shortsolution}
\end{subproblem}
\begin{subproblem}
	There is an exponential function that decreases at a constant rate of $5$.
	\begin{shortsolution}
		False; exponential functions do not decrease (nor increase) at a constant rate.
	\end{shortsolution}
\end{subproblem}
\end{problem}
			
%===================================
%   Author: Hughes
%   Date:   Dec 2011
%===================================
\begin{problem}[Classify that function!]
\pccname{Carlos} and \pccname{Anita} are playing a game the call, `Classify that function!' One of them 
describes how to plot the points or features of the graph, and the other has to say if it is linear 
or exponential. Help them decide if the following describe 
linear or exponential functions.
\begin{subproblem}
	Over 2 up 3, over 2 up 3, over 2 up 3, $\ldots$
	\begin{shortsolution}
		Linear, with $m=\nicefrac{3}{2}$.
	\end{shortsolution}
\end{subproblem}
\begin{subproblem}
	Start negative.  Over $1$, $5$ times farther down, over $1$, $5$ times farther down, over $1$, $5$ times farther down, $\ldots$
	\begin{shortsolution}
		Exponential, with $b=5$.
	\end{shortsolution}
\end{subproblem}
\begin{subproblem}
	Left 5 up 1, left 10 up 2, left 15 up 3, left 20 up 4, $\ldots$
	\begin{shortsolution}
		Linear.
	\end{shortsolution}
\end{subproblem}
\begin{subproblem}
	A straight line that goes through the points $(0,0)$ and $(20,19)$.
	\begin{shortsolution}
		Linear.
	\end{shortsolution}
\end{subproblem}
\begin{subproblem}
	A function that is concave up, and has a horizontal asymptote of $y=0$ as $x\to -\infty$.
	\begin{shortsolution}
		Exponential.
	\end{shortsolution}
\end{subproblem}
\end{problem}
			
%===================================
%   Author: Hughes
%   Date:   Dec 2011
%===================================
\begin{problem}[Matching stories with formulas]
Match each of the following formulas with one of the given statements. 
Note that $y$ and $x$ have deliberately been used in each formula to avoid 
any extra hints; you will also notice that there are more choices than questions so you 
will not be able to use all choices.
\begin{multicols}{4}
	\begin{enumerate}[label=(\roman*)]
		\item $y=2\pi^x$
		\item $y=2\pi x$
		\item $y=2^x$
		\item $y=2x$
		\item $y=\frac{5}{9}\left( x-32 \right)$
		\item $y=\frac{5}{9}x-32 $
		\item $y=\frac{9}{5}x+32$
		\item $y=100(0.9)^x$
		\item $y=100-10x$
	\end{enumerate}
\end{multicols}
\begin{subproblem}
	To convert from Fahrenheit to Celsius, subtract 32 and then multiply by $\nicefrac{5}{9}$.
	\begin{shortsolution}
		$y=\frac{5}{9}\left( x-32 \right)$
	\end{shortsolution}
\end{subproblem}
\begin{subproblem}
	A population starts with 100 people, and decreases by $\SI{10}{\percent}$ per year.
	\begin{shortsolution}
		$y=100(0.9)^x$
	\end{shortsolution}
\end{subproblem}
\begin{subproblem}
	The circumference of a circle is calculated by multiplying the radius by $2\pi$.
	\begin{shortsolution}
		$y=2\pi x$
	\end{shortsolution}
\end{subproblem}
\begin{subproblem}
	What is the biggest city in the World? Dublin, because it keeps on doublin' and doublin', and$\ldots$
	\begin{shortsolution}
		$y=2^x$
	\end{shortsolution}
\end{subproblem}
\begin{subproblem}
	To convert from Celsius to Fahrenheit, multiply by $\nicefrac95$ and then add 32.
	\begin{shortsolution}
		$y=\frac{9}{5}x+32$
	\end{shortsolution}
\end{subproblem}
\begin{subproblem}
	A population starts with 100 people and decreases by 10 people per year.
	\begin{shortsolution}
		$y=100-10x$
	\end{shortsolution}
\end{subproblem}
\end{problem}
			
%===================================
%   Author: Jordan
%   Date:   Dec 2011
%===================================
\begin{problem}
Did you complete \cref{exp:prob:tapfishtable} (from \cref{exp:sec:findformula}) about the Tapfish app?  If you did, do the values of $F(t)$ suggest that an exponential model might be appropriate?  Use successive ratios to decide.
			
\begin{shortsolution}
	\begin{tabular}[t]{S[table-format=1.0]S[table-format=2.0]l}
		\beforeheading
		\heading{$t$} & \heading{$F(t)$} & \heading{successive ratio} \\
		\afterheading
		0             & 2                &                            \\\normalline 
		1             & 3                & $1.5$                      \\\normalline 
		2             & 4                & $1.33\ldots$               \\\normalline 
		3             & 6                & $1.5$                      \\\normalline
		4             & 9                & $1.5$                      \\\normalline 
		5             & 14               & $1.55\ldots$               \\\normalline
		6             & 21               & $1.5$                      \\\normalline
		7             & 31               & $1.48\ldots$               \\\normalline
		8             & 47               & $1.52\ldots$               \\\lastline
	\end{tabular}
											
	The successive ratios are fairly close to $1.5$, although there is some variation.  This suggests that an exponential model might be appropriate with a base of $1.5$.
											
\end{shortsolution}
\end{problem}
			
			
%===================================
%   Author: Hughes
%   Date:   Dec 2011
%===================================
\begin{problem}[Long-run behavior of linear and exponential functions]
We are going to explore long-run behavior of exponential functions that have base less than one ($b<1$).
\begin{subproblem}
	Let $f$ and $n$ be the functions that have formulas
	\[
		f(x)=\left( \frac{1}{4} \right)^x, \qquad n(x)=-3x
	\]
	which are shown in \cref{exp:fig:complinexpfn}. Describe the behavior
	of $f$ and $n$ as $x\to\infty$ and $x\to-\infty$.
	\begin{shortsolution}
		$f(x)\to 0$ as $x\to\infty$ and $f(x)\to\infty$ as $x\to-\infty$.
		$n(x)\to-\infty$ as $x\to\infty$ and $n(x)\to\infty$ as $x\to-\infty$.
	\end{shortsolution}
\end{subproblem}
\begin{subproblem}
	Let $g$ and $m$ be the functions that have formulas
	\[
		g(x)= -\left( \frac{1}{5} \right)^x, \qquad m(x)=6x
	\]
	which are shown in \cref{exp:fig:complinexpgm}. Describe the behavior
	of $g$ and $m$ as $x\to\infty$ and $x\to-\infty$.
	\begin{shortsolution}
		$g(x)\to 0$ as $x\to\infty$ and $g(x)\to-\infty$ as $x\to-\infty$.
		$m(x)\to\infty$ as $x\to\infty$ and $m(x)\to-\infty$ as $x\to-\infty$.
	\end{shortsolution}
\end{subproblem}
\end{problem}
\begin{figure}[!htb]
	\begin{widepage}
	\centering
	\hfill
	\begin{minipage}{0.3\textwidth}
		\begin{tikzpicture}
			\begin{axis}[
					framed,
					xmin=-5,xmax=5,
					ymin=-40,ymax=100,
					xtick={-4,...,4},
					ytick={-20,0,...,80},
					minor ytick={-30,-20,...,90},
					grid=both,
					legend pos=north east,
				]
				\addplot expression[domain=-3.3219:5]{(1/4)^x};
				\addplot expression[domain=-5:5]{-3*x};
				\legend{$f$, $n$};
			\end{axis}
		\end{tikzpicture}
		\caption{$f$ and $n$}
		\label{exp:fig:complinexpfn}
	\end{minipage}
	\hfill
	\begin{minipage}{0.3\textwidth}
		\begin{tikzpicture}
			\begin{axis}[
					framed,
					xmin=-5,xmax=5,
					ymin=-100,ymax=40,
					xtick={-4,...,4},
					ytick={-80,-60,...,20},
					minor ytick={-90,-70,...,30},
					grid=both,
					legend pos=south east,
				]
				\addplot expression[domain=-2.8613:5]{-(1/5)^x};
				\addplot expression[domain=-5:5]{6*x};
				\legend{$g$,$m$};
			\end{axis}
		\end{tikzpicture}
		\caption{$g$ and $m$}
		\label{exp:fig:complinexpgm}
	\end{minipage}
	\hfill\null
	\end{widepage}
\end{figure}
			
\FloatBarrier
%===================================
%   Author: Hughes
%   Date:   Dec 2011
%===================================
\begin{problem}[Match formulas to graphs]
Match each of the following formulas with one of the graphs in 
\cref{exp:fig:matchlinexp}. Note that axis ticks and scaling have deliberately
been omitted to encourage you to think about long-run behavior. 
\begin{multicols}{4}
	\begin{enumerate}[label=(\roman*)]
		\item $y=-4^x$
		\item $y=\left( \dfrac{1}{2} \right)^x$
		\item $y=-\pi x$
		\item $y=4$
		\item $y=10x+2$
		\item $y=10^x$
		\item $y=x$
		\item $y=-\left( \dfrac{1}{2} \right)^x$
	\end{enumerate}
\end{multicols}
\begin{shortsolution}
	\begin{itemize}
		\item \Cref{exp:fig:matchlinexp1}: $y=10^x$
		\item \Cref{exp:fig:matchlinexp2}: $y=-\pi x$
		\item \Cref{exp:fig:matchlinexp3}: $y=\left( \dfrac{1}{2} \right)^x$
		\item \Cref{exp:fig:matchlinexp4}: $y=x$
		\item \Cref{exp:fig:matchlinexp5}: $y=4$
		\item \Cref{exp:fig:matchlinexp6}: $y=-\left( \dfrac{1}{2} \right)^x$
		\item \Cref{exp:fig:matchlinexp7}: $y=10x+2$
		\item \Cref{exp:fig:matchlinexp8}: $y=-4^x$
	\end{itemize}
\end{shortsolution}
\end{problem}
			
\begin{figure}[!htb]
	\begin{widepage}
	\setlength{\figurewidth}{0.2\textwidth}
	\centering
	\begin{subfigure}{.2\textwidth}
		\begin{tikzpicture}
			\begin{axis}[
					framed,
					xmin=-5,xmax=5,
					ymin=-25,ymax=50,
					xtick={-10},ytick={-101},
				]
				\addplot expression[domain=-5:5]{2^x};
			\end{axis}
		\end{tikzpicture}
		\caption{}
		\label{exp:fig:matchlinexp1}
	\end{subfigure}
	\hfill
	\begin{subfigure}{.2\textwidth}
		\begin{tikzpicture}
			\begin{axis}[
					framed,
					xmin=-5,xmax=5,
					ymin=-100,ymax=100,
					xtick={-10},ytick={-101},
				]
				\addplot expression[domain=-5:5]{-20*x};
			\end{axis}
		\end{tikzpicture}
		\caption{}
		\label{exp:fig:matchlinexp2}
	\end{subfigure}
	\hfill
	\begin{subfigure}{.2\textwidth}
		\begin{tikzpicture}
			\begin{axis}[
					framed,
					xmin=-5,xmax=5,
					ymin=-25,ymax=100,
					xtick={-10},ytick={-101},
				]
				\addplot expression[domain=-3.3219:5]{(1/4)^x};
			\end{axis}
		\end{tikzpicture}
		\caption{}
		\label{exp:fig:matchlinexp3}
	\end{subfigure}
	\hfill
	\begin{subfigure}{.2\textwidth}
		\begin{tikzpicture}
			\begin{axis}[
					framed,
					xmin=-5,xmax=5,
					ymin=-15,ymax=20,
					xtick={-10},ytick={-101},
				]
				\addplot expression[domain=-5:5]{3*x};
			\end{axis}
		\end{tikzpicture}
		\caption{}
		\label{exp:fig:matchlinexp4}
	\end{subfigure}\\
	\vspace{0.5cm}
	\begin{subfigure}{.2\textwidth}
		\begin{tikzpicture}
			\begin{axis}[
					framed,
					xmin=-5,xmax=5,
					ymin=-10,ymax=10,
					xtick={-10},ytick={-101},
				]
				\addplot expression[domain=-5:5]{5};
			\end{axis}
		\end{tikzpicture}
		\caption{}
		\label{exp:fig:matchlinexp5}
	\end{subfigure}
	\hfill
	\begin{subfigure}{.2\textwidth}
		\begin{tikzpicture}
			\begin{axis}[
					framed,
					xmin=-5,xmax=5,
					ymin=-10,ymax=10,
					xtick={-10},ytick={-101},
				]
				\addplot expression[domain=-3.32193:5]{-(1/2)^x};
			\end{axis}
		\end{tikzpicture}
		\caption{}
		\label{exp:fig:matchlinexp6}
	\end{subfigure}
	\hfill
	\begin{subfigure}{.2\textwidth}
		\begin{tikzpicture}
			\begin{axis}[
					framed,
					xmin=-5,xmax=5,
					ymin=-10,ymax=10,
					xtick={-10},ytick={-101},
				]
				\addplot expression[domain=-5:5]{x+2};
			\end{axis}
		\end{tikzpicture}
		\caption{}
		\label{exp:fig:matchlinexp7}
	\end{subfigure}
	\hfill
	\begin{subfigure}{.2\textwidth}
		\begin{tikzpicture}
			\begin{axis}[
					framed,
					xmin=-5,xmax=5,
					ymin=-100,ymax=40,
					xtick={-10},ytick={-101},
				]
				\addplot expression[domain=-5:4.1918]{-3^x};
			\end{axis}
		\end{tikzpicture}
		\caption{}
		\label{exp:fig:matchlinexp8}
	\end{subfigure}
	\caption{}
	\label{exp:fig:matchlinexp}
	\end{widepage}
\end{figure}
\end{exercises}
			
\section{Extensions}
\begin{outcomes}
	\begin{outcomelist}
		\item Investigate the Logistic model for population growth and decay.
		\item Revisit composition and piecewise defined functions.
	\end{outcomelist}
\end{outcomes}
\subsection*{A more realistic population model}
We considered population models in \cref{exp:sec:populationmodels}. 
Each model had the form 
\[
	P(t) = a\,b^t
\]
and therefore implied that as $t\to\infty$, the population either decays to zero, or grows without bound.
			
Intuitively, these models are unrealistic.  A decreasing population does not necessarily decay to zero, and an 
increasing population encounters limitations on food and other resources that will prevent it from growing 
without bound.  The Logistic Model takes resource limitations into account.
			
\begin{pccdefinition}[The Logistic Model]\label{exp:def:logistic}
	According to the Logistic Model, the population, $P(t)$, $t$ years after the population started its logistic growth is 
	given by the formula
	\[
		P(t) = \frac{MP_0}{P_0+(M-P_0)e^{-kt}}
	\]
	where
	\begin{itemize}
		\item $P_0$ is the initial population:  $P_0=P(0)$;
		\item $M$ is the {\em carrying capacity}, the maximum population that can be supported by the 
		available resources;
		\item $k$ approximates the relative growth rate when the population is small, relative to the carrying capacity, 
		interpreted as a percent per year.       
	\end{itemize}
\end{pccdefinition}
			
%===================================
%   Author: Hughes
%   Date:   May 2011
%===================================
\begin{pccexample}\label{exp:ex:logistic}
	Let $P(t)$ represent the population of a country at time $t$ (in years) since 2000.
	Use \cref{exp:def:logistic}, with $k=0.08$ and $M=1000$, to study the effect 
	of changing $P_0$ from 100 to 1400. What happens to $P$ as $t\to\infty$?
	\begin{pccsolution}
		Using $k=0.08$ and $M=1000$ in \cref{exp:def:logistic}, we have
		\[
			P(t) = \frac{1000P_0}{P_0+(1000-P_0)e^{-0.08t}}
		\]
		If we put $P_0=100$ and then $P_0=1400$ then we have, respectively,
		\[
			P(t)=\frac{1000}{1+9e^{-0.08t}}, \qquad P(t)= \frac{7000}{7-2e^{-0.08t}}
		\]
		graphs of which are shown in \cref{exp:fig:logistics1,exp:fig:logistics2}.
																	
		Notice that in both models, $P(t)\to 1000$ as $t\to\infty$. Remember 
		that we called $M$ the carrying capacity in \cref{exp:def:logistic}, 
		which represents the maximum population that the environment can support. 
																	
		In contrast to the models presented in \cref{exp:sec:populationmodels}, the populations 
		neither grow without bound nor decay toward zero. 
	\end{pccsolution}
\end{pccexample}
			
\begin{figure}[!htb]
	\begin{minipage}{.4\textwidth}
		\centering
		\begin{tikzpicture}
			\begin{axis}[
					framed,
					xmin=-15,xmax=100,
					ymin=-130,ymax=1500,
					xtick={10,20,...,90},
					ytick={200,400,...,1400},
					minor ytick={100,300,...,1300},
					xlabel={$t$},
					grid=both,
				]
				\addplot+[->]expression[domain=0:95]{1000/(1+9*exp(-0.08*x))};
				\addplot[asymptote,domain=0:100]({x},{1000});
				\addplot[soldot]coordinates{(0,100)};
			\end{axis}
		\end{tikzpicture}
		\caption{$\dd P(t)=\frac{1000}{1+9e^{-0.08t}}$}
		\label{exp:fig:logistics1}
	\end{minipage}
	\hfill
	\begin{minipage}{.4\textwidth}
		\centering
		\begin{tikzpicture}
			\begin{axis}[
					framed,
					xmin=-15,xmax=100,
					ymin=-130,ymax=1500,
					xtick={10,20,...,90},
					ytick={200,400,...,1400},
					minor ytick={100,300,...,1300},
					xlabel={$t$},
					grid=both,
				]
				\addplot+[->]expression[domain=0:95]{7000/(7-2*exp(-0.08*x))};
				\addplot[asymptote,domain=0:100]({x},1000);
				\addplot[soldot]coordinates{(0,1400)};
			\end{axis}
		\end{tikzpicture}
		\caption{$\dd P(t)= \frac{7000}{7-2e^{-0.08t}}$}
		\label{exp:fig:logistics2}
	\end{minipage}
\end{figure}
			
\begin{exercises}
%===================================
%   Author: Hughes
%   Date:   May 2011
%===================================
\begin{problem}[Logistic model]
Use \cref{exp:def:logistic} and \cref{exp:ex:logistic} to guide you in this 
problem. Assume that $k=0.05$, $P_0=100$ and $M=800$.
\begin{subproblem}
	Find a logistic population model, $P(t)$.
	\begin{shortsolution}
		$P(t) = \frac{800\cdot 100}{100+(800-100)e^{-0.05t}}=\frac{800}{1+7e^{-0.05t}}$
	\end{shortsolution}
\end{subproblem}
\begin{subproblem}
	Find $P(1)$ and determine the relative growth over the first year.  How does this compare to $k$?
	\begin{shortsolution}
		$P(1)\approx 104.46$. We calculate $\frac{P(1)}{P(0)}\approx 1.0446$. The 
		relative growth over the first year is approximately $\SI{4.46}{\percent}$.
	\end{shortsolution}
\end{subproblem}
\begin{subproblem}
	Graph your function $P$ on your calculator and describe its behavior as $t\to\infty$.
	\begin{shortsolution}
		$P(t)\to 800$ as $t\to\infty$, as shown below.
																			
		\begin{tikzpicture}
			\begin{axis}[
					framed,
					xmin=-25,xmax=100,
					ymin=-200,ymax=1200,
					xlabel={$t$},
					xtick={20,40,...,80},
					ytick={200,400,...,1000},
					minor ytick={10,30,...,90},
					minor ytick={100,300,...,1100},
					grid=both,
				]
				\addplot+[->]expression[domain=0:95]{800/(1+7*exp(-0.05*x))};
				\addplot[asymptote,domain=-25:100]({x},800);
				\addplot[soldot]coordinates{(0,100)};
			\end{axis}
		\end{tikzpicture}
	\end{shortsolution}
\end{subproblem}
\begin{subproblem}
	Is your function $P$ increasing or decreasing?
	\begin{shortsolution}
		$P$ is increasing.
	\end{shortsolution}
\end{subproblem}
\end{problem}
			
			
%===================================
%   Author: Hughes
%   Date:   2011
%===================================
\begin{problem}[Rats!]
In the 1800s, a ship landed on a remote island and 35 black rats (\textit{Rattus rattus}) escaped to colonize the island.  
\begin{subproblem} \label{exp:prob:ratisland}
	If the initial relative growth rate was about \SI{90}{\percent} per year and the island has a carrying capacity of 20,000 rats, find a formula for $P(t)$, the number of rats on the island at time $t$, assuming logistic growth.
	\begin{shortsolution}
		$P(t)\approx\frac{700000}{35+19965e^{-0.9t}}$.
	\end{shortsolution}
\end{subproblem}
\begin{subproblem}
	Use a graphing calculator to determine
	how long will it take for the population to reach 15,000 rats.  
	\begin{shortsolution}
		About 8.3 years.
	\end{shortsolution}
\end{subproblem}
\end{problem}
			
%===================================
%   Author: Jordan
%   Date:   2011
%===================================
\begin{problem}[The invasive blackberry]
Himalayan blackberry is an invasive species.  
Some seeds found their way into a remote valley and grew into $\SI{8}{\kilo\gram}$ of blackberry biomass by the next year.  
Suppose that the initial relative growth rate of blackberry biomass was \SI{350}{\percent} per year and that the valley has a carrying capacity of $\SI{50000}{\kilo\gram}$.  
\begin{subproblem} \label{exp:prob:blackberries}
	If $t=0$ corresponds to the time when there was $\SI{8}{\kilo\gram}$ of biomass, find a formula for the amount of 
	blackberry biomass in the valley after $t$ years, assuming logistic growth.  
	\begin{shortsolution}
		$P(t)\approx\frac{400000}{8+399992e^{-3.5t}}$.
	\end{shortsolution}
\end{subproblem}
			
\begin{subproblem}
	Use a graphing calculator to determine how long will it take for the blackberry biomass to reach half of the valley's carrying capacity.  
	\begin{shortsolution}
		About 3.1 years.
	\end{shortsolution}
\end{subproblem}
\end{problem}
%===================================
%   Author: Hughes
%   Date:   2011
%===================================
\begin{problem}
\begin{subproblem}
	Suppose that $P$ is a decreasing logistic function with $k=0.05$ and $P(t)\to 800$ as $t\to\infty$. 
	Find a formula for $P$ and graph the function on your calculator.
	\begin{shortsolution}
		There are infinitely many choices available to us- we just need to choose $P_0>800$. With $P_0=1000$, we have 
		$P(t)=\frac{4000}{5-e^{-0.05t}}$.
																			
		\begin{tikzpicture}
			\begin{axis}[
					framed,
					xmin=-25,xmax=100,
					ymin=-200,ymax=1200,
					xlabel={$t$},
					xtick={20,40,...,80},
					ytick={200,400,...,1000},
					minor ytick={10,30,...,90},
					minor ytick={100,300,...,1100},
					grid=both,
				]
				\addplot+[->]expression[domain=0:95]{4000/(5-exp(-0.05*x))};
				\addplot[asymptote,domain=-25:100]({x},800);
				\addplot[soldot]coordinates{(0,1000)};
			\end{axis}
		\end{tikzpicture}
	\end{shortsolution}
\end{subproblem}
\begin{subproblem}
	What does your formula become if $P_0=800$? Is $P$ increasing or decreasing in this case?
	\begin{shortsolution}
		When $P_0=800$, $P(t)=800$. $P$ is neither increasing nor decreasing, it is constant!
	\end{shortsolution}
\end{subproblem}
\end{problem}
%===================================
%   Author: Hughes
%   Date:   March 2012
%===================================
\begin{problem}[Composition]
Let $f$ and $g$ be functions that have formulas $f(x)=2^x$ and $g(x)=3^x$. Find each 
of the following.
\begin{multicols}{3}
	\begin{subproblem}[core]
		$(f\circ g)(1)$
		\begin{shortsolution}
			$8$ 
		\end{shortsolution}
	\end{subproblem}
	\begin{subproblem}[core]
		$(g\circ f)(2)$
		\begin{shortsolution}
			$81$ 
		\end{shortsolution}
	\end{subproblem}
	\begin{subproblem}[core]
		$(f\circ g)(0)$
		\begin{shortsolution}
			$2$ 
		\end{shortsolution}
	\end{subproblem}
	\begin{subproblem}
		$(g\circ f)(0)$
		\begin{shortsolution}
			$3$ 
		\end{shortsolution}
	\end{subproblem}
	\begin{subproblem}
		$(f\circ g)(x)$
		\begin{shortsolution}
			$2^{3^x}$ 
		\end{shortsolution}
	\end{subproblem}
	\begin{subproblem}
		$(g\circ f)(x)$
		\begin{shortsolution}
			$3^{2^x}$ 
		\end{shortsolution}
	\end{subproblem}
\end{multicols}
\end{problem}
%===================================
%   Author: Hughes
%   Date:   March 2012
%===================================
\begin{problem}[Decomposition]
In each of the following problems, you are given a formula for function  
$h$. Decompose $h$ into two functions $f$ and $g$ such that $h=f\circ g$.
\begin{multicols}{4}
	\begin{subproblem}
		$h(x)=2^{x^2}$
		\begin{shortsolution}
			$f(x)=2^x$, $g(x)=x^2$ 
		\end{shortsolution}
	\end{subproblem}
	\begin{subproblem}
		$h(x)=-4^{x^3+2x}$
		\begin{shortsolution}
			$f(x)=-4^x$, $g(x)=x^3+2x$ 
		\end{shortsolution}
	\end{subproblem}
	\begin{subproblem}
		$h(x)=2^{x^2}+3^{x^2}$
		\begin{shortsolution}
			$f(x)=2^x+3^x$, $g(x)=x^2$
		\end{shortsolution}
	\end{subproblem}
	\begin{subproblem}
		$h(x)=e^{-x^2+2}$
		\begin{shortsolution}
			$f(x)=e^x$, $g(x)=-x^2+2$ 
		\end{shortsolution}
	\end{subproblem}
\end{multicols}
\end{problem}
%===================================
%   Author: Hughes
%   Date:   August 2012
%===================================
\begin{problem}[Inverse function evaluation]
The function $f$ that has formula $f(x)=2^x$ is invertible. Evaluate 
each of the following.
\begin{multicols}{4}
	\begin{subproblem}
		$f^{-1}(8)$ 
		\begin{shortsolution}
			$3$ 
		\end{shortsolution}
	\end{subproblem}
	\begin{subproblem}
		$f^{-1}(16)$ 
		\begin{shortsolution}
			$4$ 
		\end{shortsolution}
	\end{subproblem}
	\begin{subproblem}
		$f^{-1}\left( \frac{1}{4} \right)$ 
		\begin{shortsolution}
			$-2$ 
		\end{shortsolution}
	\end{subproblem}
	\begin{subproblem}
		$f^{-1}(1)$ 
		\begin{shortsolution}
			$0$ 
		\end{shortsolution}
	\end{subproblem}
\end{multicols}
\end{problem}
%===================================
%   Author: Hughes
%   Date:   March 2012
%===================================
\begin{problem}[Piecewise functions]
Let $k$ be the function that has formula
\[
	k(t)=
	\begin{cases}
		2^t,  & t<-5        \\
		-10,  & -5\leq t< 3 \\
		6t,   & 3<t< 7      \\
		-4^t, & t>7         
	\end{cases}
\]
Evaluate each of the following, and leave your answers in exact form.
\begin{multicols}{4}
	\begin{subproblem}
		$k(-6)$
		\begin{shortsolution}
			$64$ 
		\end{shortsolution}
	\end{subproblem}
	\begin{subproblem}
		$k(-4)$
		\begin{shortsolution}
			$10$ 
		\end{shortsolution}
	\end{subproblem}
	\begin{subproblem}
		$k(0)$
		\begin{shortsolution}
			$-10$ 
		\end{shortsolution}
	\end{subproblem}
	\begin{subproblem}
		$k(2.99)$
		\begin{shortsolution}
			$-10$ 
		\end{shortsolution}
	\end{subproblem}
	\begin{subproblem}
		$k(5)$
		\begin{shortsolution}
			$30$ 
		\end{shortsolution}
	\end{subproblem}
	\begin{subproblem}
		$k(6)$
		\begin{shortsolution}
			$36$ 
		\end{shortsolution}
	\end{subproblem}
	\begin{subproblem}
		$k(7)$
		\begin{shortsolution}
			$k(7)$ is undefined. 
		\end{shortsolution}
	\end{subproblem}
	\begin{subproblem}
		$k(8)$
		\begin{shortsolution}
			$-65536$ 
		\end{shortsolution}
	\end{subproblem}
\end{multicols}
\end{problem}
			
%===================================
%   Author: Hughes
%   Date:   July 2012
%===================================
\begin{problem}[Function algebra]
Let $f$ and $g$ be the exponential functions that have formulas
\[
	f(x)=3^x, \qquad g(x)=\left( \frac{1}{4} \right)^x
\]
Evaluate each of the following (if possible).
\begin{multicols}{4}
	\begin{subproblem}
		$(f+g)(0)$ 
		\begin{shortsolution}
			$1$ 
		\end{shortsolution}
	\end{subproblem}
	\begin{subproblem}
		$(f-g)(2)$ 
		\begin{shortsolution}
			$\frac{143}{16}$ 
		\end{shortsolution}
	\end{subproblem}
	\begin{subproblem}
		$(f\cdot g)(-2)$ 
		\begin{shortsolution}
			$\frac{16}{9}$ 
		\end{shortsolution}
	\end{subproblem}
	\begin{subproblem}
		$\left( \frac{f}{g} \right)(1)$ 
		\begin{shortsolution}
			$12$ 
		\end{shortsolution}
	\end{subproblem}
\end{multicols}
\end{problem}
			
%===================================
%   Author: Hughes
%   Date:   July 2012
%===================================
\begin{problem}[Transformations: given the transformation, find the formula]
Let $f$ be the exponential function that has formula $f(x)=7^x$. In each of the following 
problems apply the given transformation to the function $f$ and 
write a formula for the transformed version of $f$.
\begin{multicols}{2}
	\begin{subproblem}
		Shift $f$ to the right by $2$ units. 
		\begin{shortsolution}
			$f(x-2)=7^{x-2}$
		\end{shortsolution}
	\end{subproblem}
	\begin{subproblem}
		Shift $f$ to the left by $5$ units. 
		\begin{shortsolution}
			$f(x+5)=7^{x+5}$
		\end{shortsolution}
	\end{subproblem}
	\begin{subproblem}
		Shift $f$ up by $11$ units. 
		\begin{shortsolution}
			$f(x)+11=7^x+11$
		\end{shortsolution}
	\end{subproblem}
	\begin{subproblem}
		Shift $f$ down by $1$ unit. 
		\begin{shortsolution}
			$f(x)-1=7^x-1$
		\end{shortsolution}
	\end{subproblem}
	\begin{subproblem}
		Reflect $f$ over the horizontal axis.
		\begin{shortsolution}
			$-f(x)=-7^x$ 
		\end{shortsolution}
	\end{subproblem}
	\begin{subproblem}
		Reflect $f$ over the vertical axis.
		\begin{shortsolution}
			$f(-x)=7^{-x}$ 
		\end{shortsolution}
	\end{subproblem}
\end{multicols}
\end{problem}
%===================================
%   Author: Hughes
%   Date:   July 2012
%===================================
\begin{problem}[Transformations: given the formula, describe the transformation]
Describe each of the functions defined by the following formulas in terms of transformations 
of the exponential function $f$ that has formula $f(x)=\left( \frac{2}{3} \right)^x$. 
\begin{multicols}{4}
	\begin{subproblem}
		$g(x)=\left( \frac{2}{3} \right)^{x+7}$ 
		\begin{shortsolution}
			$g$ is the function $f$ shifted to the left by $7$ units. 
		\end{shortsolution}
	\end{subproblem}
	\begin{subproblem}
		$h(x)=\left( \frac{2}{3} \right)^{x-13}$ 
		\begin{shortsolution}
			$h$ is the function $f$ shifted to the right by $13$ units. 
		\end{shortsolution}
	\end{subproblem}
	\begin{subproblem}
		$j(x)=\left( \frac{2}{3} \right)^{2(x+9)}$ 
		\begin{shortsolution}
			$j$ is the function $f$ horizontally compressed by a factor of $2$, 
			and shifted to the left by $9$ units. 
		\end{shortsolution}
	\end{subproblem}
	\begin{subproblem}
		$k(x)=7\left( \frac{2}{3} \right)^{-x}$ 
		\begin{shortsolution}
			$k$ is the function $f$ reflected across the vertical axis, and vertically
			stretched by a factor of $7$. 
		\end{shortsolution}
	\end{subproblem}
\end{multicols}
\end{problem}
\end{exercises}
% We'll come back to this before presenting to the SAC (hopefully)  (Chris, 9/30/11)
%\section{Glossary}
%\fixthis{Chris: Add the glossary}

%\chapter{Functions}
\minitoc

\section{Function algebra}
%===================================
%   Author: Hughes
%   Date:   October 2012
%===================================
\begin{pccdefinition}[Function algebra]
Given two functions $f$ and $g$, we may combine the two functions to 
form new functions
\[
	f+g, \qquad f-g, \qquad f\cdot g, \qquad \frac{f}{g}
\]
The formula for each function can be found using
\begin{align*}
	(f+g)(x)                      & =f(x)+g(x)         \\ 
	(f-g)(x)                      & =f(x)-g(x)         \\ 
	(f\cdot g)(x)                 & =f(x)\cdot g(x)    \\ 
	\left( \frac{f}{g} \right)(x) & =\frac{f(x)}{g(x)} 
\end{align*}
The domain of each of the functions $f+g$, $f-g$, and $f\cdot g$ is
\[
	(\text{domain of }f)\cap (\text{domain of }g)
\]
The domain of the function $\frac{f}{g}$ is
\[
	(\text{domain of }f)\cap (\text{domain of }g \text{ such that }g(x)\ne 0)
\]
\end{pccdefinition}
%===================================
%   Author: Hughes
%   Date:   October 2012
%===================================
\begin{pccexample}[Function algebra domain]
In each of the following cases you are given the formulas for two 
functions $f$ and $g$. In each case, find the domain of $f+g$
and $\frac{f}{g}$.
\begin{enumerate}
	\item $f(x)=\sqrt{x}$, $g(x)=\sqrt{1-x}$
	\item $f(x)=\sqrt{x-1}$, $g(x)=\sqrt{1-x}$
	\item $f(x)=\dfrac{1}{x+3}$, $g(x)=\sqrt{x+5}$
\end{enumerate}
\begin{pccsolution}
\begin{enumerate}
	\item The domain of $f$ is $[0,\infty)$ and the domain of $g$ is 
	$(-\infty,1]$.  Therefore, the domain of the function $f+g$ is
	\[
		[0,\infty)\cap (-\infty,1]=[0,1]
	\]
	The domain of the function $\frac{f}{g}$ is found in a similar 
	way, except we must have the additional condition that $g(x)\ne 0$;
	we therefore must exclude $1$ from the domain. The domain 
	of $\frac{f}{g}$ is therefore
	\[
		[0,1)
	\]
	\item The domain of $f$ is $[1,\infty)$ and the domain of $g$ is
	$(-\infty,1]$. Therefore the domain of $f+g$ is 
	\[
		[1,\infty)\cap (-\infty,1]= \{ 1\}
	\]
	The domain of the function $\frac{f}{g}$ is found in a simlar way, but
	we must exclude all values of $x$ that make $g(x)=0$. Since $g(1)=0$
	we must exclude $1$ from the domain of $\frac{f}{g}$; we therefore 
	conclude that the domain of $\frac{f}{g}$ is the empty set, $\emptyset$.
	\item The domain of $f$ is $(-\infty,-3)\cup (-3,\infty)$ and the 
	domain of $g$ is $[-5,\infty]$. The domain of $f+g$ is therefore
	\[
		((-\infty,-3)\cup (-3,\infty))\cap [-5,\infty ) = [-5,-3)\cup (-3,\infty)
	\]
	We must exclude $-5$ from the domain of $\frac{f}{g}$ since $g(-5)=0$; 
	the domain of $\frac{f}{g}$ is
	\[
		(-5,-3)\cup (-3,\infty)
	\]
\end{enumerate}
\end{pccsolution}
\end{pccexample}

\begin{exercises}
%===================================
%   Author: Hughes
%   Date:   October 2012
%===================================
\begin{problem}[Function algebra using formulas]
In each of the following problems you are given formulas for 
functions $f$ and $g$. Find the domain of $f\cdot g$ and 
$\frac{f}{g}$ in each case.
\begin{multicols}{2}
	\begin{subproblem}
		$f(x)=x$, $g(x)=x^2+1$ 
		\begin{shortsolution}
			Domain of $f\cdot g$: $(-\infty,\infty)$; domain of $\frac{f}{g}$: $(-\infty,\infty)$.  
		\end{shortsolution}
	\end{subproblem}
	\begin{subproblem}
		$f(x)=3x+2$, $g(x)=\sqrt{x}$  
		\begin{shortsolution}
			Domain of $f\cdot g$: $[0,\infty)$; domain of $\frac{f}{g}$: $(0,\infty)$.
		\end{shortsolution}
	\end{subproblem}
	\begin{subproblem}
		$f(x)=\sqrt[4]{x-1}$, $g(x)=x^2+5x+4$  
		\begin{shortsolution}
			Domain of $f\cdot g$: $[1,\infty)$; domain of $\frac{f}{g}$: $[1,\infty)$  
		\end{shortsolution}
	\end{subproblem}
	\begin{subproblem}
		$f(x)=\sqrt[5]{x}$, $g(x)=x^2-9x-10$  
		\begin{shortsolution}
			Domain of $f\cdot g$: $(-\infty,\infty)$; domain of $\frac{f}{g}$: $(-\infty,-1)\cup (-1,10)\cup (10,\infty)$. 
		\end{shortsolution}
	\end{subproblem}
\end{multicols}
\end{problem}
%===================================
%   Author: Hughes
%   Date:   October 2012
%===================================
\begin{problem}[Function algebra numerically]\label{fun:prob:combine}
Values of the functions $f$, $g$, $h$, and $j$ are shown in 
\crefrange{fun:tab:combinef}{fun:tab:combinej}

\begin{table}[!htb]
	\centering
	\begin{widepage}
	\caption{Tables for \cref{fun:prob:combine}}
	\label{fun:tab:combine}
	\begin{subtable}{.2\textwidth}
		\centering
		\caption{$y=f(x)$}
		\label{fun:tab:combinef}
		\begin{tabular}{S[table-format=1.0]S[table-format=2.0]}
			\beforeheading
			\heading{$x$} & \heading{$y$} \\            
			\afterheading
			-4          & -56         \\\normalline 
			-3          & -18         \\\normalline  
			-2          & 0           \\\normalline   
			-1          & 4           \\\normalline  
			0           & 0           \\\normalline   
			1           & -6          \\\normalline   
			2           & -8          \\\normalline  
			3           & 0           \\\normalline  
			4           & 24          \\\lastline    
		\end{tabular}
	\end{subtable}
	\hfill
	\begin{subtable}{.2\textwidth}
		\centering
		\caption{$y=g(x)$}
		\label{fun:tab:combineg}
		\begin{tabular}{S[table-format=1.0]S[table-format=3.0]}
			\beforeheading
			\heading{$x$} & \heading{$y$} \\ \afterheading 
			-4          & -16         \\\normalline       
			-3          & -3          \\\normalline        
			-2          & 0           \\\normalline         
			-1          & -1          \\\normalline        
			0           & 0           \\\normalline         
			1           & 9           \\\normalline         
			2           & 32          \\\normalline        
			3           & 75          \\\normalline        
			4           & 144         \\\lastline          
		\end{tabular}
	\end{subtable}
	\hfill
	\begin{subtable}{.2\textwidth}
		\centering
		\caption{$y=h(x)$}
		\label{fun:tab:combineh}
		\begin{tabular}{S[table-format=1.0]S[table-format=2.0]}
			\beforeheading
			\heading{$x$} & \heading{$y$} \\ \afterheading 
			-4          & 2           \\\normalline       
			-3          & 4           \\\normalline         
			-2          & 6           \\\normalline       
			-1          & 8           \\\normalline         
			0           & 10          \\\normalline         
			1           & 12          \\\normalline         
			2           & 14          \\\normalline       
			3           & 16          \\\normalline         
			4           & 18          \\\lastline          
		\end{tabular}
	\end{subtable}
	\hfill
	\begin{subtable}{.2\textwidth}
		\centering
		\caption{$y=j(x)$}
		\label{fun:tab:combinej}
		\begin{tabular}{S[table-format=1.0]S[table-format=3.0]}
			\beforeheading
			\heading{$x$} & \heading{$y$} \\ \afterheading 
			-4          & 30          \\\normalline        
			-3          & 21          \\\normalline         
			-2          & 12          \\\normalline        
			-1          & 3           \\\normalline         
			0           & -6          \\\normalline         
			1           & -15         \\\normalline         
			2           & 15          \\\normalline        
			3           & 96          \\\normalline        
			4           & 760         \\\lastline          
		\end{tabular}
	\end{subtable}
	\end{widepage}
\end{table}

Construct a table of values for each of the following functions, marking
with an X any that undefined.
\begin{multicols}{6}
	\begin{subproblem}
		$f+g$
		\begin{shortsolution}
		\begin{tabular}[t]{S[table-format=1.0]S[table-format=3.0]}
				\beforeheading
				\heading{$x$} & \heading{$(f+g)(x)$} \\ \afterheading 
				-4          & -72                \\\normalline        
				-3          & -21                \\\normalline         
				-2          & 0                  \\\normalline        
				-1          & 3                  \\\normalline         
				0           & 0                  \\\normalline         
				1           & 3                  \\\normalline         
				2           & 24                 \\\normalline        
				3           & 75                 \\\normalline        
				4           & 168                \\\lastline          
			\end{tabular}
		\end{shortsolution}
	\end{subproblem}
	\begin{subproblem}
		$f-g$  
		\begin{shortsolution}
		\begin{tabular}[t]{S[table-format=1.0]S[table-format=3.0]}
				\beforeheading
				\heading{$x$} & \heading{$(f-g)(x)$} \\ \afterheading 
				-4          & -40                \\\normalline        
				-3          & -15                \\\normalline         
				-2          & 0                  \\\normalline        
				-1          & 5                  \\\normalline         
				0           & 0                  \\\normalline         
				1           & -15                \\\normalline         
				2           & -40                \\\normalline        
				3           & -75                \\\normalline        
				4           & -120               \\\lastline          
			\end{tabular}
		\end{shortsolution}
	\end{subproblem}
	\begin{subproblem}
		$g\cdot h$  
		\begin{shortsolution}
		\begin{tabular}[t]{S[table-format=1.0]S[table-format=4.0]}
				\beforeheading
				\heading{$x$} & \heading{$(g\cdot h)(x)$} \\ \afterheading 
				-4          & -32                     \\\normalline        
				-3          & -12                     \\\normalline         
				-2          & 0                       \\\normalline        
				-1          & -8                      \\\normalline         
				0           & 0                       \\\normalline         
				1           & 108                     \\\normalline         
				2           & 448                     \\\normalline        
				3           & 1200                    \\\normalline        
				4           & 2592                    \\\lastline          
			\end{tabular}
		\end{shortsolution}
	\end{subproblem}
	\begin{subproblem}
		$h+j$  
		\begin{shortsolution}
		\begin{tabular}[t]{S[table-format=1.0]S[table-format=3.0]}
				\beforeheading
				\heading{$x$} & \heading{$(h+j)(x)$} \\ \afterheading 
				-4          & 32                 \\\normalline        
				-3          & 25                 \\\normalline         
				-2          & 18                 \\\normalline        
				-1          & 11                 \\\normalline         
				0           & 4                  \\\normalline         
				1           & -3                 \\\normalline         
				2           & 29                 \\\normalline        
				3           & 112                \\\normalline        
				4           & 778                \\\lastline          
			\end{tabular}
		\end{shortsolution}
	\end{subproblem}
	\begin{subproblem}
		$\left( \frac{j}{h} \right)$  
		\begin{shortsolution}
		\begin{tabular}[t]{S[table-format=1.0]S[table-format=2.0]}
				\beforeheading
				\heading{$x$} & \heading{$\left( \frac{j}{h} \right)(x)$} \\ \afterheading 
				-4          & 15                                      \\\normalline        
				-3          & \num{21/4}                            \\\normalline         
				-2          & 2                                       \\\normalline        
				-1          & \num{3/8}                             \\\normalline         
				0           & \num{-3/5}                            \\\normalline         
				1           & \num{-5/4}                            \\\normalline         
				2           & \num{15/14}                           \\\normalline        
				3           & 6                                       \\\normalline        
				4           & \num{380/9}                           \\\lastline          
			\end{tabular}
		\end{shortsolution}
	\end{subproblem}
	\begin{subproblem}
		$\left( \frac{j}{f} \right)$  
		\begin{shortsolution}
		\begin{tabular}[t]{S[table-format=1.0]S[table-format=1.0]}
				\beforeheading
				\heading{$x$} & \heading{$\left( \frac{j}{f} \right)(x)$} \\ \afterheading 
				-4          & \num{-15/28}                          \\\normalline        
				-3          & \num{-7/6}                            \\\normalline         
				-2          & X                                         \\\normalline        
				-1          & \num{3/4}                             \\\normalline         
				0           & X                                         \\\normalline         
				1           & \num{5/2}                             \\\normalline         
				2           & \num{-15/8}                           \\\normalline        
				3           & X                                         \\\normalline        
				4           & \num{95/3}                            \\\lastline          
			\end{tabular}
		\end{shortsolution}
	\end{subproblem}
\end{multicols}
\end{problem}



%===================================
%   Author: Hughes
%   Date:   October 2012
%===================================
\begin{problem}[Function algebra graphically]
Consider the functions $F$, $G$, $H$, and $J$ that have been graped in 
\cref{fun:fig:algebra}. Use the graphs to plot each of the following 
functions.
\begin{multicols}{4}
	\begin{subproblem}
		$F+G$ 
		\begin{shortsolution}
			The function $F+G$ is shown below.
			
			\begin{tikzpicture}
				\begin{axis}[
					framed,
					xmin=-3,xmax=3,
					ymin=-4,ymax=3,
					xtick={-2,0,...,2},
					ytick={-2,0,...,2},
					minor xtick={-1,1},
					minor ytick={-3,-1,1},
					grid=both,
					]
					\addplot[pccplot,-] expression[domain=-2:-1]{0};
					\addplot[pccplot,-] expression[domain=-1:0]{0};
					\addplot[pccplot,-] expression[domain=0:1]{0};
					\addplot[pccplot,-] expression[domain=1:2]{-3};
					\addplot[holdot]coordinates{(-2,0)(-1,0)(0,0)(1,0)(1,-3)(2,-3)};
				\end{axis}
			\end{tikzpicture}
		\end{shortsolution}
	\end{subproblem}
	\begin{subproblem}
		$G\cdot H$  
		\begin{shortsolution}
			The function $G\cdot H$ is shown below.
			
			\begin{tikzpicture}
				\begin{axis}[
					framed,
					xmin=-3,xmax=3,
					ymin=-3,ymax=3,
					xtick={-2,0,...,2},
					ytick={-2,0,...,2},
					minor xtick={-1,1},
					minor ytick={-1,1},
					grid=both,
					]
					\addplot[pccplot,-] expression[domain=-2:-1]{1};
					\addplot[pccplot,-] expression[domain=-1:0]{0};
					\addplot[pccplot,-] expression[domain=0:1]{1};
					\addplot[pccplot,-] expression[domain=1:2]{-2};
					\addplot[holdot]coordinates{(-2,1)(-1,1)(-1,0)(0,0)(0,1)(1,1)(1,-2)(2,-2)};
				\end{axis}
			\end{tikzpicture}
		\end{shortsolution}
	\end{subproblem}
	\begin{subproblem}
		$\frac{H}{J}$  
		\begin{shortsolution}
			The function $\frac{H}{J}$ is shown below; note that this function is
			undefined on the interval $(0,1)$.
			
			\begin{tikzpicture}
				\begin{axis}[
					framed,
					xmin=-3,xmax=3,
					ymin=-3,ymax=3,
					xtick={-2,0,...,2},
					ytick={-2,0,...,2},
					minor xtick={-1,1},
					minor ytick={-1,1},
					grid=both,
					]
					\addplot[pccplot,-] expression[domain=-2:-1]{-.5};
					\addplot[pccplot,-] expression[domain=-1:0]{0};
					\addplot[pccplot,-] expression[domain=1:2]{-1};
					\addplot[holdot]coordinates{(-2,-0.5)(-1,-0.5)(-1,0)(0,0)(1,-1)(2,-1)};
				\end{axis}
			\end{tikzpicture}
		\end{shortsolution}
	\end{subproblem}
	\begin{subproblem}
		$J-F$  
		\begin{shortsolution}
			The function $J-F$ is shown below.
			
			\begin{tikzpicture}
				\begin{axis}[
					framed,
					xmin=-3,xmax=3,
					ymin=-3,ymax=3,
					xtick={-2,0,...,2},
					ytick={-2,0,...,2},
					minor xtick={-1,1},
					minor ytick={-1,1},
					grid=both,
					]
					\addplot[pccplot,-] expression[domain=-2:-1]{1};
					\addplot[pccplot,-] expression[domain=-1:0]{-1};
					\addplot[pccplot,-] expression[domain=0:1]{1};
					\addplot[pccplot,-] expression[domain=1:2]{-1};
					\addplot[holdot]coordinates{(-2,1)(-1,1)(-1,-1)(0,-1)(0,1)(1,1)(1,-1)(2,-1)};
				\end{axis}
			\end{tikzpicture}
		\end{shortsolution}
	\end{subproblem}
\end{multicols}

\begin{figure}[!htb]
	\begin{widepage}
	\setlength{\figurewidth}{.2\textwidth}
	\centering
	\begin{subfigure}{\figurewidth}
		\begin{tikzpicture}
			\begin{axis}[
				framed,
				xmin=-3,xmax=3,
				ymin=-3,ymax=3,
				xtick={-2,0,...,2},
				ytick={-2,0,...,2},
				minor xtick={-1,1},
				minor ytick={-1,1},
				grid=both,
				]
				\addplot[pccplot,-] expression[domain=-2:-1]{1};
				\addplot[pccplot,-] expression[domain=-1:0]{2};
				\addplot[pccplot,-] expression[domain=0:1]{-1};
				\addplot[pccplot,-] expression[domain=1:2]{-2};
				\addplot[holdot]coordinates{(-2,1)(-1,1)(-1,2)(0,2)(0,-1)(1,-1)(1,-2)(2,-2)};
			\end{axis}
		\end{tikzpicture}
		\caption{$y=F(x)$}
		\label{fun:fig:algebra1}
	\end{subfigure}
	\hfill
	\begin{subfigure}{\figurewidth}
		\begin{tikzpicture}
			\begin{axis}[
				framed,
				xmin=-3,xmax=3,
				ymin=-3,ymax=3,
				xtick={-2,0,...,2},
				ytick={-2,0,...,2},
				minor xtick={-1,1},
				minor ytick={-1,1},
				grid=both,
				]
				\addplot[pccplot,-] expression[domain=-2:-1]{-1};
				\addplot[pccplot,-] expression[domain=-1:0]{-2};
				\addplot[pccplot,-] expression[domain=0:1]{1};
				\addplot[pccplot,-] expression[domain=1:2]{-1};
				\addplot[holdot]coordinates{(-2,-1)(-1,-1)(-1,-2)(0,-2)(0,1)(1,1)(1,-1)(2,-1)};
			\end{axis}
		\end{tikzpicture}
		\caption{$y=G(x)$}
		\label{fun:fig:algebra2}
	\end{subfigure}
	\hfill
	\begin{subfigure}{\figurewidth}
		\begin{tikzpicture}
			\begin{axis}[
				framed,
				xmin=-3,xmax=3,
				ymin=-3,ymax=3,
				xtick={-2,0,...,2},
				ytick={-2,0,...,2},
				minor xtick={-1,1},
				minor ytick={-1,1},
				grid=both,
				]
				\addplot[pccplot,-] expression[domain=-2:-1]{-1};
				\addplot[pccplot,-] expression[domain=-1:0]{0};
				\addplot[pccplot,-] expression[domain=0:1]{1};
				\addplot[pccplot,-] expression[domain=1:2]{2};
				\addplot[holdot]coordinates{(-2,-1)(-1,0)(-1,-1)(0,0)(0,1)(1,1)(1,2)(2,2)};
			\end{axis}
		\end{tikzpicture}
		\caption{$y=H(x)$}
		\label{fun:fig:algebra3}
	\end{subfigure}
	\hfill
	\begin{subfigure}{\figurewidth}
		\begin{tikzpicture}
			\begin{axis}[
				framed,
				xmin=-3,xmax=3,
				ymin=-3,ymax=3,
				xtick={-2,0,...,2},
				ytick={-2,0,...,2},
				minor xtick={-1,1},
				minor ytick={-1,1},
				grid=both,
				]
				\addplot[pccplot,-] expression[domain=-2:-1]{2};
				\addplot[pccplot,-] expression[domain=-1:0]{1};
				\addplot[pccplot,-] expression[domain=0:1]{0};
				\addplot[pccplot,-] expression[domain=1:2]{-2};
				\addplot[holdot]coordinates{(-2,2)(-1,2)(-1,1)(0,1)(0,0)(1,0)(1,-2)(2,-2)};
			\end{axis}
		\end{tikzpicture}
		\caption{$y=J(x)$}
		\label{fun:fig:algebra4}
	\end{subfigure}
	\caption{}
	\label{fun:fig:algebra}
	\end{widepage}
\end{figure}
\end{problem}


%===================================
%   Author: Hughes
%   Date:   October 2012
%===================================
\begin{problem}[Function algebra numerically]
\Cref{fun:tab:algebranum} shows some values of the functions $f$, $g$, 
and some functions obtained by using some function algebra on $f$ and $g$. 
Use the given values to complete \cref{fun:tab:algebranum}.
\begin{shortsolution}
	\begin{tabular}[t]{cS[table-format=2.0]*{4}S[table-format=1.0]S[parse-numbers=false]S[table-format=2.0]}
		\beforeheading
		$x$                             & -6          & -4 & -2 & 0  & 2  & 4     & 6  \\\normalline
		$f(x)$                          & 2           & 1  & 3  & 0  & 2  & \pi   & 12 \\\normalline
		$g(x)$                          & 8           & 1  & 3  & 5  & -1 & -1    & 2  \\
		\afterheading
		$(f+g)(x)$                      & 10          & 2  & 6  & 5  & 1  & \pi-1 & 14 \\\normalline 
		$(f-g)(x)$                      & -6          & 0  & 0  & -5 & 3  & \pi+1 & 10 \\\normalline
		$(f\cdot g)(x)$                 & 16          & 1  & 9  & 0  & -2 & -\pi  & 24 \\\normalline
		$\left( \frac{f}{g} \right)(x)$ & \num{1/4} & 1  & 1  & 0  & -2 & -\pi  & 6  \\\lastline
	\end{tabular}
\end{shortsolution}

\begin{table}[htb]
	\centering
	\caption{}
	\label{fun:tab:algebranum}
	\begin{tabular}{c*{7}S[table-format=1.0]}
		\beforeheading
		$x$                             & -6 & -4 & -2 & 0 & 2 & 4   & 6  \\\normalline
		$f(x)$                          & 2  &      & 3  &     &     & \pi &      \\\normalline
		$g(x)$                          & 8  & 1  &      & 5 &     & -1  &      \\
		\afterheading
		$(f+g)(x)$                      &      & 2  &      &     & 1 &       &      \\\normalline 
		$(f-g)(x)$                      &      &      &      &     & 3 &       & 10 \\\normalline
		$(f\cdot g)(x)$                 &      &      &      & 0 &     &       &      \\\normalline
		$\left( \frac{f}{g} \right)(x)$ &      &      & 1  &     &     &       & 6  \\\lastline
	\end{tabular}
\end{table}
\end{problem}

%===================================
%   Author: Hughes
%   Date:   October 2012
%===================================
\begin{problem}[Function algebra graphically]
Consider the functions $\alpha$, $\beta$, $\gamma$, and $\delta$ which 
have been graphed in \cref{fun:fig:algvarious}. 
Evaluate each of the following.
\begin{multicols}{4}
	\begin{subproblem}
		$(\alpha+\beta)(0)$
		\begin{shortsolution}
			$-2$ 
		\end{shortsolution}
	\end{subproblem}
	\begin{subproblem}
		$(\beta-\gamma)(3)$ 
		\begin{shortsolution}
			$5$ 
		\end{shortsolution}
	\end{subproblem}
	\begin{subproblem}
		$(\gamma\cdot\delta)(2)$ 
		\begin{shortsolution}
			$-2$ 
		\end{shortsolution}
	\end{subproblem}
	\begin{subproblem}
		$\left( \frac{\delta}{\alpha} \right)(0)$ 
		\begin{shortsolution}
			$\frac{1}{2}$ 
		\end{shortsolution}
	\end{subproblem}
\end{multicols}

\begin{figure}[!htb]
	\begin{widepage}
	\setlength{\figurewidth}{.2\textwidth}
	\centering
	\begin{subfigure}{\figurewidth}
		\begin{tikzpicture}
			\begin{axis}[
				framed,
				xmin=-5,xmax=5,
				ymin=-5,ymax=5,
				xtick={-4,-2,...,4},
				ytick={-4,-2,...,4},
				minor xtick={-3,-1,...,3},
				minor ytick={-3,-1,...,3},
				grid=both,
				]
				\addplot expression[domain=-3:3]{(x+2)*(x-2)};
			\end{axis}
		\end{tikzpicture}
		\caption{$y=\alpha(x)$}
	\end{subfigure}
	\hfill
	\begin{subfigure}{\figurewidth}
		\begin{tikzpicture}
			\begin{axis}[
				framed,
				xmin=-5,xmax=5,
				ymin=-5,ymax=5,
				xtick={-4,-2,...,4},
				ytick={-4,-2,...,4},
				minor xtick={-3,-1,...,3},
				minor ytick={-3,-1,...,3},
				grid=both,
				]
				\addplot expression[domain=-5:5]{2};
			\end{axis}
		\end{tikzpicture}
		\caption{$y=\beta(x)$}
	\end{subfigure}
	\hfill
	\begin{subfigure}{\figurewidth}
		\begin{tikzpicture}
			\begin{axis}[
				framed,
				xmin=-5,xmax=5,
				ymin=-5,ymax=5,
				xtick={-4,-2,...,4},
				ytick={-4,-2,...,4},
				minor xtick={-3,-1,...,3},
				minor ytick={-3,-1,...,3},
				grid=both,
				]
				\addplot expression[domain=-5:5]{-x};
			\end{axis}
		\end{tikzpicture}
		\caption{$y=\gamma(x)$}
	\end{subfigure}
	\hfill
	\begin{subfigure}{\figurewidth}
		\begin{tikzpicture}
			\begin{axis}[
				framed,
				xmin=-5,xmax=5,
				ymin=-5,ymax=5,
				xtick={-4,-2,...,4},
				ytick={-4,-2,...,4},
				minor xtick={-3,-1,...,3},
				minor ytick={-3,-1,...,3},
				grid=both,
				]
				\addplot expression[domain=-5:2.7]{2^x-3};
			\end{axis}
		\end{tikzpicture}
		\caption{$y=\delta(x)$}
	\end{subfigure}
	\caption{}
	\label{fun:fig:algvarious}
	\end{widepage}
\end{figure}
\end{problem}
\end{exercises}


\section{Piecewise-defined functions}
The functions that we have considered so far have had just one formula 
throughout their domain; for example, the quadratic function $q$ that 
has formula
\[
    q(x)=5-3x^2
\]
is defined for all real numbers.
\begin{marginfigure}
  \centering
\begin{tikzpicture}
  \begin{axis}[
    framed,
    xmin=-1,xmax=5,
    ymin=-1,ymax=2,
    xtick={1,2,...,4},
    ytick={1},
    xlabel={$t$},
    grid=major,
    ]
    \addplot+[-]expression[domain=0:1]{0};
    \addplot[pccplot,-]expression[domain=1:2]{1};
    \addplot[pccplot,-]expression[domain=2:3]{0};
    \addplot[pccplot,-]expression[domain=3:4]{1};
    \addplot[holdot]coordinates{(1,0)(2,1)(3,0)(4,1)};
    \addplot[soldot]coordinates{(0,0)(1,1)(2,0)(3,1)};
  \end{axis}
\end{tikzpicture}
\captionof{figure}{A switch function}
\label{fun:fig:electric}
\end{marginfigure}

There are many applications for which this is 
too restrictive; for example, electrical engineers often work with switches 
that are turned on (with a value of $1$) and off (with a value of $0$). 
An example of a function that might model such a switch over time, $t$, is shown 
in \cref{fun:fig:electric}.
It is clear that this function takes the value $0$ on some intervals, 
and $1$ on other intervals. We can write a formula for such a function
by first noting that is a \emph{piecewise-defined} function.


\begin{pccdefinition}[Piecewise-defined functions]
  A piecewise-defined function has different formulas for different parts
  of its domain. 

  The formula for a piecewise-defined function is written using a \emph{left brace}
  $\{$ and is read from top to bottom as we move from left to right 
  through its domain.
\end{pccdefinition}

%===================================
%   Author: Hughes
%   Date:   November 2012
%===================================
\begin{pccexample}
Find a formula for the function that is graphed in \cref{fun:fig:electric}. 
\begin{pccsolution}
  Let's assume that the function shown in \cref{fun:fig:electric} is called $f$. 
  It seems that $f(t)$ takes the value $0$ on the intervals $[0,1)$ and $[2,3)$; 
  similarly, $f(t)$ takes the value $1$ on the intervals $[1,2)$ and $[3,4)$.
  We can translate this into a formula for the function $f$ as follows
  \[
        f(t)=
        \begin{cases}
          0,& 0\leq t <1\\
          1,& 1\leq t<2\\
          0,& 2\leq t<3\\
          1,& 3\leq t<4
        \end{cases}
  \]
  Note that we use the left brace, $\{$, to link the formula together. Note 
  also that the domain of $f$ is $[0,4)$ and that as we read the formula from top
  to bottom, the values of $t$ go from left to right. This will be true in 
  every piecewise-defined function that we see.
\end{pccsolution}
\end{pccexample}

%===================================
%   Author: Hughes/Fresh/Barkin
%   Date:   November 2012
%===================================
\begin{pccexample}[Coupons] 
  \pccname{Wendy} is going shopping at \pccname{Jessica}'s
  beauty salon. Wendy has the coupons shown in \cref{fun:fig:coupons}. 
  Wendy is very interested in modeling the total amount of money that 
  she will spend after applying the discounts from the coupons. 

\begin{figure}[!htb]
\begin{widepage}
  \centering
\begin{subfigure}{.4\textwidth}
\resizebox{\textwidth}{!}{\begin{tikzpicture}
\node (reduction) at (-1,0)[scale=7] {\$5};
\node [right=of reduction](description)[scale=2]{\$5 off any purchase less than \$15};
\begin{pgfonlayer}{background}
\node (jessica) at (5,0.25)[scale=5,text=black!20] {\begin{tabular}{c}Jessica's\\ \emph{beauty}\\ salon\end{tabular}};
\node [fit=(reduction)(description)(jessica),
				draw=blue,thick,
				rounded corners,fill=red!15]{};
\end{pgfonlayer}
\end{tikzpicture}}
\caption{}
\end{subfigure}%
\hfill
\begin{subfigure}{.4\textwidth}
\resizebox{\textwidth}{!}{\begin{tikzpicture}
\node (reduction) at (-1,0)[scale=7] {20\%};
\node [right=of reduction](description)[scale=2]{20\% off any purchase \$15 or more};
\begin{pgfonlayer}{background}
\node [fit=(reduction)(description)(jessica),
				draw=blue,thick,
				rounded corners,fill=purple!35]{};
\node (jessica) at (5,0.25)[scale=5,text=black!20] {\begin{tabular}{c}Jessica's\\ 
\emph{beauty}\\salon \end{tabular}};
\end{pgfonlayer}
\end{tikzpicture}}
\caption{}
\end{subfigure}
\caption{Wendy's coupons}
\label{fun:fig:coupons}
\end{widepage}
\end{figure}

  Wendy observes that the amount of money that she will save depends 
  on the total cost of the items. She decides to let the function $d$ 
  represent the cost of the items after applying the discount to 
  items that cost $x$ dollars initially. Wendy realizes that she needs one
   formula for items that cost below \$15, and one for items 
   that cost \$15 or more; she decides to write a formula for $d(x)$ 
   using a piecewise-defined formula 
\[
    d(x)=
    \begin{cases}
      x-5, & 0<x<15\\
      0.8x, & x\geq 15
    \end{cases}
\]
Wendy decides to test her formula by finding how much an item that 
costs \$13 initially will cost after using the coupon. She evaluates $d(13)$
\begin{align*}
  d(13)&=13-5\\
  &=8
\end{align*}
The item will cost her \$8.

Wendy also uses her formula to find her savings on a \$40 item by 
evaluating
\begin{align*}
  d(40)&=0.8\cdot 40\\
  &=32
\end{align*}
and concludes that she will save \$8 using her coupon.
\end{pccexample}

%===================================
%   Author: Hughes
%   Date:   November 2012
%===================================
\begin{pccexample}[Function evaluation]
Let $g$ be the piecewise-defined function that has formula  
\[
    g(x)=
    \begin{cases}
      -13, & x\leq -4\\
      2, & -4<x < 3\\
      7, &  x>3
    \end{cases}
\]
Evaluate each of the following
\begin{multicols}{5}
  \begin{enumerate}
    \item $g(-5)$
    \item $g(-4)$
    \item $g(0)$
    \item $g(3)$
    \item $g(53)$
  \end{enumerate}
\end{multicols}
\begin{pccsolution}
  \begin{enumerate}
    \item To evaluate $g(-5)$ we first need to identify which part of the 
      domain is appropriate. Since $-5\leq -4$, we use the formula in the 
      \emph{first} row of $g(x)$, and therefore
      \[
            g(-5)=-13
      \]
    \item Since $-4\leq -4$, we use the \emph{first} row in the formula for $g(x)$ again, so
      \[
            g(-4)=-13
      \]
    \item Since $-4<0<3$ we need to use the \emph{second} row in the formula for $g(x)$, so
      \[
            g(0)=2
      \]
    \item To evaluate $g(3)$ we need to find the appropriate interval in the formula
      for $g(x)$. Notice that $3$ does not fall into any of the intervals! This means
      that $g(3)$ is undefined.
    \item We note that $53>3$, so we need to use the \emph{third} row of the formula for $g(x)$, so
      \[
            g(53)=7
      \]
  \end{enumerate}
\end{pccsolution}
\end{pccexample}

%===================================
%   Author: Hughes
%   Date:   November 2012
%===================================
\begin{pccexample}
  \fixthis{more complictaed piecewise formula using radicals, quadratics etc} 
  \[
        f(t)=
        \begin{cases}
          t^2, & t<-3\\
          4-5t, & -3\leq t< 6\\
          \sqrt{t} & t>6
        \end{cases}
  \]
\end{pccexample}

\begin{exercises}
%===================================
%   Author: Hughes
%   Date:   October 2012
%===================================
\begin{problem}[Find a formula from a graph]
Consider the functions $F$, $G$, $H$, and $J$ that have been graped in 
\cref{fun:fig:piecewise}. Find a formula for 
each function.
\begin{multicols}{4}
	\begin{subproblem}
		$F$  
		\begin{shortsolution}
			$
			F(x)=
			\begin{cases}
				1,  & -2<x<-1 \\
				2,  & -1<x<0  \\
				-1, & 0<x<1   \\
				-2, & 1<x<2   
			\end{cases}
			$
		\end{shortsolution}
	\end{subproblem}
	\begin{subproblem}
		$G$
		\begin{shortsolution}
			$
			G(x)=
			\begin{cases}
				-1, & -2<x<-1 \\
				-2, & -1<x<0  \\
				1,  & 0<x<1   \\
				-1, & 1<x<2   
			\end{cases}
			$
		\end{shortsolution}
	\end{subproblem}
	\begin{subproblem}
		$H$
		\begin{shortsolution}
			$
			H(x)=
			\begin{cases}
				-1, & -2<x<-1 \\
				0,  & -1<x<0  \\
				1,  & 0<x<1   \\
				2,  & 1<x<2   
			\end{cases}
			$
		\end{shortsolution}
	\end{subproblem}
	\begin{subproblem}
		$J$
		\begin{shortsolution}
			$
			J(x)=
			\begin{cases}
				2,  & -2<x<-1 \\
				1,  & -1<x<0  \\
				0,  & 0<x<1   \\
				-2, & 1<x<2   
			\end{cases}
			$
		\end{shortsolution}
	\end{subproblem}
\end{multicols}

\begin{figure}[!htb]
	\begin{widepage}
	\setlength{\figurewidth}{.2\textwidth}
	\centering
	\begin{subfigure}{\figurewidth}
		\begin{tikzpicture}
			\begin{axis}[
				framed,
				xmin=-3,xmax=3,
				ymin=-3,ymax=3,
				xtick={-2,0,...,2},
				ytick={-2,0,...,2},
				minor xtick={-1,1},
				minor ytick={-1,1},
				grid=both,
				]
				\addplot[pccplot,-] expression[domain=-2:-1]{1};
				\addplot[pccplot,-] expression[domain=-1:0]{2};
				\addplot[pccplot,-] expression[domain=0:1]{-1};
				\addplot[pccplot,-] expression[domain=1:2]{-2};
				\addplot[holdot]coordinates{(-2,1)(-1,1)(-1,2)(0,2)(0,-1)(1,-1)(1,-2)(2,-2)};
			\end{axis}
		\end{tikzpicture}
		\caption{$y=F(x)$}
		\label{fun:fig:piecewise1}
	\end{subfigure}
	\hfill
	\begin{subfigure}{\figurewidth}
		\begin{tikzpicture}
			\begin{axis}[
				framed,
				xmin=-3,xmax=3,
				ymin=-3,ymax=3,
				xtick={-2,0,...,2},
				ytick={-2,0,...,2},
				minor xtick={-1,1},
				minor ytick={-1,1},
				grid=both,
				]
				\addplot[pccplot,-] expression[domain=-2:-1]{-1};
				\addplot[pccplot,-] expression[domain=-1:0]{-2};
				\addplot[pccplot,-] expression[domain=0:1]{1};
				\addplot[pccplot,-] expression[domain=1:2]{-1};
				\addplot[holdot]coordinates{(-2,-1)(-1,-1)(-1,-2)(0,-2)(0,1)(1,1)(1,-1)(2,-1)};
			\end{axis}
		\end{tikzpicture}
		\caption{$y=G(x)$}
		\label{fun:fig:piecewise2}
	\end{subfigure}
	\hfill
	\begin{subfigure}{\figurewidth}
		\begin{tikzpicture}
			\begin{axis}[
				framed,
				xmin=-3,xmax=3,
				ymin=-3,ymax=3,
				xtick={-2,0,...,2},
				ytick={-2,0,...,2},
				minor xtick={-1,1},
				minor ytick={-1,1},
				grid=both,
				]
				\addplot[pccplot,-] expression[domain=-2:-1]{-1};
				\addplot[pccplot,-] expression[domain=-1:0]{0};
				\addplot[pccplot,-] expression[domain=0:1]{1};
				\addplot[pccplot,-] expression[domain=1:2]{2};
				\addplot[holdot]coordinates{(-2,-1)(-1,0)(-1,-1)(0,0)(0,1)(1,1)(1,2)(2,2)};
			\end{axis}
		\end{tikzpicture}
		\caption{$y=H(x)$}
		\label{fun:fig:piecewise3}
	\end{subfigure}
	\hfill
	\begin{subfigure}{\figurewidth}
		\begin{tikzpicture}
			\begin{axis}[
				framed,
				xmin=-3,xmax=3,
				ymin=-3,ymax=3,
				xtick={-2,0,...,2},
				ytick={-2,0,...,2},
				minor xtick={-1,1},
				minor ytick={-1,1},
				grid=both,
				]
				\addplot[pccplot,-] expression[domain=-2:-1]{2};
				\addplot[pccplot,-] expression[domain=-1:0]{1};
				\addplot[pccplot,-] expression[domain=0:1]{0};
				\addplot[pccplot,-] expression[domain=1:2]{-2};
				\addplot[holdot]coordinates{(-2,2)(-1,2)(-1,1)(0,1)(0,0)(1,0)(1,-2)(2,-2)};
			\end{axis}
		\end{tikzpicture}
		\caption{$y=J(x)$}
		\label{fun:fig:piecewise4}
	\end{subfigure}
	\caption{}
	\label{fun:fig:piecewise}
	\end{widepage}
\end{figure}
\end{problem}
\end{exercises}
%%+*** 111,112document.tex
% arara: indent: {overwrite: on, trace: on, localSettings: yes}
%===================================
%
%   Last edited: Hughes
%                11/18/12 (v9)
%
%===================================
\chapter{Logarithms}
\minitoc
\section{Logarithmic functions}
In our chapter on exponential functions we considered applications that 
lead to equations such as
\[
	10^x=19
\]
We can approximate solutions to such equations using graphical and 
numerical techniques. How can we solve these equations \emph{algebraically} 
though? The answer is to use \emph{logarithmic} functions.

\begin{pccdefinition}[The logarithm function]\label{log:def:logfunctions}
	The logarithmic function, base $b$, where $b>0$ and $b\ne 1$, is 
	defined by
	\[
		y=\log_b(x)
	\]
	if, and only if, 
	\[
		b^y=x
	\]
	The domain of the logarithmic function $y=\log_b(x)$ is $(0,\infty)$, and 
	the range is $(-\infty,\infty)$.
\end{pccdefinition}

\Cref{log:def:logfunctions} says that if we are given a \emph{logarithmic} 
expression then we can convert it into an equivalent \emph{exponential} 
expression. This is useful when evaluating logarithmic expressions.

%===================================
%   Author: Hughes
%   Date:   July 2012
%===================================
\begin{pccexample}
	Use a sentence to describe each of the following logarithmic expressions, and 
	then evaluate each expression
	\begin{multicols}{4}
		\begin{enumerate}
			\item $\log_2(32)$
			\item $\log_3(81)$
			\item $\log_5(25)$
			\item $\log_{73}(1)$
		\end{enumerate}
	\end{multicols}
	\begin{pccsolution}
		\begin{enumerate}
			\item The logarithm, base $2$, of $32$. In order to evaluate the expression, we need 
			to answer the question: what power do we raise $2$ to get $32$? The answer is $5$, so
			\[
				\log_2(32)=5
			\]
			\item The logarithm, base $3$, of $81$. What power do we raise $3$ to get $81$? The 
			answer is $4$, so
			\[
				\log_3(81)=4
			\]
			\item The logarithm, base $5$, of $25$. What power do we raise $5$ to get $25$? The 
			answer is $2$, so
			\[
				\log_5(25)=2
			\]
			\item The logarithm, base $73$, of $1$. We need to raise $73$ to the power $0$ 
			to get $1$, so
			\[
				\log_{73}(1)=0
			\]
		\end{enumerate}
	\end{pccsolution}
\end{pccexample}

%===================================
%   Author: Hughes
%   Date:   July 2012
%===================================
\begin{pccexample} 
	Convert each of the following exponential equations into their equivalent logarithm form
	\begin{multicols}{4}
		\begin{enumerate}
			\item $3^{5}=243$
			\item $7^{0}=1$
			\item $16^{\nicefrac{1}{2}}=4$
			\item $33^{-1}=\frac{1}{33}$
		\end{enumerate}
	\end{multicols}
	\begin{pccsolution}
		\begin{enumerate}
			\item $3^{5}=243$ is equivalent to 
			\[
				\log_3(243)=5
			\]
			\item $7^{0}=1$ is equivalent to 
			\[
				\log_7(1)=0
			\]
			\item $16^{\nicefrac{1}{2}}=4$ is equivalent to 
			\[
				\log_{16}(4)=\frac{1}{2}
			\]
			\item $33^{-1}=\frac{1}{33}$ is equivalent to
			\[
				\log_{33}\left( \frac{1}{33} \right)=-1
			\]
		\end{enumerate}
	\end{pccsolution}
\end{pccexample}
%===================================
%   Author: Hughes
%   Date:   July 2012
%===================================
\begin{pccexample}
	Convert each of the following logarithmic equations into their equivalent exponential form
	\begin{multicols}{4}
		\begin{enumerate}
			\item $\log_4\left( \frac{1}{4} \right)=-1$
			\item $\log_{6}(36)=2$
			\item $\log_{\frac{1}{2}}(4)=-2$
			\item $\log_{e}(e^8)=8$
		\end{enumerate}
	\end{multicols}
	\begin{pccsolution}
		\begin{enumerate}
			\item $\log_4\left( \frac{1}{4} \right)=-1$ is equivalent to
			\[
				4^{-1}=\frac{1}{4}
			\]
			\item $\log_{6}(36)=2$ is equivalent to
			\[
				6^2=36
			\]
			\item $\log_{\frac{1}{2}}(4)=-2$ is equivalent to 
			\[
				\left(\frac{1}{2}\right)^{-2}=4
			\]
			\item $\log_{e}(e^8)=8$ is equivalent to
			\[
				e^8=e^8
			\]
			In fact when evaluating a logarithm base $e$ we use a special notation, as we'll soon 
			see.
		\end{enumerate}
		\mbox{}
	\end{pccsolution}
\end{pccexample}

We have been able to perform all of our calculations so far using our knowledge of
arithmetic and exponents.  When faced with a 
logarithmic calculation that goes beyond this, we need to use a calculator
to compute the value. Most modern calculators can work in any base, 
but of all the possible choices that we have available there are  
two bases that are particularly important.

\begin{pccdefinition}[The common and natural logarithm functions]
	When working with logarithmic functions that have base $b$ and formula $y=\log_b(x)$,
	\begin{itemize}
		\item the \emph{common} logarithmic function has base $10$ and is 
		written as
		\[
			y=\log(x)
		\]
		\item the \emph{natural} logarithmic function has base $e$ and is
		written as
		\[
			y=\ln(x)
		\]
		It may help to recall from \vref{exp:def:e} that $e$ is called the \emph{natural} base.
	\end{itemize}
\end{pccdefinition}
%===================================
%   Author: Hughes
%   Date:   July 2012
%===================================
\begin{pccexample}[Domain]
	Find the domain of each the functions implied by the following formulas
	\begin{multicols}{2}
		\begin{enumerate}
			\item $f(x)=\log(x)$
			\item $g(x)=\log_3(2+x)$
			\item $h(x)=\ln(4x-5)$
			\item $j(x)=\log_7(x^2)$
		\end{enumerate}
	\end{multicols}
	\begin{pccsolution}
		\begin{enumerate}
			\item The domain of $f$ is $(0,\infty)$. Note that the base of $f$ is $10$; 
			$f$ is the common logarithmic function.
			\item To find the domain of $g$ we need to solve the inequality $2+x>0$. 
			The domain of $g$ is, therefore, $(-2,\infty)$.
			\item To find the domain of $h$ we need to solve the inequality $4x-5>0$.
			The domain of $h$ is $\left( \frac{5}{4},\infty \right)$. Note that the 
			base of $h$ is $e$; $h$ is the natural logarithmic function.
			\item To find the domain of $j$ we need to solve the inequality $x^2>0$. 
			The domain of $g$ is therefore $(-\infty,0)\cup (0,\infty)$.
		\end{enumerate}
	\end{pccsolution}
\end{pccexample}

One of the implications of \cref{log:def:logfunctions} is that there is 
a relationship between logarithmic functions and exponential functions. 
Explicitly, if $f$ is the exponential function that has formula
\[
	f(x)=b^x
\]
then the inverse function, $f^{-1}$, has formula
\[
	f^{-1}(x)=\log_b(x)
\]
We can use our knowledge of inverse functions (see \fixthis{insert vref reference to inverse function
section- doesn't exist yet!}) to help us graph logarithmic functions.

%===================================
%   Author: Hughes
%   Date:   July 2012
%===================================
\begin{pccexample}[Graphing]\label{log:ex:graphing}
	Use your knowledge of the function $f$ that has formula $f(x)=2^x$ 
	to help you graph its inverse function, $f^{-1}$, that has formula
	$f^{-1}(x)=\log_2(x)$.
	\begin{pccsolution}
		Let's start by constructing a table of values of the function $f$ in 
		\cref{log:tab:fandinverse}. We can easily construct a table of 
		values of $f^{-1}(x)$ by simply swapping the input and output values, 
		which we have done in \cref{log:tab:finverse}.
		
		\begin{table}[!htb]
			\renewcommand{\arraystretch}{1.25}
			\begin{minipage}{.5\textwidth}
				\centering
				\caption{$f$}
				\label{log:tab:fandinverse}
				\begin{tabular}{S[table-format=1.0]S[table-format=1.0]}
					\beforeheading 
					\heading{$x$} & \heading{$f(x)$} \\ \afterheading   
					\afterheading
					-3            & \num{1/8}        \\\normalline
					-2            & \num{1/4}        \\\normalline
					-1            & \num{1/2}        \\\normalline
					0             & 1                \\\normalline
					1             & 2                \\\normalline
					2             & 4                \\\normalline
					3             & 8                \\\lastline
				\end{tabular}
			\end{minipage}%
			\begin{minipage}{.5\textwidth}
				\centering
				\caption{$f^{-1}$}
				\label{log:tab:finverse}
				\begin{tabular}{S[table-format=1.0]S[table-format=1.0]}
					\beforeheading 
					\heading{$x$} & \heading{$f^{-1}(x)$} \\
					\afterheading
					\num{1/8}     & -3                    \\\normalline
					\num{1/4}     & -2                    \\\normalline
					\num{1/2}     & -1                    \\\normalline
					1             & 0                     \\\normalline
					2             & 1                     \\\normalline
					4             & 2                     \\\normalline
					8             & 3                     \\\lastline
				\end{tabular}
			\end{minipage}
		\end{table}
		
		If we plot the values we obtained in \cref{log:tab:fandinverse,log:tab:finverse} 
		and connect them using a smooth curve, then we obtain the curves given in \cref{log:fig:fandinverse}.
		\begin{figure}[!htb]
			\centering
			\begin{tikzpicture}
				\begin{axis}[
						xmin=-10,xmax=10,
						ymin=-10,ymax=10,
						width=.5\textwidth,
					]
					\addplot expression[domain=-10:3.32192]{2^x}node[axisnode,anchor=north west]{$y=2^x$};
					\addplot[soldot] coordinates{(-3,1/8)(-2,1/4)(-1,1/2)(0,1)(1,2)(2,4)(3,8)};
					\addplot[pccplot] expression[domain=0.001:10,samples=100]{ln(x)/ln(2)}node[axisnode,anchor=south east]{$y=\log_2(x)$};
					\addplot[soldot] coordinates{(1/8,-3)(1/4,-2)(1/2,-1)(1,0)(2,1)(4,2)(8,3)};
					\addplot[pccplot,dashed] expression[domain=-10:10]{x}node[axisnode,anchor=east,pos=0.2]{$y=x$};
				\end{axis}
			\end{tikzpicture}
			\caption{}
			\label{log:fig:fandinverse}
		\end{figure}
		
		There are a few more observations that we can make about $f$ and its inverse, using \cref{log:fig:fandinverse} 
		as a guide:
		\begin{itemize}
			\item the domain of $f$ is $(-\infty,\infty)$, and the range of $f$ is $(0,\infty)$; this means
			that the domain of $f^{-1}$ is $(0,\infty)$, and the range of $f^{-1}$ is $(-\infty,\infty)$;
			\item the function $f$ has a \emph{horizontal} asymptote with equation $y=0$; this necessarily
			means that the function $f^{-1}$ has a \emph{vertical} asymptote with equation $x=0$;
			\item the function $f$ does not have a \emph{vertical} asymptote| this therefore
			implies that the function $f^{-1}$ does not have a \emph{horizontal} asymptote;
			\item the curves of $f$ and $f^{-1}$ are symmetric about the line $y=x$.
		\end{itemize}
	\end{pccsolution}
\end{pccexample}


\begin{doyouunderstand}
	\begin{problem}
	Repeat \cref{log:ex:graphing} using the function $f$ that has formula $f(x)=3^x$.
	\begin{shortsolution}
		\begin{tabular}[t]{S[table-format=1.0]S[table-format=2.0]}
			\beforeheading 
			\heading{$x$} & \heading{$f(x)$} \\
			\afterheading
			-3            & \num{1/27}       \\\normalline
			-2            & \num{1/9}        \\\normalline
			-1            & \num{1/3}        \\\normalline
			0             & 1                \\\normalline
			1             & 3                \\\normalline
			2             & 9                \\\normalline
			3             & 27               \\\lastline
		\end{tabular}
				
		\begin{tabular}{S[table-format=1.0]S[table-format=1.0]}
			\beforeheading 
			\heading{$x$} & \heading{$f^{-1}(x)$} \\
			\afterheading
			\num{1/27}    & -3                    \\\normalline
			\num{1/9}     & -2                    \\\normalline
			\num{1/3}     & -1                    \\\normalline
			1             & 0                     \\\normalline
			3             & 1                     \\\normalline
			9             & 2                     \\\normalline
			27            & 3                     \\\lastline
		\end{tabular}
				
		\begin{tikzpicture}
			\begin{axis}[
					xmin=-5,xmax=5,
					ymin=-30,ymax=30,
				]
				\addplot expression[domain=-5:3.0959]{3^x}node[axisnode,anchor=north west]{$y=3^x$};
				\addplot[soldot] coordinates{(-3,1/27)(-2,1/9)(-1,1/3)(0,1)(1,3)(2,9)(3,27)};
				\addplot[pccplot] expression[domain=0.0000000001:5,samples=100]{ln(x)/ln(3)}node[axisnode,pos=0,anchor=south west]{$y=\log_3(x)$};
				\addplot[soldot] coordinates{(1/27,-3)(1/9,-2)(1/3,-1)(1,0)(3,1)(9,2)(27,3)};
			\end{axis}
		\end{tikzpicture}
	\end{shortsolution}
	\end{problem}
\end{doyouunderstand}

There is a strong relationship between the logarithmic function $f$ that has formula
$f(x)=\log_b(x)$ and its inverse exponential function $f^{-1}$ that has formula
$f^{-1}(x)=b^x$. We can think of both functions as a type of \emph{mapping} from
their domains to their respective ranges. There are many possible ways to visualize
the mapping| one such image is shown in \cref{log:fig:mapping}. Notice that
the mapping lends itself well to highlighting \cref{log:prop:inv1,log:prop:inv2}, 
which detail the composition of logarithmic and exponential functions
\[
	(f\circ f^{-1})(x)=(f^{-1}\circ f)(x)=x
\]

\begin{figure}[!htb]
	\centering
	\begin{tikzpicture}
		% set up nodes
		\node (domain) at (0,0){$(0,\infty)$};
		\node (range) at (7,0){$(-\infty,\infty)$};
		\node[text=blue] (f) at (3.5,2) {$f$};
		\node[text=red] (finv) at (3.5,-2) {$f^{-1}$};
		% connect them
		\draw[blue,very thick] (domain) to[bend left=25] (f);
		\draw[->,very thick,blue] (f) to[bend left=25] (range);
		\draw[red,very thick] (range) to[bend left=25] (finv);
		\draw[->,red,very thick] (finv) to[bend left=25] (domain);
	\end{tikzpicture}
	\caption{Visualizing the mappings of $f$ and $f^{-1}$, where $f$ 
		has formula $f(x)=\log_b(x)$ and $f^{-1}$ has formula $f^{-1}(x)=b^x$.}
	\label{log:fig:mapping}
\end{figure}

Our examples so far have concentrated on familiarizing ourselves
with logarithmic functions but we have yet to see an application. 
The logarithmic functions have a myriad of applications| in particular, 
they can be used to help us study examples  that otherwise could 
only be attempted from a graphical or numerical perpesctive.

%===================================
%   Author: Hughes
%   Date:   July 2012
%===================================
\begin{pccexample}
	The number of radioactive atoms in a sample of Carbon-14 decays according to the model
	\[
		Q(t)= Q_0\,e^{-0.000120968t},
	\]
	where $Q_0$ is the initial mass of the radioactive atoms and $Q(t)$ is the mass of radioactive atoms $t$ years after the sample was established.
	
	What is the half-life of the sample?
	\begin{pccsolution}
		We need to find the value of $t$ that satisfies the equation $Q(t)=\frac{Q_0}{2}$. 
		We proceed using the following steps
		\begin{align*}
			\frac{Q_0}{2}=Q_0\,e^{-0.000120968t} & \Rightarrow \frac{1}{2}=e^{-0.000120968t}                            \\
			                                     & \Rightarrow \ln\left( \frac{1}{2} \right) = -0.000120968t            \\
			                                     & \Rightarrow t = -\frac{1}{-0.000120968}\ln\left( \frac{1}{2} \right) \\
			                                     & \phantom{ {}\Rightarrow t} = 5370                                    
		\end{align*}
		We conclude that the half-life of the sample is $5370$ years.
	\end{pccsolution}
\end{pccexample}

%===================================
%   Author: Neft (Hughes)
%   Date:   August 2012
%===================================
\begin{pccexample}[The RC circuit]
	A \emph{capacitor} is a device that stores electrical energy in the form of charged particles.
	The \emph{voltage} on the capacitor is a result of the electric field created by the particles and
	is proportional to the amount of charge stored. A \emph{resistor} is a device that dissipates
	electrical energy. If a capacitor is charged up and then connected across a resistor, the
	capacitor discharges and the voltage drops.
	
	The voltage (in \si{\volt}), on the capacitor as it is being discharged is modeled by the
	function $V$ that has formula 
	\[
		V(t)=V_0e^{-\frac{t}{RC}}
	\]
	where $V_0$ is the initial capacitor voltage, $R$ is the value of the
	resistor (in \si{\ohm}), $C$ is the value of the capacitor (in \si{\farad}) and $t$ is time (in \si{\second}).
	\begin{enumerate}
		\item Suppose that a $1.0\times 10^{-6}$-farad capacitor, initially charged to 
		$\SI{12}{\volt}$, is connected across a $\SI{10,000}{\ohm}$-resistor. 
		How long will it take for the voltage on the capacitor to drop to half 
		of its original value?
		\item Suppose the capacitor is initially charged to $\SI{20}{\volt}$. How long
		will it take for the voltage to drop to one half of its original value?
		\item Suppose the capacitor is initially charged up to $\SI{100}{\volt}$. How 
		long will it take for the voltage to drop to one half of its original value?
		\item What effect will doubling the \emph{resistance} have on the time it takes for the voltage to
		drop to one half of its initial value?
	\end{enumerate}
	\begin{pccsolution}
		\begin{enumerate}
			\item We need to solve the equation $\frac{1}{2}V_0=V_0e^{-\frac{t}{RC}}$:
			\begin{align*}
				6 = 12 e^{-100t} & \Rightarrow \frac{1}{2}=e^{-100t}                       \\
				                 & \Rightarrow \ln\left(\frac{1}{2}\right)=-100t           \\
				                 & \Rightarrow t=-\frac{1}{100}\ln\left(\frac{1}{2}\right) \\
				                 & \phantom{ {}\Rightarrow t}\approx 0.007                 
			\end{align*}
			It takes about $\SI{0.007}{\second}$ for the voltage of the capacitor to 
			reach one half of its initial value.
			\item We need to solve the equation $\frac{1}{2}V_0=V_0e^{-\frac{t}{RC}}$:
			\begin{align*}
				10 = 20 e^{-100t} & \Rightarrow \frac{1}{2}=e^{-100t}                       \\
				                  & \Rightarrow \ln\left(\frac{1}{2}\right)=-100t           \\
				                  & \Rightarrow t=-\frac{1}{100}\ln\left(\frac{1}{2}\right) \\
				                  & \phantom{ {}\Rightarrow t}\approx 0.007                 
			\end{align*}
			It takes about $\SI{0.007}{\second}$ for the voltage of the capacitor to 
			reach one half of its initial value. Does this sound familiar?
			\item We need to solve the equation $\frac{1}{2}V_0=V_0e^{-\frac{t}{RC}}$:
			\begin{align*}
				50 = 100 e^{-100t} & \Rightarrow \frac{1}{2}=e^{-100t}                       \\
				                   & \Rightarrow \ln\left(\frac{1}{2}\right)=-100t           \\
				                   & \Rightarrow t=-\frac{1}{100}\ln\left(\frac{1}{2}\right) \\
				                   & \phantom{ {}\Rightarrow t}\approx 0.007                 
			\end{align*}
			It takes about $\SI{0.007}{\second}$ for the voltage of the capacitor to 
			reach one half of its initial value. There seems to be a pattern here\ldots
			\item If we double the resistance to $\SI{20,000}{\ohm}$ then we need 
			to solve the equation $\frac{1}{2}V_0=V_0e^{-50t}$; note that the value of $V_0$
			does not affect our calculations
			\begin{align*}
				\frac{1}{2}V_0 = V_0 e^{-50t} & \Rightarrow \frac{1}{2}=e^{-50t}                       \\
				                              & \Rightarrow \ln\left(\frac{1}{2}\right)=-50t           \\
				                              & \Rightarrow t=-\frac{1}{50}\ln\left(\frac{1}{2}\right) \\
				                              & \phantom{ {}\Rightarrow t}\approx 0.014                
			\end{align*}
			We conclude that doubling the resistance doubles the time  it takes (to about $\SI{0.014}{\second}$) 
			for the voltage on the capacitor to reach half of its initial value.
		\end{enumerate}
	\end{pccsolution}
\end{pccexample}


\begin{exercises}
%===================================
%   Author: Hughes
%   Date:   July 2012
%===================================
\begin{problem}[Domain]
Find the domain of each of the functions implied by the following formulas. 
\begin{multicols}{4}
	\begin{subproblem}
		$f(x)=\log_4(x+7)$ 
		\begin{shortsolution}
			$(-7,\infty)$
		\end{shortsolution}
	\end{subproblem}
	\begin{subproblem}
		$g(x)=\log_9(x-2)$ 
		\begin{shortsolution}
			$(2,\infty)$
		\end{shortsolution}
	\end{subproblem}
	\begin{subproblem}
		$h(x)=5\log(3x)$ 
		\begin{shortsolution}
			$(0,\infty)$
		\end{shortsolution}
	\end{subproblem}
	\begin{subproblem}
		$j(x)=8-\log_2(4x+3)$ 
		\begin{shortsolution}
			$\left( -\frac{3}{4},\infty \right)$
		\end{shortsolution}
	\end{subproblem}
	\begin{subproblem}
		$k(x)=\log_6(x^2-9)$ 
		\begin{shortsolution}
			$(-\infty,-3)\cup (3,\infty)$
		\end{shortsolution}
	\end{subproblem}
	\begin{subproblem}
		$l(x)=3\log_8(4-2x^2)$ 
		\begin{shortsolution}
			$(-\sqrt{2},\sqrt{2})$
		\end{shortsolution}
	\end{subproblem}
	\begin{subproblem}
		$m(x)=\ln(2^x)$ 
		\begin{shortsolution}
			$(-\infty,\infty)$
		\end{shortsolution}
	\end{subproblem}
	\begin{subproblem}
		$n(x)=2^{\log(x)}$ 
		\begin{shortsolution}
			$(0,\infty)$
		\end{shortsolution}
	\end{subproblem}
\end{multicols}
\end{problem}
%===================================
%   Author: Hughes
%   Date:   July 2012
%===================================
\begin{problem}[Transformations: given the formula, describe the transformation]
Describe each of the functions $g$, $h$, $j$, and $k$ in terms of transformations 
of the logarithmic function $f$ that has formula $f(x)=\log(x)$. State the domain 
of each function.
\begin{multicols}{4}
	\begin{subproblem}
		$g(x)=\log(x+3)$ 
		\begin{shortsolution}
			$g$ is the function $f$ shifted to the left by $3$ units. The domain
			of $g$ is $(-3,\infty)$.
		\end{shortsolution}
	\end{subproblem}
	\begin{subproblem}
		$h(x)=\log(x-5)$ 
		\begin{shortsolution}
			$h$ is the function $f$ shifted to the right by $5$ units. The domain
			of $h$ is $(5,\infty)$.
		\end{shortsolution}
	\end{subproblem}
	\begin{subproblem}
		$j(x)=\log(2(x+7))$ 
		\begin{shortsolution}
			$j$ is the function $f$ horizontally compressed by a factor of $2$, 
			and shifted to the left by $7$ units. The domain of $j$ is $(-7,\infty)$.
		\end{shortsolution}
	\end{subproblem}
	\begin{subproblem}
		$k(x)=5\log(-x)$ 
		\begin{shortsolution}
			$k$ is the function $f$ reflected across the vertical axis, and vertically
			stretched by a factor of $5$. The domain of $k$ is $(-\infty,0)$.
		\end{shortsolution}
	\end{subproblem}
\end{multicols}
\end{problem}

%===================================
%   Author: Hughes
%   Date:   July 2012
%===================================
\begin{problem}[Transformations: given the transformation, find the formula]
Let $f$ be the function that has formula $f(x)=\log(x)$. In each of the following 
problems apply the given transformation to the function $f$ and 
write a formula for the transformed version of $f$.
\begin{multicols}{2}
	\begin{subproblem}
		Shift $f$ to the right by $2$ units. 
		\begin{shortsolution}
			$f(x-2)=\log(x-2)$
		\end{shortsolution}
	\end{subproblem}
	\begin{subproblem}
		Shift $f$ to the left by $5$ units. 
		\begin{shortsolution}
			$f(x+5)=\log(x+5)$
		\end{shortsolution}
	\end{subproblem}
	\begin{subproblem}
		Shift $f$ up by 11 units. 
		\begin{shortsolution}
			$f(x)+11=\log(x)+11$
		\end{shortsolution}
	\end{subproblem}
	\begin{subproblem}
		Shift $f$ down by 1 unit. 
		\begin{shortsolution}
			$f(x)-1=\log(x)-1$
		\end{shortsolution}
	\end{subproblem}
\end{multicols}
\end{problem}

%===================================
%   Author: Hughes
%   Date:   July 2012
%===================================
\begin{problem}[Find the base from graphs]\label{log:prob:findbase}
Consider the functions graphed in \cref{log:fig:findbase}. Each function 
has a formula of the form $y=\log_b(x+a)$, where $b$ is the base, and 
$a$ is given for each function. Use the ordered pair given in each graph to find the base, $b$.
\begin{shortsolution}
	\begin{itemize}
		\item \Vref{log:fig:findbase1}: $b=2$, so $y=\log_2(x)$;
		\item \Vref{log:fig:findbase2}: $b=4$, so $y=\log_4(x+3)$;
		\item \Vref{log:fig:findbase3}: $b=\frac{1}{2}$, so $y=\log_{\frac{1}{2}}(x-4)$;
		\item \Vref{log:fig:findbase4}: $b=\frac{1}{3}$, so $y=\log_{\frac{1}{3}}(x+2)$.
	\end{itemize}
\end{shortsolution}

\begin{figure}[htb]
	\begin{widepage}
	\begin{subfigure}{.2\textwidth}
		\centering
		\begin{tikzpicture}
			\begin{axis}[
					framed,
					minor xtick={-8,-4,4,8},
					xtick={-4},
					minor ytick={-8,-4,4,8},
					ytick={4},
					grid=both,
					xmin=-10,xmax=10,
					ymin=-10,ymax=10,
				]
				\addplot expression[domain=0.001:10,samples=50]{ln(x)/ln(2)};
				\addplot[soldot]coordinates{(8,3)}node[axisnode,anchor=south]{$(8,3)$};
			\end{axis}
		\end{tikzpicture}
		\caption{$y=\log_b(x)$}
		\label{log:fig:findbase1}
	\end{subfigure}%
	\hfill
	\begin{subfigure}{.2\textwidth}
		\centering
		\begin{tikzpicture}
			\begin{axis}[
					framed,
					minor xtick={-8,-4,4,8},
					xtick={4},
					minor ytick={-8,-4,4,8},
					ytick={-4},
					grid=both,
					xmin=-10,xmax=10,
					ymin=-10,ymax=10,
				]
				\addplot expression[domain=-2.999999:10,samples=100]{ln(x+3)/ln(4)};
				\addplot[soldot]coordinates{(-1,1/2)}node[axisnode,anchor=south east]{$\left(-1,\frac{1}{2}\right)$};
			\end{axis}
		\end{tikzpicture}
		\caption{$y=\log_b(x+3)$}
		\label{log:fig:findbase2}
	\end{subfigure}%
	\hfill
	\begin{subfigure}{.2\textwidth}
		\centering
		\begin{tikzpicture}
			\begin{axis}[
					framed,
					minor xtick={-8,-4,4,8},
					xtick={-4},
					minor ytick={-8,-4,4,8},
					ytick={4},
					grid=both,
					xmin=-10,xmax=10,
					ymin=-10,ymax=10,
				]
				\addplot expression[domain=4.001:10,samples=50]{ln(x-4)/ln(1/2)};
				\addplot[soldot]coordinates{(6,-1)}node[axisnode,anchor=north]{$(6,-1)$};
			\end{axis}
		\end{tikzpicture}
		\caption{$y=\log_b(x-4)$}
		\label{log:fig:findbase3}
	\end{subfigure}%
	\hfill
	\begin{subfigure}{.2\textwidth}
		\centering
		\begin{tikzpicture}
			\begin{axis}[
					framed,
					minor xtick={-8,-4,4,8},
					xtick={-4},
					minor ytick={-8,-4,4,8},
					ytick={-4},
					grid=both,
					xmin=-10,xmax=10,
					ymin=-10,ymax=10,
				]
				\addplot expression[domain=-1.99997:10,samples=100]{ln(x+2)/ln(1/3)};
				\addplot[soldot]coordinates{(7,-2)}node[axisnode,anchor=north]{$(7,-2)$};
			\end{axis}
		\end{tikzpicture}
		\caption{$y=\log_b(x+2)$}
		\label{log:fig:findbase4}
	\end{subfigure}%
	\caption{Graphs for \cref{log:prob:findbase}}
	\label{log:fig:findbase}
	\end{widepage}
\end{figure} 
\end{problem}

%===================================
%   Author: Hughes
%   Date:   July 2012
%===================================
\begin{problem}[Solving exponential equations with base $10$ and base $e$]
Use \cref{log:def:logfunctions} to solve each of the following equations. Give both 
the exact and an approximate solution.
\begin{multicols}{4}
	\begin{subproblem}
		$e^x=7$ 
		\begin{shortsolution}
			$x=\ln(7)\approx 1.95$
		\end{shortsolution}
	\end{subproblem}
	\begin{subproblem}
		$e^x+5=10$ 
		\begin{shortsolution}
			$x=\ln(5)\approx 1.61$
		\end{shortsolution}
	\end{subproblem}
	\begin{subproblem}
		$e^{x+5}=10$  
		\begin{shortsolution}
			$x=\ln(10)-5\approx -2.70$
		\end{shortsolution}
	\end{subproblem}
	\begin{subproblem}
		$e^{5x+7}-4=2$ 
		\begin{shortsolution}
			$x=\frac{1}{5}(\ln(6)-7)\approx -1.04$
		\end{shortsolution}
	\end{subproblem}
	\begin{subproblem}
		$10^x=1$ 
		\begin{shortsolution}
			$x=0$
		\end{shortsolution}
	\end{subproblem}
	\begin{subproblem}
		$10^{x+1}=11$ 
		\begin{shortsolution}
			$x=\log(11)-1\approx 0.04$
		\end{shortsolution}
	\end{subproblem}
	\begin{subproblem}
		$10^{2x}=4$ 
		\begin{shortsolution}
			$x=\frac{\log(4)}{2}\approx 0.30$
		\end{shortsolution}
	\end{subproblem}
	\begin{subproblem}
		$10^{4-x}=21$ 
		\begin{shortsolution}
			$x=4-\log(21)\approx 2.68$
		\end{shortsolution}
	\end{subproblem}
	\begin{subproblem}
		$5e^{2x}+1=10$ 
		\begin{shortsolution}
			$x=\frac{1}{2}\ln\left( \frac{9}{5} \right)\approx 0.29$
		\end{shortsolution}
	\end{subproblem}
	\begin{subproblem}
		$8-7e^{-3x}=-10$ 
		\begin{shortsolution}
			$x=-\frac{1}{3}\ln\left( \frac{18}{7} \right)\approx -0.31$
		\end{shortsolution}
	\end{subproblem}
	\begin{subproblem}
		$9e^{5-x}-1=0$ 
		\begin{shortsolution}
			$x=5-\ln\left( \frac{1}{9} \right)\approx 7.20$
		\end{shortsolution}
	\end{subproblem}
	\begin{subproblem}
		$e^{3x}-4=-5e^{3x}$ 
		\begin{shortsolution}
			$x=\frac{1}{3}\ln\left( \frac{2}{3} \right)\approx -0.14$
		\end{shortsolution}
	\end{subproblem}
\end{multicols}
\end{problem}

%===================================
%   Author: Hughes
%   Date:   July 2012
%===================================
\begin{problem}[Solving logarithmic equations with base $10$ and base $e$]
Use \cref{log:def:logfunctions} to solve each of the following equations. Give both 
the exact and an approximate solution.
\begin{multicols}{4}
	\begin{subproblem}
		$\ln(x)=7$ 
		\begin{shortsolution}
			$x=e^7\approx 1096.63$
		\end{shortsolution}
	\end{subproblem}
	\begin{subproblem}
		$2\ln(x)=-3$ 
		\begin{shortsolution}
			$x=e^{-\nicefrac{3}{2}}\approx 0.22$
		\end{shortsolution}
	\end{subproblem}
	\begin{subproblem}
		$5-4\ln(2x)=13$ 
		\begin{shortsolution}
			$x=\frac{e^{-2}}{2}\approx 0.07$
		\end{shortsolution}
	\end{subproblem}
	\begin{subproblem}
		$(\ln(x))^2=5$ 
		\begin{shortsolution}
			$x=e^{\sqrt{5}}\approx 9.36$ and $x=e^{-\sqrt{5}}\approx 0.11$
		\end{shortsolution}
	\end{subproblem}
	\begin{subproblem}
		$\log(x)=7$ 
		\begin{shortsolution}
			$x=10^7$
		\end{shortsolution}
	\end{subproblem}
	\begin{subproblem}
		$3-\log(x)=0$ 
		\begin{shortsolution}
			$x=1000$
		\end{shortsolution}
	\end{subproblem}
	\begin{subproblem}
		$\log(5x+2)=-3$ 
		\begin{shortsolution}
			$x=\frac{1}{5}\left( 10^{-3}-2 \right)\approx -0.40$
		\end{shortsolution}
	\end{subproblem}
	\begin{subproblem}
		$\log(5-x)=\log_2(8)$ 
		\begin{shortsolution}
			$x=-995$
		\end{shortsolution}
	\end{subproblem}
\end{multicols}
\end{problem}


%===================================
%   Author: Hughes
%   Date:   July 2012
%===================================
\begin{problem}[Find the base from tables]\label{log:prob:findbasetabs}
\Crefrange{log:tab:findbase1}{log:tab:findbase4} show values of four 
different functions; each function has the form $y=\log_b(ax)$ where
$b$ is the base, and $a$ is given for each function. 
Use any ordered pair you wish from each table to find the base, $b$, 
for each function.
\begin{shortsolution}
	\begin{itemize}
		\item \Vref{log:tab:findbase1}: $b=2$, so $y=\log_2(3x)$;
		\item \Vref{log:tab:findbase2}: $b=3$, so $y=\log_3(5x)$;
		\item \Vref{log:tab:findbase3}: $b=\frac{1}{4}$, so $y=\log_{\frac{1}{4}}(x)$;
		\item \Vref{log:tab:findbase4}: $b=\frac{1}{3}$, so $y=\log_{\frac{2}{3}}(-2x)$.
	\end{itemize}
\end{shortsolution}

\end{problem}
\begin{table}[htb]
	\renewcommand{\arraystretch}{1.25}
	\begin{widepage}
	\caption{Tables for \cref{log:prob:findbasetabs}}
	\label{log:tab:findbase}
	\begin{subtable}{.2\textwidth}
		\centering
		\caption{$y=\log_b(3x)$}
		\label{log:tab:findbase1}
		\begin{tabular}{S[table-format=1.0]S[table-format=1.0]}
			\beforeheading 
			\heading{$x$} & \heading{$y$} \\
			\afterheading
			\num{1/16}    & 1             \\  \normalline
			\num{1/3}     & 0             \\  \normalline
			\num{2/3}     & 1             \\  \normalline
			\num{4/3}     & 2             \\  \normalline
			\num{8/3}     & 3             \\  \lastline
		\end{tabular}
	\end{subtable}%
	\hfill
	\begin{subtable}{.2\textwidth}
		\centering
		\caption{$y=\log_b(5x)$}
		\label{log:tab:findbase2}
		\begin{tabular}{S[table-format=1.0]S[table-format=1.0]}
			\beforeheading 
			\heading{$x$} & \heading{$y$} \\
			\afterheading
			\num{1/135}   & -3            \\  \normalline
			\num{1/45}    & -2            \\  \normalline
			\num{1/15}    & -1            \\  \normalline
			\num{1/5}     & 0             \\  \normalline
			\num{3/5}     & 1             \\  \lastline
		\end{tabular}
	\end{subtable}%
	\hfill
	\begin{subtable}{.2\textwidth}
		\centering
		\caption{$y=\log_b(x)$}
		\label{log:tab:findbase3}
		\begin{tabular}{S[table-format=1.0]S[table-format=1.0]}
			\beforeheading 
			\heading{$x$} & \heading{$y$} \\
			\afterheading
			16            & -2            \\  \normalline
			4             & -1            \\  \normalline
			1             & 0             \\  \normalline
			\num{1/4}     & 1             \\  \normalline
			\num{1/16}    & 2             \\  \lastline
		\end{tabular}
	\end{subtable}%
	\hfill
	\begin{subtable}{.2\textwidth}
		\centering
		\caption{$y=\log_b(-2x)$}
		\label{log:tab:findbase4}
		\begin{tabular}{S[table-format=1.0]S[table-format=1.0]}
			\beforeheading 
			\heading{$x$} & \heading{$y$} \\
			\afterheading
			\num{-9/8}    & -2            \\  \normalline
			\num{-3/4}    & -1            \\  \normalline
			-1            & 0             \\  \normalline
			\num{-1/3}    & 1             \\  \normalline
			\num{-2/9}    & 2             \\  \lastline
		\end{tabular}
	\end{subtable}%
	\end{widepage}
\end{table} 


%===================================
%   Author: Hughes
%   Date:   July 2012
%===================================
\begin{problem}[Inverse functions]
Let $f$ be the function that has formula $f(x)=4^x$. 
\begin{subproblem}\label{log:prob:invconstruct}
	Construct a table of values of $f$, allowing $x$ to take integer values on the interval $[-3,3]$.
	\begin{shortsolution}
		\begin{tabular}[t]{S[table-format=1.0]S[table-format=1.0]}
			\beforeheading
			\heading{$x$} & \heading{$f(x)$} \\
			\afterheading
			-3            & \num{1/64}       \\\normalline
			-2            & \num{1/16}       \\\normalline
			-1            & \num{1/4}        \\\normalline
			0             & 1                \\\normalline
			1             & 4                \\\normalline
			2             & 16               \\\normalline
			3             & 64               \\\lastline
		\end{tabular}
	\end{shortsolution}
\end{subproblem}
\begin{subproblem}\label{log:prob:invconstructuse}
	Use your answer to \cref{log:prob:invconstruct} to construct a table of values 
	of the function $f^{-1}$.
	\begin{shortsolution}
		\begin{tabular}[t]{S[table-format=1.0]S[table-format=1.0]}
			\beforeheading
			\heading{$x$} & \heading{$f^{-1}(x)$} \\
			\afterheading
			\num{1/64}    & -3                    \\\normalline
			\num{1/16}    & -2                    \\\normalline
			\num{1/4}     & -1                    \\\normalline
			1             & 0                     \\\normalline
			4             & 1                     \\\normalline
			16            & 2                     \\\normalline
			64            & 3                     \\\lastline
		\end{tabular}
	\end{shortsolution}
\end{subproblem}
\begin{subproblem}
	Use your answer to \cref{log:prob:invconstructuse} to evaluate each of the following
	\begin{multicols}{4}
		\begin{enumerate}
			\item $f^{-1}(4)$
			\item $f^{-1}(16)$
			\item $f^{-1}\left( \frac{1}{4} \right)$
			\item $f^{-1}\left( \frac{1}{16} \right)$
		\end{enumerate}
	\end{multicols}
	\begin{shortsolution}
		\begin{enumerate}
			\item $f^{-1}(4) = 1$
			\item $f^{-1}(16)= 2$
			\item $f^{-1}\left( \frac{1}{4} \right)=-1$
			\item $f^{-1}\left( \frac{1}{16} \right)=-2$
		\end{enumerate}
	\end{shortsolution}
\end{subproblem}
\begin{subproblem}
	Give the formula for $f^{-1}(x)$.
	\begin{shortsolution}
		$f^{-1}(x)=\log_{4}(x)$ 
	\end{shortsolution}
\end{subproblem}
\end{problem}

%===================================
%   Author: Hughes
%   Date:   July 2012
%===================================
\begin{problem}[Inverse functions]
Each of the functions defined by the following formulas are invertable. For each function
\begin{enumerate}
	\item state its domain and range;
	\item find its inverse;
	\item state the domain and range of the inverse function.
\end{enumerate}
\begin{multicols}{4}
	\begin{subproblem}
		$f(x)=2^{5x}$ 
		\begin{shortsolution}
			\begin{enumerate}
				\item domain: $(-\infty,\infty)$, range: $(0,\infty)$.
				\item $f^{-1}(x)=\frac{1}{5}\log_2(x)$ 
				\item domain: $(0,\infty)$, range: $(-\infty,\infty)$.
			\end{enumerate}
		\end{shortsolution}
	\end{subproblem}
	\begin{subproblem}
		$g(t)=e^{3t+4}$ 
		\begin{shortsolution}
			\begin{enumerate}
				\item domain: $(-\infty,\infty)$, range: $(0,\infty)$.
				\item $g^{-1}(t)=\frac{1}{3}(\ln(t)-4)$ 
				\item domain: $(0,\infty)$, range: $(-\infty,\infty)$.
			\end{enumerate}
		\end{shortsolution}
	\end{subproblem}
	\begin{subproblem}
		$h(s)=5-4^{s-7}$ 
		\begin{shortsolution}
			\begin{enumerate}
				\item domain: $(-\infty,\infty)$, range: $(-\infty,5)$.
				\item $h^{-1}(s)=7+\log_4(5-s)$ 
				\item domain: $(-\infty,5)$, range: $(-\infty,\infty)$.
			\end{enumerate}
		\end{shortsolution}
	\end{subproblem}
	\begin{subproblem}
		$j(u)=3\cdot 5^{-u}$ 
		\begin{shortsolution}
			\begin{enumerate}
				\item domain: $(-\infty,\infty)$, range: $(0,\infty)$.
				\item $j^{-1}(u)=\log_5\left( \frac{3}{u} \right)$ 
				\item domain: $(0,\infty)$, range: $(-\infty,\infty)$.
			\end{enumerate}
		\end{shortsolution}
	\end{subproblem}
	\begin{subproblem}
		$k(v)=\ln(3v+2)$ 
		\begin{shortsolution}
			\begin{enumerate}
				\item domain: $\left( -\frac{2}{3},\infty \right)$, range: $(-\infty,\infty)$.
				\item $k^{-1}(v)=\frac{1}{3}(e^v-2)$ 
				\item domain: $(-\infty,\infty)$, range: $\left( -\frac{2}{3},\infty \right)$.
			\end{enumerate}
		\end{shortsolution}
	\end{subproblem}
	\begin{subproblem}
		$l(w)=5\log(2-7w)$ 
		\begin{shortsolution}
			\begin{enumerate}
				\item domain: $\left( -\infty,\frac{2}{7} \right)$, range: $(-\infty,\infty)$.
				\item $l^{-1}(w)=\frac{1}{7}\left(2-10^{\frac{w}{5}}\right)$ 
				\item domain: $(-\infty,\infty)$, range: $\left( -\infty,\frac{2}{7} \right)$.
			\end{enumerate}
		\end{shortsolution}
	\end{subproblem}
	\begin{subproblem}
		$m(\alpha)=\log_8(2\alpha-1)$ 
		\begin{shortsolution}
			\begin{enumerate}
				\item domain: $\left( \frac{1}{2},\infty \right)$, range: $(-\infty,\infty)$.
				\item $m^{-1}(\alpha)=\frac{1}{2}\left( 8^\alpha+1 \right)$ 
				\item domain: $(-\infty,\infty)$, range: $\left( \frac{1}{2},\infty \right)$.
			\end{enumerate}
		\end{shortsolution}
	\end{subproblem}
	\begin{subproblem}
		$n(\beta)=\frac{2}{3}\log_3(4\beta-7)$ 
		\begin{shortsolution}
			\begin{enumerate}
				\item domain: $\left( \frac{7}{4},\infty \right)$, range: $(-\infty,\infty)$.
				\item $n^{-1}(\beta)=\frac{1}{4}\left( 3^{\frac{3\beta}{2}}+7\right)$ 
				\item domain: $(-\infty,\infty)$, range: $\left( \frac{7}{4},\infty \right)$.
			\end{enumerate}
		\end{shortsolution}
	\end{subproblem}
\end{multicols}
\end{problem}

%===================================
%   Author: Neft (Hughes)
%   Date:   August 2012
%===================================
\begin{problem}[The RL Circuit]
An \emph{inductor} is a device that stores electrical energy in a magnetic field. As long as
\emph{current} is flowing through the inductor, energy is stored. When 
connected to a voltage source, such as a battery, and a resistor, the current, (in \si{\ampere}), 
through the inductor is modeled by the function $I$ that has formula
\[
	I(t)=\frac{V}{R}\left( 1-e^{-\frac{R}{L}t} \right)
\]
where $V$ is voltage of the source (in \si{\volt}), $R$ is the value of 
the resistor (in \si{\ohm}), $L$ is the value of the inductor (in \si{\henry}), 
and $t$ is time (in \si{\second}) since the inductor was connected 
to the voltage source.
\begin{subproblem}
	A switch is thrown in a circuit that connects a $5$-henry inductor 
	to a $200$-ohm resistor and a $12$-volt battery. Write the formula 
	for $I(t)$.
	\begin{shortsolution}
		$I(t)=\frac{3}{50}\left( 1-e^{-40t} \right)$ 
	\end{shortsolution}
\end{subproblem}
\begin{subproblem}
	What is the current in the circuit after $\SI{0.025}{\second}$? After $\SI{0.05}{\second}$?
	\begin{shortsolution}
		\begin{itemize}
			\item $I(0.025)\approx 0.038$; the current in the circuit after $\SI{0.025}{\second}$ is 
			approximately $\SI{0.038}{\ampere}$;
			\item $I(0.05)\approx \SI{0.052}{\ampere}$; the current in the circuit after $\SI{0.05}{\second}$
			is approximately $\SI{0.052}{\ampere}$.
		\end{itemize}
	\end{shortsolution}
\end{subproblem}
\begin{subproblem}
	What is the maximum value the current will reach? 
	\begin{shortsolution}
		The maximum value the current could reach is $\SI{3/50}{\ampere}=\SI{0.06}{\ampere}$.
	\end{shortsolution}
\end{subproblem}
\begin{subproblem}
	Since we know that, mathematically, our model will never actually reach this value,
	how long will it take the current to reach $\SI{95}{\percent}$ of this value?
	\begin{shortsolution}
		We need to solve the equation $0.95\cdot 0.06=I(t)$; so $t\approx 0.075$. The current reaches
		$\SI{95}{\percent}$ of its maximum value after about $\SI{0.075}{\second}$.
	\end{shortsolution}
\end{subproblem}
\end{problem}

%===================================
%   Author: Hughes
%   Date:   July 2012
%===================================
\begin{problem}[Factoring]
Use your factoring skills to solve the following exponential equations 
(if possible). Give both the exact and an approximate solution.
\begin{multicols}{4}
	\begin{subproblem}
		% (e^x-2)(e^x-4)=0
		$e^{2x}-6e^x+8=0$ 
		\begin{shortsolution}
			$x=\ln(2)\approx 0.69$ and $x=\ln(4)\approx 1.39$.
		\end{shortsolution}
	\end{subproblem}
	\begin{subproblem}
		% (e^x-3)(e^x-1)=0
		$e^{2x}-4e^x-3=0$ 
		\begin{shortsolution}
			$x=\ln(3)\approx 1.10$ and $x=\ln(1)=0$.
		\end{shortsolution}
	\end{subproblem}
	\begin{subproblem}
		% (e^x-10)(e^x+2)=0 
		$e^{2x}-8e^x-20=0$
		\begin{shortsolution}
			$x=\ln(10)\approx 2.30$ (there are no solutions to the equation $e^x=-2$).
		\end{shortsolution}
	\end{subproblem}
	\begin{subproblem}
		% (e^x+5)(e^x+6)=0 
		$e^{2x}+11e^x+30=0$
		\begin{shortsolution}
			There are no solutions to this equation.
		\end{shortsolution}
	\end{subproblem}
\end{multicols}
\begin{multicols}{4}
	\begin{subproblem}
		% (10^x-2)(10^x-4)=0
		$10^{2x}-6\cdot 10^x+8=0$ 
		\begin{shortsolution}
			$x=\log(2)\approx 0.30$ and $x=\log(4)\approx 0.60$.
		\end{shortsolution}
	\end{subproblem}
	\begin{subproblem}
		% (10^x-3)(10^x-1)=0
		$10^{2x}-4\cdot 10^x-3=0$ 
		\begin{shortsolution}
			$x=\log(3)\approx 0.48$ and $x=\log(1)=0$.
		\end{shortsolution}
	\end{subproblem}
	\begin{subproblem}
		% (10^x-10)(10^x+2)=0 
		$10^{2x}-8\cdot 10^x-20=0$
		\begin{shortsolution}
			$x=\log(10)=1$ (there are no solutions to the equation $10^x=-2$).
		\end{shortsolution}
	\end{subproblem}
	\begin{subproblem}
		% (10^x+5)(10^x+6)=0 
		$10^{2x}+11\cdot 10^x+30=0$
		\begin{shortsolution}
			There are no solutions to this equation.
		\end{shortsolution}
	\end{subproblem}
\end{multicols}
\end{problem}

%===================================
%   Author: Hughes
%   Date:   July 2012
%===================================
\begin{problem}[Factoring with logarithms]
Use your factoring skills to solve the following 
logarithmic equations (if possible). Give both the exact and an approximate solution.
\begin{multicols}{3}
	\begin{subproblem}
		% (ln(x)+2)(ln(x)+1)=0
		$(\ln(x))^2+3\ln(x)+2=0$ 
		\begin{shortsolution}
			$x=e^{-2}\approx 0.14$ and $x=e^{-1}\approx 0.37$.
		\end{shortsolution}
	\end{subproblem}
	\begin{subproblem}
		% (ln(x)-2)(ln(x)-1)=0
		$(\ln(x))^2-3\ln(x)+2=0$ 
		\begin{shortsolution}
			$x=e^2\approx 7.39$ and $x=e\approx 2.72$.
		\end{shortsolution}
	\end{subproblem}
	\begin{subproblem}
		% (ln(x)-4)(ln(x)+4)=0 
		$(\ln(x))^2-16=0$
		\begin{shortsolution}
			$x=e^4\approx 54.60$ and $x=e^{-4}\approx 0.02$. 
		\end{shortsolution}
	\end{subproblem}
\end{multicols}
\begin{multicols}{3}
	\begin{subproblem}
		% (log(x)+7)(log(x)-1)=0
		$(\log(x))^2+6\log(x)-7=0$ 
		\begin{shortsolution}
			$x=10^{-7}$ and $x=10$.
		\end{shortsolution}
	\end{subproblem}
	\begin{subproblem}
		% (log(x)+4)(log(x)+3)=0 
		$(\log(x))^2+7\log(x)+12=0$
		\begin{shortsolution}
			$x=10^{-4}$ and $x=10^{-3}$.
		\end{shortsolution}
	\end{subproblem}
	\begin{subproblem}
		% (log(x)-1) (log(x)+1)=0
		$(\log(x))^2-1=0$
		\begin{shortsolution}
			$x=10$ and $x=10^{-1}$.
		\end{shortsolution}
	\end{subproblem}
\end{multicols}
\end{problem}
\end{exercises}

\section{Properties of logarithms}
\begin{pccspecialcomment}[Properties of logarithms]
	Assuming that $x$, $y$, and $b$ are any positive real numbers (where $b$ 
	is the base) then the following properties of logarithms hold
	\reformatpropslist{l}
	\begin{props}
		\item\label{log:prop:add} $\log_b(x)+\log_b(y)=\log_b(xy)$ 
		\item\label{log:prop:sub} $\log_b(x)-\log_b(y)=\log_b\left( \frac{x}{y} \right)$ 
		\item\label{log:prop:pow} $\log_b(x^t)=t\log_b(x)$ where $t$ is any real number
		\item\label{log:prop:obv} $x=y\Leftrightarrow \log_b(x)=\log_b(y)$
		\item\label{log:prop:inv1} $b^{\log_b(x)}=x$
		\item\label{log:prop:inv2} $\log_b(b^x)=x$
	\end{props}
	Note that, in particular, \cref{log:prop:inv1,log:prop:inv2} say that if $f$ and $g$ 
	are functions that have formulas
	\[
		f(x)=b^x, \qquad g(x)=\log_b(x)
	\]
	then $f$ and $g$ are inverse functions. 
\end{pccspecialcomment}
%===================================
%   Author: Hughes
%   Date:   July 2012
%===================================
\begin{pccexample}
	Use the properties of logarithms to help solve the equation
	\begin{equation}\label{log:eq:solve}
		\ln(x-3)+\ln(x+6)=\ln(10)
	\end{equation}
	\begin{pccsolution}
		When solving equations such as \cref{log:eq:solve}, it is often helpful to use the 
		symbol $\Rightarrow$ which means, `implies that'; this also allows
		us to annotate each line (if necessary).
		\begin{align*}
			\ln(x-3)+\ln(x+6)=\ln(10)
			  & \Rightarrow \ln( (x-3)(x+6))=\ln(10) &   & \text{\cref{log:prop:add}} \\
			  & \Rightarrow \ln( x^2+3x-18)=\ln(10)  &   & \text{distribute}          \\
			  & \Rightarrow x^2+3x-18=10             &   & \text{\cref{log:prop:obv}} \\
			  & \Rightarrow x^2+3x-28=0              &   &                            \\
			  & \Rightarrow (x+7)(x-4)=0             &   & \text{factor}              \\
			  & \Rightarrow x=-7,4                   &   &                            
		\end{align*}
		It seems that we have two solutions| we need to check both of them by 
		substituting each into \cref{log:eq:solve}:
		\begin{align*}
			\ln(-7-3)+\ln(-7+6) & \stackrel{?}{=} \ln(10) & \ln(4-3)+\ln(4+6) & \stackrel{?}{=}\ln(10) \\
			\ln(-10)+\ln(-1)    & \stackrel{?}{=}\ln(10)  & \ln(1)+\ln(10)    & \stackrel{?}{=}\ln(10) \\
			                    & \text{domain error!}    & \ln(10)           & \stackrel{?}{=}\ln(10) \\
			                    &                         &                   & \text{true}            
		\end{align*}
		Since $-7$ gives a domain error when substituted into \cref{log:eq:solve} and $4$ \emph{does} satisfy \cref{log:eq:solve}, we conclude
		that $4$ is the only solution to the equation.
	\end{pccsolution}
\end{pccexample}

%===================================
%   Author: Hughes
%   Date:   August 2012
%===================================
\fixthis{finish example}
\begin{pccexample}
	another solving equation problem 
\end{pccexample}

\begin{pccspecialcomment}[The change of base formula]
	The change of base formula for logarithms is
	\begin{equation}\label{log:eq:changebase}
		\log_a(x)=\frac{\log_b(x)}{\log_b(a)}
	\end{equation}
	where $a$, $b$, and $x$ are real, positive numbers.
\end{pccspecialcomment}

The change of base formula may seem like a little strange, 
but it is fairly simple to derive, as we
show in the following steps
\begin{align*}
	y=\log_a(x) & \Rightarrow a^y=x                           \\
	            & \Rightarrow \log_b(a^y)=\log_b(x)           \\
	            & \Rightarrow y\log_b(a) = \log_b(x)          \\
	            & \Rightarrow y = \frac{\log_b(x)}{\log_b(a)} 
\end{align*}

The change of base formula is particularly useful when calculating logarithms
that have a base other than $e$ or $10$. We can use it to help us explore 
graphical and numerical features of logarithmic functions with such bases; 
even though most modern calculators can evaluate logarithmic expressions of 
any base, the principle remains useful.
%===================================
%   Author: Hughes
%   Date:   July 2012
%===================================
\begin{pccexample}
	Use the change of base formula, \cref{log:eq:changebase}, to help you 
	graph the function $f$ that has formula
	\[
		f(x)=\log_{\frac{1}{4}}(x)
	\]
	Compare the graph with that of the function $g$ that has formula $g(x)=\left( \frac{1}{4} \right)^x$.
	\begin{pccsolution}
		We begin by using \cref{log:eq:changebase} to rewrite the formula for $f$
		\begin{align*}
			f(x) & =\log_{\frac{1}{4}}(x)                        \\
			     & =\frac{\ln(x)}{\ln\left( \frac{1}{4} \right)} 
		\end{align*}
		Note that the change of base formula allows us to use \emph{any base we choose}; we have
		chosen to use the \emph{natural} base simply because the function $\ln(x)$ 
		is easily accessible on most calculators. Typically we will use either $\ln(x)$
		or $\log(x)$ when changing base. We have plotted $f$ and $g$ (which has formula
		$g(x)=\left( \frac{1}{4} \right)^x$) in \cref{log:fig:quarterand4}.
		
		\begin{figure}[!htb]
			\centering
			\begin{tikzpicture}
				\begin{axis}[
						xmin=-2,xmax=5,
						ymin=-2,ymax=5,
						width=.5\textwidth,
						legend pos=outer north east,
					]
					\addplot expression[domain=0.000977:5,samples=100]{ln(x)/ln(1/4)};
					\addlegendentry{$y=\log_{\frac{1}{4}}(x)$};
					\addplot expression[domain=-1.16096:5,samples=100]{(1/4)^x};
					\addlegendentry{$y=\left( \frac{1}{4} \right)^x$};
					\addplot[pccplot,dashed] expression[domain=-2:5]{x}node[axisnode,anchor=east,pos=0.9]{$y=x$};
				\end{axis}
			\end{tikzpicture}
			\caption{}
			\label{log:fig:quarterand4}
		\end{figure}
		
		We can make observations about the graphs of $f$ and $g=f^{-1}$ (which are 
		similar to the observations we made in \vref{log:ex:graphing})
		\begin{itemize}
			\item the domain of $f$ is $(0,\infty)$, and the range of $f$ is $(-\infty,\infty)$; this means
			that the domain of $f^{-1}$ is $(-\infty,\infty)$, and the range of $f^{-1}$ is $(0,\infty)$;
			\item the function $f$ has a \emph{vertical} asymptote with equation $x=0$; this necessarily
			means that the function $f^{-1}$ has a \emph{horizontal} asymptote with equation $y=0$;
			\item the function $f$ does not have a \emph{horizontal} asymptote| this therefore
			implies that the function $f^{-1}$ does not have a \emph{vertical} asymptote;
			\item the curves of $f$ and $f^{-1}$ are symmetric about the line $y=x$.
		\end{itemize}
	\end{pccsolution}
\end{pccexample}

%===================================
%   Author: Hughes
%   Date:   July 2012
%===================================
\begin{pccexample}[Investing in an account]
	You have \$2000 to invest in an account that accrues interest at a nominal rate of $\SI{3.75}{\percent}$
	Assuming that $A(t)$ is the amount of money in the account $t$ years 
	after opening the account, calculate the amount of time it will take 
	the money in each account to reach \$3000 when the interest is 
	compounded in each of the following ways
	\begin{multicols}{4}
		\begin{enumerate}
			\item Annually. 
			\item Monthly 
			\item Daily 
			\item Continuously 
		\end{enumerate}
	\end{multicols}
	You may like to refresh your knowledge about compound interest using \vref{exp:def:compoundint}.
	\begin{pccsolution}
		\begin{enumerate}
			\item $A(t)=2000(1.0375)^t$; to calculate when $A(t)=3000$ we need to solve the 
			equation
			\begin{align*}
				3000=2000(1.0375)^t & \Rightarrow \frac{3}{2}=(1.0375)^t                              \\
				                    & \Rightarrow \ln\left( \frac{3}{2} \right)=\ln(1.0375)^t         \\
				                    & \Rightarrow \ln\left( \frac{3}{2} \right)=t\ln(1.0375)          \\
				                    & \Rightarrow t=\frac{\ln\left( \frac{3}{2} \right)}{\ln(1.0375)} \\
				                    & \phantom{ {}\Rightarrow t}\approx 11.0139                       
			\end{align*}
			If the interest is compounded \emph{annually}, it will take about $11$ years for the 
			initial investment to reach \$3000.
			\item $A(t)=2000\left( 1+\frac{0.0375}{12} \right)^{12t}$; to calculate when $A(t)=3000$ we need to solve the 
			equation
			\begin{align*}
				3000=2000\left( 1+\frac{0.0375}{12} \right)^{12t} & \Rightarrow \frac{3}{2}=\left( 1+\frac{0.0375}{12} \right)^{12t}                                           \\
				                                                  & \Rightarrow \ln\left(\frac{3}{2}\right)=\ln \left( 1+\frac{0.0375}{12}\right)^{12t}                        \\
				                                                  & \Rightarrow \ln\left(\frac{3}{2}\right)=12t\ln\left( 1+\frac{0.0375}{12} \right)                           \\
				                                                  & \Rightarrow 12t= \frac{\ln\left(\frac{3}{2}\right)}{\ln\left( 1+\frac{0.0375}{12} \right)}                 \\
				                                                  & \Rightarrow t= \frac{1}{12}\cdot \frac{\ln\left(\frac{3}{2}\right)}{\ln\left( 1+\frac{0.0375}{12} \right)} \\
				                                                  & \phantom{ {}\Rightarrow t}\approx 10.8293                                                                  
			\end{align*}
			If the interest is compounded \emph{monthly}, it will take just under $11$ years for the 
			initial investment to reach \$3000.
			\item $A(t)=2000\left( 1+\frac{0.0375}{365} \right)^{365t}$; to calculate when $A(t)=3000$ we need to solve the 
			equation
			\begin{align*}
				3000=2000\left( 1+\frac{0.0375}{365} \right)^{365t} & \Rightarrow \frac{3}{2}=\left( 1+\frac{0.0375}{365} \right)^{365t}                                           \\
				                                                    & \Rightarrow \ln\left( \frac{3}{2} \right)=\ln\left( 1+\frac{0.0375}{365} \right)^{365t}                      \\
				                                                    & \Rightarrow \ln\left( \frac{3}{2} \right)=365t \ln\left( 1+\frac{0.0375}{365} \right)                        \\
				                                                    & \Rightarrow 365t = \frac{\ln\left( \frac{3}{2} \right)}{\left( 1+\frac{0.0375}{365} \right)}                 \\
				                                                    & \Rightarrow t = \frac{1}{365}\cdot \frac{\ln\left( \frac{3}{2} \right)}{\left( 1+\frac{0.0375}{365} \right)} \\
				                                                    & \phantom{ {}\Rightarrow t}\approx 10.8130                                                                    
			\end{align*}
			If the interest is compounded \emph{daily}, it will take just under $11$ years for the 
			initial investment to reach \$3000.
			\item $A(t)=2000e^{0.0375t}$; to calculate when $A(t)=3000$ we need to solve the 
			equation
			\begin{align*}
				3000=2000e^{0.0375t} & \Rightarrow \frac{3}{2}=e^{0.0375t}                           \\
				                     & \Rightarrow \ln\left( \frac{3}{2} \right)=0.0375t             \\
				                     & \Rightarrow t = \frac{1}{0.0375}\ln\left( \frac{3}{2} \right) \\
				                     & \phantom{ {}\Rightarrow t}\approx 10.8124                     
			\end{align*}
			If the interest is compounded \emph{continuously}, it will take just under $11$ years for the 
			initial investment to reach \$3000.
		\end{enumerate}
	\end{pccsolution}
\end{pccexample}

%===================================
%   Author: Hughes
%   Date:   July 2012
%===================================
\begin{pccexample}[A cautionary tale]
	\pccname{Tyrell} and \pccname{Latisha} are studying the equation
	\begin{equation}\label{log:eq:oneortwo}
		\ln(x^2)=3
	\end{equation}
	Tyrell uses \cref{log:prop:pow} to solve the equation
	\begin{align*}
		\ln(x^2)=3 & \Rightarrow 2\ln(x)=3          \\ 
		           & \Rightarrow \ln(x)=\frac{3}{2} \\
		           & \Rightarrow x=e^{\frac{3}{2}}  
	\end{align*}
	Latisha takes a different approach:
	\begin{align*}
		\ln(x^2)=3 & \Rightarrow x^2=e^3                                 \\
		           & \Rightarrow x  =\pm\sqrt{e^3}                       \\
		           & \phantom{ {}\Rightarrow x} =\pm e^{\frac{3}{2}} % ugly hack! :) cmh  
	\end{align*}
	Note that Latisha has $2$ solutions, and Tyrell only has $1$! Who has
	the correct solution set?
	\begin{pccsolution}
		Let's begin by exploring \cref{log:prop:pow}; it is certainly true to 
		say
		\[
			\ln(3^2)=2\ln(3)
		\]
		but it is not true to say that
		\[
			\ln( (-3)^2)=2\ln(-3)
		\]
		since we can not take the logarithm of negative number. These two examples
		illustrate that Tyrell's application of \cref{log:prop:pow} was not 
		appropriate, since we can input both positive \emph{and} negative 
		numbers into \cref{log:eq:oneortwo}.
		
		We conclude that Latisha has the correct solution set.
	\end{pccsolution}
\end{pccexample}


The properties of logarithms may seem a little mysterious. Remembering that 
logarithmic expressions are closely related to exponential expressions (see \cref{log:def:logfunctions}), it should 
sound reasonable that the properties of logarithms are somewhat related to the 
properties of exponents. Let's see if we can tie the two ideas together, and prove \cref{log:prop:add}. 

%===================================
%   Author: Hughes
%   Date:   August 2012
%===================================
\begin{pccexample}[Proving that $\log_b(xy)=\log_b(x)+\log_b(y)$]\label{log:ex:prooveprop}
	When proving such an identify, we have a 
	few options:
	\begin{itemize}
		\item We could start with one side of the identity, and try to work
		toward the other side of it.
		\item We could start with one side of the identity, simplify it, and 
		then try to reach the same expression by working with the other side of 
		the identity.
	\end{itemize}
	We will demonstrate a proof using the second of these options.
	\begin{pccsolution}
		We'll start by writing
		\begin{equation}\label{log:eq:prooveprops}
			m=\log_b(x), \qquad n=\log_b(y)
		\end{equation}
		We can write the equations in \cref{log:eq:prooveprops} in their equivalent
		exponential form
		\[
			b^m=x, \qquad b^n=y 
		\]
		We clearly see that $b^m\cdot b^n= xy$ and so
		\[
			xy=b^{m+n}
		\]
		Let's write an equivalent logarithmic equation to the exponential equation $xy=b^{m+n}$
		\begin{equation}\label{log:eq:prooveprops1}
			\log_b(xy)=m+n
		\end{equation}
		Notice that this equation contains the left hand side of \cref{log:prop:add}; we 
		are at the half-way point of our proof| let's see if we can meet here using the right hand side 
		of \cref{log:prop:add}. 
		
		We can write $\log_b(x)+\log_b(y)$ in terms of $m$ and $n$
		\begin{equation}\label{log:eq:prooveprops2}
			\log_b(x)+\log_b(y)=m+n 
		\end{equation}
		Combining \cref{log:eq:prooveprops1,log:eq:prooveprops2} gives
		the desired result
		\begin{align*}
			\log_b(x)+\log_b(y) & =m+n         \\
			                    & = \log_b(xy) 
		\end{align*}
	\end{pccsolution}
\end{pccexample}
%===================================
%   Author: Hughes
%   Date:   August 2012
%===================================
\begin{doyouunderstand}
	\begin{problem}
	Use \cref{log:ex:prooveprop} to help guide you in prooving \cref{log:prop:sub}.
	\begin{shortsolution}
		Put $m=\log_b(x)$ and $n=\log_b(y)$ so that $b^m=x$ and $b^n=y$. Therefore
		$b^{m-n}=\frac{x}{y}$, and equivalently $\log_b\left( \frac{x}{y} \right)=m-n$. 
		Also, $\log_b(x)-\log_b(y)=m-n$. The result follows.
	\end{shortsolution}
	\end{problem}
\end{doyouunderstand}

\investigation*{}
%===================================
%   Author: Hughes
%   Date:   July 2012
%===================================
\begin{problem}[True or false?]
\begin{subproblem}\label{log:prob:truefalse}
	Use the change of base formula and a calculator to help plot the function $f$ that has formula $f(x)=\log_b(x)$ for each of
	the following values of $b$
	\begin{multicols}{4}
		\begin{enumerate}
			\item $b=2$
			\item $b=\frac{1}{4}$
			\item $b=5$
			\item $b=\frac{1}{3}$
		\end{enumerate}
	\end{multicols}
	\begin{shortsolution}
		\begin{enumerate}
			\item $b=2$
						
			\begin{tikzpicture}
				\begin{axis}[
						xmin=-10,xmax=10,
						ymin=-10,ymax=10,
					]
					\addplot expression[domain=0.001:10,samples=50]{ln(x)/ln(2)};
					\addlegendentry{$y=\log_2(x)$}
				\end{axis}
			\end{tikzpicture}
			\item $b=\frac{1}{4}$
						
			\begin{tikzpicture}
				\begin{axis}[
						xmin=-10,xmax=10,
						ymin=-10,ymax=10,
					]
					\addplot expression[domain=0.00001:10,samples=50]{ln(x)/ln(1/4)};
					\addlegendentry{$y=\log_{\frac{1}{4}}(x)$}
				\end{axis}
			\end{tikzpicture}
			\item $b=5$
						
			\begin{tikzpicture}
				\begin{axis}[
						xmin=-10,xmax=10,
						ymin=-10,ymax=10,
					]
					\addplot expression[domain=0.00001:10,samples=50]{ln(x)/ln(5)};
					\addlegendentry{$y=\log_5(x)$}
				\end{axis}
			\end{tikzpicture}
			\item $b=\frac{1}{3}$
						
			\begin{tikzpicture}
				\begin{axis}[
						xmin=-10,xmax=10,
						ymin=-10,ymax=10,
					]
					\addplot expression[domain=0.001:10,samples=50]{ln(x)/ln(1/3)};
					\addlegendentry{$y=\log_{\frac{1}{3}}(x)$}
				\end{axis}
			\end{tikzpicture}
		\end{enumerate}
	\end{shortsolution}
\end{subproblem}
Use your answer to \cref{log:prob:truefalse} to help you determine if 
each of the following statements are true or false for all values of $b$; if you believe that 
the statement is false, provide an example that supports it.
\begin{subproblem}
	The function $f$ is increasing.  
	\begin{shortsolution}
		False; consider $b=\frac{1}{4}$ or $b=\frac{1}{3}$, or any other value of $b$ such that $0<b<1$.
	\end{shortsolution}
\end{subproblem}
\begin{subproblem}
	The function $f$ is decreasing. 
	\begin{shortsolution}
		False; consider $b=2$ or $b=5$, or any other value of $b$ such that $b>1$.
	\end{shortsolution}
\end{subproblem}
\begin{subproblem}
	The function $f$ has a vertical asymptote at $0$. 
	\begin{shortsolution}
		True.
	\end{shortsolution}
\end{subproblem}
\begin{subproblem}
	The function $f$ is concave up.     
	\begin{shortsolution}
		False; consider $b=2$ or $b=5$, or any other value of $b$ such that $b>1$.
	\end{shortsolution}
\end{subproblem}
\begin{subproblem}
	The function $f$ has a zero at $1$. 
	\begin{shortsolution}
		True. 
	\end{shortsolution}
\end{subproblem}
\begin{subproblem}
	The function $f$ has a vertical intercept. 
	\begin{shortsolution}
		False; consider any value of $b$. 
	\end{shortsolution}
\end{subproblem}
\begin{subproblem}
	The function $f$ is concave down. 
	\begin{shortsolution}
		False; consider $b=\frac{1}{4}$ or $b=\frac{1}{3}$, or any other value of $b$ such that $0<b<1$.
	\end{shortsolution}
\end{subproblem}
\end{problem}

%===================================
%   Author: Hughes
%   Date:   July 2012
%===================================
\begin{problem}[\Cref{log:prop:add,log:prop:sub}]
\begin{subproblem}\label{log:prob:props}
	Use \cref{log:prop:add,log:prop:sub} to help you complete \cref{log:tab:props}; note that 
	you should be able to do so without using a calculator.
	\begin{shortsolution}
		\begin{tabular}[t]{S[table-format=2.3,parse-numbers=false]S[table-format=5.0,parse-numbers=false]S[table-format=1.0,parse-numbers=false]*{4}{S[table-format=1.0]}}
			\beforeheading
			\heading{$A$} & \heading{$B$}    & \heading{$b$}   & \heading{$\log_b(A)$} & \heading{$\log_b(B)$} & \heading{$\log_b(AB)$} & \heading{$\log_b\left( \frac{A}{B} \right)$} \\ 
			\afterheading
			1             & 2                & 2               & 0                     & 1                     & 1                      & -1                                           \\\normalline
			e^5           & e^3              & e               & 5                     & 3                     & 8                      & 2                                            \\\normalline
			36            & \sqrt[3]{6}      & 6               & 2                     & \num{1/3}             & \num{7/3}              & \num{5/3}                                    \\\normalline
			0.001         & 10000            & 10              & -3                    & 4                     & 1                      & -7                                           \\\normalline
			4             & \nicefrac{1}{16} & \nicefrac{1}{4} & -1                    & 2                     & 1                      & -3                                           \\\lastline
		\end{tabular}
	\end{shortsolution}
\end{subproblem}

\begin{table}[!htb]
	\centering
	\caption{\Cref{log:prop:add,log:prop:sub}}
	\label{log:tab:props}
	\begin{tabular}{S[table-format=2.3,parse-numbers=false]S[table-format=5.0,parse-numbers=false]S[table-format=2.0,parse-numbers=false]*{4}{c}}
		\beforeheading
		\heading{$A$} & \heading{$B$}    & \heading{$b$}   & \heading{$\log_b(A)$} & \heading{$\log_b(B)$} & \heading{$\log_b(AB)$} & \heading{$\log_b\left( \frac{A}{B} \right)$} \\ 
		\afterheading
		1             & 2                & 2               &                       &                       &                        &                                              \\\normalline
		e^5           & e^3              & e               &                       &                       &                        &                                              \\\normalline
		36            & \sqrt[3]{6}      & 6               &                       &                       &                        &                                              \\\normalline
		0.001         & 10000            & 10              &                       &                       &                        &                                              \\\normalline
		4             & \nicefrac{1}{16} & \nicefrac{1}{4} &                       &                       &                        &                                              \\\lastline
	\end{tabular}
\end{table}

Use your answer to \cref{log:prob:props} to help you decide if the following 
properties of logarithms are true or false.
\begin{subproblem}
	$\log_b(AB)=\log_b(A)\cdot\log_b(B)$    
	\begin{shortsolution}
		False.
	\end{shortsolution}
\end{subproblem}
\begin{subproblem}
	$\log_b(A+B)=\log_b(A)+\log_b(B)$ 
	\begin{shortsolution}
		False.
	\end{shortsolution}
\end{subproblem}
\begin{subproblem}
	$\log_b(AB)=\log_b(A)+\log_b(B)$ 
	\begin{shortsolution}
		True.
	\end{shortsolution}
\end{subproblem}
\begin{subproblem}
	$\log_b\left( \frac{A}{B} \right)=\frac{\log_b(A)}{\log_b(B)}$ 
	\begin{shortsolution}
		False.
	\end{shortsolution}
\end{subproblem}
\begin{subproblem}
	$\log_b\left( \frac{A}{B} \right)=\log_b(A)-\log_b(B)$
	\begin{shortsolution}
		True.
	\end{shortsolution}
\end{subproblem}
\end{problem}
\begin{exercises}
%===================================
%   Author: Hughes
%   Date:   July 2012
%===================================
\begin{problem}[Change of base]
Use the change of base formula, \cref{log:eq:changebase}, and a calculator to approximate each 
of the following (if possible).
\begin{multicols}{4}
	\begin{subproblem}
		$\log_2(3)$ 
		\begin{shortsolution}
			$\frac{\ln(3)}{\ln(2)}\approx 1.58$
		\end{shortsolution}
	\end{subproblem}
	\begin{subproblem}
		$\log_{23}(-2)$ 
		\begin{shortsolution}
			Undefined since the argument is negative. 
		\end{shortsolution}
	\end{subproblem}
	\begin{subproblem}
		$\log_3(7)$ 
		\begin{shortsolution}
			$\frac{\ln(7)}{\ln(3)}\approx 1.77$
		\end{shortsolution}
	\end{subproblem}
	\begin{subproblem}
		$\log_{\frac{1}{2}}(13)$ 
		\begin{shortsolution}
			$\frac{\ln(13)}{\ln\left( \frac{1}{2} \right)}\approx -3.70$
		\end{shortsolution}
	\end{subproblem}
	\begin{subproblem}
		$\log_8(2)$ 
		\begin{shortsolution}
			$\frac{\ln(2)}{\ln(8)}\approx .33$
		\end{shortsolution}
	\end{subproblem}
	\begin{subproblem}
		$\log_{-1}(5)$ 
		\begin{shortsolution}
			Undefined since the base is negative.
		\end{shortsolution}
	\end{subproblem}
	\begin{subproblem}
		$\log_\pi(5)$ 
		\begin{shortsolution}
			$\frac{\ln(5)}{\ln(\pi)}\approx 1.41$
		\end{shortsolution}
	\end{subproblem}
	\begin{subproblem}
		$\log_2(0)$ 
		\begin{shortsolution}
			Undefined since the argument is $0$.
		\end{shortsolution}
	\end{subproblem}
\end{multicols}
\end{problem}

%===================================
%   Author: Hughes
%   Date:   July 2012
%===================================
\begin{problem}[Expand logarithmic expressions]
Use the properties of logarithms to write each of the following 
expressions as the sum and/or difference of logarithms; leave
your answer in exact form.
\begin{multicols}{4}
	\begin{subproblem}
		$\log(2x)$ 
		\begin{shortsolution}
			$\log(2)+\log(x)$
		\end{shortsolution}
	\end{subproblem}
	\begin{subproblem}
		$\log_3\left( \frac{4}{x} \right)$ 
		\begin{shortsolution}
			$\log_3(4)-\log_3(x)$
		\end{shortsolution}
	\end{subproblem}
	\begin{subproblem}
		$\log_5(x^7)$ 
		\begin{shortsolution}
			$7\log_5(x)$ 
		\end{shortsolution}
	\end{subproblem}
	\begin{subproblem}
		$\log_9(4x^3)$ 
		\begin{shortsolution}
			$\log_9(4)+3\log_9(x)$
		\end{shortsolution}
	\end{subproblem}
	\begin{subproblem}
		$\ln(\sqrt{x})$ 
		\begin{shortsolution}
			$\frac{1}{2}\ln(x)$
		\end{shortsolution}
	\end{subproblem}
	\begin{subproblem}
		$\ln\left( \sqrt[7]{\frac{x^3}{x+2}} \right)$ 
		\begin{shortsolution}
			$\frac{3}{7}\ln(x)-\frac{1}{7}\ln(x+2)$
		\end{shortsolution}
	\end{subproblem}
	\begin{subproblem}
		$\log_\pi\left( \frac{x^2}{4} \right)$ 
		\begin{shortsolution}
			$2\log_\pi(x)-\log_\pi(4)$
		\end{shortsolution}
	\end{subproblem}
	\begin{subproblem}
		$3\log(10x)$ 
		\begin{shortsolution}
			$3+3\log(x)$
		\end{shortsolution}
	\end{subproblem}
\end{multicols}
\end{problem}

%===================================
%   Author: Hughes
%   Date:   July 2012
%===================================
\begin{problem}[Condense logarithmic expressions]
Use the properties of logarithms to write each of the following 
expressions as a single logarithm.
\fixthis{start and finish} 
\end{problem}

%===================================
%   Author: Hughes
%   Date:   July 2012
%===================================
\begin{problem}[Solving equations involving logarithms]
Use the properties of logarithms to help you solve the following 
equations.
\begin{multicols}{2} 
	\begin{subproblem}
		$\log_2(x)+\log_2(7)=3$ 
		\begin{shortsolution}
			$\frac{8}{7}$
		\end{shortsolution}
	\end{subproblem}
	\begin{subproblem}
		$\log_4(x)-\log_4(3)=-1$ 
		\begin{shortsolution}
			$\frac{3}{4}$ 
		\end{shortsolution}
	\end{subproblem}
	\begin{subproblem}
		$\log_3(x)+\log_3(9)=-2$ 
		\begin{shortsolution}
			$\frac{1}{81}$
		\end{shortsolution}
	\end{subproblem}
	\begin{subproblem}
		$\ln(2x)-\ln(9)=0$ 
		\begin{shortsolution}
			$\frac{9}{2}$
		\end{shortsolution}
	\end{subproblem}
\end{multicols}
\end{problem}
%===================================
%   Author: Hughes
%   Date:   July 2012
%===================================
\begin{problem}[Solving equations involving logarithms]
Use the properties of logarithms to help you solve the following 
equations.
\begin{multicols}{2}
	\begin{subproblem}
		$\log_2(x-2)+\log_2(x+9)=\log_2(12)$ 
		\begin{shortsolution}
			$3$
		\end{shortsolution}
	\end{subproblem}
	\begin{subproblem}
		$\ln(x+6)-\ln(x-2)=\ln(5)$ 
		\begin{shortsolution}
			$4$
		\end{shortsolution}
	\end{subproblem}
	\begin{subproblem}
		$\log(x+2)+\log(x-4)=\log(7)$ 
		\begin{shortsolution}
			$5$
		\end{shortsolution}
	\end{subproblem}
	\begin{subproblem}
		$\log_5(x+32)-\log_5(x)=\log_5(5)$ 
		\begin{shortsolution}
			$8$
		\end{shortsolution}
	\end{subproblem}
	\begin{subproblem}
		$\log_3(x-5)+\log_3(x)=\log_3(24)$ 
		\begin{shortsolution}
			$8$
		\end{shortsolution}
	\end{subproblem}
	\begin{subproblem}
		$\ln(x+76)-\ln(x+4)=\ln(9)$ 
		\begin{shortsolution}
			$5$
		\end{shortsolution}
	\end{subproblem}
	\begin{subproblem}
		$\log_{13}(x+3)+\log_{13}(x+1)=\log_{13}(24)$ 
		\begin{shortsolution}
			$3$
		\end{shortsolution}
	\end{subproblem}
	\begin{subproblem}
		$\log_{\pi}(x+58)-\log_{\pi}(x+7)=\log_{\pi}(4)$ 
		\begin{shortsolution}
			$10$
		\end{shortsolution}
	\end{subproblem}
\end{multicols}
\end{problem}

%===================================
%   Author: Hughes
%   Date:   July 2012
%===================================
\begin{problem}[Solving equations involving logarithms]
Use the properties of logarithms to help you solve the following 
equations.
\begin{multicols}{2}
	\begin{subproblem}
		$\log_2(x^2)+\log_2(7)=3$ 
		\begin{shortsolution}
			$\pm\frac{8}{7}$
		\end{shortsolution}
	\end{subproblem}
	\begin{subproblem}
		$\log(\sqrt[3]{x+1})+\log(x+1)=\log_2(16)$ 
		\begin{shortsolution}
			$999$
		\end{shortsolution}
	\end{subproblem}
	\begin{subproblem}
		$\log_3(\sqrt{x})-\log_3(5)=-2$ 
		\begin{shortsolution}
			$\frac{25}{81}$
		\end{shortsolution}
	\end{subproblem}
	\begin{subproblem}
		$\frac{1}{4}\ln(x+1) +\frac{1}{4}\ln(x)=1$
		\begin{shortsolution}
			$\frac{-1+\sqrt{1+4e^4}}{2}\approx 6.91$
		\end{shortsolution}
	\end{subproblem}
\end{multicols}
\end{problem}

%===================================
%   Author: Hughes
%   Date:   July 2012
%===================================
\begin{problem}[Piecewise logarithmic functions]
Consider the function $f$ that has formula
\[
	f(x)=
	\begin{cases}
		\log(x^2) & x <-2 \\
		\ln(x+2)  & x>-2  
	\end{cases}
\]
Evaluate each of the following (if possible), giving both the exact and an approximate answer.
\begin{multicols}{4}
	\begin{subproblem}
		$f(-5)$ 
		\begin{shortsolution}
			$\log(25)\approx 1.40$ 
		\end{shortsolution}
	\end{subproblem}
	\begin{subproblem}
		$f(-1)$ 
		\begin{shortsolution}
			$\ln(1)=0$ 
		\end{shortsolution}
	\end{subproblem}
	\begin{subproblem}
		$f(0)$ 
		\begin{shortsolution}
			$\ln(2)\approx 0.69$ 
		\end{shortsolution}
	\end{subproblem}
	\begin{subproblem}
		$f(-2)$ 
		\begin{shortsolution}
			Undefined. 
		\end{shortsolution}
	\end{subproblem}
\end{multicols}
\end{problem}

%===================================
%   Author: Hughes
%   Date:   July 2012
%===================================
\begin{problem}[Composition of logarithmic functions]
Let $f$ and $g$ be functions that have the following formulas
\[
	f(x)=\log(x+5), \qquad g(x)=\ln(x)
\]
Evaluate each of the following (if possible), giving both the exact 
and an approximate answer.
\begin{multicols}{4}
	\begin{subproblem}
		$(f\circ g)(1)$ 
		\begin{shortsolution}
			$\log(5)\approx 0.70$ 
		\end{shortsolution}
	\end{subproblem}
	\begin{subproblem}
		$(g\circ f)(1)$ 
		\begin{shortsolution}
			$-0.25$ 
		\end{shortsolution}
	\end{subproblem}
	\begin{subproblem}
		$\left(f\circ g \vphantom{e^{-5}}\right)\left( e^{-5} \right)$ 
		\begin{shortsolution}
			Undefined. 
		\end{shortsolution}
	\end{subproblem}
	\begin{subproblem}
		$(g\circ f)(-4)$ 
		\begin{shortsolution}
			Undefined. 
		\end{shortsolution}
	\end{subproblem}
	\begin{subproblem}
		$(f\circ g)\left( \frac{1}{2} \right)$ 
		\begin{shortsolution}
			$\log\left( \ln\left( \frac{1}{2} \right)+5 \right)\approx .63$ 
		\end{shortsolution}
	\end{subproblem}
	\begin{subproblem}
		$(f\circ g)(-3)$ 
		\begin{shortsolution}
			$\ln(\log(2))\approx -1.20$ 
		\end{shortsolution}
	\end{subproblem}
	\begin{subproblem}
		$(f\circ g)(x)$ 
		\begin{shortsolution}
			$\log(\ln(x)+5)$ 
		\end{shortsolution}
	\end{subproblem}
	\begin{subproblem}
		$(g\circ f)(x)$
		\begin{shortsolution}
			$\ln(\log(x+5))$ 
		\end{shortsolution}
	\end{subproblem}
\end{multicols}
\end{problem}

%===================================
%   Author: Hughes
%   Date:   August 2012
%===================================
\begin{problem}[Decomposition]
In each of the following problems, you are given a formula for function  
$h$. Decompose $h$ into two functions $f$ and $g$ such that $h=f\circ g$.
\begin{multicols}{4}
	\begin{subproblem}
		$h(x)=\log(3x^2)$
		\begin{shortsolution}
			$f(x)=\log(x)$, $g(x)=3x^2$
		\end{shortsolution}
	\end{subproblem}
	\begin{subproblem}
		$h(x)=-2\ln(5-x)$
		\begin{shortsolution}
			$f(x)=-2\ln(x)$, $g(x)=5-x$
		\end{shortsolution}
	\end{subproblem}
	\begin{subproblem}
		$h(x)=\log_3\left( \sqrt[3]{x} \right)$
		\begin{shortsolution}
			$f(x)=\log_3(x)$, $g(x)=\sqrt[3]{x}$
		\end{shortsolution}
	\end{subproblem}
	\begin{subproblem}
		$h(x)=\log_5(x^2)+7^{x^2}$
		\begin{shortsolution}
			$f(x)=\log_5(x)+7^x$, $g(x)=x^2$
		\end{shortsolution}
	\end{subproblem}
\end{multicols}
\end{problem}

%===================================
%   Author: Hughes
%   Date:   July 2012
%===================================
\begin{problem}[Function algebra]
Let $f$ and $g$ be the functions that have formulas
\[
	f(x)=\log(x), \qquad g(x)=\ln(x)
\]
Evaluate each of the following (if possible), giving the exact
and an approximate solution (where appropriate).
\begin{multicols}{4}
	\begin{subproblem}
		$(f+g)(1)$ 
		\begin{shortsolution}
			$2$ 
		\end{shortsolution}
	\end{subproblem}
	\begin{subproblem}
		$(f-g)(1)$ 
		\begin{shortsolution}
			$0$ 
		\end{shortsolution}
	\end{subproblem}
	\begin{subproblem}
		$(f\cdot g)(1)$ 
		\begin{shortsolution}
			$1$ 
		\end{shortsolution}
	\end{subproblem}
	\begin{subproblem}
		$\left( \frac{f}{g} \right)(1)$ 
		\begin{shortsolution}
			Undefined. 
		\end{shortsolution}
	\end{subproblem}
	\begin{subproblem}
		$(f+g)(5)$ 
		\begin{shortsolution}
			$\log(5)+\ln(5)\approx 2.31$ 
		\end{shortsolution}
	\end{subproblem}
	\begin{subproblem}
		$(f-g)(\pi)$ 
		\begin{shortsolution}
			$\log(\pi)-\ln(\pi)\approx -0.65$       
		\end{shortsolution}
	\end{subproblem}
	\begin{subproblem}
		$(f\cdot g)(e)$ 
		\begin{shortsolution}
			$\log(e)\approx .43$  
		\end{shortsolution}
	\end{subproblem}
	\begin{subproblem}
		$\left( \frac{f}{g} \right)\left( \frac{1}{2} \right)$ 
		\begin{shortsolution}
			$\frac{\log\left( \frac{1}{2} \right)}{\ln\left( \frac{1}{2} \right)}\approx 0.43$ 
		\end{shortsolution}
	\end{subproblem}
\end{multicols}
\end{problem}

\end{exercises}

%%+*** 111,112document.tex
% arara: indent: {overwrite: on, trace: on, localSettings: yes}
%===================================
%
%   Last edited: Hughes
%                11/18/12 (v19)
%
%===================================
\chapter{Polynomial and Rational Functions}
\minitoc
\section{Polynomial functions}
\reformatstepslist{P} % the steps list should be P1, P2, \ldots
In your previous mathematics classes you have studied \emph{linear} and 
\emph{quadratic} functions. The most general forms of these types of 
functions can be represented (respectively) by  the functions $f$ 
and $g$ that have formulas
\begin{equation}\label{poly:eq:linquad}
	f(x)=mx+b, \qquad g(x)=ax^2+bx+c
\end{equation}
We know that $m$ is the slope of $f$, and that $a$ is the \emph{leading coefficient} 
of $g$. We also know that the \emph{signs} of $m$ and $a$ completely 
determine the behavior of the functions $f$ and $g$. For example, if $m>0$
then $f$ is an \emph{increasing} function, and if $m<0$ then $f$ is 
a \emph{decreasing} function.  Similarly, if $a>0$ then $g$ is 
\emph{concave up} and if $a<0$ then $g$ is \emph{concave down}. Graphical 
representations of these statements are given in \cref{poly:fig:linquad}.

\begin{figure}[!htb]
	\setlength{\figurewidth}{.2\textwidth}
	\begin{subfigure}{\figurewidth}
		\begin{tikzpicture}
			\begin{axis}[
					framed,
					xmin=-10,xmax=10,
					ymin=-10,ymax=10,
					minor xtick={-8,-4,...,8},
					minor ytick={-8,-4,...,8},
					xtick={-11},
					ytick={-11},
					grid=minor,
				]
				\addplot expression[domain=-10:8]{(x+2)};
			\end{axis}
		\end{tikzpicture}
		\caption{$m>0$}
	\end{subfigure}
	\hfill
	\begin{subfigure}{\figurewidth}
		\begin{tikzpicture}
			\begin{axis}[
					framed,
					xmin=-10,xmax=10,
					ymin=-10,ymax=10,
					minor xtick={-8,-4,...,8},
					minor ytick={-8,-4,...,8},
					xtick={-11},
					ytick={-11},
					grid=minor,
				]
				\addplot expression[domain=-10:8]{-(x+2)};
			\end{axis}
		\end{tikzpicture}
		\caption{$m<0$}
	\end{subfigure}
	\hfill
	\begin{subfigure}{\figurewidth}
		\begin{tikzpicture}
			\begin{axis}[
					framed,
					xmin=-10,xmax=10,
					ymin=-10,ymax=10,
					minor xtick={-8,-4,...,8},
					minor ytick={-8,-4,...,8},
					xtick={-11},
					ytick={-11},
					grid=minor,
				]
				\addplot expression[domain=-4:4]{(x^2-6)};
			\end{axis}
		\end{tikzpicture}
		\caption{$a>0$}
	\end{subfigure}
	\hfill
	\begin{subfigure}{\figurewidth}
		\begin{tikzpicture}
			\begin{axis}[
					framed,
					xmin=-10,xmax=10,
					ymin=-10,ymax=10,
					minor xtick={-8,-4,...,8},
					minor ytick={-8,-4,...,8},
					xtick={-11},
					ytick={-11},
					grid=minor,
				]
				\addplot expression[domain=-4:4]{-(x^2-6)};
			\end{axis}
		\end{tikzpicture}
		\caption{$a<0$}
	\end{subfigure}
	\caption{Typical graphs of linear and quadratic functions.}
	\label{poly:fig:linquad}
\end{figure}

Let's look a little more closely at the formulas for $f$ and $g$ in 
\cref{poly:eq:linquad}. Note that the \emph{degree} 
of $f$ is $1$ since the highest power of $x$ that is present in the 
formula for $f(x)$ is $1$. Since $f$ has $2$ terms, we may call it 
a \emph{bi}nomial function. Similarly, the degree of $g$ is $2$ since
the highest power of $x$ that is present in the formula for $g(x)$ 
is $2$. Since $g$ has $3$ terms, we may call it a \emph{tri}nomial 
function.

In this section we will build upon our knowledge of these elementary
functions. In particular, we will generalize our knowledege of 
the functions $f$ and $g$ 
to the study of a \emph{poly}nomial function $p$ that has any degree (and 
any number of terms) that we wish. The only restriction that we will 
enforce is that the degree of $p$ must be an integer.

%===================================
%   Author: Hughes
%   Date:   March 2012
%===================================
\begin{essentialskills}
	%===================================
	%   Author: Hughes
	%   Date:   March 2012
	%===================================
	\begin{problem}[Quadratic functions]
	Every quadratic function has the form $y=ax^2+bx+c$; state the value 
	of $a$ for each of the following functions, and hence decide if the 
	parabola that represents the function opens upward or downward.
	\begin{multicols}{2}
		\begin{subproblem}
			$F(x)=x^2+3$ 
			\begin{shortsolution}
				$a=1$; the parabola opens upward. 
			\end{shortsolution}
		\end{subproblem}
		\begin{subproblem}
			$G(t)=4-5t^2$ 
			\begin{shortsolution}
				$a=-5$; the parabola opens downward. 
			\end{shortsolution}
		\end{subproblem}
		\begin{subproblem}
			$H(y)=4y^2-96y+8$ 
			\begin{shortsolution}
				$a=4$; the parabola opens upward. 
			\end{shortsolution}
		\end{subproblem}
		\begin{subproblem}
			$K(z)=-19z^2$ 
			\begin{shortsolution}
				$m=-19$; the parabola opens downward. 
			\end{shortsolution}
		\end{subproblem}
	\end{multicols}
	Now let's generalize our findings for the most general quadratic function $g$
	that has formula $g(x)=a_2x^2+a_1x+a_0$. Complete the following sentences.
	\begin{subproblem}
		When $a_2>0$, the parabola that represents $y=g(x)$ opens $\ldots$ 
		\begin{shortsolution}
			When $a_2>0$, the parabola that represents the function opens upward.
		\end{shortsolution}
	\end{subproblem}
	\begin{subproblem}
		When $a_2<0$, the parabola that represents $y=g(x)$ opens $\ldots$ 
		\begin{shortsolution}
			When $a_2<0$, the parabola that represents the function opens downward.
		\end{shortsolution}
	\end{subproblem}
	\end{problem}
\end{essentialskills}

\subsection*{Power functions with positive exponents}
The study of polynomials will rely upon a good knowledge 
of power functions| you may reasonably ask, what is a power function?
\begin{pccdefinition}[Power functions]
	The most general formula for a power functions is
	\[
		f(x) = a_n x^n
	\]
	where $n$ can be any real number.
	
	Note that for this section we will only be concerned with the 
	case when $n$ is a positive integer.
\end{pccdefinition}

You may find assurance in the fact that you are already very comfortable 
with power functions that have $n=1$ (linear) and $n=2$ (quadratic). Let's 
explore some power functions that you might not be so familiar with.
As you read \cref{poly:ex:oddpow,poly:ex:evenpow}, try and spot 
as many patterns and similarities as you can.

%===================================
%   Author: Hughes
%   Date:   March 2012
%===================================
\begin{pccexample}[Power functions with odd positive exponents]
	\label{poly:ex:oddpow}
	Graph each the functions $f$, $g$, and $h$ that have 
	formulas
	\[
		f(x)=x^3,   \qquad  g(x)=x^5, \qquad h(x)=x^7
	\]
	and state their domain, and their long-run behavior as $x\rightarrow\pm\infty$
	\begin{pccsolution}
		The functions $f$, $g$, and $h$ are plotted in \cref{poly:fig:oddpow}.
		The domain of each of the functions $f$, $g$, and $h$ is $(-\infty,\infty)$. Note that 
		the long-run behavior of each of the functions is the same, and in particular
		\begin{align*}
			f(x)\rightarrow\infty                           & \text{ as } x\rightarrow\infty  \\ 
			\mathllap{\text{and }}   f(x)\rightarrow-\infty & \text{ as } x\rightarrow-\infty 
		\end{align*}
		The same results hold for $g$ and $h$. Note that the range of each of the 
		functions $f$, $g$, and $h$ is $(-\infty,\infty)$.
		
		It appears from \cref{poly:fig:oddpow} that each of the functions $f$, $g$, 
		and $h$ are symmetric about the origin. Remember from REF 
		\fixthis{need reference to definition about even and odd functions- doesn't
		exist yet}
		that a function that exhibits this behavior is called \emph{odd}. We can 
		test a function algebraically to see if it is odd by evaluating $f(-x)$; let's
		do that for each of the functions $f$, $g$, and $h$:
		\begin{align*}
			f(-x) & =(-x)^3 & g(-x) & =(-x)^5 & h(-x) & =(-x)^7 \\ 
			      & =-x^3   &       & =-x^5   &       & =-x^7   \\   
			      & =-f(x)  &       & =-g(x)  &       & =-h(x)  
		\end{align*}
		We conclude that each of the functions $f$, $g$, and $h$ are odd.
	\end{pccsolution}
\end{pccexample}

\begin{figure}[!htb]
	\begin{minipage}{.45\textwidth}
		\begin{tikzpicture}
			\begin{axis}[
					framed,
					xmin=-1.5,xmax=1.5,
					ymin=-5,ymax=5,
					xtick={-1.0,-0.5,...,1.0},
					minor ytick={-3,-1,...,3},
					grid=both,
					legend pos=north west,
				]
				\addplot expression[domain=-1.5:1.5]{x^3};
				\addplot expression[domain=-1.379:1.379]{x^5};
				\addplot expression[domain=-1.258:1.258]{x^7};
				\addplot[soldot]coordinates{(-1,-1)} node[axisnode,anchor=north west]{$(-1,-1)$};
				\addplot[soldot]coordinates{(1,1)} node[axisnode,anchor=south east]{$(1,1)$};
				\legend{$f$,$g$,$h$}
			\end{axis}
		\end{tikzpicture}
		\caption{Odd power functions}
		\label{poly:fig:oddpow}
	\end{minipage}%
	\hfill
	\begin{minipage}{.45\textwidth}
		\begin{tikzpicture}
			\begin{axis}[
					framed,
					xmin=-2.5,xmax=2.5,
					ymin=-5,ymax=5,
					xtick={-2.0,-1.5,...,2.0},
					minor ytick={-3,-1,...,3},
					grid=both,
					legend pos=south east,
				]
				\addplot expression[domain=-2.236:2.236]{x^2};
				\addplot expression[domain=-1.495:1.495]{x^4};
				\addplot expression[domain=-1.307:1.307]{x^6};
				\addplot[soldot]coordinates{(-1,1)} node[axisnode,anchor=east]{$(-1,1)$};
				\addplot[soldot]coordinates{(1,1)} node[axisnode,anchor=west]{$(1,1)$};
				\legend{$F$,$G$,$H$}
			\end{axis}
		\end{tikzpicture}
		\caption{Even power functions}
		\label{poly:fig:evenpow}
	\end{minipage}%
\end{figure}

%===================================
%   Author: Hughes
%   Date:   March 2012
%===================================
\begin{pccexample}[Power functions with even positive exponents]\label{poly:ex:evenpow}%
	Graph each the functions $F$, $G$, and $H$ that 
	have formulas
	\[
		F(x)=x^2, \qquad G(x)=x^4, \qquad H(x)=x^6
	\]
	and  state their domain, and their long-run behavior as $x\rightarrow\pm\infty$
	\begin{pccsolution}
		The functions $F$, $G$, and $H$ are plotted in \cref{poly:fig:evenpow}. The domain
		of each of the functions is $(-\infty,\infty)$. Note that the long-run behavior 
		of each of the functions is the same, and in particular
		\begin{align*}
			F(x)\rightarrow\infty                          & \text{ as } x\rightarrow\infty  \\ 
			\mathllap{\text{and }}   F(x)\rightarrow\infty & \text{ as } x\rightarrow-\infty 
		\end{align*}
		The same result holds for $G$ and $H$. Note that the range of each of 
		the functions $F$, $G$, and $H$ is $[0,\infty)$.
		
		It appears from \cref{poly:fig:evenpow} that each of the functions $F$, $G$, 
		and $H$ are symmetric across the vertical axis. Remember from REF 
		\fixthis{need reference to definition about even and odd functions- doesn't
		exist yet}
		that a function that exhibits this behavior is called \emph{even}. We can 
		test a function algebraically to see if it is even by evaluating $f(-x)$; let;s
		do that for each of the functions $F$, $G$, and $H$:
		\begin{align*}
			F(-x) & =(-x)^2 & G(-x) & =(-x)^4 & H(-x) & =(-x)^6 \\ 
			      & =x^2    &       & =x^4    &       & =x^6    \\    
			      & =F(x)   &       & =G(x)   &       & =H(x)   
		\end{align*}
		We conclude that each of the functions $F$, $G$, and $H$ are even.
	\end{pccsolution}
\end{pccexample}

\begin{doyouunderstand}
	\begin{problem}
	Repeat \cref{poly:ex:oddpow,poly:ex:evenpow} using (respectively) the 
	functions that have the following formulas.
	\begin{subproblem}
		$f(x)=-x^3,   \qquad  g(x)=-x^5, \qquad h(x)=-x^7$
		\begin{shortsolution}
			The functions $f$, $g$, and $h$ have domain $(-\infty,\infty)$ and 
			are graphed below.
						
			\begin{tikzpicture}
				\begin{axis}[
						framed,
						xmin=-1.5,xmax=1.5,
						ymin=-5,ymax=5,
						xtick={-1.0,-0.5,...,0.5},
						minor ytick={-3,-1,...,3},
						grid=both,
						legend pos=north east,
					]
					\addplot expression[domain=-1.5:1.5]{-x^3};
					\addplot expression[domain=-1.379:1.379]{-x^5};
					\addplot expression[domain=-1.258:1.258]{-x^7};
					\legend{$f$,$g$,$h$}
				\end{axis}
			\end{tikzpicture}
						
			Note that
			\begin{align*}
				f(x)\rightarrow-\infty                         & \text{ as } x\rightarrow\infty  \\ 
				\mathllap{\text{and }}   f(x)\rightarrow\infty & \text{ as } x\rightarrow-\infty 
			\end{align*}
			The same is true for $g$ and $h$. The range of $f$, $g$, and $h$ 
			is $(-\infty,\infty)$. 
						
			Each of the functions $f$, $g$, and $h$ are odd
			\begin{align*}
				f(-x) & =-(-x)^3 & g(-x) & =-(-x)^5 & h(-x) & =-(-x)^7 \\ 
				      & =x^3     &       & =x^5     &       & =x^7     \\     
				      & =-f(x)   &       & =-g(x)   &       & =-h(x)   
			\end{align*}
		\end{shortsolution}
	\end{subproblem}
	\begin{subproblem}
		$F(x)=-x^2,   \qquad  G(x)=-x^4, \qquad H(x)=-x^6$
		\begin{shortsolution}
			The functions $F$, $G$, and $H$ have domain $(-\infty,\infty)$ and 
			are graphed below.
						
			\begin{tikzpicture}
				\begin{axis}[
						framed,
						xmin=-2.5,xmax=2.5,
						ymin=-5,ymax=5,
						xtick={-1.0,-0.5,...,0.5},
						minor ytick={-3,-1,...,3},
						grid=both,
						legend pos=north east,
					]
					\addplot expression[domain=-2.236:2.236]{-x^2};
					\addplot expression[domain=-1.495:1.495]{-x^4};
					\addplot expression[domain=-1.307:1.307]{-x^6};
					\legend{$F$,$G$,$H$}
				\end{axis}
			\end{tikzpicture}
						
			Note that
			\begin{align*}
				F(x)\rightarrow-\infty                          & \text{ as } x\rightarrow\infty  \\ 
				\mathllap{\text{and }}   F(x)\rightarrow-\infty & \text{ as } x\rightarrow-\infty 
			\end{align*}
			The same is true for $G$ and $H$. The range of $F$, $G$, and $H$ 
			is $(-\infty,0]$.
						
			Each of the functions $F$, $G$, and $H$ are even
			\begin{align*}
				F(-x) & =-(-x)^2 & G(-x) & =-(-x)^4 & H(-x) & =-(-x)^6 \\ 
				      & =-x^2    &       & =-x^4    &       & =-x^6    \\    
				      & =F(x)    &       & =G(x)    &       & =H(x)    
			\end{align*}
		\end{shortsolution}
	\end{subproblem}
	\end{problem}
\end{doyouunderstand}

\subsection*{Polynomial functions}
Now that we have a little more familiarity with power functions, 
we can define polynomial functions. Provided that you were comfortable
with our opening discussion about linear and quadratic functions (see 
$f$ and $g$ in \cref{poly:eq:linquad}) then there is every chance 
that you'll be able to master polynomial functions as well; just remember
that polynomial functions are a natural generalization of linear
and quadratic functions. Once you've studied the examples and problems
in this section, you'll hopefully agree that polynomial functions
are remarkably predictable.

%===================================
%   Author: Hughes
%   Date:   May 2011
%===================================
\begin{pccdefinition}[Polynomial functions]
	The most general formula for a polynomial function, $p$, is
	\[
		p(x)=a_nx^n+a_{n-1}x^{n-1}+\ldots+a_1x+a_0
	\]
	where $a_n$, $a_{n-1}$, $a_{n-2}$, \ldots, $a_0$ are real numbers.
	\begin{itemize}
		\item We call $n$ the degree of the polynomial, and require that $n$
		is a non-negative integer;
		\item $a_n$, $a_{n-1}$, $a_{n-2}$, \ldots, $a_0$ are called the coefficients;
		\item We typically write polynomial functions in descending powers of $x$.
	\end{itemize}
	In particular, we call $a_n$ the \emph{leading} coefficient, and $a_nx^n$ the 
	\emph{leading term}.
	
	Note that if a polynomial is given in factored form, then the degree can be found 
	by counting the number of linear factors.
\end{pccdefinition}

%===================================
%   Author: Hughes
%   Date:   March 2012
%===================================
\begin{pccexample}[Polynomial or not]
	Decide if the following formulas correspond to polynomial functions 
	or not; if so, state the degree of the polynomial.
	\begin{multicols}{3}
		\begin{enumerate}
			\item $p(x)=x^2-3$     
			\item $q(x)=-4x^{\nicefrac{1}{2}}+10$     
			\item $r(x)=10x^5$
			\item $s(x)=x^{-2}+x^{23}$
			\item $f(x)=-8$
			\item $g(x)=3^x$
			\item $h(x)=\sqrt[3]{x^7}-x^2+x$
			\item $k(x)=4x(x+2)(x-3)$
			\item $j(x)=x^2(x-4)(5-x)$
		\end{enumerate}
	\end{multicols}
	\begin{pccsolution}
		\begin{enumerate}
			\item $p$ is a polynomial, and its degree is $2$.
			\item $q$ is \emph{not} a polynomial, because $\frac{1}{2}$ is not an integer.
			\item $r$ is a polynomial, and its degree is $5$.
			\item $s$ is \emph{not} a polynomial, because $-2$ is not a positive integer.
			\item $f$ is a polynomial, and its degree is $0$.
			\item $g$ is \emph{not} a polynomial, because the independent 
			variable, $x$, is in the exponent.
			\item $h$ is \emph{not} a polynomial, because $\frac{7}{3}$ is not an integer.
			\item $k$ is a polynomial, and its degree is $3$.
			\item $j$ is a polynomial, and its degree is $4$.
		\end{enumerate}
	\end{pccsolution}
\end{pccexample}

%===================================
%   Author: Hughes
%   Date:   March 2012
%===================================
\begin{pccexample}[Typical graphs]\label{poly:ex:typical}
	\Cref{poly:fig:typical} shows graphs of some polynomial functions;
	the ticks have deliberately been left off the axis to allow us to concentrate
	on the features of each graph. Note in particular that:
	\begin{itemize}
		\item \cref{poly:fig:typical1} shows a degree-$1$ polynomial (you might also 
		classify the function as linear) whose leading coefficient, $a_1$, is positive.
		\item \cref{poly:fig:typical2} shows a degree-$2$ polynomial (you might also
		classify the function as quadratic) whose leading coefficient, $a_2$, is positive.
		\item \cref{poly:fig:typical3} shows a degree-$3$ polynomial whose leading coefficient, $a_3$,
		is positive| compare its overall
		shape and long-run behavior to the functions described in \cref{poly:ex:oddpow}.
		\item \cref{poly:fig:typical4} shows a degree-$4$ polynomial whose leading coefficient, $a_4$,
		is positive|compare its overall shape and long-run behavior to the functions described in \cref{poly:ex:evenpow}.
		\item \cref{poly:fig:typical5} shows a degree-$5$ polynomial whose leading coefficient, $a_5$,
		is positive| compare its overall
		shape and long-run behavior to the functions described in \cref{poly:ex:oddpow}.
	\end{itemize}
\end{pccexample}

%===================================
%   Author: Hughes
%   Date:   May 2011
%===================================
\begin{figure}[!htb]
	\begin{widepage}
	\setlength{\figurewidth}{\textwidth/6}
	\begin{subfigure}{\figurewidth}
		\begin{tikzpicture}
			\begin{axis}[
					framed,
					xmin=-10,xmax=10,
					ymin=-10,ymax=10,
					minor xtick={-8,-4,...,8},
					minor ytick={-8,-4,...,8},
					xtick={-11},
					ytick={-11},
					grid=minor,
				]
				\addplot expression[domain=-10:8]{(x+2)};
			\end{axis}
		\end{tikzpicture}
		\caption{$a_1>0$}
		\label{poly:fig:typical1}
	\end{subfigure}
	\hfill
	\begin{subfigure}{\figurewidth}
		\begin{tikzpicture}
			\begin{axis}[
					framed,
					xmin=-10,xmax=10,
					ymin=-10,ymax=10,
					minor xtick={-8,-4,...,8},
					minor ytick={-8,-4,...,8},
					xtick={-11},
					ytick={-11},
					grid=minor,
				]
				\addplot expression[domain=-4:4]{(x^2-6)};
			\end{axis}
		\end{tikzpicture}
		\caption{$a_2>0$}
		\label{poly:fig:typical2}
	\end{subfigure}
	\hfill
	\begin{subfigure}{\figurewidth}
		\begin{tikzpicture}
			\begin{axis}[
					framed,
					xmin=-10,xmax=10,
					ymin=-10,ymax=10,
					minor xtick={-8,-4,...,8},
					minor ytick={-8,-4,...,8},
					xtick={-11},
					ytick={-11},
					grid=minor,
				]
				\addplot expression[domain=-7.5:7.5]{0.05*(x+6)*x*(x-6)};
			\end{axis}
		\end{tikzpicture}
		\caption{$a_3>0$}
		\label{poly:fig:typical3}
	\end{subfigure}
	\hfill
	\begin{subfigure}{\figurewidth}
		\begin{tikzpicture}
			\begin{axis}[
					framed,
					xmin=-10,xmax=10,
					ymin=-10,ymax=10,
					minor xtick={-8,-4,...,8},
					minor ytick={-8,-4,...,8},
					xtick={-11},
					ytick={-11},
					grid=minor,
				]
				\addplot expression[domain=-2.35:5.35,samples=100]{0.2*(x-5)*x*(x-3)*(x+2)};
			\end{axis}
		\end{tikzpicture}
		\caption{$a_4>0$}
		\label{poly:fig:typical4}
	\end{subfigure}
	\hfill
	\begin{subfigure}{\figurewidth}
		\begin{tikzpicture}
			\begin{axis}[
					framed,
					xmin=-10,xmax=10,
					ymin=-10,ymax=10,
					minor xtick={-8,-4,...,8},
					minor ytick={-8,-4,...,8},
					xtick={-11},
					ytick={-11},
					grid=minor,
				]
				\addplot expression[domain=-5.5:6.3,samples=100]{0.01*(x+2)*x*(x-3)*(x+5)*(x-6)};
			\end{axis}
		\end{tikzpicture}
		\caption{$a_5>0$}
		\label{poly:fig:typical5}
	\end{subfigure}
	\end{widepage}
	\caption{Graphs to illustrate typical curves of polynomial functions.}
	\label{poly:fig:typical}
\end{figure}

%===================================
%   Author: Hughes
%   Date:   March 2012
%===================================
\begin{doyouunderstand}
	\begin{problem}
	Use \cref{poly:ex:typical} and \cref{poly:fig:typical} to help you sketch 
	the graphs of polynomial functions that have negative leading coefficients| note
	that there are many ways to do this! The intention with this problem
	is to use your knowledge of transformations- in particular, \emph{reflections}- 
	to guide you.
	\begin{shortsolution}
		$a_1<0$:
				
		\begin{tikzpicture}
			\begin{axis}[
					framed,
					xmin=-10,xmax=10,
					ymin=-10,ymax=10,
					xtick={-11},
					ytick={-11},
				]
				\addplot expression[domain=-10:8]{-(x+2)};
			\end{axis}
		\end{tikzpicture}
				
		$a_2<0$
				
		\begin{tikzpicture}
			\begin{axis}[
					framed,
					xmin=-10,xmax=10,
					ymin=-10,ymax=10,
					xtick={-11},
					ytick={-11},
				]
				\addplot expression[domain=-4:4]{-(x^2-6)};
			\end{axis}
		\end{tikzpicture}
				
		$a_3<0$
				
		\begin{tikzpicture}
			\begin{axis}[
					framed,
					xmin=-10,xmax=10,
					ymin=-10,ymax=10,
					xtick={-11},
					ytick={-11},
				]
				\addplot expression[domain=-7.5:7.5]{-0.05*(x+6)*x*(x-6)};
			\end{axis}
		\end{tikzpicture}
				
		$a_4<0$
				
		\begin{tikzpicture}
			\begin{axis}[
					framed,
					xmin=-10,xmax=10,
					ymin=-10,ymax=10,
					xtick={-11},
					ytick={-11},
				]
				\addplot expression[domain=-2.35:5.35,samples=100]{-0.2*(x-5)*x*(x-3)*(x+2)};
			\end{axis}
		\end{tikzpicture}
				
		$a_5<0$
				
		\begin{tikzpicture}
			\begin{axis}[
					framed,
					xmin=-10,xmax=10,
					ymin=-10,ymax=10,
					xtick={-11},
					ytick={-11},
				]
				\addplot expression[domain=-5.5:6.3,samples=100]{-0.01*(x+2)*x*(x-3)*(x+5)*(x-6)};
			\end{axis}
		\end{tikzpicture}
	\end{shortsolution}
	\end{problem}
\end{doyouunderstand}

The main intention behind \cref{poly:ex:typical} was to provide sketches
of some typical polynomial functions. The graphs in \cref{poly:fig:typical}
do not have much detail| in \cref{poly:ex:detail} we study two polynomial
functions in much more depth.

%===================================
%   Author: Hughes
%   Date:   August 2012
%===================================
\begin{pccexample}\label{poly:ex:detail}
	Study the graphs of the polynomial functions $p$ and $q$ defined by the following 
	formulas:
	\begin{multicols}{2}
		\begin{enumerate}
			\item $p(x)=\frac{1}{8}(x+6)(x+1)(x-5)$
			\item $q(x)=\frac{x}{20}(x+4)(x-3)(x-6)$
		\end{enumerate}
	\end{multicols}
	Describe the long-run behavior, the intervals of increase and decrease, 
	and the intervals of concavity of each function.
	Determine if each function is odd, even, or neither.
	\begin{pccsolution}
		\begin{enumerate}
			\item The first observation we note about the function $p$ is that 
			since it has three linear factors, the degree of $p$ is $3$. We
			can illustrate this further by expanding the formula for $p(x)$
			\[
				p(x)=\frac{x^3}{8}+\frac{x^2}{4}-\frac{29x}{8}-\frac{15}{4}
			\]  
			The curve $y=p(x)$ is graphed in \cref{poly:fig:detailed1}. There are
			three zeros of $p$: $-6$, $-1$, and $5$. 
				
			In order to determine the long-rung behavior of $p$, we examine the leading
			term of $p(x)$ which is $\frac{x^3}{8}$. If we view \cref{poly:fig:detailed1} 
			on a larger viewing window (imagine zooming out), then we can visualize that the overall
			shape of the curve $y=p(x)$ will look like the curve $y=\frac{x^3}{8}$ (see \cref{poly:fig:oddpow}).
				
			We can approximate the intervals of increase and decrease using 
			\cref{poly:fig:detailed1}. $p$ is increasing on (approximately) the 
			interval $(-\infty,-3.9)\cup (2.2,\infty)$ and decreasing on (approximately)
			the interval $(-3.9,2.2)$.
				
			We may similarly approximate the intervals of concavity: $p$ is concave down
			on (approximately) the interval $(-\infty,-0.5)$ and is concave up on 
			(approximately) the interval $(-0.5,\infty)$.
				
			Remember that all of the power functions in \cref{poly:ex:oddpow} 
			have \emph{odd} exponents and are \emph{odd} functions. Does it therefore
			follow that since $p$ is a degree-$3$ polynomial and $3$ is an odd number, 
			that $p$ is an odd function? Let's evaluate $p(-x)$ to find out
			\begin{align*}
				p(-x) & =\frac{1}{8}(-x+6)(-x+1)(-x-5) \\ 
				      & \ne -p(x) \text{ or } p(x)     
			\end{align*}
			We therefore conclude that $p$ is neither odd nor even; this is confirmed visually in 
			\cref{poly:fig:detailed1} since the curve $y=p(x)$ is not symmetric about the origin 
			nor about the vertical axis.
				
			\begin{figure}[!htb]
				\setlength{\figurewidth}{.4\textwidth}
				\begin{subfigure}{\figurewidth}
					\begin{tikzpicture}
						\begin{axis}[
								framed,
								xmin=-10,xmax=10,
								ymin=-10,ymax=10,
								xtick={-8,-6,...,8},
								minor xtick={-9,-7,...,9},
								ytick={-8,-6,...,8},
								minor ytick={-9,-7,...,9},
								grid=major,
							]
							\addplot expression[domain=-7.0872:5.9608,samples=50]{1/8*(x+6)*(x+1)*(x-5)};
							\addplot[soldot]coordinates{(-6,0)(-1,0)(5,0)};
						\end{axis}
					\end{tikzpicture}
					\caption{$y=\frac{1}{8}(x+6)(x+1)(x-5)$}
					\label{poly:fig:detailed1}
				\end{subfigure}
				\hfill
				\begin{subfigure}{\figurewidth}
					\begin{tikzpicture}
						\begin{axis}[
								framed,
								xmin=-10,xmax=10,
								ymin=-10,ymax=10,
								xtick={-8,-6,...,8},
								minor xtick={-9,-7,...,9},
								ytick={-8,-6,...,8},
								minor ytick={-9,-7,...,9},
								grid=major,
							]
							\addplot expression[domain=-4.551:6.739,samples=50]{x/20*(x+4)*(x-3)*(x-6)};
							\addplot[soldot]coordinates{(-4,0)(0,0)(3,0)(6,0)};
						\end{axis}
					\end{tikzpicture}
					\caption{$y=\frac{x}{20}(x+4)(x-3)(x-6)$}
					\label{poly:fig:detailed2}
				\end{subfigure}
				\caption{The functions $p$ and $q$.}
				\label{poly:fig:detailed}
			\end{figure}
			\item The degree of $q$ is $4$ since it has four linear factors. $q$ has 
			four zeros: $-4$, $0$, $3$, and $6$. Furthermore,
			me may expand the formula for $q(x)$
			\[
				q(x)=\frac{x^4}{20}-\frac{x^3}{4}-\frac{9x^2}{10}+\frac{18x}{5}
			\]
			which allows to see that the leading term of $q$ is $\frac{x^4}{20}$. If
			we imagine viewing \cref{poly:fig:detailed2} on a larger viewing 
			window, then we can visualize that the overall shape of the curve $y=q(x)$
			will look like $y=\frac{x^4}{20}$ (see \cref{poly:fig:evenpow}).
				
			Using \cref{poly:fig:detailed2} as a guide, we see that $q$ is increasing
			on (approximately) the interval $(-2.2,1.5)\cup (4.8,\infty)$ and decreasing
			on (approximately) the interval $(-\infty,-2.2)\cup (1.5,4.8)$.
				
			We can also approximate the intervals of concavity: $q$ is concave up on 
			(approximately) the interval $(-\infty,-1)\cup (3.1,\infty)$, and is concave 
			down on (approximately) the interval $(-1,3.1)$.
				
			The power functions in \cref{poly:ex:evenpow} have \emph{even} 
			exponents and are \emph{even} functions. Since $q$ has degree $4$, which 
			is an even number, does it therefore follow that $q$ is an even function?
			Let's evaluate $q(-x)$ to find out
			\begin{align*}
				q(-x) & =-\frac{x}{20}(-x+4)(-x-3)(-x-6) \\ 
				      & \ne q(x) \text{ or } -q(x)       
			\end{align*}
			We conclude that $q$ is neither even nor odd; this is confirmed visually in 
			\cref{poly:fig:detailed2}, since the curve $y=q(x)$ is not symmetric about 
			the vertical axis nor about the origin.
		\end{enumerate}
	\end{pccsolution}
\end{pccexample}

The polynomial functions in \cref{poly:ex:detail} had many differences, 
but they also had one feature in common| at each of their zeros, the 
curve of the function \emph{crossed through} the horizontal axis. Not 
all polynomial functions exhibit this behavior as we shall see in 
the next example.

%===================================
%   Author: Hughes
%   Date:   May 2011
%===================================
\begin{pccexample}[Multiple zeros]
	Consider the polynomial functions $p$, $q$, and $r$ which are 
	graphed in \cref{poly:fig:moremultiple}. 
	The formulas for $p$, $q$, and $r$ are as follows
	\begin{align*}
		p(x) & =(x-3)^2(x+4)^2       \\       
		q(x) & =x(x+2)^2(x-1)^2(x-3) \\ 
		r(x) & =x(x-3)^3(x+1)^2      
	\end{align*}
	Find the degree of $p$, $q$, and $r$, and decide if the functions bounce off or cut 
	through the horizontal axis at each of their zeros.
	\begin{pccsolution}
		The degree of $p$ is 4. Referring to \cref{poly:fig:bouncep}, 
		the curve bounces off the horizontal axis at both zeros, $3$ and $4$.
		
		The degree of $q$ is 6. Referring to \cref{poly:fig:bounceq},
		the curve bounces off the horizontal axis at $-2$ and $1$, and cuts 
		through the horizontal axis at $0$ and $3$.
		
		The degree of $r$ is 6. Referring to \cref{poly:fig:bouncer},
		the curve bounces off the horizontal axis at $-1$, and cuts through 
		the horizontal axis at $0$ and at $3$, although is flattened immediately to the left and right of $3$.
	\end{pccsolution}
\end{pccexample}

\setlength{\figurewidth}{0.25\textwidth}
\begin{figure}[!htb]
	\begin{subfigure}{\figurewidth}
		\begin{tikzpicture}
			\begin{axis}[
					xmin=-6,xmax=5,
					ymin=-30,ymax=200,
					xtick={-4,-2,...,4},
				]
				\addplot expression[domain=-5.63733:4.63733,samples=50]{(x-3)^2*(x+4)^2};
				\addplot[soldot]coordinates{(3,0)(-4,0)};
			\end{axis}
		\end{tikzpicture}
		\caption{$y=p(x)$}
		\label{poly:fig:bouncep}
	\end{subfigure}
	\hfill
	\begin{subfigure}{\figurewidth}
		\begin{tikzpicture}
			\begin{axis}[
					xmin=-3,xmax=4,
					xtick={-2,...,3},
					ymin=-60,ymax=40,
				]
				\addplot+[samples=50] expression[domain=-2.49011:3.11054]{x*(x+2)^2*(x-1)^2*(x-3)};
				\addplot[soldot]coordinates{(-2,0)(0,0)(1,0)(3,0)};
			\end{axis}
		\end{tikzpicture}
		\caption{$y=q(x)$}
		\label{poly:fig:bounceq}
	\end{subfigure}
	\hfill
	\begin{subfigure}{\figurewidth}
		\begin{tikzpicture}
			\begin{axis}[
					xmin=-2,xmax=4,
					xtick={-1,...,3},
					ymin=-40,ymax=40,
				]
				\addplot expression[domain=-1.53024:3.77464,samples=50]{x*(x-3)^3*(x+1)^2};
				\addplot[soldot]coordinates{(-1,0)(0,0)(3,0)};
			\end{axis}
		\end{tikzpicture}
		\caption{$y=r(x)$}
		\label{poly:fig:bouncer}
	\end{subfigure}
	\caption{}
	\label{poly:fig:moremultiple}
\end{figure}

\begin{pccdefinition}[Multiple zeros]\label{poly:def:multzero}
	Let $p$ be a polynomial that has a repeated linear factor $(x-a)^n$. Then we say
	that $p$ has a multiple zero at $a$ of multiplicity $n$ and 
	\begin{itemize}
		\item if the factor $(x-a)$ is repeated an even number of times, the graph of $y=p(x)$ does not
		cross the $x$ axis at $a$, but `bounces' off the horizontal axis at $a$.
		\item if the factor $(x-a)$ is repeated an odd number of times, the graph of $y=p(x)$ crosses the
		horizontal axis at $a$, but it looks `flattened' there
	\end{itemize}
	If $n=1$, then we say that $p$ has a \emph{simple} zero at $a$.
\end{pccdefinition}

%===================================
%   Author: Hughes
%   Date:   May 2011
%===================================
\begin{pccexample}[Find a formula]
	Find formulas for the polynomial functions, $p$ and $q$, graphed in \cref{poly:fig:findformulademoboth}.
	\begin{figure}[!htb]
		\begin{subfigure}{.45\textwidth}
			\begin{tikzpicture}
				\begin{axis}[framed,
						xmin=-5,xmax=5,
						ymin=-10,ymax=10,
						xtick={-4,-2,...,4},
						minor xtick={-3,-1,...,3},
						ytick={-8,-6,...,8},
					grid=both]
					\addplot expression[domain=-3.25842:2.25842,samples=50]{-x*(x-2)*(x+3)*(x+1)};
					\addplot[soldot]coordinates{(1,8)}node[axisnode,inner sep=.35cm,anchor=west]{$(1,8)$};
					\addplot[soldot]coordinates{(-3,0)(-1,0)(0,0)(2,0)};
				\end{axis}
			\end{tikzpicture}
			\caption{$p$}
			\label{poly:fig:findformulademo}
		\end{subfigure}
		\hfill
		\begin{subfigure}{.45\textwidth}
			\begin{tikzpicture}
				\begin{axis}[framed,
						xmin=-5,xmax=5,
						ymin=-10,ymax=10,
						xtick={-4,-2,...,4},
						minor xtick={-3,-1,...,3},
						ytick={-8,-6,...,8},
					grid=both]
					\addplot expression[domain=-4.33:4.08152]{-.25*(x+2)^2*(x-3)};
					\addplot[soldot]coordinates{(2,4)}node[axisnode,anchor=south west]{$(2,4)$};
					\addplot[soldot]coordinates{(-2,0)(3,0)};
				\end{axis}
			\end{tikzpicture}
			\caption{$q$}
			\label{poly:fig:findformulademo1}
		\end{subfigure}
		\caption{}
		\label{poly:fig:findformulademoboth}
	\end{figure}
	\begin{pccsolution}
		\begin{enumerate}
			\item We begin by noting that the horizontal intercepts of $p$ are $(-3,0)$, $(-1,0)$, $(0,0)$ and $(2,0)$. 
			We also note that each zero is simple (multiplicity $1$).
			If we assume that $p$ has no other zeros, then we can start by writing
			\begin{align*}
				p(x) & =(x+3)(x+1)(x-0)(x-2) \\ 
				     & =x(x+3)(x+1)(x-2)     \\     
			\end{align*}
			According to \cref{poly:fig:findformulademo}, the point $(1,8)$ lies 
			on the curve $y=p(x)$.
			Let's check if the formula we have written satisfies this requirement
			\begin{align*}
				p(1) & = (1)(4)(2)(-1) \\ 
				     & = -8            
			\end{align*}
			which is clearly not correct| it is close though. We can correct this by 
			multiplying $p$ by a constant $k$; so let's assume that
			\[
				p(x)=kx(x+3)(x+1)(x-2)
			\]
			Then $p(1)=-8k$, and if this is to equal $8$, then $k=-1$. Therefore
			the formula for $p(x)$ is
			\[
				p(x)=-x(x+3)(x+1)(x-2)
			\]
			\item The function $q$ has a zero at $-2$ of multiplicity $2$, and zero of 
			multiplicity $1$ at $3$ (so $3$ is a simple zero of $q$); we can therefore assume that $q$ has the form
			\[
				q(x)=k(x+2)^2(x-3)
			\]
			where $k$ is some real number. In order to find $k$, we use the given ordered pair, $(2,4)$, and
			evaluate $p(2)$
			\begin{align*}
				p(2) & =k(4)^2(-1) \\ 
				     & =-16k       
			\end{align*}
			We solve the equation $4=-8k$ and obtain $k=-\frac{1}{4}$ and conclude that the 
			formula for $q(x)$ is
			\[
				q(x)=-\frac{1}{4}(x+2)^2(x-3)
			\]
		\end{enumerate}
	\end{pccsolution}
\end{pccexample}


\fixthis{Chris: need sketching polynomial problems}
\subsection*{Sketching polynomial functions}
In the examples that we have considered so far, we have been provided 
with a formula for a polynomial function and its corresponding graph. Of course, we may not always
be fortunate enough to have access to the graph of the relevant function, 
and may need to construct it ourselves. In such a scenario, we can 
use \crefrange{poly:step:first}{poly:step:last} to guide us.

\begin{pccspecialcomment}[Steps to follow when sketching polynomial functions]
	\begin{steps}
		\item \label{poly:step:first} Determine the degree of the polynomial, 
		its leading term and leading coefficient, and hence determine
		the long-run behavior of the polynomial| does it behave like $\pm x^2$ or $\pm x^3$ 
		as $x\rightarrow\pm\infty$?
		\item Determine the zeros and their multiplicity. Mark all zeros 
		and the vertical intercept on the graph using solid circles $\bullet$.
		\item \label{poly:step:last}  Deduce the overall shape of the curve, and sketch it. If there isn't
		enough information from the previous steps, then construct a table of values.
	\end{steps}
	Remember that until we have the tools of calculus, we won't be able to 
	find the exact coordinates of local minimums, local maximums, and points
	of inflection.
\end{pccspecialcomment}
Before we demonstrate some examples, it is important to remember the following:
\begin{itemize}
	\item our sketches will give a good representation of the overall 
	shape of the graph, but until we have the tools of calculus (from MTH 251)
	we can not find local minimums, local maximums, and inflection points algebraically. This
	means that we will make our best guess as to where these points are.
	\item we will not concern ourselves too much with the vertical scale (because of 
	our previous point)| we will, however, mark the vertical intercept (assuming there is one), 
	and any horizontal asymptotes.
\end{itemize}
%===================================
%   Author: Hughes
%   Date:   May 2012
%===================================
\begin{pccexample}\label{poly:ex:simplecubic}
	Use \crefrange{poly:step:first}{poly:step:last} to sketch a graph of the function $p$ 
	that has formula
	\[
		p(x)=\frac{1}{2}(x-4)(x-1)(x+3)
	\]
	\begin{pccsolution}
		\begin{steps}
			\item $p$ has degree $3$. The leading term of $p$ is $\frac{1}{2}x^3$, so the leading coefficient of $p$ 
			is $\frac{1}{2}$. The long-run behavior of $p$ is therefore similar to that of $x^3$.
			\item The zeros of $p$ are $-3$, $1$, and $4$; each zero is simple (i.e, it has multiplicity $1$).
			This means that the curve of $p$ cuts the horizontal axis at each zero. The vertical 
			intercept of $p$ is $(0,6)$.
			\item We draw the details we have obtained so far on \cref{poly:fig:simplecubicp1}. Given 
			that the curve of $p$ looks like the curve of $x^3$ in the long-run, we are able to complete a sketch of the 
			graph of $p$ in \cref{poly:fig:simplecubicp2}.
				
			Note that we can not find the coordinates of the local minimums, local maximums, and inflection
			points| for the moment we make reasonable guesses as to where these points are (you'll find how
			to do this in calculus).
		\end{steps}
		
		\begin{figure}[!htbp]
			\begin{subfigure}{.45\textwidth}
				\begin{tikzpicture}
					\begin{axis}[
							xmin=-10,xmax=10,
							ymin=-10,ymax=15,
							xtick={-8,-6,...,8},
							ytick={-5,5},
						]
						\addplot[soldot] coordinates{(-3,0)(1,0)(4,0)(0,6)}node[axisnode,anchor=south west]{$(0,6)$};
					\end{axis}
				\end{tikzpicture}
				\caption{}
				\label{poly:fig:simplecubicp1}
			\end{subfigure}%
			\hfill
			\begin{subfigure}{.45\textwidth}
				\begin{tikzpicture}
					\begin{axis}[
							xmin=-10,xmax=10,
							ymin=-10,ymax=15,
							xtick={-8,-6,...,8},
							ytick={-5,5},
						]
						\addplot[soldot] coordinates{(-3,0)(1,0)(4,0)(0,6)}node[axisnode,anchor=south west]{$(0,6)$};
						\addplot[pccplot] expression[domain=-3.57675:4.95392,samples=100]{.5*(x-4)*(x-1)*(x+3)};
					\end{axis}
				\end{tikzpicture}
				\caption{}
				\label{poly:fig:simplecubicp2}
			\end{subfigure}%
			\caption{$y=\dfrac{1}{2}(x-4)(x-1)(x+3)$}
			\label{poly:fig:simplecubic}
		\end{figure}
	\end{pccsolution}
\end{pccexample}

%===================================
%   Author: Hughes
%   Date:   May 2012
%===================================
\begin{pccexample}\label{poly:ex:degree5}
	Use \crefrange{poly:step:first}{poly:step:last} to sketch a graph of the function $q$ 
	that has formula
	\[
		q(x)=\frac{1}{200}(x+7)^2(2-x)(x-6)^2
	\]
	\begin{pccsolution}
		\begin{steps}
			\item $q$ has degree $4$. The leading term of $q$ is 
			\[
				-\frac{1}{200}x^5
			\]
			so the leading coefficient of $q$ is $-\frac{1}{200}$. The long-run behavior of $q$
			is therefore similar to that of $-x^5$.
			\item The zeros of $q$ are $-7$ (multiplicity 2), $2$ (simple), and $6$ (multiplicity $2$).
			The curve of $q$ bounces off the horizontal axis at the zeros with multiplicity $2$ and
			cuts the horizontal axis at the simple zeros. The vertical intercept of $q$ is $\left( 0,\frac{441}{25} \right)$.
			\item We mark the details we have found so far on \cref{poly:fig:degree5p1}. Given that 
			the curve of $q$ looks like the curve of $-x^5$ in the long-run, we can complete \cref{poly:fig:degree5p2}.
		\end{steps}
		
		\begin{figure}[!htbp]
			\begin{subfigure}{.45\textwidth}
				\begin{tikzpicture}
					\begin{axis}[
							xmin=-10,xmax=10,
							ymin=-10,ymax=40,
							xtick={-8,-6,...,8},
							ytick={-5,0,...,35},
						]
						\addplot[soldot] coordinates{(-7,0)(2,0)(6,0)(0,441/25)}node[axisnode,anchor=south west]{$\left( 0, \frac{441}{25} \right)$};
					\end{axis}
				\end{tikzpicture}
				\caption{}
				\label{poly:fig:degree5p1}
			\end{subfigure}%
			\hfill
			\begin{subfigure}{.45\textwidth}
				\begin{tikzpicture}
					\begin{axis}[
							xmin=-10,xmax=10,
							ymin=-10,ymax=40,
							xtick={-8,-6,...,8},
							ytick={-5,0,...,35},
						]
						\addplot[soldot] coordinates{(-7,0)(2,0)(6,0)(0,441/25)}node[axisnode,anchor=south west]{$\left( 0, \frac{441}{25} \right)$};
						\addplot[pccplot] expression[domain=-8.83223:7.34784,samples=50]{1/200*(x+7)^2*(2-x)*(x-6)^2};
					\end{axis}
				\end{tikzpicture}
				\caption{}
				\label{poly:fig:degree5p2}
			\end{subfigure}%
			\caption{$y=\dfrac{1}{200}(x+7)^2(2-x)(x-6)^2$}
			\label{poly:fig:degree5}
		\end{figure}
	\end{pccsolution}
\end{pccexample}

%===================================
%   Author: Hughes
%   Date:   May 2012
%===================================
\begin{pccexample}
	Use \crefrange{poly:step:first}{poly:step:last} to sketch a graph of the function $r$ 
	that has formula
	\[
		r(x)=\frac{1}{100}x^3(x+4)(x-4)(x-6)
	\]
	\begin{pccsolution}
		\begin{steps}
			\item $r$ has degree $6$. The leading term of $r$ is 
			\[
				\frac{1}{100}x^6
			\]
			so the leading coefficient of $r$ is $\frac{1}{100}$. The long-run behavior of $r$
			is therefore similar to that of $x^6$.
			\item The zeros of $r$ are $-4$ (simple), $0$ (multiplicity $3$), $4$ (simple), 
			and $6$ (simple). The vertical intercept of $r$ is $(0,0)$. The curve of $r$
			cuts the horizontal axis at the simple zeros, and goes through the axis
			at $(0,0)$, but does so in a flattened way.
			\item We mark the zeros and vertical intercept on \cref{poly:fig:degree6p1}. Given that
			the curve of $r$ looks like the curve of $x^6$ in the long-run, we complete the graph
			of $r$ in \cref{poly:fig:degree6p2}.
		\end{steps}
		
		\begin{figure}[!htbp]
			\begin{subfigure}{.45\textwidth}
				\begin{tikzpicture}
					\begin{axis}[
							xmin=-5,xmax=10,
							ymin=-20,ymax=10,
							xtick={-4,-2,...,8},
							ytick={-15,-10,...,5},
						]
						\addplot[soldot] coordinates{(-4,0)(0,0)(4,0)(6,0)};
					\end{axis}
				\end{tikzpicture}
				\caption{}
				\label{poly:fig:degree6p1}
			\end{subfigure}%
			\hfill
			\begin{subfigure}{.45\textwidth}
				\begin{tikzpicture}
					\begin{axis}[
							xmin=-5,xmax=10,
							ymin=-20,ymax=10,
							xtick={-4,-2,...,8},
							ytick={-15,-10,...,5},
						]
						\addplot[soldot] coordinates{(-4,0)(0,0)(4,0)(6,0)};
						\addplot[pccplot] expression[domain=-4.16652:6.18911,samples=100]{1/100*(x+4)*x^3*(x-4)*(x-6)};
					\end{axis}
				\end{tikzpicture}
				\caption{}
				\label{poly:fig:degree6p2}
			\end{subfigure}%
			\caption{$y=\dfrac{1}{100}(x+4)x^3(x-4)(x-6)$}
		\end{figure}
	\end{pccsolution}
\end{pccexample}

%===================================
%   Author: Hughes
%   Date:   March 2012
%===================================
\begin{pccexample}[An open-topped box] 
	A cardboard company makes open-topped boxes for their clients. The specifications
	dictate that the box must have a square base, and that it must be open-topped. 
	The company uses sheets of cardboard that are $\SI{1200}{\centi\meter\squared}$. Assuming that 
	the base of each box has side $x$ (measured in cm), it can be shown that the volume of each box, $V(x)$,
	has formula
	\[
		V(x)=\frac{x}{4}(1200-x^2)
	\]
	Find the dimensions of the box that maximize the volume.
	\begin{pccsolution}
		We graph $y=V(x)$ in \cref{poly:fig:opentoppedbox}. Note that because 
		$x$ represents the length of a side, and $V(x)$ represents the volume
		of the box, we necessarily require both values to be positive; we illustrate
		the part of the curve that applies to this problem using a solid line. 
		
		\begin{figure}[!htb]
			\centering
			\begin{tikzpicture}
				\begin{axis}[framed,
						xmin=-50,xmax=50,
						ymin=-5000,ymax=5000,
						xtick={-40,-30,...,40},
						minor xtick={-45,-35,...,45},
						minor ytick={-3000,-1000,1000,3000},
						width=.75\textwidth,
						height=.5\textwidth,
					grid=both]
					\addplot[pccplot,dashed,<-] expression[domain=-40:0,samples=50]{x/4*(1200-x^2)};
					\addplot[pccplot,-] expression[domain=0:34.64,samples=50]{x/4*(1200-x^2)};
					\addplot[pccplot,dashed,->] expression[domain=34.64:40,samples=50]{x/4*(1200-x^2)};
					\addplot[soldot] coordinates{(20,4000)};
				\end{axis}
			\end{tikzpicture}
			\caption{$y=V(x)$}
			\label{poly:fig:opentoppedbox}
		\end{figure}
		
		According to \cref{poly:fig:opentoppedbox}, the maximum volume of such a box is 
		approximately $\SI{4000}{\centi\meter\squared}$, and we achieve it using a base of length 
		approximately $\SI{20}{\centi\meter}$. Since the base is square and each sheet of cardboard
		is $\SI{1200}{\centi\meter\squared}$, we conclude that the dimensions of each box are $\SI{20}{\centi\meter}\times\SI{20}{\centi\meter}\times\SI{30}{\centi\meter}$.
	\end{pccsolution}
\end{pccexample}

\subsection*{Complex zeros}
There has been a pattern to all of the examples that we have seen so far|
the degree of the polynomial has dictated the number of \emph{real} zeros that the 
polynomial has. For example, the function $p$ in \cref{poly:ex:simplecubic}
has degree $3$, and $p$ has three real zeros; the function $q$ in \cref{poly:ex:degree5}
has degree $5$ and $q$ has five real zeros.

You may wonder if this result can be generalized| does every polynomial that
has degree $n$ have $n$ real zeros? Before we tackle the general result, 
let's consider an example that may help motivate it.
%===================================
%   Author: Hughes
%   Date:   June 2012
%===================================
\begin{pccexample}\label{poly:ex:complx}
	Consider the polynomial function $c$ that has formula
	\[
		c(x)=x(x^2+1)
	\]
	It is clear that $c$ has degree $3$, and that $c$ has a (simple) zero at $0$. Does
	$c$ have any other zeros, i.e, can we find any values of $x$ that satisfy the equation
	\begin{equation}\label{poly:eq:complx}
		x^2+1=0
	\end{equation}
	The solutions to \cref{poly:eq:complx} are $\pm i$.
	
	We conclude that $c$ has three zeros: $0$ and $\pm i$; we note that \emph{not 
	all of them are real}.
\end{pccexample}
\Cref{poly:ex:complx} shows that not every degree-$3$ polynomial has $3$
\emph{real} zeros; however, if we are prepared to venture into the complex numbers, 
then we can state the following theorem.
%===================================
%   Author: Hughes
%   Date:   June 2012
%===================================
\begin{pccspecialcomment}[The fundamental theorem of algebra]
	Every polynomial function of degree $n$ has $n$ roots, some of which may 
	be complex, and some may be repeated.
\end{pccspecialcomment}
\fixthis{Fundamental theorem of algebra: is this wording ok? do we want
it as a theorem?}
%===================================
%   Author: Hughes
%   Date:   June 2012
%===================================
\begin{pccexample}
	Find all the zeros of the polynomial function $p$ that has formula to solve it
	\[
		p(x)=x^4-2x^3+5x^2
	\]
	\begin{pccsolution}
		We begin by factoring $p$
		\begin{align*}
			p(x) & =x^4-2x^3+5x^2 \\ 
			     & =x^2(x^2-2x+5) 
		\end{align*}
		We note that $0$ is a zero of $p$ with multiplicity $2$. The other zeros of $p$ 
		can be found by solving the equation
		\[
			x^2-2x+5=0
		\]
		This equation can not be factored, so we use the quadratic formula
		\begin{align*}
			x & =\frac{2\pm\sqrt{(-2)^2}-20}{2(1)} \\ 
			  & =\frac{2\pm\sqrt{-16}}{2}          \\          
			  & =1\pm 2i                           
		\end{align*}
		We conclude that $p$ has four zeros: $0$ (multiplicity $2$), and $1\pm 2i$ (simple).
	\end{pccsolution}
\end{pccexample}
%===================================
%   Author: Hughes
%   Date:   June 2012
%===================================
\begin{pccexample}
	Find a polynomial that has zeros at $2\pm i\sqrt{2}$. 
	\begin{pccsolution}
		We know that the zeros of a polynomial can be found by analyzing the linear
		factors. We are given the zeros, and have to work backwards to find the 
		linear factors. 
		
		We begin by assuming that $p$ has the form
		\begin{align*}
			p(x) & =(x-(2-i\sqrt{2}))(x-(2+i\sqrt{2}))                           \\                           
			     & =x^2-x(2+i\sqrt{2})-x(2-i\sqrt{2})+(2-i\sqrt{2})(2+i\sqrt{2}) \\ 
			     & =x^2-4x+(4-2i^2)                                              \\                                              
			     & =x^2-4x+6                                                     
		\end{align*}
		We conclude that a possible formula for a polynomial function, $p$, 
		that has zeros at $2\pm i\sqrt{2}$ is
		\[
			p(x)=x^2-4x+6
		\]
		Note that we could multiply $p$ by any real number and still ensure
		that $p$ has the same zeros.
	\end{pccsolution}
\end{pccexample}
\investigation*{}
%===================================
%   Author: Hughes
%   Date:   May 2011
%===================================
\begin{problem}[Find a formula from a graph]
For each of the polynomial functions that are represented in \cref{poly:fig:findformula}
\begin{enumerate}
	\item count the number of times the curve turns round, and cuts/bounces off the $x$ axis;
	\item approximate the degree of the polynomial;
	\item use your information to find the linear factors of each polynomial, and therefore write a possible formula for each;
	\item make sure your polynomial goes through the given ordered pair.
\end{enumerate}
\begin{shortsolution}
	\Vref{poly:fig:findformdeg2}: 
	\begin{enumerate}
		\item the curve turns round once;
		\item the degree could be 2;
		\item based on the zeros, the linear factors are $(x+5)$ and $(x-3)$; since the 
		graph opens downwards, we will assume the leading coefficient is negative: $p(x)=-k(x+5)(x-3)$;
		\item $p$ goes through $(2,2)$, so we need to solve $2=-k(7)(-1)$ and therefore $k=\nicefrac{2}{7}$, so
		\[
			p(x)=-\frac{2}{7}(x+5)(x-3)
		\]
	\end{enumerate}
	\Vref{poly:fig:findformdeg3}:
	\begin{enumerate}
		\item the curve turns around twice;
		\item the degree could be 3;
		\item based on the zeros, the linear factors are $(x+2)^2$, and $(x-1)$;
		based on the behavior of $p$, we assume that the leading coefficient is positive, and try $p(x)=k(x+2)^2(x-1)$;
		\item $p$ goes through $(0,-2)$, so we need to solve $-2=k(4)(-1)$ and therefore $k=\nicefrac{1}{2}$, so
		\[
			p(x)=\frac{1}{2}(x+2)^2(x-1)
		\]
	\end{enumerate}
	\Vref{poly:fig:findformdeg5}:
	\begin{enumerate}
		\item the curve turns around 4 times;
		\item the degree could be 5;
		\item based on the zeros, the linear factors are $(x+5)^2$, $(x+1)$, $(x-2)$, $(x-3)$;
		based on the behavior of $p$, we assume that the leading coefficient is positive, and try $p(x)=k(x+5)^2(x+1)(x-2)(x-3)$;
		\item $p$ goes through $(-3,-50)$, so we need to solve $-50=k(64)(-2)(-5)(-6)$ and therefore $k=\nicefrac{5}{384}$, so
		\[
			p(x)=\frac{5}{384}(x+5)^2(x+1)(x-2)(x-3)
		\]
	\end{enumerate}
\end{shortsolution}
\end{problem}


\begin{figure}[!htb]
	\setlength{\figurewidth}{0.3\textwidth}
	\begin{subfigure}{\figurewidth}
		\begin{tikzpicture}
			\begin{axis}[
					xmin=-5,xmax=5,
					ymin=-2,ymax=5,
				]
				\addplot expression[domain=-4.5:3.75]{-1/3*(x+4)*(x-3)};
				\addplot[soldot] coordinates{(-4,0)(3,0)(2,2)} node[axisnode,above right]{$(2,2)$};
			\end{axis}
		\end{tikzpicture}
		\caption{}
		\label{poly:fig:findformdeg2}
	\end{subfigure}
	\hfill
	\begin{subfigure}{\figurewidth}
		\begin{tikzpicture}
			\begin{axis}[
					xmin=-3,xmax=2,
					ymin=-2,ymax=4,
					xtick={-2,...,1},
				]
				\addplot expression[domain=-2.95:1.75]{1/3*(x+2)^2*(x-1)};
				\addplot[soldot]coordinates{(-2,0)(1,0)(0,-1.33)}node[axisnode,anchor=north west]{$(0,-2)$};
			\end{axis}
		\end{tikzpicture}
		\caption{}
		\label{poly:fig:findformdeg3}
	\end{subfigure}
	\hfill
	\begin{subfigure}{\figurewidth}
		\begin{tikzpicture}
			\begin{axis}[
					xmin=-5,xmax=5,
					ymin=-100,ymax=150,
				]
				\addplot expression[domain=-4.5:3.4,samples=50]{(x+4)^2*(x+1)*(x-2)*(x-3)};
				\addplot[soldot]coordinates{(-4,0)(-1,0)(2,0)(3,0)(-3,-60)}node[axisnode,anchor=north]{$(-3,-50)$};
			\end{axis}
		\end{tikzpicture}
		\caption{}
		\label{poly:fig:findformdeg5}
	\end{subfigure}
	\caption{}
	\label{poly:fig:findformula}
\end{figure}




\begin{exercises}
%===================================
%   Author: Hughes
%   Date:   March 2012
%===================================
\begin{problem}[Prerequisite classifacation skills]
Decide if the following formulas correspond to linear or 
quadratic functions.
\begin{multicols}{3}
	\begin{subproblem}
		$f(x)=2x+3$ 
		\begin{shortsolution}
			$f$ is linear. 
		\end{shortsolution}
	\end{subproblem}
	\begin{subproblem}
		$g(x)=10-7x$ 
		\begin{shortsolution}
			$g$ is linear 
		\end{shortsolution}
	\end{subproblem}
	\begin{subproblem}
		$h(x)=-x^2+3x-9$ 
		\begin{shortsolution}
			$h$ is quadratic. 
		\end{shortsolution}
	\end{subproblem}
	\begin{subproblem}
		$k(x)=-17$ 
		\begin{shortsolution}
			$k$ is linear.
		\end{shortsolution}
	\end{subproblem}
	\begin{subproblem}
		$l(x)=-82x^2-4$
		\begin{shortsolution}
			$l$ is quadratic 
		\end{shortsolution}
	\end{subproblem}
	\begin{subproblem}
		$m(x)=6^2x-8$ 
		\begin{shortsolution}
			$m$ is linear. 
		\end{shortsolution}
	\end{subproblem}
\end{multicols}
\end{problem}
%===================================
%   Author: Hughes
%   Date:   March 2012
%===================================
\begin{problem}[Prerequisite slope identification]
The following formulas correspond to the linear functions $\alpha$, 
$\beta$, $\gamma$, and $\delta$. 
State the slope of each function, and hence decide if each function is increasing or decreasing.
\begin{multicols}{4}
	\begin{subproblem}
		$\alpha(x)=4x+1$ 
		\begin{shortsolution}
			$m=4$; $\alpha$ is increasing. 
		\end{shortsolution}
	\end{subproblem}
	\begin{subproblem}
		$\beta(x)=-9x$ 
		\begin{shortsolution}
			$m=-9$; $\beta$ is decreasing. 
		\end{shortsolution}
	\end{subproblem}
	\begin{subproblem}
		$\gamma(t)=18t+100$ 
		\begin{shortsolution}
			$m=18$; $\gamma$ is increasing.
		\end{shortsolution}
	\end{subproblem}
	\begin{subproblem}
		$\delta(y)=23-y$ 
		\begin{shortsolution}
			$m=-1$; $\delta$ is decreasing. 
		\end{shortsolution}
	\end{subproblem}
\end{multicols}
Now let's generalize our findings for the most general linear function $f$
that has formula $f(x)=mx+b$. Complete the following sentences.
\begin{subproblem}
	When $m>0$, the function $f$ is $\ldots$  
	\begin{shortsolution}
		When $m>0$, the function $f$ is $\ldots$  \emph{increasing}.
	\end{shortsolution}
\end{subproblem}
\begin{subproblem}
	When $m<0$, the function $f$ is $\ldots$  
	\begin{shortsolution}
		When $m<0$, the function $f$ is $\ldots$  \emph{decreasing}.
	\end{shortsolution}
\end{subproblem}
\end{problem}
%===================================
%   Author: Hughes
%   Date:   May 2011
%===================================
\begin{problem}[Polynomial or not?]
Decide if the following formulas correspond to polynomial functions 
or not; if so, state the degree of the polynomial.
\begin{multicols}{3}
	\begin{subproblem}
		$p(x)=2x+1$
		\begin{shortsolution}
			$p$ is a polynomial (you might also describe $p$ as linear). The degree of $p$ is 1.
		\end{shortsolution}
	\end{subproblem}
	\begin{subproblem}
		$p(x)=7x^2+4x$
		\begin{shortsolution}
			$p$ is a polynomial (you might also describe $p$ as quadratic). The degree of $p$ is 2.
		\end{shortsolution}
	\end{subproblem}
	\begin{subproblem}
		$p(x)=\sqrt{x}+2x+1$
		\begin{shortsolution}
			$p$ is not a polynomial; we require the powers of $x$ to be integer values.
		\end{shortsolution}
	\end{subproblem}
	\begin{subproblem}
		$p(x)=2^x-45$
		\begin{shortsolution}
			$p$ is not a polynomial; the $2^x$ term is exponential.
		\end{shortsolution}
	\end{subproblem}
	\begin{subproblem}
		$p(x)=6x^4-5x^3+9$
		\begin{shortsolution}
			$p$ is a polynomial, and the degree of $p$ is $6$.
		\end{shortsolution}
	\end{subproblem}
	\begin{subproblem}
		$p(x)=-5x^{17}+9x+2$
		\begin{shortsolution}
			$p$ is a polynomial, and the degree of $p$ is 17.
		\end{shortsolution}
	\end{subproblem}
	\begin{subproblem}
		$p(x)=4x(x+7)^2(x-3)^3$ 
		\begin{shortsolution}
			$p$ is a polynomial, and the degree of $p$ is $6$.     
		\end{shortsolution}
	\end{subproblem}
	\begin{subproblem}
		$p(x)=4x^{-5}-x^2+x$ 
		\begin{shortsolution}
			$p$ is not a polynomial because $-5$ is not a positive integer. 
		\end{shortsolution}
	\end{subproblem}
	\begin{subproblem}
		$p(x)=-x^6(x^2+1)(x^3-2)$ 
		\begin{shortsolution}
			$p$ is a polynomial, and the degree of $p$ is $11$.
		\end{shortsolution}
	\end{subproblem}
\end{multicols}
\end{problem}
%===================================
%   Author: Hughes
%   Date:   May 2011
%===================================
\begin{problem}[Polynomial graphs]
Three polynomial functions $p$, $m$, and $n$ are shown in \crefrange{poly:fig:functionp}{poly:fig:functionn}.
The functions have the following formulas
\begin{align*}
	p(x) & = (x-1)(x+2)(x-3)           \\         
	m(x) & = -(x-1)(x+2)(x-3)          \\        
	n(x) & = (x-1)(x+2)(x-3)(x+1)(x+4) 
\end{align*}
Note that for our present purposes we are not concerned with the vertical scale of the graphs.
\begin{subproblem}
	Identify both on the graph {\em and} algebraically, the zeros of each polynomial.
	\begin{shortsolution}
		$y=p(x)$ is shown below.
				
		\begin{tikzpicture}
			\begin{axis}[
					xmin=-5,xmax=5,
					ymin=-10,ymax=10,
				]
				\addplot expression[domain=-2.5:3.5,samples=50]{(x-1)*(x+2)*(x-3)};
				\addplot[soldot] coordinates{(-2,0)(1,0)(3,0)};
			\end{axis}
		\end{tikzpicture}
				
		$y=m(x)$ is shown below.
				
		\begin{tikzpicture}
			\begin{axis}[
					xmin=-5,xmax=5,
					ymin=-10,ymax=10,
				]
				\addplot expression[domain=-2.5:3.5,samples=50]{-1*(x-1)*(x+2)*(x-3)};
				\addplot[soldot] coordinates{(-2,0)(1,0)(3,0)};
			\end{axis}
		\end{tikzpicture}
				
		$y=n(x)$ is shown below.
				
		\begin{tikzpicture}
			\begin{axis}[
					xmin=-5,xmax=5,
					ymin=-90,ymax=70,
				]
				\addplot expression[domain=-4.15:3.15,samples=50]{(x-1)*(x+2)*(x-3)*(x+1)*(x+4)};
				\addplot[soldot] coordinates{(-4,0)(-2,0)(-1,0)(1,0)(3,0)};
			\end{axis}
		\end{tikzpicture}
				
		The zeros of $p$ are $-2$, $1$, and $3$; the zeros of $m$ are $-2$, $1$, and $3$; the zeros of $n$ are 
		$-4$, $-2$, $-1$, and $3$.
	\end{shortsolution}
\end{subproblem}
\begin{subproblem}
	Write down the degree, how many times the curve of each function `turns around', 
	and how many zeros it has
	\begin{shortsolution}
		\begin{itemize}
			\item The degree of $p$ is 3, and the curve $y=p(x)$ turns around twice. 
			\item The degree of $q$ is also 3, and the curve $y=q(x)$ turns around twice. 
			\item The degree of $n$ is $5$, and the curve $y=n(x)$ turns around 4 times.
		\end{itemize}
	\end{shortsolution}
\end{subproblem}
\end{problem}

\begin{figure}[!htb]
	\begin{widepage}
	\setlength{\figurewidth}{0.3\textwidth}
	\begin{subfigure}{\figurewidth}
		\begin{tikzpicture}
			\begin{axis}[
					xmin=-5,xmax=5,
					ymin=-10,ymax=10,
					ytick={-5,5},
				]
				\addplot expression[domain=-2.5:3.5,samples=50]{(x-1)*(x+2)*(x-3)};
				\addplot[soldot]coordinates{(-2,0)(1,0)(3,0)};
			\end{axis}
		\end{tikzpicture}
		\caption{$y=p(x)$}
		\label{poly:fig:functionp}
	\end{subfigure}
	\hfill
	\begin{subfigure}{\figurewidth}
		\begin{tikzpicture}
			\begin{axis}[
					xmin=-5,xmax=5,
					ymin=-10,ymax=10,
					ytick={-5,5},
				]
				\addplot expression[domain=-2.5:3.5,samples=50]{-1*(x-1)*(x+2)*(x-3)};
				\addplot[soldot]coordinates{(-2,0)(1,0)(3,0)};
			\end{axis}
		\end{tikzpicture}
		\caption{$y=m(x)$}
		\label{poly:fig:functionm}
	\end{subfigure}
	\hfill
	\begin{subfigure}{\figurewidth}
		\begin{tikzpicture}
			\begin{axis}[
					xmin=-5,xmax=5,
					ymin=-90,ymax=70,
				]
				\addplot expression[domain=-4.15:3.15,samples=100]{(x-1)*(x+2)*(x-3)*(x+1)*(x+4)};
				\addplot[soldot]coordinates{(-4,0)(-2,0)(-1,0)(1,0)(3,0)};
			\end{axis}
		\end{tikzpicture}
		\caption{$y=n(x)$}
		\label{poly:fig:functionn}
	\end{subfigure}
	\caption{}
	\end{widepage}
\end{figure}
%===================================
%   Author: Hughes
%   Date:   May 2011
%===================================
\begin{problem}[Horizontal intercepts]\label{poly:prob:matchpolys}%
The following formulas correspond to the polynomial functions $p$, $q$, 
$r$, and $s$.  State the horizontal intercepts (as ordered pairs) of 
each function.
\begin{multicols}{2}
	\begin{subproblem}\label{poly:prob:degree5}
		$p(x)=(x-1)(x+2)(x-3)(x+1)(x+4)$
		\begin{shortsolution}
			$(-4,0)$, $(-2,0)$, $(-1,0)$, $(1,0)$, $(3,0)$
		\end{shortsolution}
	\end{subproblem}
	\begin{subproblem}
		$q(x)=-(x-1)(x+2)(x-3)$
		\begin{shortsolution}
			$(-2,0)$, $(1,0)$, $(3,0)$
		\end{shortsolution}
	\end{subproblem}
	\begin{subproblem}
		$r(x)=(x-1)(x+2)(x-3)$
		\begin{shortsolution}
			$(-2,0)$, $(1,0)$, $(3,0)$
		\end{shortsolution}
	\end{subproblem}
	\begin{subproblem}\label{poly:prob:degree2}
		$s(x)=(x-2)(x+2)$
		\begin{shortsolution}
			$(-2,0)$, $(2,0)$
		\end{shortsolution}
	\end{subproblem}
\end{multicols}
\end{problem}
%===================================
%   Author: Hughes
%   Date:   March 2012
%===================================
\begin{problem}[Minimums, maximums, and concavity]\label{poly:prob:incdec}
Four polynomial functions are graphed in \cref{poly:fig:incdec}. The formulas
for these functions are (not respectively)
\begin{gather*}
	p(x)=\frac{x^3}{6}-\frac{x^2}{4}-3x, \qquad q(x)=\frac{x^4}{20}+\frac{x^3}{15}-\frac{6}{5}x^2+1\\
	r(x)=-\frac{x^5}{50}-\frac{x^4}{40}+\frac{2x^3}{5}+6, \qquad s(x)=-\frac{x^6}{6000}-\frac{x^5}{2500}+\frac{67x^4}{4000}+\frac{17x^3}{750}-\frac{42x^2}{125}
\end{gather*}
\begin{figure}[!htb]
	\begin{widepage}
	\setlength{\figurewidth}{.23\textwidth}
	\centering
	\begin{subfigure}{\figurewidth}
		\begin{tikzpicture}
			\begin{axis}[
					framed,
					xmin=-10,xmax=10,
					ymin=-10,ymax=10,
					xtick={-8,-6,...,8},
					ytick={-8,-6,...,8},
					grid=major,
				]
				\addplot expression[domain=-5.28:4.68,samples=50]{-x^5/50-x^4/40+2*x^3/5+6};
			\end{axis}
		\end{tikzpicture}
		\caption{}
		\label{poly:fig:incdec3}
	\end{subfigure}
	\hfill
	\begin{subfigure}{\figurewidth}
		\begin{tikzpicture}
			\begin{axis}[
					framed,
					xmin=-10,xmax=10,ymin=-10,ymax=10,
					xtick={-8,-6,...,8},
					ytick={-8,-6,...,8},
					grid=major,
				]
				\addplot expression[domain=-6.08:4.967,samples=50]{x^4/20+x^3/15-6/5*x^2+1};
			\end{axis}
		\end{tikzpicture}
		\caption{}
		\label{poly:fig:incdec2}
	\end{subfigure}
	\hfill
	\begin{subfigure}{\figurewidth}
		\begin{tikzpicture}
			\begin{axis}[
					framed,
					xmin=-6,xmax=8,ymin=-10,ymax=10,
					xtick={-4,-2,...,6},
					ytick={-8,-4,4,8},
					minor ytick={-6,-2,...,6},
					grid=both,
				]
				\addplot expression[domain=-4.818:6.081,samples=50]{x^3/6-x^2/4-3*x};
			\end{axis}
		\end{tikzpicture}
		\caption{}
		\label{poly:fig:incdec1}
	\end{subfigure}
	\hfill
	\begin{subfigure}{\figurewidth}
		\begin{tikzpicture}
			\begin{axis}[
					framed,
					xmin=-10,xmax=10,ymin=-10,ymax=10,
					xtick={-8,-4,4,8},
					ytick={-8,-4,4,8},
					minor xtick={-6,-2,...,6},
					minor ytick={-6,-2,...,6},
					grid=both,
				]
				\addplot expression[domain=-9.77:8.866,samples=50]{-x^6/6000-x^5/2500+67*x^4/4000+17/750*x^3-42/125*x^2};
			\end{axis}
		\end{tikzpicture}
		\caption{}
		\label{poly:fig:incdec4}
	\end{subfigure}
	\caption{Graphs for \cref{poly:prob:incdec}.}
	\label{poly:fig:incdec}
	\end{widepage}
\end{figure}
\begin{subproblem}
	Match each of the formulas with one of the given graphs.
	\begin{shortsolution}
		\begin{itemize}
			\item $p$ is graphed in \vref{poly:fig:incdec1};
			\item $q$ is graphed in \vref{poly:fig:incdec2};
			\item $r$ is graphed in \vref{poly:fig:incdec3};
			\item $s$ is graphed in \vref{poly:fig:incdec4}.
		\end{itemize}
	\end{shortsolution}
\end{subproblem}
\begin{subproblem}
	Approximate the zeros of each function using the appropriate graph.
	\begin{shortsolution}
		\begin{itemize}
			\item $p$ has simple zeros at about $-3.8$, $0$, and $5$.
			\item $q$ has simple zeros at about $-5.9$, $-1$, $1$, and $4$.
			\item $r$ has simple zeros at about $-5$, $-2.9$, and $4.1$.
			\item $s$ has simple zeros at about $-9$, $-6$, $4.2$, $8.1$, and a zero of multiplicity $2$ at $0$.
		\end{itemize}
	\end{shortsolution}
\end{subproblem}
\begin{subproblem}
	Approximate the local maximums and minimums of each of the functions.
	\begin{shortsolution}
		\begin{itemize}
			\item $p$ has a local maximum of approximately $3.9$ at $-2$, and a local minimum of approximately $-6.5$ at $3$.
			\item $q$ has a local minimum of approximately $-10$ at $-4$, and $-4$ at $3$; $q$ has a local maximum of approximately $1$ at $0$.
			\item $r$ has a local minimum of approximately $-5.5$ at $-4$, and a local maximum of approximately $10$ at $3$.
			\item $s$ has a local maximum of approximately $5$ at $-8$, $0$ at $0$, and $5$ at  $7$; $s$ has local minimums 
			of approximately $-3$ at $-4$, and $-1$ at $3$.
		\end{itemize}
	\end{shortsolution}
\end{subproblem}
\begin{subproblem}
	Approximate the global maximums and minimums of each of the functions.
	\begin{shortsolution}
		\begin{itemize}
			\item $p$ does not have a global maximum, nor a global minimum.
			\item $q$ has a global minimum of approximately $-10$; it does not have a global maximum.
			\item $r$ does not have a global maximum, nor a global minimum.
			\item $s$ has a global maximum of approximately $5$; it does not have a global minimum.
		\end{itemize}
	\end{shortsolution}
\end{subproblem}
\begin{subproblem}
	Approximate the intervals on which each function is increasing and decreasing.
	\begin{shortsolution}
		\begin{itemize}
			\item $p$ is increasing on $(-\infty,-2)\cup (3,\infty)$, and decreasing on $(-2,3)$.
			\item $q$ is increasing on $(-4,0)\cup (3,\infty)$, and decreasing on $(-\infty,-4)\cup (0,3)$.
			\item $r$ is increasing on $(-4,3)$, and decreasing on $(-\infty,-4)\cup (3,\infty)$.
			\item $s$ is increasing on $(-\infty,-8)\cup (-4,0)\cup (3,5)$, and decreasing on $(-8,-4)\cup (0,3)\cup (5,\infty)$.
		\end{itemize}
	\end{shortsolution}
\end{subproblem}
\begin{subproblem}
	Approximate the intervals on which each function is concave up and concave down.
	\begin{shortsolution}
		\begin{itemize}
			\item $p$ is concave up on  $(1,\infty)$, and concave down on  $(-\infty,1)$.
			\item $q$ is concave up on $(-\infty,-1)\cup (1,\infty)$, and concave down on $(-1,1)$.
			\item $r$ is concave up on $(-\infty,-3)\cup (0,2)$, and concave down on $(-3,0)\cup (2,\infty)$.
			\item $s$ is concave up on $(-6,-2)\cup (2,5)$, and concave down on $(-\infty,-6)\cup (-2,2)\cup (5,\infty)$.
		\end{itemize}
	\end{shortsolution}
\end{subproblem}
\begin{subproblem}
	The degree of $q$ is $5$. Assuming that all of the real zeros of $q$ are
	shown in its graph, how many complex zeros does $q$ have?
	\begin{shortsolution}
		\Vref{poly:fig:incdec2} shows that $q$ has $3$ real zeros 
		since the curve of $q$ cuts the horizontal axis $3$ times. 
		Since $q$ has degree $5$, $q$ must have $2$ complex zeros.
	\end{shortsolution}
\end{subproblem}
\end{problem}

%===================================
%   Author: Hughes
%   Date:   May 2011
%===================================
\begin{problem}[Long-run behaviour of polynomials]
Describe the long-run behavior of each of polynomial functions in 
\crefrange{poly:prob:degree5}{poly:prob:degree2}.
\begin{shortsolution}
	$\dd\lim_{x\rightarrow-\infty}p(x)=-\infty$,
	$\dd\lim_{x\rightarrow\infty}p(x)=\infty$,
	$\dd\lim_{x\rightarrow-\infty}q(x)=\infty$,
	$\dd\lim_{x\rightarrow\infty}q(x)=-\infty$,
	$\dd\lim_{x\rightarrow-\infty}r(x)=-\infty$,
	$\dd\lim_{x\rightarrow\infty}r(x)=\infty$,
	$\dd\lim_{x\rightarrow-\infty}s(x)=\infty$,
	$\dd\lim_{x\rightarrow\infty}s(x)=\infty$,
\end{shortsolution}
\end{problem}

%===================================
%   Author: Hughes
%   Date:   May 2011
%===================================
\begin{problem}[True of false?]
Let $p$ be a polynomial function. 
Label each of the following statements as true (T) or false (F); if they are false, 
provide an example that supports your answer.
\begin{subproblem}
	If $p$ has degree $3$, then $p$ has $3$ distinct zeros.
	\begin{shortsolution}
		False. Consider $p(x)=x^2(x+1)$ which has only 2 distinct zeros.
	\end{shortsolution}
\end{subproblem}
\begin{subproblem}
	If $p$ has degree $4$, then $\dd\lim_{x\rightarrow-\infty}p(x)=\infty$ and $\dd\lim_{x\rightarrow\infty}p(x)=\infty$.
	\begin{shortsolution}
		False. Consider $p(x)=-x^4$.
	\end{shortsolution}
\end{subproblem}
\begin{subproblem}
	If $p$ has even degree, then it is possible that $p$ can have no real zeros.
	\begin{shortsolution}
		True.
	\end{shortsolution}
\end{subproblem}
\begin{subproblem}
	If $p$ has odd degree, then it is possible that $p$ can have no real zeros.
	\begin{shortsolution}
		False. All odd degree polynomials will cut the horizontal axis at least once.
	\end{shortsolution}
\end{subproblem}
\end{problem}
%===================================
%   Author: Hughes
%   Date:   May 2011
%===================================
\begin{problem}[Find a formula from a description]
In each of the following problems, give a possible formula for a polynomial 
function that has the specified properties.
\begin{subproblem}
	Degree 2 and has zeros at $4$ and $5$.
	\begin{shortsolution}
		Possible option: $p(x)=(x-4)(x-5)$. Note we could multiply $p$ by any real number, and still meet the requirements.
	\end{shortsolution}
\end{subproblem}
\begin{subproblem}
	Degree 3 and has zeros at $4$,$5$ and $-3$.
	\begin{shortsolution}
		Possible option: $p(x)=(x-4)(x-5)(x+3)$. Note we could multiply $p$ by any real number, and still meet the requirements.
	\end{shortsolution}
\end{subproblem}
\begin{subproblem}
	Degree 4 and has zeros at $0$, $4$, $5$, $-3$.
	\begin{shortsolution}
		Possible option: $p(x)=x(x-4)(x-5)(x+3)$. Note we could multiply $p$ by any real number, and still meet the requirements.
	\end{shortsolution}
\end{subproblem}
\begin{subproblem}
	Degree 4, with zeros that make the graph cut at $2$, $-5$, and a zero that makes the graph touch at $-2$;
	\begin{shortsolution}
		Possible option: $p(x)=(x-2)(x+5)(x+2)^2$. Note we could multiply $p$ by any real number, and still meet the requirements.
	\end{shortsolution}
\end{subproblem}
\begin{subproblem}
	Degree 3, with only one zero at $-1$.
	\begin{shortsolution}
		Possible option: $p(x)=(x+1)^3$. Note we could multiply $p$ by any real number, and still meet the requirements.
	\end{shortsolution}
\end{subproblem}
\end{problem}
%===================================
%   Author: Hughes
%   Date:   June 2012
%===================================
\begin{problem}[\Cref{poly:step:last}]
\pccname{Saheed} is graphing a polynomial function, $p$. 
He is following \crefrange{poly:step:first}{poly:step:last} and has so far
marked the zeros of $p$ on \cref{poly:fig:optionsp1}. Saheed tells you that 
$p$ has degree $3$, but does \emph{not} say if the leading coefficient 
of $p$ is positive or negative.
\begin{figure}[!htbp]
	\begin{widepage}
	\begin{subfigure}{.45\textwidth}
		\begin{tikzpicture}
			\begin{axis}[
					xmin=-10,xmax=10,
					ymin=-10,ymax=10,
					xtick={-8,-6,...,8},
					ytick={-15},
					height=.5\textwidth,
				]
				\addplot[soldot] coordinates{(-5,0)(2,0)(6,0)};
			\end{axis}
		\end{tikzpicture}
		\caption{}
		\label{poly:fig:optionsp1}
	\end{subfigure}%
	\hfill
	\begin{subfigure}{.45\textwidth}
		\begin{tikzpicture}
			\begin{axis}[
					xmin=-10,xmax=10,
					ymin=-10,ymax=10,
					xtick={-8,-6,...,8},
					ytick={-15},
					height=.5\textwidth,
				]
				\addplot[soldot] coordinates{(-5,0)(6,0)};
			\end{axis}
		\end{tikzpicture}
		\caption{}
		\label{poly:fig:optionsp2}
	\end{subfigure}%
	\caption{}
	\end{widepage}
\end{figure}
\begin{subproblem}
	Use the information in \cref{poly:fig:optionsp1} to help sketch $p$, assuming that the leading coefficient
	is positive.
	\begin{shortsolution}
		Assuming that $a_3>0$:
				
		\begin{tikzpicture}
			\begin{axis}[
					xmin=-10,xmax=10,
					ymin=-10,ymax=10,
					xtick={-8,-6,...,8},
					ytick={-15},
				]
				\addplot expression[domain=-6.78179:8.35598,samples=50]{1/20*(x+5)*(x-2)*(x-6)};
				\addplot[soldot] coordinates{(-5,0)(2,0)(6,0)};
			\end{axis}
		\end{tikzpicture}
	\end{shortsolution}
\end{subproblem}
\begin{subproblem}
	Use the information in \cref{poly:fig:optionsp1} to help sketch $p$, assuming that the leading coefficient
	is negative.
	\begin{shortsolution}
		Assuming that $a_3<0$:
				
		\begin{tikzpicture}
			\begin{axis}[
					xmin=-10,xmax=10,
					ymin=-10,ymax=10,
					xtick={-8,-6,...,8},
					ytick={-15},
				]
				\addplot expression[domain=-6.78179:8.35598,samples=50]{-1/20*(x+5)*(x-2)*(x-6)};
				\addplot[soldot] coordinates{(-5,0)(2,0)(6,0)};
			\end{axis}
		\end{tikzpicture}
	\end{shortsolution}
\end{subproblem}
Saheed now turns his attention to another polynomial function, $q$. He finds
the zeros of $q$ (there are only $2$) and marks them on \cref{poly:fig:optionsp2}. 
Saheed knows that $q$ has degree $3$, but doesn't know if the leading 
coefficient is positive or negative.
\begin{subproblem}
	Use the information in \cref{poly:fig:optionsp2} to help sketch $q$, assuming that the leading 
	coefficient of $q$ is positive. Hint: only one of the zeros is simple.
	\begin{shortsolution}
		Assuming that $a_4>0$ there are $2$ different options:
				
		\begin{tikzpicture}
			\begin{axis}[
					xmin=-10,xmax=10,
					ymin=-10,ymax=10,
					xtick={-8,-6,...,8},
					ytick={-15},
				]
				\addplot expression[domain=-8.68983:7.31809,samples=50]{1/20*(x+5)^2*(x-6)};
				\addplot expression[domain=-6.31809:9.68893,samples=50]{1/20*(x+5)*(x-6)^2};
				\addplot[soldot] coordinates{(-5,0)(6,0)};
			\end{axis}
		\end{tikzpicture}
	\end{shortsolution}
\end{subproblem}
\begin{subproblem}
	Use the information in \cref{poly:fig:optionsp2} to help sketch $q$, assuming that the leading 
	coefficient of $q$ is negative.
	\begin{shortsolution}
		Assuming that $a_4<0$ there are $2$ different options:
				
		\begin{tikzpicture}
			\begin{axis}[
					xmin=-10,xmax=10,
					ymin=-10,ymax=10,
					xtick={-8,-6,...,8},
					ytick={-15},
				]
				\addplot expression[domain=-8.68983:7.31809,samples=50]{-1/20*(x+5)^2*(x-6)};
				\addplot expression[domain=-6.31809:9.68893,samples=50]{-1/20*(x+5)*(x-6)^2};
				\addplot[soldot] coordinates{(-5,0)(6,0)};
			\end{axis}
		\end{tikzpicture}
	\end{shortsolution}
\end{subproblem}
\end{problem}
%===================================
%   Author: Hughes
%   Date:   June 2012
%===================================
\begin{problem}[Zeros]
Find all zeros of each of the following polynomial functions, making 
sure to detail their multiplicity. Note that
you may need to use factoring, or the quadratic formula, or both! Also note 
that some zeros may be repeated, and some may be complex.
\begin{multicols}{3}
	\begin{subproblem}
		$p(x)=x^2+1$ 
		\begin{shortsolution}
			$\pm i$ (simple).
		\end{shortsolution}
	\end{subproblem}
	\begin{subproblem}
		$q(y)=(y^2-9)(y^2-7)$ 
		\begin{shortsolution}
			$\pm 3$, $\pm \sqrt{7}$ (all are simple).
		\end{shortsolution}
	\end{subproblem}
	\begin{subproblem}
		$r(z)=-4z^3(z^2+3)(z^2+64)$ 
		\begin{shortsolution}
			$0$ (multiplicity $3$), $\pm\sqrt{3}$ (simple), $\pm\sqrt{8}$ (simple).
		\end{shortsolution}
	\end{subproblem}
	\begin{subproblem}
		$a(x)=x^4-81$ 
		\begin{shortsolution}
			$\pm 3$, $\pm 3i$ (all are simple).
		\end{shortsolution}
	\end{subproblem}
	\begin{subproblem}
		$b(y)=y^3-8$ 
		\begin{shortsolution}
			$2$, $-1\pm i\sqrt{3}$ (all are simple).
		\end{shortsolution}
	\end{subproblem}
	\begin{subproblem}
		$c(m)=m^3-m^2$
		\begin{shortsolution}
			$0$ (multiplicity $2$), $1$ (simple).
		\end{shortsolution}
	\end{subproblem}
	\begin{subproblem}
		$h(n)=(n+1)(n^2+4)$ 
		\begin{shortsolution}
			$-1$, $\pm 2i$ (all are simple).
		\end{shortsolution}
	\end{subproblem}
	\begin{subproblem}
		$f(\alpha)=(\alpha^2-16)(\alpha^2-5\alpha+4)$ 
		\begin{shortsolution}
			$-4$ (simple), $4$ (multiplicity $2$), $1$ (simple).
		\end{shortsolution}
	\end{subproblem}
	\begin{subproblem}
		$g(\beta)=(\beta^2-25)(\beta^2-5\beta-4)$ 
		\begin{shortsolution}
			$\pm 5$, $\dfrac{5\pm\sqrt{41}}{2}$ (all are simple).
		\end{shortsolution}
	\end{subproblem}
\end{multicols}
\end{problem}
%===================================
%   Author: Hughes
%   Date:   June 2012
%===================================
\begin{problem}[Given zeros, find a formula]
In each of the following problems you are given the zeros of a polynomial. 
Write a possible formula for each polynomial| you may leave your 
answer in factored form, but it may not contain complex numbers. Unless
otherwise stated, assume that the zeros are simple.
\begin{multicols}{3}
	\begin{subproblem}
		$1$, $2$ 
		\begin{shortsolution}
			$p(x)=(x-1)(x-2)$
		\end{shortsolution}
	\end{subproblem}
	\begin{subproblem}
		$0$, $5$, $13$ 
		\begin{shortsolution}
			$p(x)=x(x-5)(x-13)$
		\end{shortsolution}
	\end{subproblem}
	\begin{subproblem}
		$-7$, $2$ (multiplicity $3$), $5$ 
		\begin{shortsolution}
			$p(x)=(x+7)(x-2)^3(x-5)$
		\end{shortsolution}
	\end{subproblem}
	\begin{subproblem}
		$0$, $\pm i$ 
		\begin{shortsolution}
			$p(x)=x(x^2+1)$
		\end{shortsolution}
	\end{subproblem}
	\begin{subproblem}
		$\pm 2i$, $\pm 7$ 
		\begin{shortsolution}
			$p(x)=(x^2+4)(x^2-49)$
		\end{shortsolution}
	\end{subproblem}
	\begin{subproblem}
		$-2\pm i\sqrt{6}$ 
	\end{subproblem}
\end{multicols}
\end{problem}
%===================================
%   Author: Hughes
%   Date:   June 2012
%===================================
\begin{problem}[Composition of polynomials]
Let $p$ and $q$ be polynomial functions that have formulas
\[
	p(x)=(x+1)(x+2)(x+5), \qquad q(x)=3-x^4
\]
Evaluate each of the following.
\begin{multicols}{4}
	\begin{subproblem}
		$(p\circ q)(0)$ 
		\begin{shortsolution}
			$160$
		\end{shortsolution}
	\end{subproblem}
	\begin{subproblem}
		$(q\circ p)(0)$ 
		\begin{shortsolution}
			$-9997$
		\end{shortsolution}
	\end{subproblem}
	\begin{subproblem}
		$(p\circ q)(1)$ 
		\begin{shortsolution}
			$84$
		\end{shortsolution}
	\end{subproblem}
	\begin{subproblem}
		$(p\circ p)(0)$ 
		\begin{shortsolution}
			$1980$
		\end{shortsolution}
	\end{subproblem}
\end{multicols}
\end{problem}
%===================================
%   Author: Hughes
%   Date:   June 2012
%===================================
\begin{problem}[Piecewise polynomial functions]
Let $P$ be the piecewise-defined function with formula
\[
	P(x)=\begin{cases}
	(1-x)(2x+5)(x^2+1), & x\leq -3 \\ 
	4-x^2,              & -3<x < 4 \\ 
	x^3                 & x\geq 4  
	\end{cases}
\]
Evaluate each of the following 
\begin{multicols}{5}
	\begin{subproblem}
		$P(-4)$ 
		\begin{shortsolution}
			$-255$
		\end{shortsolution}
	\end{subproblem}
	\begin{subproblem}
		$P(0)$ 
		\begin{shortsolution}
			$4$
		\end{shortsolution}
	\end{subproblem}
	\begin{subproblem}
		$P(4)$ 
		\begin{shortsolution}
			$64$
		\end{shortsolution}
	\end{subproblem}
	\begin{subproblem}
		$P(-3)$ 
		\begin{shortsolution}
			$-40$
		\end{shortsolution}
	\end{subproblem}
	\begin{subproblem}
		$(P\circ P)(0)$ 
		\begin{shortsolution}
			$64$
		\end{shortsolution}
	\end{subproblem}
\end{multicols}
\end{problem}

%===================================
%   Author: Hughes
%   Date:   July 2012
%===================================
\begin{problem}[Function algebra]
Let $p$ and $q$ be the polynomial functions that have formulas
\[
	p(x)=x(x+1)(x-3)^2, \qquad q(x)=7-x^2
\]
Evaluate each of the following (if possible).
\begin{multicols}{4}
	\begin{subproblem}
		$(p+q)(1)$ 
		\begin{shortsolution}
			$14$ 
		\end{shortsolution}
	\end{subproblem}
	\begin{subproblem}
		$(p-q)(0)$ 
		\begin{shortsolution}
			$7$ 
		\end{shortsolution}
	\end{subproblem}
	\begin{subproblem}
		$(p\cdot q)(\sqrt{7})$ 
		\begin{shortsolution}
			$0$ 
		\end{shortsolution}
	\end{subproblem}
	\begin{subproblem}
		$\left( \frac{q}{p} \right)(1)$ 
		\begin{shortsolution}
			$\frac{3}{4}$ 
		\end{shortsolution}
	\end{subproblem}
\end{multicols}
\begin{subproblem}
	What is the domain of the function $\frac{q}{p}$?
	\begin{shortsolution}
		$(-\infty,-1)\cup (-1,0)\cup (0,3)\cup (3,\infty)$ 
	\end{shortsolution}
\end{subproblem}
\end{problem}

%===================================
%   Author: Hughes
%   Date:   July 2012
%===================================
\begin{problem}[Transformations: given the transformation, find the formula]
Let $p$ be the polynomial function that has formula.
\[
	p(x)=4x(x^2-1)(x+3)
\]
In each of the following 
problems apply the given transformation to the function $p$ and 
write a formula for the transformed version of $p$.
\begin{multicols}{2}
	\begin{subproblem}
		Shift $p$ to the right by $5$ units. 
		\begin{shortsolution}
			$p(x-5)=4(x-5)(x-2)(x^2-10x+24)$
		\end{shortsolution}
	\end{subproblem}
	\begin{subproblem}
		Shift $p$ to the left by $6$ units. 
		\begin{shortsolution}
			$p(x+6)=4(x+6)(x+9)(x^2+12x+35)$
		\end{shortsolution}
	\end{subproblem}
	\begin{subproblem}
		Shift $p$ up by $12$ units. 
		\begin{shortsolution}
			$p(x)+12=4x(x^2-1)(x+3)+12$
		\end{shortsolution}
	\end{subproblem}
	\begin{subproblem}
		Shift $p$ down by $2$ units. 
		\begin{shortsolution}
			$p(x)-2=4x(x^2-1)(x+3)-2$
		\end{shortsolution}
	\end{subproblem}
	\begin{subproblem}
		Reflect $p$ over the horizontal axis.
		\begin{shortsolution}
			$-p(x)=-4x(x^2-1)(x+3)$
		\end{shortsolution}
	\end{subproblem}
	\begin{subproblem}
		Reflect $p$ over the vertical axis.
		\begin{shortsolution}
			$p(-x)=-4x(x^2-1)(3-x)$ 
		\end{shortsolution}
	\end{subproblem}
\end{multicols}
\end{problem}

%===================================
%   Author: Hughes
%   Date:   May 2011
%===================================
\begin{problem}[Find a formula from a table]\label{poly:prob:findformula}
\Crefrange{poly:tab:findformulap}{poly:tab:findformulas} show values of polynomial functions, $p$, $q$, 
$r$, and $s$.

\begin{table}[!htb]
	\centering
	\begin{widepage}
	\caption{Tables for \cref{poly:prob:findformula}}
	\label{poly:tab:findformula}
	\begin{subtable}{.2\textwidth}
		\centering
		\caption{$y=p(x)$}
		\label{poly:tab:findformulap}
		\begin{tabular}{S[table-format=1.0]S[table-format=2.0]}
			\beforeheading
			\heading{$x$} & \heading{$y$} \\            
			\afterheading
			-4            & -56           \\\normalline 
			-3            & -18           \\\normalline  
			-2            & 0             \\\normalline   
			-1            & 4             \\\normalline  
			0             & 0             \\\normalline   
			1             & -6            \\\normalline   
			2             & -8            \\\normalline  
			3             & 0             \\\normalline  
			4             & 24            \\\lastline    
		\end{tabular}
	\end{subtable}
	\hfill
	\begin{subtable}{.2\textwidth}
		\centering
		\caption{$y=q(x)$}
		\label{poly:tab:findformulaq}
		\begin{tabular}{S[table-format=1.0]S[table-format=3.0]}
			\beforeheading
			\heading{$x$} & \heading{$y$} \\ \afterheading 
			-4            & -16           \\\normalline       
			-3            & -3            \\\normalline        
			-2            & 0             \\\normalline         
			-1            & -1            \\\normalline        
			0             & 0             \\\normalline         
			1             & 9             \\\normalline         
			2             & 32            \\\normalline        
			3             & 75            \\\normalline        
			4             & 144           \\\lastline          
		\end{tabular}
	\end{subtable}
	\hfill
	\begin{subtable}{.2\textwidth}
		\centering
		\caption{$y=r(x)$}
		\label{poly:tab:findformular}
		\begin{tabular}{S[table-format=1.0]S[table-format=3.0]}
			\beforeheading
			\heading{$x$} & \heading{$y$} \\ \afterheading 
			-4            & 105           \\\normalline       
			-3            & 0             \\\normalline         
			-2            & -15           \\\normalline       
			-1            & 0             \\\normalline         
			0             & 9             \\\normalline         
			1             & 0             \\\normalline         
			2             & -15           \\\normalline       
			3             & 0             \\\normalline         
			4             & 105           \\\lastline          
		\end{tabular}
	\end{subtable}
	\hfill
	\begin{subtable}{.2\textwidth}
		\centering
		\caption{$y=s(x)$}
		\label{poly:tab:findformulas}
		\begin{tabular}{S[table-format=1.0]S[table-format=3.0]}
			\beforeheading
			\heading{$x$} & \heading{$y$} \\ \afterheading 
			-4            & 75            \\\normalline        
			-3            & 0             \\\normalline         
			-2            & -9            \\\normalline        
			-1            & 0             \\\normalline         
			0             & 3             \\\normalline         
			1             & 0             \\\normalline         
			2             & 15            \\\normalline        
			3             & 96            \\\normalline        
			4             & 760           \\\lastline          
		\end{tabular}
	\end{subtable}
	\end{widepage}
\end{table}

\begin{subproblem}
	Assuming that all of the zeros of $p$ are shown (in \cref{poly:tab:findformulap}), how many zeros does $p$ have? 
	\begin{shortsolution}
		$p$ has 3 zeros.
	\end{shortsolution}
\end{subproblem}
\begin{subproblem}
	What is the degree of $p$?
	\begin{shortsolution}
		$p$ is degree 3.
	\end{shortsolution}
\end{subproblem}
\begin{subproblem}
	Write a formula for $p(x)$.
	\begin{shortsolution}
		$p(x)=x(x+2)(x-3)$
	\end{shortsolution}
\end{subproblem}
\begin{subproblem}
	Assuming that all of the zeros of $q$ are shown (in \cref{poly:tab:findformulaq}), how many zeros does $q$ have?
	\begin{shortsolution}
		$q$ has 2 zeros.
	\end{shortsolution}
\end{subproblem}
\begin{subproblem}
	Describe the difference in behavior of $p$ and $q$ at $-2$.
	\begin{shortsolution}
		$p$ changes sign at $-2$, and $q$ does not change sign at $-2$.
	\end{shortsolution}
\end{subproblem}
\begin{subproblem}
	Given that $q$ is a degree-$3$ polynomial, write a formula for $q(x)$.
	\begin{shortsolution}
		$q(x)=x(x+2)^2$
	\end{shortsolution}
\end{subproblem}
\begin{subproblem}
	Assuming that all of the zeros of $r$ are shown (in \cref{poly:tab:findformular}), find a formula for $r(x)$.
	\begin{shortsolution}
		$r(x)=(x+3)(x+1)(x-1)(x-3)$ 
	\end{shortsolution}
\end{subproblem}
\begin{subproblem}
	Assuming that all of the zeros of $s$ are shown (in \cref{poly:tab:findformulas}), find a formula for $s(x)$.
	\begin{shortsolution}
		$s(x)=(x+3)(x+1)(x-1)^2$ 
	\end{shortsolution}
\end{subproblem}
\end{problem}
\end{exercises}

\section{Rational functions}
\subsection*{Power functions with negative exponents}
The study of rational functions will rely upon a good knowledge 
of power functions with negative exponents. \Cref{rat:ex:oddpow,rat:ex:evenpow} are 
simple but fundamental to understanding the behavior of rational functions.
%===================================
%   Author: Hughes
%   Date:   May 2011
%===================================
\begin{pccexample}[Power functions with odd negative exponents]\label{rat:ex:oddpow}
	Graph each the functions $f$, $g$, and $h$ that have 
	formulas
	\[
		f(x)=\frac{1}{x},\qquad g(x)=\dfrac{1}{x^3},\qquad h(x)=\dfrac{1}{x^5}
	\]
	and state their domain in interval notation, and their 
	behavior as $x\rightarrow 0^-$ and $x\rightarrow 0^+$.
	\begin{pccsolution}
		The functions $f$, $g$, and $k$ are plotted in \cref{rat:fig:oddpow}.
		The domain of each of the functions $f$, $g$, and $h$ is $(-\infty,0)\cup (0,\infty)$. Note that 
		the long-run behavior of each of the functions is the same, and in particular
		\begin{align*}
			f(x)\rightarrow 0                         & \text{ as } x\rightarrow\infty  \\ 
			\mathllap{\text{and }}  f(x)\rightarrow 0 & \text{ as } x\rightarrow-\infty 
		\end{align*}
		The same results hold for $g$ and $h$. Note also that each of the functions
		has a \emph{vertical asymptote} at $0$. We see that
		\begin{align*}
			                       & f(x)\rightarrow -\infty                          \text{ as } x\rightarrow 0^- \\ 
			\mathllap{\text{and }} & f(x)\rightarrow \infty  \text{ as } x\rightarrow 0^+                          
		\end{align*}
		The same results hold for $g$ and $h$. Note that the range of each of the functions
		$f$, $g$, and $h$ is $(-\infty,0)\cup (0,\infty)$.
		
		The curve of a function that has a vertical asymptote is necessarily separated 
		into \emph{branches}| each of the functions $f$, $g$, and $h$ have two branches. 
		
		It appears from \cref{rat:fig:oddpow} that each curve is symmetric
		about the origin| perhaps each function is \emph{odd}. 
		\fixthis{need reference to definition about even and odd functions- doesn't
		exist yet}
		Let's test each function to see if they are odd or not:
		\begin{align*}
			f(-x) & =\frac{1}{(-x)^3} & g(-x) & =\frac{1}{(-x)^5} & h(-x) & =\frac{1}{(-x)^7} \\ 
			      & =\frac{1}{-x^3}   &       & =\frac{1}{-x^5}   &       & =\frac{1}{-x^7}   \\   
			      & =-\frac{1}{x^3}   &       & =-\frac{1}{x^5}   &       & =-\frac{1}{x^7}   \\   
			      & =-f(x)            &       & =-g(x)            &       & =-h(x)            
		\end{align*}
		We conclude that each of the functions $f$, $g$, and $h$ \emph{are} odd.
	\end{pccsolution}
\end{pccexample}

\begin{figure}[!htb]
	\begin{minipage}{.45\textwidth}
		\begin{tikzpicture}
			\begin{axis}[
					framed,
					xmin=-3,xmax=3,
					ymin=-5,ymax=5,
					xtick={-2,-1,...,2},
					minor ytick={-3,-1,...,3},
					grid=both,
					legend pos=north west,
				]
				\addplot expression[domain=-3:-0.2]{1/x};
				\addplot expression[domain=-3:-0.584]{1/x^3};
				\addplot expression[domain=-3:-0.724]{1/x^5};
				\addplot expression[domain=0.2:3]{1/x};
				\addplot expression[domain=0.584:3]{1/x^3};
				\addplot expression[domain=0.724:3]{1/x^5};
				\addplot[soldot]coordinates{(-1,-1)}node[axisnode,anchor=north east]{$(-1,-1)$};
				\addplot[soldot]coordinates{(1,1)}node[axisnode,anchor=south west]{$(1,1)$};
				\legend{$f$,$g$,$h$}
			\end{axis}
		\end{tikzpicture}
		\caption{}
		\label{rat:fig:oddpow}
	\end{minipage}%
	\hfill
	\begin{minipage}{.45\textwidth}
		\begin{tikzpicture}
			\begin{axis}[
					framed,
					xmin=-3,xmax=3,
					ymin=-5,ymax=5,
					xtick={-2,-1,...,2},
					minor ytick={-3,-1,...,3},
					grid=both,
					legend pos=south east,
				]
				\addplot expression[domain=-3:-0.447]{1/x^2};
				\addplot expression[domain=-3:-0.668]{1/x^4};
				\addplot expression[domain=-3:-0.764]{1/x^6};
				\addplot expression[domain=0.447:3]{1/x^2};
				\addplot expression[domain=0.668:3]{1/x^4};
				\addplot expression[domain=0.764:3]{1/x^6};
				\addplot[soldot]coordinates{(-1,1)}node[axisnode,anchor=south east]{$(-1,1)$};
				\addplot[soldot]coordinates{(1,1)}node[axisnode,anchor=south west]{$(1,1)$};
				\legend{$F$,$G$,$H$}
			\end{axis}
		\end{tikzpicture}
		\caption{}
		\label{rat:fig:evenpow}
	\end{minipage}%
\end{figure}


%===================================
%   Author: Hughes
%   Date:   May 2011
%===================================
\begin{pccexample}[Power functions with even negative exponents]\label{rat:ex:evenpow}%
	Graph each the functions $F$, $G$, and $H$ that 
	have formulas
	\[
		F(x)=\frac{1}{x^2},\qquad G(x)=\frac{1}{x^4},\qquad H(x)=\frac{1}{x^6}
	\]
	and state their domain, and their behavior as $x\rightarrow 0^-$ and $x\rightarrow 0^+$.
	\begin{pccsolution}
		The functions $F$, $G$, and $H$ are plotted in \cref{rat:fig:evenpow}.
		The domain of each of the functions $F$, $G$, and $H$ is $(-\infty,0)\cup (0,\infty)$. Note that 
		the long-run behavior of each of the functions is the same, and in particular
		\begin{align*}
			F(x)\rightarrow 0                           & \text{ as } x\rightarrow\infty  \\ 
			\mathllap{\text{and }}    F(x)\rightarrow 0 & \text{ as } x\rightarrow-\infty 
		\end{align*}
		As in \cref{rat:ex:oddpow}, $F$ has a horizontal asymptote  that 
		has equation $y=0$.
		The same results hold for $G$ and $H$. Note also that each of the functions
		has a \emph{vertical asymptote} at $0$. We see that
		\begin{align*}
			F(x)\rightarrow \infty                          & \text{ as } x\rightarrow 0^- \\ 
			\mathllap{\text{and }}   F(x)\rightarrow \infty & \text{ as } x\rightarrow 0^+ 
		\end{align*}
		The same results hold for $G$ and $H$. Each of the functions $F$, $G$, and $H$ 
		have two branches, and the range of each function is $(0,\infty)$.
		
		It appears from \cref{rat:fig:evenpow} that each curve is symmetric
		about the vertical axis| perhaps each function is \emph{even}. 
		\fixthis{need reference to definition about even and odd functions- doesn't
		exist yet}
		Let's test each function to see if they are even or not:
		\begin{align*}
			F(-x) & =\frac{1}{(-x)^2} & G(-x) & =\frac{1}{(-x)^4} & H(-x) & =\frac{1}{(-x)^6} \\ 
			      & =\frac{1}{x^2}    &       & =\frac{1}{x^4}    &       & =\frac{1}{x^6}    \\    
			      & =F(x)             &       & =G(x)             &       & =H(x)             
		\end{align*}
		We conclude that each of the functions $f$, $g$, and $h$ \emph{are} even.
	\end{pccsolution}
\end{pccexample}
%===================================
%   Author: Hughes
%   Date:   March 2012
%===================================
\begin{doyouunderstand}
	\begin{problem}
	Repeat \cref{rat:ex:oddpow,rat:ex:evenpow} using (respectively) the 
	functions that have the following formulas.
	\begin{subproblem}
		$k(x)=-\dfrac{1}{x}$, $ m(x)=-\dfrac{1}{x^3}$, $ n(x)=-\dfrac{1}{x^5}$
		\begin{shortsolution}
			The functions $k$, $m$, and $n$ have domain $(-\infty,0)\cup (0,\infty)$, and 
			range $(-\infty,0)\cup (0,\infty)$; the functions are graphed below.
						
			\begin{tikzpicture}
				\begin{axis}[
						framed,
						xmin=-3,xmax=3,
						ymin=-5,ymax=5,
						xtick={-2,-1,...,2},
						minor ytick={-3,-1,...,3},
						grid=both,
						legend pos=north east,
					]
					\addplot expression[domain=-3:-0.2]{-1/x};
					\addplot expression[domain=-3:-0.584]{-1/x^3};
					\addplot expression[domain=-3:-0.724]{-1/x^5};
					\addplot expression[domain=0.2:3]{-1/x};
					\addplot expression[domain=0.584:3]{-1/x^3};
					\addplot expression[domain=0.724:3]{-1/x^5};
					\legend{$k$,$m$,$n$}
				\end{axis}
			\end{tikzpicture}
						
			Note that
			\begin{align*}
				k(x)\rightarrow 0                                & \text{ as } x\rightarrow\infty  \\  
				\mathllap{\text{and }}    k(x)\rightarrow 0      & \text{ as } x\rightarrow-\infty \\ 
				\intertext{and also}
				k(x)\rightarrow \infty                           & \text{ as } x\rightarrow 0^-    \\    
				\mathllap{\text{and }}   k(x)\rightarrow -\infty & \text{ as } x\rightarrow 0^+    
			\end{align*}
			The same are true for $m$ and $n$. Note that each function is odd:
			\begin{align*}
				k(-x) & =-\frac{1}{(-x)^3} & m(-x) & =-\frac{1}{(-x)^5} & n(-x) & =-\frac{1}{(-x)^7} \\ 
				      & =-\frac{1}{-x^3}   &       & =-\frac{1}{-x^5}   &       & =-\frac{1}{-x^7}   \\   
				      & =\frac{1}{x^3}     &       & =\frac{1}{x^5}     &       & =\frac{1}{x^7}     \\   
				      & =-k(x)             &       & =-m(x)             &       & =-n(x)             
			\end{align*}
		\end{shortsolution}
	\end{subproblem}
	\begin{subproblem}
		$ K(x)=-\dfrac{1}{x^2}$, $ M(x)=-\dfrac{1}{x^4}$, $ N(x)=-\dfrac{1}{x^6}$
		\begin{shortsolution}
			The functions $K$, $M$, and $N$ have domain $(-\infty,0)\cup (0,\infty)$; the 
			range of each function is $(-\infty,0)$. The functions are 
			are graphed below.
						
			\begin{tikzpicture}
				\begin{axis}[
						framed,
						xmin=-3,xmax=3,
						ymin=-5,ymax=5,
						xtick={-2,-1,...,2},
						minor ytick={-3,-1,...,3},
						grid=both,
						legend pos=north east,
					]
					\addplot expression[domain=-3:-0.447]{-1/x^2};
					\addplot expression[domain=-3:-0.668]{-1/x^4};
					\addplot expression[domain=-3:-0.764]{-1/x^6};
					\addplot expression[domain=0.447:3]{-1/x^2};
					\addplot expression[domain=0.668:3]{-1/x^4};
					\addplot expression[domain=0.764:3]{-1/x^6};
					\legend{$K$,$M$,$N$}
				\end{axis}
			\end{tikzpicture}
						
			Note that
			\begin{align*}
				K(x)\rightarrow 0                                & \text{ as } x\rightarrow\infty  \\  
				\mathllap{\text{and }}    K(x)\rightarrow 0      & \text{ as } x\rightarrow-\infty \\ 
				\intertext{and also}
				K(x)\rightarrow -\infty                          & \text{ as } x\rightarrow 0^-    \\    
				\mathllap{\text{and }}   K(x)\rightarrow -\infty & \text{ as } x\rightarrow 0^+    
			\end{align*}
			The same are true for $M$ and $N$. Note that each function is even:
			\begin{align*}
				K(-x) & =-\frac{1}{(-x)^2} & M(-x) & =-\frac{1}{(-x)^4} & N(-x) & =-\frac{1}{(-x)^6} \\ 
				      & =-\frac{1}{x^2}    &       & =-\frac{1}{x^4}    &       & =-\frac{1}{x^6}    \\   
				      & =K(x)              &       & =M(x)              &       & =N(x)              
			\end{align*}
		\end{shortsolution}
	\end{subproblem}
	\end{problem}
\end{doyouunderstand}

\subsection*{Rational functions}
\begin{pccdefinition}[Rational functions]\label{rat:def:function}
	The most general formula for a rational function, $r$, is
	\[
		r(x) = \frac{p(x)}{q(x)}
	\]
	where both $p$ and $q$ are polynomial functions. 
	
	Note that
	\begin{itemize}
		\item the domain or $r$ will be all real numbers, except those that
		make the \emph{denominator}, $q(x)$, equal to $0$;
		\item the zeros of $r$ are the zeros of $p$, i.e the real numbers
		that make the \emph{numerator}, $p(x)$, equal to $0$.
	\end{itemize}
	
	\Cref{rat:ex:oddpow,rat:ex:evenpow} are particularly important because $r$ 
	will behave like $\frac{1}{x}$, or $\frac{1}{x^2}$ around its vertical asymptotes, 
	depending on the power that the relevant term is raised to| we will demonstrate 
	this in what follows.
\end{pccdefinition}

%===================================
%   Author: Hughes
%   Date:   May 2011
%===================================
\begin{pccexample}[Rational or not]
	Decide if the following formulas correspond to rational functions 
	or not; if the function is rational, state its domain.
	\begin{multicols}{3}
		\begin{enumerate}
			\item $r(x)=\dfrac{1}{x}$
			\item $f(x)=2^x+3$
			\item $g(x)=19$
			\item $h(x)=\dfrac{3+x}{4-x}$
			\item $k(x)=\dfrac{x^3+2x}{x-15}$
			\item $l(x)=9-4x$
			\item $m(x)=\dfrac{x+5}{(x-7)(x+9)}$
			\item $n(x)=x^2+6x+7$
			\item $q(x)=1-\dfrac{3}{x+1}$
		\end{enumerate}
	\end{multicols}
	\begin{pccsolution}
		\begin{enumerate}
			\item $r$ is rational; the domain of $r$ is $(-\infty,0)\cup(0,\infty)$.
			\item $f$ is not rational.
			\item $g$ is not rational; $g$ is constant.
			\item $h$ is rational; the domain of $h$ is $(-\infty,4)\cup(4,\infty)$.
			\item $k$ is rational; the domain of $k$ is $(-\infty,15)\cup(15,\infty)$.
			\item $l$ is not rational; $l$ is linear.
			\item $m$ is rational; the domain of $m$ is $(-\infty,-9)\cup(-9,7)\cup(7,\infty)$.
			\item $n$ is not rational; $n$ is quadratic (or you might describe $n$ as a polynomial).
			\item $q$ is rational; the domain of $q$ is $(-\infty,-1)\cup (-1,\infty)$.
		\end{enumerate}
	\end{pccsolution}
\end{pccexample}

%===================================
%   Author: Hughes
%   Date:   May 2011
%===================================
\begin{pccexample}[Match formula to graph]\label{rat:ex:match}
	The functions $r$, $q$, and $k$ that have formulas 
	\[
		r(x)=\frac{1}{x-3}, \qquad q(x)=\frac{x-2}{x+5}, \qquad k(x)=\frac{1}{(x+2)(x-3)}
	\]
	are graphed in \cref{rat:fig:whichiswhich}.  Match each formula 
	to the appropriate graph.
	
	\begin{figure}[!htb]
		\setlength{\figurewidth}{0.3\textwidth}
		\begin{subfigure}{\figurewidth}
			\begin{tikzpicture}[/pgf/declare function={f=(x-2)/(x+5);}]
				\begin{axis}[
						framed,
						xmin=-10,xmax=10,
						ymin=-6,ymax=6,
						xtick={-8,-6,...,8},
						minor ytick={-4,-3,...,4},
						grid=both,
					]
					\addplot[pccplot] expression[domain=-10:-6.37]{f};
					\addplot[pccplot] expression[domain=-3.97:10]{f};
					\addplot[soldot] coordinates{(2,0)};
					\addplot[asymptote,domain=-6:6]({-5},{x});
				\end{axis}
			\end{tikzpicture}
			\caption{}
			\label{rat:fig:which1}
		\end{subfigure}
		\hfill
		\begin{subfigure}{\figurewidth}
			\begin{tikzpicture}[/pgf/declare function={f=1/(x-3);}]
				\begin{axis}[
						framed,
						xmin=-10,xmax=10,
						ymin=-5,ymax=6,
						xtick={-8,-6,...,8},
						ytick={-4,4},
						minor ytick={-3,...,5},
						grid=both,
					]
					\addplot[pccplot] expression[domain=-10:2.8]{f};
					\addplot[pccplot] expression[domain=3.17:10]{f};
					\addplot[asymptote,domain=-6:6]({3},{x});
				\end{axis}
			\end{tikzpicture}
			\caption{}
			\label{rat:fig:which2}
		\end{subfigure}
		\hfill
		\begin{subfigure}{\figurewidth}
			\begin{tikzpicture}[/pgf/declare function={f=1/((x-3)*(x+2));}]
				\begin{axis}[
						framed,
						xmin=-10,xmax=10,
						ymin=-5,ymax=5,
						xtick={-8,-6,...,8},
						ytick={-4,4},
						minor ytick={-3,...,3},
						grid=both,
					]
					\addplot[pccplot] expression[domain=-10:-2.03969]{f};
					\addplot[pccplot] expression[domain=-1.95967:2.95967]{f};
					\addplot[pccplot] expression[domain=3.03969:10]{f};
					\addplot[asymptote,domain=-5:5]({-2},{x});
					\addplot[asymptote,domain=-5:5]({3},{x});
				\end{axis}
			\end{tikzpicture}
			\caption{}
			\label{rat:fig:which3}
		\end{subfigure}
		\caption{}
		\label{rat:fig:whichiswhich}
	\end{figure}
	
	\begin{pccsolution}
		Let's start with the function $r$. Note that domain of $r$ is $(-\infty,3)\cup(0,3)$, so 
		we search for a function that has a vertical asymptote at $3$. There 
		are two possible choices: the functions graphed in \cref{rat:fig:which2,rat:fig:which3}, 
		but note that the function in \cref{rat:fig:which3} also has a vertical asymptote at $-2$ 
		which is not consistent with the formula for $r(x)$. Therefore, $y=r(x)$
		is graphed in \cref{rat:fig:which2}. 
		
		The function $q$ has domain $(-\infty,-5)\cup(-5,\infty)$, so we search 
		for a function that has a vertical asymptote at $-5$. The only candidate 
		is the curve shown in \cref{rat:fig:which1}; note that the curve also goes through $(2,0)$, 
		which is consistent with the formula for $q(x)$, since $q(2)=0$, i.e $q$
		has a zero at $2$.
		
		The function $k$ has domain $(-\infty,-2)\cup(-2,3)\cup(3,\infty)$, and 
		has vertical asymptotes at $-2$ and $3$. This is consistent with 
		the graph in \cref{rat:fig:which3} (and is the only curve that 
		has $3$ branches).
	\end{pccsolution}
\end{pccexample}

We note that each function in \cref{rat:ex:match} behaves like $\frac{1}{x}$ around its vertical asymptotes, 
because each linear factor in each denominator is raised to the power $1$; if (for example) 
the definition of $r$ was instead
\[
	r(x)=\frac{1}{(x-3)^2}
\]
then we would see that $r$ behaves like $\frac{1}{x^2}$ around its vertical asymptote, and 
the graph of $r$ would be very different. We will deal with these cases in the examples that follow.

%===================================
%   Author: Hughes
%   Date:   May 2011
%===================================
\begin{pccexample}[Repeated factors in the denominator]
	Consider the functions $f$, $g$, and $h$ that have formulas
	\[
		f(x)=\frac{x-2}{(x-3)(x+2)}, \qquad g(x)=\frac{x-2}{(x-3)^2(x+2)}, \qquad h(x)=\frac{x-2}{(x-3)(x+2)^2}
	\]
	which are graphed in \cref{rat:fig:repfactd}. Note that each function has $2$ 
	vertical asymptotes, and the domain of each function is 
	\[
		(-\infty,-2)\cup(-2,3)\cup(3,\infty)
	\]
	so we are not surprised to see that each curve has $3$ branches. We also note that 
	the numerator of each function is the same, which tells us that each function has 
	only $1$ zero at $2$.
	
	The functions $g$ and $h$ are different from those that we have considered previously, 
	because they have a repeated factor in the denominator. Notice in particular 
	the way that the functions behave around their asymptotes:
	\begin{itemize}
		\item $f$ behaves like $\frac{1}{x}$ around both of its asymptotes;
		\item $g$ behaves like $\frac{1}{x}$ around $-2$, and like $\frac{1}{x^2}$ around $3$;
		\item $h$ behaves like $\frac{1}{x^2}$ around $-2$, and like $\frac{1}{x}$ around $3$.
	\end{itemize}
\end{pccexample}
\begin{figure}[!htb]
	\setlength{\figurewidth}{0.3\textwidth}
	\begin{subfigure}{\figurewidth}
		\begin{tikzpicture}[/pgf/declare function={f=(x-2)/((x+2)*(x-3));}]
			\begin{axis}[
					%                    framed,
					xmin=-5,xmax=5,
					ymin=-4,ymax=4,
					xtick={-4,-2,...,4},
					ytick={-2,2},
					%                    grid=both,
				]
				\addplot[pccplot] expression[domain=-5:-2.201]{f};
				\addplot[pccplot] expression[domain=-1.802:2.951]{f};
				\addplot[pccplot] expression[domain=3.052:5]{f};
				\addplot[soldot] coordinates{(2,0)};
				%                 \addplot[asymptote,domain=-6:6]({-2},{x});
				%                 \addplot[asymptote,domain=-6:6]({3},{x});
			\end{axis}
		\end{tikzpicture}
		\caption{$y=\dfrac{x-2}{(x+2)(x-3)}$}
		\label{rat:fig:repfactd1}
	\end{subfigure}
	\hfill
	\begin{subfigure}{\figurewidth}
		\begin{tikzpicture}[/pgf/declare function={f=(x-2)/((x+2)*(x-3)^2);}]
			\begin{axis}[
					%                    framed,
					xmin=-5,xmax=5,
					ymin=-4,ymax=4,
					xtick={-4,-2,...,4},
					ytick={-2,2},
					%                    grid=both,
				]
				\addplot[pccplot] expression[domain=-5:-2.039]{f};
				\addplot[pccplot] expression[domain=-1.959:2.796]{f};
				\addplot[pccplot] expression[domain=3.243:5]{f};
				\addplot[soldot] coordinates{(2,0)};
				%                 \addplot[asymptote,domain=-4:4]({-2},{x});
				%                 \addplot[asymptote,domain=-4:4]({3},{x});
			\end{axis}
		\end{tikzpicture}
		\caption{$y=\dfrac{x-2}{(x+2)(x-3)^2}$}
		\label{rat:fig:repfactd2}
	\end{subfigure}
	\hfill
	\begin{subfigure}{\figurewidth}
		\begin{tikzpicture}[/pgf/declare function={f=(x-2)/((x+2)^2*(x-3));}]
			\begin{axis}[
					%                    framed,
					xmin=-5,xmax=5,
					ymin=-4,ymax=4,
					xtick={-4,-2,...,2},
					ytick={-2,2},
					%                    grid=both,
				]
				\addplot[pccplot] expression[domain=-5:-2.451]{f};
				\addplot[pccplot] expression[domain=-1.558:2.990]{f};
				\addplot[pccplot] expression[domain=3.010:6]{f};
				\addplot[soldot] coordinates{(2,0)};
				%                 \addplot[asymptote,domain=-4:4]({-2},{x});
				%                 \addplot[asymptote,domain=-4:4]({3},{x});
			\end{axis}
		\end{tikzpicture}
		\caption{$y=\dfrac{x-2}{(x+2)^2(x-3)}$}
		\label{rat:fig:repfactd3}
	\end{subfigure}
	\caption{}
	\label{rat:fig:repfactd}
\end{figure}

\Cref{rat:def:function} says that the zeros of 
the rational function $r$ that has formula $r(x)=\frac{p(x)}{q(x)}$ are 
the zeros of $p$. Let's explore this a little more.
%===================================
%   Author: Hughes
%   Date:   May 2012
%===================================
\begin{pccexample}[Zeros] 
	Find the zeros of the functions $\alpha$, $\beta$, and $\gamma$ that
	formulas
	\[
		\alpha(x)=\frac{x+5}{3x-7}, \qquad \beta(x)=\frac{9-x}{x+1}, \qquad \gamma(x)=\frac{17x^2-10}{2x+1}
	\]
	\begin{pccsolution}
		We find the zeros of each function in turn by setting the numerator equal to $0$. The zeros of 
		$\alpha$ are found by solving 
		\[
			x+5=0
		\]
		The zero of $\alpha$ is $-5$.
		
		Similarly, we may solve $9-x=0$ to find the zero of $\beta$, which is clearly $9$.
		
		The zeros of $\gamma$ satisfy the equation
		\[
			17x^2-10=0
		\]
		which we can solve using the square root property to obtain
		\[
			x=\pm\frac{10}{17}
		\]
		The zeros of $\gamma$ are $\pm\frac{10}{17}$.
	\end{pccsolution}
\end{pccexample}

\subsection*{Long-run behavior}
Our focus so far has been on the behavior of rational functions around 
their \emph{vertical} asymptotes. In fact, rational functions also 
have interesting long-run behavior around their \emph{horizontal} or 
\emph{oblique} asymptotes. A rational function will always have either 
a horizontal or an oblique asymptote| the case is determined by the degree
of the numerator and the degree of the denominator.
\begin{pccdefinition}[Long-run behavior]\label{rat:def:longrun}
	Let $r$ be the rational function that has formula
	\[
		r(x) = \frac{a_n x^n + a_{n-1}x^{n-1}+\ldots + a_0}{b_m x^m + b_{m-1}x^{m-1}+\ldots+b_0}
	\]
	We can classify the long-run behavior of the rational function $r$ 
	according to the following criteria:
	\begin{itemize}
		\item if $n<m$ then  $r$ has a horizontal asymptote with equation $y=0$;
		\item if $n=m$ then $r$ has a horizontal asymptote with equation $y=\dfrac{a_n}{b_m}$;
		\item if $n>m$ then $r$ will have an oblique asymptote as $x\rightarrow\pm\infty$ (more on this in \cref{rat:sec:oblique})
	\end{itemize} 
\end{pccdefinition}
We will concentrate on functions that have horizontal asymptotes until 
we reach \cref{rat:sec:oblique}.

%===================================
%   Author: Hughes
%   Date:   May 2012
%===================================
\begin{pccexample}[Long-run behavior graphically]\label{rat:ex:horizasymp}
	\pccname{Kebede} has graphed the functions $r$, $s$, and $t$ that 
	have formulas
	\[
		r(x)=\frac{x+1}{x-3}, \qquad s(x)=\frac{2(x+1)}{x-3}, \qquad t(x)=\frac{3(x+1)}{x-3}
	\]
	in his graphing calculator and obtained the curves shown in \cref{rat:fig:horizasymp}. Kebede decides 
	to test his knowledgeable friend \pccname{Oscar}, and asks him 
	to match the formulas to the graphs.
	
	\begin{figure}[!htb]
		\setlength{\figurewidth}{0.3\textwidth}
		\begin{subfigure}{\figurewidth}
			\begin{tikzpicture}[/pgf/declare function={f=2*(x+1)/(x-3);}]
				\begin{axis}[
						framed,
						xmin=-15,xmax=15,
						ymin=-6,ymax=6,
						xtick={-12,-8,...,12},
						minor ytick={-4,-3,...,4},
						grid=both,
					]
					\addplot[pccplot] expression[domain=-15:2]{f};
					\addplot[pccplot] expression[domain=5:15]{f};
					\addplot[soldot] coordinates{(-1,0)};
					\addplot[asymptote,domain=-6:6]({3},{x});
					\addplot[asymptote,domain=-15:15]({x},{2});
				\end{axis}
			\end{tikzpicture}
			\caption{}
			\label{rat:fig:horizasymp1}
		\end{subfigure}
		\hfill
		\begin{subfigure}{\figurewidth}
			\begin{tikzpicture}[/pgf/declare function={f=(x+1)/(x-3);}]
				\begin{axis}[
						framed,
						xmin=-15,xmax=15,
						ymin=-6,ymax=6,
						xtick={-12,-8,...,12},
						minor ytick={-4,-3,...,4},
						grid=both,
					]
					\addplot[pccplot] expression[domain=-15:2.42857,samples=50]{f};
					\addplot[pccplot] expression[domain=3.8:15,samples=50]{f};
					\addplot[soldot] coordinates{(-1,0)};
					\addplot[asymptote,domain=-6:6]({3},{x});
					\addplot[asymptote,domain=-15:15]({x},{1});
				\end{axis}
			\end{tikzpicture}
			\caption{}
			\label{rat:fig:horizasymp2}
		\end{subfigure}
		\hfill
		\begin{subfigure}{\figurewidth}
			\begin{tikzpicture}[/pgf/declare function={f=3*(x+1)/(x-3);}]
				\begin{axis}[
						framed,
						xmin=-15,xmax=15,
						ymin=-6,ymax=6,
						xtick={-12,-8,...,12},
						minor ytick={-4,-3,...,4},
						grid=both,
					]
					\addplot[pccplot] expression[domain=-15:1.6666,samples=50]{f};
					\addplot[pccplot] expression[domain=7:15]{f};
					\addplot[soldot] coordinates{(-1,0)};
					\addplot[asymptote,domain=-6:6]({3},{x});
					\addplot[asymptote,domain=-15:15]({x},{3});
				\end{axis}
			\end{tikzpicture}
			\caption{}
			\label{rat:fig:horizasymp3}
		\end{subfigure}
		\caption{Horizontal asymptotes}
		\label{rat:fig:horizasymp}
	\end{figure}
	
	Oscar notices that each function has a vertical asymptote at $3$ and a zero at $-1$. 
	The main thing that catches Oscar's eye is that each function has a different 
	coefficient in the numerator, and that each curve has a different horizontal asymptote. 
	In particular, Oscar notes that:
	\begin{itemize}
		\item the curve shown in \cref{rat:fig:horizasymp1} has a horizontal asymptote with equation $y=2$;
		\item the curve shown in \cref{rat:fig:horizasymp2} has a horizontal asymptote with equation $y=1$;
		\item the curve shown in \cref{rat:fig:horizasymp3} has a horizontal asymptote with equation $y=3$.
	\end{itemize}
	Oscar is able to tie it all together for Kebede by referencing \cref{rat:def:longrun}. He says 
	that since the degree of the numerator and the degree of the denominator is the same 
	for each of the functions $r$, $s$, and $t$, the horizontal asymptote will be determined 
	by evaluating the ratio of their leading coefficients. 
	
	Oscar therefore says that $r$ should have a horizontal asymptote $y=\frac{1}{1}=1$, $s$ should 
	have a horizontal asymptote $y=\frac{2}{1}=2$, and $t$ should have a horizontal asymptote 
	$y=\frac{3}{1}=3$. Kebede is able to finish the problem from here, and says that $r$ is 
	shown in \cref{rat:fig:horizasymp2}, $s$ is shown in \cref{rat:fig:horizasymp1}, and 
	$t$ is shown in \cref{rat:fig:horizasymp3}.
\end{pccexample}

%===================================
%   Author: Hughes
%   Date:   May 2012
%===================================
\begin{pccexample}[Long-run behavior numerically]
	\pccname{Xiao} and \pccname{Dwayne} saw \cref{rat:ex:horizasymp} but are a little confused 
	about horizontal asymptotes. What does it mean to say that a function $r$ has a horizontal
	asymptote? 
	
	They decide to explore the concept by 
	constructing a table of values for the rational functions $R$ and  $S$ that have formulas
	\[
		R(x)=\frac{-5(x+1)}{x-3}, \qquad S(x)=\frac{7(x-5)}{2(x+1)}
	\]
	In \cref{rat:tab:plusinfty} they model the behavior of $R$ and $S$ as $x\rightarrow\infty$, 
	and in \cref{rat:tab:minusinfty} they model the behavior of $R$ and $S$ as $x\rightarrow-\infty$
	by substituting very large values of $|x|$ into each function.
	\begin{table}[!htb]
		\begin{minipage}{.5\textwidth}
			\centering
			\caption{$R(x)$ and $S(x)$ as $x\rightarrow\infty$}
			\label{rat:tab:plusinfty}
			\begin{tabular}{S*{2}S[table-format=1.5]}
				\beforeheading
				\heading{$x$} & \heading{$R(x)$} & \heading{$S(x)$} \\ \afterheading 
				\num{1e2 }    & -5.20619         & 3.29208          \\\normalline 
				\num{1e3}     & -5.02006         & 3.47902          \\\normalline 
				\num{1e4}     & -5.00200         & 3.49790          \\\normalline 
				\num{1e5}     & -5.00020         & 3.49979          \\\normalline 
				\num{1e6}     & -5.00002         & 3.49998          \\\lastline   
			\end{tabular}
		\end{minipage}%
		\begin{minipage}{.5\textwidth}
			\centering
			\caption{$R(x)$ and $S(x)$ as $x\rightarrow-\infty$}
			\label{rat:tab:minusinfty}
			\begin{tabular}{S*{2}S[table-format=1.5]}
				\beforeheading
				\heading{$x$} & \heading{$R(x)$} & \heading{$S(x)$} \\ \afterheading 
				\num{-1e2}    & -4.80583         & 3.71212          \\\normalline 
				\num{-1e3}    & -4.98006         & 3.52102          \\\normalline 
				\num{-1e4}    & -4.99800         & 3.50210          \\\normalline 
				\num{-1e5}    & -4.99980         & 3.50021          \\\normalline 
				\num{-1e6}    & -4.99998         & 3.50002          \\\lastline   
			\end{tabular}
		\end{minipage}
	\end{table}
	
	Xiao and Dwayne study \cref{rat:tab:plusinfty,rat:tab:minusinfty} and decide that 
	the functions $R$ and $S$ never actually touch their horizontal asymptotes, but they 
	do get infinitely close. They also feel as if they have a better understanding of 
	what it means to study the behavior of a function as $x\rightarrow\pm\infty$.
\end{pccexample}

%===================================
%   Author: Hughes
%   Date:   May 2011
%===================================
\begin{pccexample}[Repeated factors in the numerator]
	Consider the functions $f$, $g$, and $h$ that have formulas
	\[
		f(x)=\frac{(x-2)^2}{(x-3)(x+1)}, \qquad g(x)=\frac{x-2}{(x-3)(x+1)}, \qquad h(x)=\frac{(x-2)^3}{(x-3)(x+1)}
	\]
	which are graphed in \cref{rat:fig:repfactn}. We note that each function has vertical
	asymptotes at $-1$ and $3$, and so the domain of each function is
	\[
		(-\infty,-1)\cup(-1,3)\cup(3,\infty)
	\]
	We also notice that the numerators of each function are quite similar| indeed, each 
	function has a zero at $2$, but how does each function behave around their zero?
	
	Using \cref{rat:fig:repfactn} to guide us, we note that
	\begin{itemize}
		\item $f$ has a horizontal intercept $(2,0)$, but the curve of 
		$f$ does not cut the horizontal axis| it bounces off it;
		\item $g$ also has a horizontal intercept $(2,0)$, and the curve 
		of $g$ \emph{does} cut the horizontal axis;
		\item $h$ has a horizontal intercept $(2,0)$, and the curve of $h$ 
		also cuts the axis, but appears flattened as it does so.
	\end{itemize}
	
	We can further enrich our study by discussing the long-run behavior of each function. 
	Using the tools of \cref{rat:def:longrun}, we can deduce that 
	\begin{itemize}
		\item $f$ has a horizontal asymptote with equation $y=1$;
		\item $g$ has a horizontal asymptote with equation $y=0$;
		\item $h$ does \emph{not} have a horizontal asymptote| it has an oblique asymptote (we'll 
		study this more in \cref{rat:sec:oblique}).
	\end{itemize}
\end{pccexample}

\begin{figure}[!htb]
	\setlength{\figurewidth}{0.3\textwidth}
	\begin{subfigure}{\figurewidth}
		\begin{tikzpicture}[/pgf/declare function={f=(x-2)^2/((x+1)*(x-3));}]
			\begin{axis}[
					%                    framed,
					xmin=-5,xmax=5,
					ymin=-10,ymax=10,
					xtick={-4,-2,...,4},
					ytick={-8,-4,...,8},
					%                    grid=both,
					width=\figurewidth,
				]
				\addplot[pccplot] expression[domain=-5:-1.248,samples=50]{f};
				\addplot[pccplot] expression[domain=-0.794:2.976,samples=50]{f};
				\addplot[pccplot] expression[domain=3.026:5,samples=50]{f};
				\addplot[soldot] coordinates{(2,0)};
				%                 \addplot[asymptote,domain=-6:6]({-1},{x});
				%                 \addplot[asymptote,domain=-6:6]({3},{x});
			\end{axis}
		\end{tikzpicture}
		\caption{$y=\dfrac{(x-2)^2}{(x+1)(x-3)}$}
		\label{rat:fig:repfactn1}
	\end{subfigure}
	\hfill
	\begin{subfigure}{\figurewidth}
		\begin{tikzpicture}[/pgf/declare function={f=(x-2)/((x+1)*(x-3));}]
			\begin{axis}[
					%                    framed,
					xmin=-5,xmax=5,
					ymin=-10,ymax=10,
					xtick={-4,-2,...,4},
					ytick={-8,-4,...,8},
					%                    grid=both,
					width=\figurewidth,
				]
				\addplot[pccplot] expression[domain=-5:-1.075]{f};
				\addplot[pccplot] expression[domain=-0.925:2.975]{f};
				\addplot[pccplot] expression[domain=3.025:5]{f};
				\addplot[soldot] coordinates{(2,0)};
				%                 \addplot[asymptote,domain=-6:6]({-1},{x});
				%                 \addplot[asymptote,domain=-6:6]({3},{x});
			\end{axis}
		\end{tikzpicture}
		\caption{$y=\dfrac{x-2}{(x+1)(x-3)}$}
		\label{rat:fig:repfactn2}
	\end{subfigure}
	\hfill
	\begin{subfigure}{\figurewidth}
		\begin{tikzpicture}[/pgf/declare function={f=(x-2)^3/((x+1)*(x-3));}]
			\begin{axis}[
					%                    framed,
					xmin=-5,xmax=5,
					xtick={-8,-6,...,8},
					%                    grid=both,
					ymin=-30,ymax=30,
					width=\figurewidth,
				]
				\addplot[pccplot] expression[domain=-5:-1.27]{f};
				\addplot[pccplot] expression[domain=-0.806:2.99185]{f};
				\addplot[pccplot] expression[domain=3.0085:5]{f};
				\addplot[soldot] coordinates{(2,0)};
				%                 \addplot[asymptote,domain=-30:30]({-1},{x});
				%                 \addplot[asymptote,domain=-30:30]({3},{x});
			\end{axis}
		\end{tikzpicture}
		\caption{$y=\dfrac{(x-2)^3}{(x+1)(x-3)}$}
		\label{rat:fig:repfactn3}
	\end{subfigure}
	\caption{}
	\label{rat:fig:repfactn}
\end{figure}

\subsection*{Holes}
Rational functions have a vertical asymptote at $a$ if the denominator is $0$ at $a$. 
What happens if the numerator is $0$ at the same place? In this case, we say that the rational 
function has a \emph{hole} at $a$.
\begin{pccdefinition}[Holes]
	The rational function
	\[
		r(x)=\frac{p(x)}{q(x)}
	\]
	has a hole at $a$ if $p(a)=q(a)=0$. Note that holes are different from 
	a vertical asymptotes. We represent that $r$ has a hole at the point 
	$(a,r(a))$ on the curve $y=r(x)$ by 
	using a hollow circle, $\circ$.
\end{pccdefinition}

%===================================
%   Author: Hughes
%   Date:   March 2012
%===================================
\begin{pccexample}
	\pccname{Mohammed} and \pccname{Sue} have graphed the function $r$ that has formula
	\[
		r(x)=\frac{x^2+x-6}{(x-2)}
	\]
	in their calculators, and can not decide if the correct graph 
	is \cref{rat:fig:hole} or \cref{rat:fig:hole1}.
	
	Luckily for them, Oscar is nearby, and can help them settle the debate. 
	Oscar demonstrates that
	\begin{align*}
		r(x) & =\frac{(x+3)(x-2)}{(x-2)} \\ 
		     & = x+3                     
	\end{align*}
	but only when $x\ne 2$, because the function is undefined at $2$. Oscar 
	says that this necessarily means that the domain or $r$ is 
	\[
		(-\infty,2)\cup(2,\infty)
	\]
	and that $r$ must have a hole at $2$. 
	
	Mohammed and Sue are very grateful for the clarification, and conclude that 
	the graph of $r$ is shown in \cref{rat:fig:hole1}.
	\begin{figure}[!htb]
		\begin{minipage}{.45\textwidth}
			\begin{tikzpicture}
				\begin{axis}[
						framed,
						xmin=-10,xmax=10,
						ymin=-10,ymax=10,
						xtick={-8,-4,...,8},
						ytick={-8,-4,...,8},
						grid=both,
					]
					\addplot expression[domain=-10:7]{x+3};
					\addplot[soldot] coordinates{(-3,0)};
				\end{axis}
			\end{tikzpicture}
			\caption{}
			\label{rat:fig:hole}
		\end{minipage}%
		\hfill
		\begin{minipage}{.45\textwidth}
			\begin{tikzpicture}
				\begin{axis}[
						framed,
						xmin=-10,xmax=10,
						ymin=-10,ymax=10,
						xtick={-8,-4,...,8},
						ytick={-8,-4,...,8},
						grid=both,
					]
					\addplot expression[domain=-10:7]{x+3};
					\addplot[holdot] coordinates{(2,5)};
					\addplot[soldot] coordinates{(-3,0)};
				\end{axis}
			\end{tikzpicture}
			\caption{}
			\label{rat:fig:hole1}
		\end{minipage}%
	\end{figure}
\end{pccexample}

%===================================
%   Author: Hughes
%   Date:   May 2011
%===================================
\begin{pccexample}
	Consider the function $f$ that has formula
	\[
		f(x)=\frac{x(x+3)}{x^2-4x}
	\]
	The domain of $f$ is $(-\infty,0)\cup(0,4)\cup(4,\infty)$ because both $0$ and $4$ 
	make the denominator equal to $0$. Notice that
	\begin{align*}
		f(x) & = \frac{x(x+3)}{x(x-4)} \\ 
		     & = \frac{x+3}{x-4}       
	\end{align*}
	provided that $x\ne 0$. Since $0$ makes the numerator 
	and the denominator 0 at the same time, we say that $f$ has a hole at $(0,-\nicefrac{3}{4})$. 
	Note that this necessarily means that $f$ does not have a vertical intercept.
	
	We also note $f$ has a vertical asymptote at $4$; the function is graphed in \cref{rat:fig:holeex}.
	\begin{figure}[!htb]
		\centering
		\begin{tikzpicture}[/pgf/declare function={f=(x+3)/(x-4);}]
			\begin{axis}[
					framed,
					xmin=-10,xmax=10,
					ymin=-10,ymax=10,
					xtick={-8,-6,...,8},
					ytick={-8,-6,...,8},
					grid=both,
				]
				\addplot[pccplot] expression[domain=-10:3.36364,samples=50]{f};
				\addplot[pccplot] expression[domain=4.77:10]{f};
				\addplot[asymptote,domain=-10:10]({4},{x});
				\addplot[holdot]coordinates{(0,-0.75)};
				\addplot[soldot] coordinates{(-3,0)};
			\end{axis}
		\end{tikzpicture}
		\caption{$y=\dfrac{x(x+3)}{x^2-4x}$}
		\label{rat:fig:holeex}
	\end{figure}
\end{pccexample}



%===================================
%   Author: Hughes
%   Date:   March 2012
%===================================
\begin{pccexample}[Minimums and maximums]
	\pccname{Seamus} and \pccname{Trang} are discussing rational functions. Seamus says that 
	if a rational function has a vertical asymptote, then it can 
	not possibly have local minimums and maximums, nor can it have 
	global minimums and maximums.
	
	Trang says this statement is not always true. She plots the functions 
	$f$ and $g$ that have formulas
	\[
		f(x)=-\frac{32(x-1)(x+1)}{(x-2)^2(x+2)^2}, \qquad g(x)=\frac{32(x-1)(x+1)}{(x-2)^2(x+2)^2}
	\]
	in \cref{rat:fig:minmax1,rat:fig:minmax2} and shows them to Seamus. On seeing the graphs, 
	Seamus quickly corrects himself, and says  that $f$ has a local (and global) 
	maximum of $2$ at $0$, and that $g$ has a local (and global) minimum of $-2$ at $0$.
	
	\begin{figure}[!htb]
		\begin{minipage}{.45\textwidth}
			\begin{tikzpicture}[/pgf/declare function={f=-32*(x-1)*(x+1)/(( x-2)^2*(x+2)^2);}]
				\begin{axis}[
						framed,
						xmin=-10,xmax=10,
						ymin=-10,ymax=10,
						xtick={-8,-6,...,8},
						ytick={-8,-6,...,8},
						grid=both,
					]
					\addplot[pccplot] expression[domain=-10:-3.01]{f};
					\addplot[pccplot] expression[domain=-1.45:1.45]{f};
					\addplot[pccplot] expression[domain=3.01:10]{f};
					\addplot[soldot] coordinates{(-1,0)(1,0)};
				\end{axis}
			\end{tikzpicture}
			\caption{$y=f(x)$}
			\label{rat:fig:minmax1}
		\end{minipage}%
		\hfill
		\begin{minipage}{.45\textwidth}
			\begin{tikzpicture}[/pgf/declare function={f=32*(x-1)*(x+1)/(( x-2)^2*(x+2)^2);}]
				\begin{axis}[
						framed,
						xmin=-10,xmax=10,
						ymin=-10,ymax=10,
						xtick={-8,-6,...,8},
						ytick={-8,-6,...,8},
						grid=both,
					]
					\addplot[pccplot] expression[domain=-10:-3.01]{f};
					\addplot[pccplot] expression[domain=-1.45:1.45]{f};
					\addplot[pccplot] expression[domain=3.01:10]{f};
					\addplot[soldot] coordinates{(-1,0)(1,0)};
				\end{axis}
			\end{tikzpicture}
			\caption{$y=g(x)$}
			\label{rat:fig:minmax2}
		\end{minipage}%
	\end{figure}
	
	Seamus also notes that (in its domain) the function $f$ is always concave down, and 
	that (in its domain) the function $g$ is always concave up. Furthermore, Trang
	observes that each function behaves like $\frac{1}{x^2}$ around each of its vertical 
	asymptotes, because each linear factor in the denominator is raised to the power $2$.
	
	\pccname{Oscar} stops by and reminds both students about the long-run behavior; according 
	to \cref{rat:def:longrun} since the degree of the denominator is greater than the
	degree of the numerator (in both functions), each function has a horizontal asymptote
	at $y=0$.
\end{pccexample}


\investigation*{}
%===================================
%   Author: Pettit/Hughes
%   Date:   March 2012
%===================================
\begin{problem}[The spaghetti incident]
The same Queen from \vref{exp:prob:queenschessboard} has recovered from 
the rice experiments, and has called her loyal jester for another challenge.

The jester has an $11-$\si{\inch} piece of uncooked spaghetti that he puts on a table; 
he uses a book to cover $\SI{1}{\inch}$ of it so that 
$\SI{10}{\inch}$ hang over the edge. The jester then produces a box of $\si{\milli\gram}$
weights that can be hung from the spaghetti.

The jester says it will take $y\si{\milli\gram}$ to break the spaghetti when hung
$x\si{\inch}$ from the edge, according to the rule $y=\frac{100}{x}$.
\begin{margintable}
	\centering
	\captionof{table}{}
	\label{rat:tab:spaghetti}
	\begin{tabular}{S[table-format=2]c}
		\beforeheading
		\heading{$x$} & \heading{$y$} \\ 
		\afterheading
		1             &               \\\normalline    
		2             &               \\\normalline    
		3             &               \\\normalline    
		4             &               \\\normalline    
		5             &               \\\normalline    
		6             &               \\\normalline    
		7             &               \\\normalline    
		8             &               \\\normalline    
		9             &               \\\normalline    
		10            &               \\\lastline      
	\end{tabular}
\end{margintable}
\begin{subproblem}\label{rat:prob:spaggt1}
	Help the Queen complete \cref{rat:tab:spaghetti}, and use $2$ digits after the decimal
	where appropriate.
	\begin{shortsolution}
		\begin{tabular}[t]{S[table-format=2]S[table-format=3.2]}
			\beforeheading
			\heading{$x$} & \heading{$y$} \\    
			\afterheading
			1             & 100           \\\normalline  
			2             & 50            \\\normalline    
			3             & 33.33         \\\normalline 
			4             & 25            \\\normalline    
			5             & 20            \\\normalline    
			6             & 16.67         \\\normalline 
			7             & 14.29         \\\normalline  
			8             & 12.50         \\\normalline 
			9             & 11.11         \\\normalline 
			10            & 10            \\\lastline     
		\end{tabular}
	\end{shortsolution}
\end{subproblem}
\begin{subproblem}
	What do you notice about the number of $\si{\milli\gram}$ that it takes to break 
	the spaghetti as $x$ increases?
	\begin{shortsolution}
		It seems that the number of $\si{\milli\gram}$ that it takes to break the spaghetti decreases
		as $x$ increases.
	\end{shortsolution}
\end{subproblem}
\begin{subproblem}\label{rat:prob:spaglt1}
	The Queen wonders what happens when $x$ gets very small| help the Queen construct 
	a table of values for $x$ and $y$ when $x=0.0001, 0.001, 0.01, 0.1, 0.5, 1$.
	\begin{shortsolution}
		\begin{tabular}[t]{S[table-format=1.4]S[table-format=7.0]}
			\beforeheading
			\heading{$x$} & \heading{$y$} \\        
			\afterheading
			0.0001        & 1000000       \\\normalline 
			0.001         & 100000        \\\normalline  
			0.01          & 10000         \\\normalline   
			0.1           & 1000          \\\normalline    
			0.5           & 200           \\\normalline     
			1             & 100           \\\lastline      
		\end{tabular}
	\end{shortsolution}
\end{subproblem}
\begin{subproblem}
	What do you notice about the number of $\si{\milli\gram}$ that it takes to break the spaghetti
	as $x\rightarrow 0$? Would it ever make sense to let $x=0$?
	\begin{shortsolution}
		The number of $\si{\milli\gram}$ required to break the spaghetti increases as $x\rightarrow 0$.
		We can not allow $x$ to be $0$, as we can not divide by $0$, and we can not 
		be $0$ inches from the edge of the table.
	\end{shortsolution}
\end{subproblem}
\begin{subproblem}
	Plot your results from \cref{rat:prob:spaggt1,rat:prob:spaglt1} on the same graph, 
	and join the points using a smooth curve| set the maximum value of $y$ as $200$, and 
	note that this necessarily means that you will not be able to plot all of the points.
	\begin{shortsolution}
		The graph of $y=\frac{100}{x}$ is shown below.
				
		\begin{tikzpicture}
			\begin{axis}[
					framed,
					xmin=-2,xmax=11,
					ymin=-20,ymax=200,
					xtick={2,4,...,10},
					ytick={20,40,...,180},
					grid=major,
				]
				\addplot+[-] expression[domain=0.5:10]{100/x};
				\addplot[soldot] coordinates{(0.5,200)(1,100)(2,50)(3,33.33)
				(4,25)(5,20)(16.67)(7,14.29)(8,12.50)(9,11.11)(10,10)};
			\end{axis}
		\end{tikzpicture}
	\end{shortsolution}
\end{subproblem}
\begin{subproblem}
	Using your graph, observe what happens to $y$ as $x$ increases. If we could somehow
	construct a piece of uncooked spaghetti that was $\SI{101}{\inch}$ long, how many 
	$\si{\milli\gram}$ would it take to break the spaghetti?
	\begin{shortsolution}
		As $x$ increases, $y\rightarrow 0$. If we could construct a piece of spaghetti 
		$\SI{101}{\inch}$ long, it would only take $\SI{1}{\milli\gram}$ to break it $\left(\frac{100}{100}=1\right)$. Of course, 
		the weight of spaghetti would probably cause it to break without the weight.
	\end{shortsolution}
\end{subproblem}
The Queen looks forward to more food-related investigations from her jester.
\end{problem}



%===================================
%   Author: Adams (Hughes)
%   Date:   March 2012
%===================================
\begin{problem}[Debt Amortization]
To amortize a debt means to pay it off in a given length of time using 
equal periodic payments. The payments include interest on the unpaid 
balance. The following formula gives the monthly payment, $M$, in dollars
that is necessary to amortize a debt of $P$ dollars in $n$ months 
at a monthly interest rate of $i$
\[
	M=\frac{P\cdot i}{1-(1+i)^{-n}}
\]
Use this formula in each of the following problems.
\begin{subproblem}
	What monthly payments are necessary on a credit card debt of \$2000 at 
	$\SI{1.5}{\percent}$ monthly if you want to pay off the debt in $2$ years?
	In one year? How much money will you save by paying off the debt in the
	shorter amount of time?
	\begin{shortsolution}
		Paying off the debt in $2$ years, we use
		\begin{align*}
			M & = \frac{2000\cdot 0.015}{1-(1+0.015)^{-24}} \\ 
			  & \approx 99.85                               
		\end{align*}
		The monthly payments are \$99.85.
				
		Paying off the debt in $1$ year, we use
		\begin{align*}
			M & = \frac{2000\cdot 0.015}{1-(1+0.015)^{-12}} \\ 
			  & \approx 183.36                              
		\end{align*}
		The monthly payments are \$183.36
				
		In the $2$-year model we would pay a total of $\$99.85\cdot 12=\$2396.40$. In the
		$1$-year model we would pay a total of $\$183.36\cdot 12=\$2200.32$. We would therefore
		save $\$196.08$ if we went with the $1$-year model instead of the $2$-year model.
	\end{shortsolution}
\end{subproblem}
\begin{subproblem}
	To purchase a home, a family needs a loan of \$300,000 at $\SI{5.2}{\percent}$ 
	annual interest.  Compare a $20$ year loan to a $30$ year loan and make 
	a recommendation for the family.
	(Note: when given an annual interest rate, it is a common business practice to divide by
	$12$ to get a monthly rate.)
	\begin{shortsolution}
		For the $20$-year loan we use
		\begin{align*}
			M & = \frac{300000\cdot \frac{0.052}{12}}{1-\left( 1+\frac{0.052}{12} \right)^{-12\cdot 20}} \\ 
			  & \approx 2013.16                                                                          
		\end{align*}
		The monthly payments are \$2013.16.
				
		For the $30$-year loan we use
		\begin{align*}
			M & = \frac{300000\cdot \frac{0.052}{12}}{1-\left( 1+\frac{0.052}{12} \right)^{-12\cdot 30}} \\ 
			  & \approx 1647.33                                                                          
		\end{align*}
		The monthly payments are \$1647.33.
				
		The total amount paid during the $20$-year loan is $\$2013.16\cdot 12\cdot 20=\$483,158.40$. 
		The total amount paid during the $30$-year loan is $\$1647.33\cdot 12\cdot 30=\$593,038.80$.
				
		Recommendation: if you can afford the payments, choose the $20$-year loan.
	\end{shortsolution}
\end{subproblem}
\begin{subproblem}
	\pccname{Ellen} wants to make monthly payments of \$100 to pay off a debt of \$3000 
	at \SI{12}{\percent} annual interest. How long will it take her to pay off the 
	debt?
	\begin{shortsolution}
		We are given $M=100$, $P=3000$, $i=0.01$, and we need to find $n$
		in the equation
		\[
			100 = \frac{3000\cdot 0.01}{1-(1+0.01)^{-n}}
		\]
		Using logarithms, we find that $n\approx 36$. It will take 
		Ellen about $3$ years to pay off the debt.
	\end{shortsolution}
\end{subproblem}
\begin{subproblem}
	\pccname{Jake} is going to buy a new car. He puts \$2000 down and wants to finance the
	remaining \$14,000. The dealer will offer him \SI{4}{\percent} annual interest for 
	$5$ years, or a \$2000
	rebate which he can use to reduce the amount of the loan and \SI{8}{\percent} 
	annual interest for 5 years. Which should he choose?
	\begin{shortsolution}
		\begin{description}
			\item[Option 1:] $\SI{4}{\percent}$ annual interest for $5$ years on \$14,000.
			This means that the monthly payments will be calculated using
			\begin{align*}
				M & = \frac{14000\cdot \frac{0.04}{12}}{1-\left( 1+\frac{0.04}{12} \right)^{-12\cdot 5}} \\ 
				  & \approx 257.83                                                                       
			\end{align*}
			The monthly payments will be $\$257.83$. The total amount paid will be
			$\$257.83\cdot 5\cdot 12=\$15,469.80$, of which $\$1469.80$ is interest.
			\item[Option 2:] $\SI{8}{\percent}$ annual interest for $5$ years on \$12,000.
			This means that the monthly payments will be calculated using
			\begin{align*}
				M & = \frac{12000\cdot \frac{0.08}{12}}{1-\left( 1+\frac{0.08}{12} \right)^{-12\cdot 5}} \\ 
				  & \approx 243.32                                                                       
			\end{align*}
			The monthly payments will be $\$243.32$. The total amount paid 
			will be $\$243.32\cdot 5\cdot 12 =\$14,599.20$, of which $\$2599.2$ is 
			interest.
		\end{description}
		Jake should choose option 1 to minimize the amount of interest 
		he has to pay.
	\end{shortsolution}
\end{subproblem}
\end{problem}

\begin{exercises}
%===================================
%   Author: Hughes
%   Date:   March 2012
%===================================
\begin{problem}[Rational or not]
Decide if the following formulas correspond to rational functions or not; 
if the function is rational, state its domain.
\begin{multicols}{3}
	\begin{subproblem}
		$r(x)=\dfrac{3}{x}$    
		\begin{shortsolution}
			$r$ is rational; the domain of $r$ is $(-\infty,0)\cup (0,\infty)$.
		\end{shortsolution}
	\end{subproblem}
	\begin{subproblem}
		$s(y)=\dfrac{y}{6}$    
		\begin{shortsolution}
			$s$ is not rational ($s$ is linear). 
		\end{shortsolution}
	\end{subproblem}
	\begin{subproblem}
		$t(z)=\dfrac{4-x}{7-8z}$    
		\begin{shortsolution}
			$t$ is rational; the domain of $t$ is $\left( -\infty,\dfrac{7}{8} \right)\cup \left( \dfrac{7}{8},\infty \right)$.
		\end{shortsolution}
	\end{subproblem}
	\begin{subproblem}
		$u(w)=\dfrac{w^2}{(w-3)(w+4)}$ 
		\begin{shortsolution}
			$u$ is rational; the domain of $w$ is $(-\infty,-4)\cup(-4,3)\cup(3,\infty)$. 
		\end{shortsolution}
	\end{subproblem}
	\begin{subproblem}
		$v(x)=\dfrac{4}{(x-2)^2}$ 
		\begin{shortsolution}
			$v$ is rational; the domain of $v$ is $(-\infty,2)\cup(2,\infty)$. 
		\end{shortsolution}
	\end{subproblem}
	\begin{subproblem}
		$w(x)=\dfrac{9-x}{x+17}$ 
		\begin{shortsolution}
			$w$ is rational; the domain of $w$ is $(-\infty,-17)\cup(-17,\infty)$. 
		\end{shortsolution}
	\end{subproblem}
	\begin{subproblem}
		$a(x)=x^2+4$ 
		\begin{shortsolution}
			$a$ is not rational ($a$ is quadratic, or a polynomial of degree $2$).
		\end{shortsolution}
	\end{subproblem}
	\begin{subproblem}
		$b(y)=3^y$
		\begin{shortsolution}
			$b$ is not rational ($b$ is exponential).
		\end{shortsolution}
	\end{subproblem}
	\begin{subproblem}
		$c(z)=\dfrac{z^2}{z^3}$ 
		\begin{shortsolution}
			$c$ is rational; the domain of $c$ is $(-\infty,0)\cup (0,\infty)$.
		\end{shortsolution}
	\end{subproblem}
	\begin{subproblem}
		$d(x)=x^2(x+3)(5x-7)$ 
		\begin{shortsolution}
			$d$ is not rational ($d$ is a polynomial).
		\end{shortsolution}
	\end{subproblem}
	\begin{subproblem}
		$e(\alpha)=\dfrac{\alpha^2}{\alpha^2-1}$ 
		\begin{shortsolution}
			$e$ is rational; the domain of $e$ is $(-\infty,-1)\cup(-1,1)\cup(1,\infty)$.
		\end{shortsolution}
	\end{subproblem}
	\begin{subproblem}
		$f(\beta)=\dfrac{3}{4}$ 
		\begin{shortsolution}
			$f$ is not rational ($f$ is constant).
		\end{shortsolution}
	\end{subproblem}
\end{multicols}
\end{problem}
%===================================
%   Author: Hughes
%   Date:   March 2012
%===================================
\begin{problem}[Function evaluation]
Let $r$ be the function that has formula 
\[
	r(x)=\frac{(x-2)(x+3)}{(x+5)(x-7)}
\]
Evaluate each of the following (if possible); if the value is undefined, 
then state so.
\begin{multicols}{4}
	\begin{subproblem}
		$r(0)$ 
		\begin{shortsolution}
			$\begin{aligned}[t]
				r(0) & =\frac{(0-2)(0+3)}{(0+5)(0-7)} \\ 
				     & =\frac{-6}{-35}                \\                
				     & =\frac{6}{35}                  
			\end{aligned}$
		\end{shortsolution}
	\end{subproblem}
	\begin{subproblem}
		$r(1)$ 
		\begin{shortsolution}
			$\begin{aligned}[t]
				r(1) & =\frac{(1-2)(1+3)}{(1+5)(1-7)} \\ 
				     & =\frac{-4}{-36}                \\                
				     & =\frac{1}{9}                   
			\end{aligned}$
		\end{shortsolution}
	\end{subproblem}
	\begin{subproblem}
		$r(2)$ 
		\begin{shortsolution}
			$\begin{aligned}[t]
				r(2) & =\frac{(2-2)(2+3)}{(2+5)(2-7)} \\ 
				     & = \frac{0}{-50}                \\                
				     & =0                             
			\end{aligned}$
		\end{shortsolution}
	\end{subproblem}
	\begin{subproblem}
		$r(4)$ 
		\begin{shortsolution}
			$\begin{aligned}[t]
				r(4) & =\frac{(4-2)(4+3)}{(4+5)(4-7)} \\ 
				     & =\frac{14}{-27}                \\                
				     & =-\frac{14}{27}                
			\end{aligned}$
		\end{shortsolution}
	\end{subproblem}
	\begin{subproblem}
		$r(7)$ 
		\begin{shortsolution}
			$\begin{aligned}[t]
				r(7) & =\frac{(7-2)(7+3)}{(7+5)(7-7)} \\ 
				     & =\frac{50}{0}                  
			\end{aligned}$
						
			$r(7)$ is undefined.
		\end{shortsolution}
	\end{subproblem}
	\begin{subproblem}
		$r(-3)$ 
		\begin{shortsolution}
			$\begin{aligned}[t]
				r(-3) & =\frac{(-3-2)(-3+3)}{(-3+5)(-3-7)} \\ 
				      & =\frac{0}{-20}                     \\                     
				      & =0                                 
			\end{aligned}$
		\end{shortsolution}
	\end{subproblem}
	\begin{subproblem}
		$r(-5)$ 
		\begin{shortsolution}
			$\begin{aligned}[t]
				r(-5) & =\frac{(-5-2)(-5+3)}{(-5+5)(-5-7)} \\ 
				      & =\frac{14}{0}                      
			\end{aligned}$
						
			$r(-5)$ is undefined.
		\end{shortsolution}
	\end{subproblem}
	\begin{subproblem}
		$r\left( \frac{1}{2} \right)$
		\begin{shortsolution}
			$\begin{aligned}[t]
				r\left( \frac{1}{2} \right) & = \frac{\left( \frac{1}{2}-2 \right)\left( \frac{1}{2}+3 \right)}{\left( \frac{1}{2}+5 \right)\left( \frac{1}{2}-7 \right)} \\ 
				                            & =\frac{-\frac{3}{2}\cdot\frac{7}{2}}{\frac{11}{2}\left( -\frac{13}{2} \right)}                                              \\                                              
				                            & =\frac{-\frac{21}{4}}{-\frac{143}{4}}                                                                                       \\                                                                                       
				                            & =\frac{37}{143}                                                                                                             
			\end{aligned}$
		\end{shortsolution}
	\end{subproblem}
\end{multicols}
\end{problem}
%===================================
%   Author: Hughes
%   Date:   March 2012
%===================================
\begin{problem}[Holes or asymptotes?]
State the domain of each of the rational functions implied by the 
following formulas. Identify any holes or asymptotes.
\begin{multicols}{3}
	\begin{subproblem}
		$f(x)=\dfrac{12}{x-2}$ 
		\begin{shortsolution}
			$f$ has a vertical asymptote at $2$; the domain of $f$ is $(-\infty,2)\cup (2,\infty)$. 
		\end{shortsolution}
	\end{subproblem}
	\begin{subproblem}
		$g(x)=\dfrac{x^2+x}{(x+1)(x-2)}$ 
		\begin{shortsolution}
			$g$ has a vertical asymptote at $2$, and a hole at $-1$; the domain of $g$ is $(-\infty,-1)\cup(-1,2)\cup(2,\infty)$. 
		\end{shortsolution}
	\end{subproblem}
	\begin{subproblem}
		$h(x)=\dfrac{x^2+5x+4}{x^2+x-12}$ 
		\begin{shortsolution}
			$h$ has a vertical asymptote at $3$, and a whole at $-4$; the domain of $h$ is $(-\infty,-4)\cup(-4,3)\cup(3,\infty)$. 
		\end{shortsolution}
	\end{subproblem}
	\begin{subproblem}
		$k(z)=\dfrac{z+2}{2z-3}$ 
		\begin{shortsolution}
			$k$ has a vertical asymptote at $\dfrac{3}{2}$; the domain of $k$ is $\left( -\infty,\dfrac{3}{2} \right)\cup\left( \dfrac{3}{2},\infty \right)$. 
		\end{shortsolution}
	\end{subproblem}
	\begin{subproblem}
		$l(w)=\dfrac{w}{w^2+1}$ 
		\begin{shortsolution}
			$l$ does not have any vertical asymptotes nor holes; the domain of $w$ is $(-\infty,\infty)$. 
		\end{shortsolution}
	\end{subproblem}
	\begin{subproblem}
		$m(t)=\dfrac{14}{13-t^2}$ 
		\begin{shortsolution}
			$m$ has vertical asymptotes at $\pm\sqrt{13}$; the domain of $m$ is $(-\infty,\sqrt{13})\cup(-\sqrt{13},\sqrt{13})\cup(\sqrt{13},\infty)$.
		\end{shortsolution}
	\end{subproblem}
\end{multicols}
\end{problem}

%===================================
%   Author: Hughes
%   Date:   May 2011
%===================================
\begin{problem}[Find a formula from a graph]
Consider the rational functions graphed in \cref{rat:fig:findformula}. Find 
the vertical asymptotes for each function, together with any zeros, and 
give a possible formula for each. 
\begin{shortsolution}
	\begin{itemize}
		\item \Vref{rat:fig:formula1}: possible formula is $r(x)=\dfrac{1}{x+5}$ 
		\item \Vref{rat:fig:formula2}: possible formula is $r(x)=\dfrac{(x+3)}{(x-5)}$
		\item \Vref{rat:fig:formula3}: possible formula is $r(x)=\dfrac{1}{(x-4)(x+3)}$.
	\end{itemize}
\end{shortsolution}
\end{problem}

\begin{figure}[!htb]
	\begin{widepage}
	\setlength{\figurewidth}{0.3\textwidth}
	\begin{subfigure}{\figurewidth}
		\begin{tikzpicture}[/pgf/declare function={f=1/(x+4);}]
			\begin{axis}[
					framed,
					xmin=-10,xmax=10,
					ymin=-6,ymax=6,
					xtick={-8,-6,...,8},
					minor ytick={-4,-3,...,4},
					grid=both,
				]
				\addplot[pccplot] expression[domain=-10:-4.16667,samples=50]{f};
				\addplot[pccplot] expression[domain=-3.83333:10,samples=50]{f};
				\addplot[asymptote,domain=-6:6]({-4},{x});
			\end{axis}
		\end{tikzpicture}
		\caption{}
		\label{rat:fig:formula1}
	\end{subfigure}
	\hfill
	\begin{subfigure}{\figurewidth}
		\begin{tikzpicture}[/pgf/declare function={f=(x+3)/(x-5);}]
			\begin{axis}[
					framed,
					xmin=-10,xmax=10,
					ymin=-6,ymax=6,
					xtick={-8,-6,...,8},
					minor ytick={-4,-3,...,4},
					grid=both,
				]
				\addplot[pccplot] expression[domain=-10:3.85714]{f};
				\addplot[pccplot] expression[domain=6.6:10]{f};
				\addplot[soldot] coordinates{(-3,0)};
				\addplot[asymptote,domain=-6:6]({5},{x});
				\addplot[asymptote,domain=-10:10]({x},{1});
			\end{axis}
		\end{tikzpicture}
		\caption{}
		\label{rat:fig:formula2}
	\end{subfigure}
	\hfill
	\begin{subfigure}{\figurewidth}
		\begin{tikzpicture}[/pgf/declare function={f=1/((x-4)*(x+3));}]
			\begin{axis}[
					framed,
					xmin=-10,xmax=10,
					ymin=-3,ymax=3,
					xtick={-8,-6,...,8},
					minor ytick={-4,-3,...,4},
					grid=both,
				]
				\addplot[pccplot] expression[domain=-10:-3.0473]{f};
				\addplot[pccplot] expression[domain=-2.95205:3.95205]{f};
				\addplot[pccplot] expression[domain=4.0473:10]{f};
				\addplot[asymptote,domain=-3:3]({-3},{x});
				\addplot[asymptote,domain=-3:3]({4},{x});
				\addplot[asymptote,domain=-10:10]({x},{0});
			\end{axis}
		\end{tikzpicture}
		\caption{}
		\label{rat:fig:formula3}
	\end{subfigure}
	\caption{}
	\label{rat:fig:findformula}
	\end{widepage}
\end{figure}

%===================================
%   Author: Hughes
%   Date:   May 2011
%===================================
\begin{problem}[Find a formula from a description]
In each of the following problems, give a formula of a rational 
function that has the listed properties.
\begin{subproblem}
	Vertical asymptote at $2$.
	\begin{shortsolution}
		Possible option: $r(x)=\dfrac{1}{x-2}$. Note that we could multiply the 
		numerator or denominator by any real number and still have the desired properties.
	\end{shortsolution}
\end{subproblem}
\begin{subproblem}
	Vertical asymptote at $5$.
	\begin{shortsolution}
		Possible option: $r(x)=\dfrac{1}{x-5}$. Note that we could multiply the 
		numerator or denominator by any real number and still have the desired properties.
	\end{shortsolution}
\end{subproblem}
\begin{subproblem}
	Vertical asymptote at $-2$, and zero at $6$.
	\begin{shortsolution}
		Possible option: $r(x)=\dfrac{x-6}{x+2}$. Note that we could multiply the 
		numerator or denominator by any real number and still have the desired properties.
	\end{shortsolution}
\end{subproblem}
\begin{subproblem}
	Zeros at $2$ and $-5$ and vertical asymptotes at $1$ and $-7$.
	\begin{shortsolution}
		Possible option: $r(x)=\dfrac{(x-2)(x+5)}{(x-1)(x+7)}$. Note that we could multiply the 
		numerator or denominator by any real number and still have the desired properties.
	\end{shortsolution}
\end{subproblem}
\end{problem}

%===================================
%   Author: Hughes
%   Date:   May 2011
%===================================
\begin{problem}[Given formula, find horizontal asymptotes]
Each of the rational functions implied by the following formulas has a 
horizontal asymptote. Write the equation of the horizontal asymptote for each function.
\begin{multicols}{3}
	\begin{subproblem}
		$f(x) = \dfrac{1}{x}$
		\begin{shortsolution}
			$y=0$
		\end{shortsolution}
	\end{subproblem}
	\begin{subproblem}
		$g(x) = \dfrac{2x+3}{x}$
		\begin{shortsolution}
			$y=2$
		\end{shortsolution}
	\end{subproblem}
	\begin{subproblem}
		$h(x) = \dfrac{x^2+2x}{x^2+3}$
		\begin{shortsolution}
			$y=1$
		\end{shortsolution}
	\end{subproblem}
	\begin{subproblem}
		$k(x) = \dfrac{x^2+7}{x}$
		\begin{shortsolution}
			$y=1$
		\end{shortsolution}
	\end{subproblem}
	\begin{subproblem}
		$l(x)=\dfrac{3x-2}{5x+8}$ 
		\begin{shortsolution}
			$y=\dfrac{3}{5}$
		\end{shortsolution}
	\end{subproblem}
	\begin{subproblem}
		$m(x)=\dfrac{3x-2}{5x^2+8}$ 
		\begin{shortsolution}
			$y=0$
		\end{shortsolution}
	\end{subproblem}
	\begin{subproblem}
		$n(x)=\dfrac{(6x+1)(x-7)}{(11x-8)(x-5)}$ 
		\begin{shortsolution}
			$y=\dfrac{6}{11}$
		\end{shortsolution}
	\end{subproblem}
	\begin{subproblem}
		$p(x)=\dfrac{19x^3}{5-x^4}$ 
		\begin{shortsolution}
			$y=0$
		\end{shortsolution}
	\end{subproblem}
	\begin{subproblem}
		$q(x)=\dfrac{14x^2+x}{1-7x^2}$ 
		\begin{shortsolution}
			$y=-2$
		\end{shortsolution}
	\end{subproblem}
\end{multicols}
\end{problem}

%===================================
%   Author: Hughes
%   Date:   May 2012
%===================================
\begin{problem}[Given horizontal asymptotes, find formula]
In each of the following problems, give a formula for a rational function that 
has the given horizontal asymptote. Note that there may be more than one option.
\begin{multicols}{4}
	\begin{subproblem}
		$y=7$ 
		\begin{shortsolution}
			Possible option: $f(x)=\dfrac{7(x-2)}{x+1}$. Note that there
			are other options, provided that the degree of the numerator is the same as the degree
			of the denominator, and that the ratio of the leading 
			coefficients is $7$.
		\end{shortsolution}
	\end{subproblem}
	\begin{subproblem}
		$y=-1$ 
		\begin{shortsolution}
			Possible option: $f(x)=\dfrac{5-x^2}{x^2+10}$. Note that there
			are other options, provided that the degree of the numerator is the same as the degree
			of the denominator, and that the ratio of the leading 
			coefficients is $10$.
		\end{shortsolution}
	\end{subproblem}
	\begin{subproblem}
		$y=53$ 
		\begin{shortsolution}
			Possible option: $f(x)=\dfrac{53x^3}{x^3+4x^2-7}$. Note that there
			are other options, provided that the degree of the numerator is the same as the degree
			of the denominator, and that the ratio of the leading 
			coefficients is $53$.
		\end{shortsolution}
	\end{subproblem}
	\begin{subproblem}
		$y=-17$ 
		\begin{shortsolution}
			Possible option: $f(x)=\dfrac{34(x+2)}{7-2x}$. Note that there
			are other options, provided that the degree of the numerator is the same as the degree
			of the denominator, and that the ratio of the leading 
			coefficients is $-17$.
		\end{shortsolution}
	\end{subproblem}
	\begin{subproblem}
		$y=\dfrac{3}{2}$ 
		\begin{shortsolution}
			Possible option: $f(x)=\dfrac{3x+4}{2(x+1)}$. Note that there
			are other options, provided that the degree of the numerator is the same as the degree
			of the denominator, and that the ratio of the leading 
			coefficients is $\dfrac{3}{2}$.
		\end{shortsolution}
	\end{subproblem}
	\begin{subproblem}
		$y=0$ 
		\begin{shortsolution}
			Possible option: $f(x)=\dfrac{4}{x}$. Note that there
			are other options, provided that the degree of the numerator is less than the degree
			of the denominator.
		\end{shortsolution}
	\end{subproblem}
	\begin{subproblem}
		$y=-1$ 
		\begin{shortsolution}
			Possible option: $f(x)=\dfrac{10x}{5-10x}$. Note that there
			are other options, provided that the degree of the numerator is the same as the degree
			of the denominator, and that the ratio of the leading 
			coefficients is $-1$.
		\end{shortsolution}
	\end{subproblem}
	\begin{subproblem}
		$y=2$ 
		\begin{shortsolution}
			Possible option: $f(x)=\dfrac{8x-3}{4x+1}$. Note that there
			are other options, provided that the degree of the numerator is the same as the degree
			of the denominator, and that the ratio of the leading 
			coefficients is $2$.
		\end{shortsolution}
	\end{subproblem}
\end{multicols}
\end{problem}

%===================================
%   Author: Hughes
%   Date:   May 2011
%===================================
\begin{problem}[Find a formula from a description]
In each of the following problems, give a formula for a rational function that 
has the prescribed properties. Note that there may be more than one option.
\begin{subproblem}
	$f(x)\rightarrow 3$ as $x\rightarrow\pm\infty$.
	\begin{shortsolution}
		Possible option: $f(x)=\dfrac{3(x-2)}{x+7}$. Note that 
		the zero and asymptote of $f$ could be changed, and $f$ would still have the desired properties.
	\end{shortsolution}
\end{subproblem}
\begin{subproblem}
	$r(x)\rightarrow -4$ as $x\rightarrow\pm\infty$.
	\begin{shortsolution}
		Possible option: $r(x)=\dfrac{-4(x-2)}{x+7}$. Note that 
		the zero and asymptote of $r$ could be changed, and $r$ would still have the desired properties.
	\end{shortsolution}
\end{subproblem}
\begin{subproblem}
	$k(x)\rightarrow 2$ as $x\rightarrow\pm\infty$, and $k$ has vertical asymptotes at $-3$ and $5$.
	\begin{shortsolution}
		Possible option: $k(x)=\dfrac{2x^2}{(x+3)(x-5)}$. Note that the denominator 
		must have the given factors; the numerator could be any degree $2$ polynomial, provided the 
		leading coefficient is $2$.
	\end{shortsolution}
\end{subproblem}
\end{problem}

%===================================
%   Author: Hughes
%   Date:   Feb 2011
%===================================
\begin{problem}
Let $r$ be the rational function that has
\[
	r(x) = \frac{(x+2)(x-1)}{(x+3)(x-4)}
\]  
Each of the following questions are in relation to this function.
\begin{subproblem}
	What is the vertical intercept of this function? State your answer as an 
	ordered pair. \index{rational functions!vertical intercept}
	\begin{shortsolution}
		$\left(0,\frac{1}{6}\right)$
	\end{shortsolution}
\end{subproblem}
\begin{subproblem}\label{rat:prob:rational}
	What values of $x$ make the denominator equal to $0$?
	\begin{shortsolution}
		$-3,4$
	\end{shortsolution}
\end{subproblem}
\begin{subproblem}
	Use your answer to \cref{rat:prob:rational} to write the domain of the function in 
	both interval, and set builder notation. %\index{rational functions!domain}\index{domain!rational functions}
	\begin{shortsolution}
		Interval notation: $(-\infty,-3)\cup (-3,4)\cup (4,\infty)$. 
		Set builder: $\{x|x\ne -3, \mathrm{and}\, x\ne 4\}$
	\end{shortsolution}
\end{subproblem}
\begin{subproblem}
	What are the vertical asymptotes of the function? State your answers in 
	the form $x=$
	\begin{shortsolution}
		$x=-3$ and $x=4$
	\end{shortsolution}
\end{subproblem}
\begin{subproblem}\label{rat:prob:zeroes}
	What values of $x$ make the numerator equal to $0$?
	\begin{shortsolution}
		$-2,1$ 
	\end{shortsolution}
\end{subproblem}
\begin{subproblem}
	Use your answer to \cref{rat:prob:zeroes} to write the horizontal intercepts of 
	$r$ as ordered pairs.
	\begin{shortsolution}
		$(-2,0)$ and $(1,0)$
	\end{shortsolution}
\end{subproblem}
\end{problem}


%===================================
%   Author: Hughes
%   Date:   May 2011
%===================================
\begin{problem}[Holes]
\pccname{Josh} and \pccname{Pedro} are discussing the function
\[
	r(x)=\frac{x^2-1}{(x+3)(x-1)}
\]
\begin{subproblem}
	What is the domain of $r$?
	\begin{shortsolution}
		The domain of $r$ is $(-\infty,-3)\cup(-3,1)\cup(1,\infty)$.
	\end{shortsolution}
\end{subproblem}
\begin{subproblem}
	Josh notices that the numerator can be factored- can you see how?
	\begin{shortsolution}
		$(x^2-1)=(x-1)(x+1)$
	\end{shortsolution}
\end{subproblem}
\begin{subproblem}
	Pedro asks, `Doesn't that just mean that 
	\[
		r(x)=\frac{x+1}{x+3}
	\]
	for all values of $x$?' Josh says, `Nearly\ldots but not for all values of $x$'. 
	What does Josh mean?
	\begin{shortsolution}
		$r(x)=\dfrac{x+1}{x+3}$ provided that $x\ne -1$. 
	\end{shortsolution}
\end{subproblem}
\begin{subproblem}
	Where does $r$ have vertical asymptotes, and where does it have holes?
	\begin{shortsolution}
		The function $r$ has a vertical asymptote at $-3$, and a hole at $1$.
	\end{shortsolution}
\end{subproblem}
\begin{subproblem}
	Sketch a graph of $r$.
	\begin{shortsolution}
		A graph of $r$ is shown below.
				
		\begin{tikzpicture}
			\begin{axis}[
					framed,
					xmin=-10,xmax=10,
					ymin=-10,ymax=10,
					xtick={-8,-6,...,8},
					ytick={-8,-6,...,8},
					grid=both,
				]
				\addplot[pccplot] expression[domain=-10:-3.25]{(x+1)/(x+3)};
				\addplot[pccplot] expression[domain=-2.75:10]{(x+1)/(x+3)};
				\addplot[asymptote,domain=-10:10]({-3},{x});
				\addplot[holdot]coordinates{(1,0.5)};
			\end{axis}
		\end{tikzpicture}
	\end{shortsolution}
\end{subproblem}
\end{problem}

%===================================
%   Author: Hughes
%   Date:   July 2012
%===================================
\begin{problem}[Function algebra]
Let $r$ and $s$ be the rational functions that have formulas
\[
	r(x)=\frac{2-x}{x+3}, \qquad s(x)=\frac{x^2}{x-4}
\]
Evaluate each of the following (if possible).
\begin{multicols}{4}
	\begin{subproblem}
		$(r+s)(5)$ 
		\begin{shortsolution}
			$\frac{197}{8}$ 
		\end{shortsolution}
	\end{subproblem}
	\begin{subproblem}
		$(r-s)(3)$
		\begin{shortsolution}
			$\frac{53}{6}$ 
		\end{shortsolution}
	\end{subproblem}
	\begin{subproblem}
		$(r\cdot s)(4)$ 
		\begin{shortsolution}
			Undefined.
		\end{shortsolution}
	\end{subproblem}
	\begin{subproblem}
		$\left( \frac{r}{s} \right)(1)$ 
		\begin{shortsolution}
			$-\frac{3}{4}$ 
		\end{shortsolution}
	\end{subproblem}
\end{multicols}
\end{problem}


%===================================
%   Author: Hughes
%   Date:   July 2012
%===================================
\begin{problem}[Transformations: given the transformation, find the formula]
Let $r$ be the rational function that has formula.
\[
	r(x)=\frac{x+5}{2x-3}
\]
In each of the following problems apply the given transformation to the function $r$ and 
write a formula for the transformed version of $r$.
\begin{multicols}{2}
	\begin{subproblem}
		Shift $r$ to the right by $3$ units. 
		\begin{shortsolution}
			$r(x-3)=\frac{x+2}{2x-9}$
		\end{shortsolution}
	\end{subproblem}
	\begin{subproblem}
		Shift $r$ to the left by $4$ units. 
		\begin{shortsolution}
			$r(x+4)=\frac{x+9}{2x+5}$
		\end{shortsolution}
	\end{subproblem}
	\begin{subproblem}
		Shift $r$ up by $\pi$ units. 
		\begin{shortsolution}
			$r(x)+\pi=\frac{x+5}{2x-3}+\pi$
		\end{shortsolution}
	\end{subproblem}
	\begin{subproblem}
		Shift $r$ down by $17$ units. 
		\begin{shortsolution}
			$r(x)-17=\frac{x+5}{2x-3}-17$
		\end{shortsolution}
	\end{subproblem}
	\begin{subproblem}
		Reflect $r$ over the horizontal axis.
		\begin{shortsolution}
			$-r(x)=-\frac{x+5}{2x-3}$
		\end{shortsolution}
	\end{subproblem}
	\begin{subproblem}
		Reflect $r$ over the vertical axis.
		\begin{shortsolution}
			$r(-x)=\frac{x-5}{2x+3}$
		\end{shortsolution}
	\end{subproblem}
\end{multicols}
\end{problem}


%===================================
%   Author: Hughes
%   Date:   May 2011
%===================================
\begin{problem}[Find a formula from a table]\label{rat:prob:findformula}
\Crefrange{rat:tab:findformular}{rat:tab:findformulau} show values of rational functions $r$, $q$, $s$, 
and $t$. Assume that any values marked with an X are undefined.

\begin{table}[!htb]
	\begin{widepage}
	\centering
	\caption{Tables for \cref{rat:prob:findformula}}
	\label{rat:tab:findformula}
	\begin{subtable}{.2\textwidth}
		\centering
		\caption{$y=r(x)$}
		\label{rat:tab:findformular}
		\begin{tabular}{S[table-format=1.0]S[table-format=1.0]}
			\beforeheading
			\heading{$x$} & \heading{$y$} \\ \afterheading   
			-4            & \num{7/2}     \\\normalline  
			-3            & -18           \\\normalline   
			-2            & X             \\\normalline 
			-1            & -4            \\\normalline   
			0             & \num{-3/2}    \\\normalline  
			1             & \num{-2/3}    \\\normalline    
			2             & \num{-1/4}    \\\normalline  
			3             & 0             \\\normalline     
			4             & \num{1/6}     \\\lastline      
		\end{tabular}
	\end{subtable}
	\hfill
	\begin{subtable}{.2\textwidth}
		\centering
		\caption{$y=s(x)$}
		\label{rat:tab:findformulas}
		\begin{tabular}{S[table-format=1.0]S[table-format=1.0]}
			\beforeheading
			\heading{$x$} & \heading{$y$} \\ \afterheading   
			-4            & \num{-2/21}   \\\normalline 
			-3            & \num{-1/12}   \\\normalline  
			-2            & 0             \\\normalline                  
			-1            & X             \\\normalline                     
			0             & \num{-2/3}    \\\normalline 
			1             & \num{-3/4}    \\\normalline   
			2             & \num{-4/3}    \\\normalline 
			3             & X             \\\normalline                
			4             & \num{6/5}     \\\lastline     
		\end{tabular}
	\end{subtable}
	\hfill
	\begin{subtable}{.2\textwidth}
		\centering
		\caption{$y=t(x)$}
		\label{rat:tab:findformulat}
		\begin{tabular}{S[table-format=1.0]S[table-format=1.0]}
			\beforeheading
			\heading{$x$} & \heading{$y$} \\ \afterheading   
			-4            & \num{3/5}     \\\normalline    
			-3            & 0             \\\normalline                   
			-2            & X             \\\normalline                     
			-1            & 3             \\\normalline                    
			0             & 3             \\\normalline                 
			1             & X             \\\normalline                     
			2             & 0             \\\normalline                 
			3             & \num{3/5}     \\\normalline 
			4             & \num{7/9}     \\\lastline      
		\end{tabular}
	\end{subtable}
	\hfill
	\begin{subtable}{.2\textwidth}
		\centering
		\caption{$y=u(x)$}
		\label{rat:tab:findformulau}
		\begin{tabular}{S[table-format=1.0]S[table-format=1.0]}
			\beforeheading
			\heading{$x$} & \heading{$y$} \\ \afterheading   
			-4            & \num{16/7}    \\\normalline  
			-3            & X             \\\normalline                    
			-2            & \num{-4/5}    \\\normalline 
			-1            & \num{-1/8}    \\\normalline 
			0             & 0             \\\normalline                      
			1             & \num{-1/8}    \\\normalline   
			2             & \num{-4/5}    \\\normalline   
			3             & X             \\\normalline                     
			4             & \num{16/7}    \\\lastline     
		\end{tabular}
	\end{subtable}
	\end{widepage}
\end{table}
\begin{subproblem}
	Given that the formula for $r(x)$ has the form $r(x)=\dfrac{x-A}{x-B}$, use \cref{rat:tab:findformular}
	to find values of $A$ and $B$.
	\begin{shortsolution}
		$A=3$ and $B=-2$, so $r(x)=\dfrac{x-3}{x+2}$.
	\end{shortsolution}
\end{subproblem}
\begin{subproblem}
	Check your formula by computing $r(x)$ at the values specified in the table.
	\begin{shortsolution}
		$\begin{aligned}[t]
			r(-4) & = \frac{-4-3}{-4+2} \\ 
			      & = \frac{7}{2}       \\       
		\end{aligned}$
				
		$r(-3)=\ldots$ etc
	\end{shortsolution}
\end{subproblem}
\begin{subproblem}
	The function $s$ in \cref{rat:tab:findformulas} has two vertical asymptotes and one zero. 
	Can you find a formula for $s(x)$?
	\begin{shortsolution}
		$s(x)=\dfrac{x+2}{(x-3)(x+1)}$
	\end{shortsolution}
\end{subproblem}
\begin{subproblem}
	Check your formula by computing $s(x)$ at the values specified in the table.
	\begin{shortsolution}
		$\begin{aligned}[t]
			s(-4) & =\frac{-4+2}{(-4-3)(-4+1)} \\ 
			      & =-\frac{2}{21}             
		\end{aligned}$
				
		$s(-3)=\ldots$ etc
	\end{shortsolution}
\end{subproblem}
\begin{subproblem}
	Given that the formula for $t(x)$ has  the form $t(x)=\dfrac{(x-A)(x-B)}{(x-C)(x-D)}$, use \cref{rat:tab:findformulat} to find the 
	values of $A$, $B$, $C$, and $D$; hence write a formula for $t(x)$.
	\begin{shortsolution}
		$t(x)=\dfrac{(x+3)(x-2)}{(x+2)(x+1)}$ 
	\end{shortsolution}
\end{subproblem}
\begin{subproblem}
	Given that the formula for $u(x)$ has  the form $u(x)=\dfrac{(x-A)^2}{(x-B)(x-C)}$, use \cref{rat:tab:findformulau} to find the 
	values of $A$, $B$, and $C$; hence write a formula for $u(x)$.
	\begin{shortsolution}
		$u(x)=\dfrac{x^2}{(x+3)(x-3)}$ 
	\end{shortsolution}
\end{subproblem}
\end{problem}
\end{exercises}

\section{Graphing rational functions (horizontal asymptotes)}
\reformatstepslist{R} % the steps list should be R1, R2, \ldots
We studied rational functions in the previous section, but were 
not asked to graph them; in this section we will demonstrate the 
steps to be followed in order to sketch graphs of the functions. 

Remember from \vref{rat:def:function} that rational functions have 
the form
\[
	r(x)=\frac{p(x)}{q(x)}
\]
In this section we will restrict attention to the case when 
\[
	\text{degree of }p\leq \text{degree of }q
\]
Note that this necessarily means that each function that we consider
in this section \emph{will have a horizontal asymptote} (see \vref{rat:def:longrun}).
The cases in which the degree of $p$ is greater than the degree of $q$ 
is covered in the next section.

Before we begin, it is important to remember the following:
\begin{itemize}
	\item Our sketches will give a good representation of the overall 
	shape of the graph, but until we have the tools of calculus (from MTH 251)
	we can not find local minimums, local maximums, and inflection points algebraically. This
	means that we will make our best guess as to where these points are.
	\item We will not concern ourselves too much with the vertical scale (because of 
	our previous point)| we will, however, mark the vertical intercept (assuming there is one), 
	and any horizontal asymptotes.
\end{itemize}
\fixthis{Jessica suggested add factoring... and seeing if the curve 
cuts the horizontal asymptote}
\begin{pccspecialcomment}[Steps to follow when sketching rational functions]\label{rat:def:stepsforsketch}
	\begin{steps}
		\item \label{rat:step:first} Find all vertical asymptotes and holes, and mark them on the 
		graph using dashed vertical lines and open circles $\circ$ respectively.
		\item Find any intercepts, and mark them using solid circles $\bullet$;
		determine if the curve cuts the axis, or bounces off it at each zero.
		\item Determine the behavior of the function around each asymptote| does
		it behave like $\frac{1}{x}$ or $\frac{1}{x^2}$?
		\item \label{rat:step:penultimate} Determine the long-run behavior of the function, and mark the horizontal 
		asymptote using a dashed horizontal line.
		\item \label{rat:step:last}  Deduce the overall shape of the curve, and sketch it. If there isn't
		enough information from the previous steps, then construct a table of values
		including sample points from each branch.
	\end{steps}
	Remember that until we have the tools of calculus, we won't be able to 
	find the exact coordinates of local minimums, local maximums, and points
	of inflection.
\end{pccspecialcomment}

The examples that follow show how \crefrange{rat:step:first}{rat:step:last} can be
applied to a variety of different rational functions.

%===================================
%   Author: Hughes
%   Date:   May 2012
%===================================
\begin{pccexample}\label{rat:ex:1overxminus2p2}
	Use \crefrange{rat:step:first}{rat:step:last} to sketch a graph of the function $r$ 
	that has formula
	\[
		r(x)=\frac{1}{x-2}
	\]
	\begin{pccsolution}
		\begin{steps}
			\item $r$ has a vertical asymptote at $2$; $r$ does not have any holes. The curve of 
			$r$ will have $2$ branches.
			\item $r$ does not have any zeros since the numerator is never equal to $0$. The
			vertical intercept of $r$ is $\left( 0,-\frac{1}{2} \right)$.
			\item $r$ behaves like $\frac{1}{x}$ around its vertical asymptote since $(x-2)$
			is raised to the power $1$.
			\item Since the degree of the numerator is less than the degree of the denominator, 
			according to \vref{rat:def:longrun} the horizontal asymptote of $r$ has equation $y=0$.
			\item We put the details we have obtained so far on \cref{rat:fig:1overxminus2p1}. Notice 
			that there is only one way to complete the graph, which we have done in \cref{rat:fig:1overxminus2p2}.
		\end{steps}
	\end{pccsolution}
\end{pccexample}

\begin{figure}[!htbp]
	\begin{subfigure}{.45\textwidth}
		\begin{tikzpicture}
			\begin{axis}[
					xmin=-5,xmax=5,
					ymin=-5,ymax=5,
				]
				\addplot[asymptote,domain=-5:5]({2},{x});
				\addplot[asymptote,domain=-5:5]({x},{0});
				\addplot[soldot] coordinates{(0,-0.5)}node[axisnode,anchor=north east]{$\left( 0,-\frac{1}{2} \right)$};
			\end{axis}
		\end{tikzpicture}
		\caption{}
		\label{rat:fig:1overxminus2p1}
	\end{subfigure}%
	\hfill
	\begin{subfigure}{.45\textwidth}
		\begin{tikzpicture}[/pgf/declare function={f=1/(x-2);}]
			\begin{axis}[
					xmin=-5,xmax=5,
					ymin=-5,ymax=5,
				]
				\addplot[pccplot] expression[domain=-5:1.8,samples=50]{f};
				\addplot[pccplot] expression[domain=2.2:5]{f};
				\addplot[asymptote,domain=-5:5]({2},{x});
				\addplot[asymptote,domain=-5:5]({x},{0});
				\addplot[soldot] coordinates{(0,-0.5)}node[axisnode,anchor=north east]{$\left( 0,-\frac{1}{2} \right)$};
			\end{axis}
		\end{tikzpicture}
		\caption{}
		\label{rat:fig:1overxminus2p2}
	\end{subfigure}%
	\caption{$y=\dfrac{1}{x-2}$}
\end{figure}

The function $r$ in \cref{rat:ex:1overxminus2p2} has a horizontal asymptote which has equation $y=0$.
This asymptote lies on the horizontal axis, and you might (understandably) find it hard 
to distinguish between the two lines (\cref{rat:fig:1overxminus2p2}). When faced
with such a situation, it is perfectly acceptable to draw the horizontal axis 
as a dashed line| just make sure to label it correctly. We will demonstrate this 
in the next example.

%===================================
%   Author: Hughes
%   Date:   May 2012
%===================================
\begin{pccexample}\label{rat:ex:1overxp1}
	Use \crefrange{rat:step:first}{rat:step:last} to sketch a graph of the function $v$ 
	that has formula
	\[
		v(x)=\frac{10}{x}
	\]
	\begin{pccsolution}
		\begin{steps}
			\item $v$ has a vertical asymptote at $0$. $v$ does not have
			any holes. The curve of $v$ will have $2$ branches.
			\item $v$ does not have any zeros (since $10\ne 0$). Furthermore, $v$ 
			does not have a vertical intercept since $v(0)$ is undefined.
			\item $v$ behaves like $\frac{1}{x}$ around its vertical asymptote.
			\item $v$ has a horizontal asymptote with equation $y=0$.
			\item We put the details we have obtained so far in \cref{rat:fig:1overxp1}. 
			We do not have enough information to sketch $v$ yet (because $v$ does 
			not have any intercepts), so let's pick a sample 
			point in either of the $2$ branches| it doesn't matter where our sample point 
			is, because we know what the overall shape will be. Let's compute $v(2)$
			\begin{align*}
				v(2) & =\dfrac{10}{2} \\ 
				     & = 5            
			\end{align*}
			We therefore mark the point $(2,5)$ on \cref{rat:fig:1overxp2}, and then complete the sketch using 
			the details we found in the previous steps.
		\end{steps}
		
		\begin{figure}[!htbp]
			\begin{subfigure}{.45\textwidth}
				\begin{tikzpicture}
					\begin{axis}[
							xmin=-10,xmax=10,
							ymin=-10,ymax=10,
							xtick={-5,5},
							ytick={-5,5},
							axis line style={color=white},
						]
						\addplot[asymptote,<->,domain=-10:10]({0},{x});
						\addplot[asymptote,<->,domain=-10:10]({x},{0});
					\end{axis}
				\end{tikzpicture}
				\caption{}
				\label{rat:fig:1overxp1}
			\end{subfigure}%
			\hfill
			\begin{subfigure}{.45\textwidth}
				\begin{tikzpicture}[/pgf/declare function={f=10/x;}]
					\begin{axis}[
							xmin=-10,xmax=10,
							ymin=-10,ymax=10,
							xtick={-5,5},
							ytick={-5,5},
							axis line style={color=white},
						]
						\addplot[pccplot] expression[domain=-10:-1]{f};
						\addplot[pccplot] expression[domain=1:10]{f};
						\addplot[soldot] coordinates{(2,5)}node[axisnode,anchor=south west]{$(2,5)$};
						\addplot[asymptote,<->,domain=-10:10]({0},{x});
						\addplot[asymptote,<->,domain=-10:10]({x},{0});
					\end{axis}
				\end{tikzpicture}
				\caption{}
				\label{rat:fig:1overxp2}
			\end{subfigure}%
			\caption{$y=\dfrac{10}{x}$}
		\end{figure}
	\end{pccsolution}
\end{pccexample}

%===================================
%   Author: Hughes
%   Date:   May 2012
%===================================
\begin{pccexample}\label{rat:ex:asympandholep1}
	Use \crefrange{rat:step:first}{rat:step:last} to sketch a graph of the function $u$ 
	that has formula
	\[
		u(x)=\frac{-4(x^2-9)}{x^2-8x+15}
	\]
	\begin{pccsolution}
		\begin{steps}
			\item We begin by factoring both the numerator and denominator of $u$ to help
			us find any vertical asymptotes or holes
			\begin{align*}
				u(x) & =\frac{-4(x^2-9)}{x^2-8x+15}     \\     
				     & =\frac{-4(x+3)(x-3)}{(x-5)(x-3)} \\ 
				     & =\frac{-4(x+3)}{x-5}             
			\end{align*}
			provided that $x\ne 3$. Therefore $u$ has a vertical asymptote at $5$ and 
			a hole at $3$. The curve of $u$ has $2$ branches.
			\item $u$ has a simple zero at $-3$. The vertical intercept of $u$ is $\left( 0,\frac{12}{5} \right)$.
			\item $u$ behaves like $\frac{1}{x}$ around its vertical asymptote at $4$.
			\item Using \vref{rat:def:longrun} the equation of the horizontal asymptote of $u$ is $y=-4$.
			\item We put the details we have obtained so far on \cref{rat:fig:1overxminus2p1}. Notice 
			that there is only one way to complete the graph, which we have done in \cref{rat:fig:1overxminus2p2}.
		\end{steps}
		
		\begin{figure}[!htbp]
			\begin{subfigure}{.45\textwidth}
				\begin{tikzpicture}
					\begin{axis}[
							xmin=-10,xmax=10,
							ymin=-20,ymax=20,
							xtick={-8,-6,...,8},
							ytick={-10,10},
						]
						\addplot[asymptote,domain=-20:20]({4},{x});
						\addplot[asymptote,domain=-10:10]({x},{-4});
						\addplot[soldot] coordinates{(-3,0)(0,2.4)}node[axisnode,anchor=south east]{$\left( 0,\frac{12}{5} \right)$};
						\addplot[holdot] coordinates{(3,12)};
					\end{axis}
				\end{tikzpicture}
				\caption{}
				\label{rat:fig:asympandholep1}
			\end{subfigure}%
			\hfill
			\begin{subfigure}{.45\textwidth}
				\begin{tikzpicture}[/pgf/declare function={f=-4*(x+3)/(x-5);}]
					\begin{axis}[
							xmin=-10,xmax=10,
							ymin=-20,ymax=20,
							xtick={-8,-6,...,8},
							ytick={-10,10},
						]
						\addplot[pccplot] expression[domain=-10:3.6666,samples=50]{f};
						\addplot[pccplot] expression[domain=7:10]{f};
						\addplot[asymptote,domain=-20:20]({5},{x});
						\addplot[asymptote,domain=-10:10]({x},{-4});
						\addplot[soldot] coordinates{(-3,0)(0,2.4)}node[axisnode,anchor=south east]{$\left( 0,\frac{12}{5} \right)$};
						\addplot[holdot] coordinates{(3,12)};
					\end{axis}
				\end{tikzpicture}
				\caption{}
				\label{rat:fig:asympandholep2}
			\end{subfigure}%
			\caption{$y=\dfrac{-4(x+3)}{x-5}$}
		\end{figure}
	\end{pccsolution}
\end{pccexample}

\Cref{rat:ex:1overxminus2p2,rat:ex:1overxp1,rat:ex:asympandholep1} have focused on functions
that only have one vertical asymptote; the remaining examples in this section
concern functions that have more than one vertical asymptote. We will demonstrate
that \crefrange{rat:step:first}{rat:step:last} still apply.

%===================================
%   Author: Hughes
%   Date:   May 2012
%===================================
\begin{pccexample}\label{rat:ex:sketchtwoasymp}
	Use \crefrange{rat:step:first}{rat:step:last} to sketch a graph of the function $w$ 
	that has formula
	\[
		w(x)=\frac{2(x+3)(x-5)}{(x+5)(x-4)}
	\]
	\begin{pccsolution}
		\begin{steps}
			\item $w$ has vertical asymptotes at $-5$ and $4$. $w$ does not have 
			any holes. The curve of $w$ will have $3$ branches.
			\item $w$ has simple zeros at $-3$ and $5$. The vertical intercept of $w$ 
			is $\left( 0,\frac{3}{2} \right)$.
			\item $w$ behaves like $\frac{1}{x}$ around both of its vertical 
			asymptotes.
			\item The degree of the numerator of $w$ is $2$ and the degree of the 
			denominator of $w$ is also $2$. Using the ratio of the leading coefficients
			of the numerator and denominator, we say that $w$ has a horizontal 
			asymptote with equation $y=\frac{2}{1}=2$.
			\item We put the details we have obtained so far on \cref{rat:fig:sketchtwoasymptp1}. 
				
			The function $w$ is a little more complicated than the functions that 
			we have considered in the previous examples because the curve has $3$
			branches. When graphing such functions, it is generally a good idea to start with the branch
			for which you have the most information| in this case, that is the \emph{middle} branch
			on the interval $(-5,4)$.
				
			Once we have drawn the middle branch, there is only one way to complete the graph 
			(because of our observations about the behavior of $w$ around its vertical asymptotes), 
			which we have done in \cref{rat:fig:sketchtwoasymptp2}.
		\end{steps}
	\end{pccsolution}
\end{pccexample}

\begin{figure}[!htbp]
	\begin{subfigure}{.45\textwidth}
		\begin{tikzpicture}
			\begin{axis}[
					xmin=-10,xmax=10,
					ymin=-10,ymax=10,
					xtick={-8,-6,...,8},
					ytick={-5,5},
				]
				\addplot[asymptote,domain=-10:10]({-5},{x});
				\addplot[asymptote,domain=-10:10]({4},{x});
				\addplot[asymptote,domain=-10:10]({x},{2});
				\addplot[soldot] coordinates{(-3,0)(5,0)};
				\addplot[soldot] coordinates{(0,1.5)}node[axisnode,anchor=north west]{$\left( 0,\frac{3}{2} \right)$};
			\end{axis}
		\end{tikzpicture}
		\caption{}
		\label{rat:fig:sketchtwoasymptp1}
	\end{subfigure}%
	\hfill
	\begin{subfigure}{.45\textwidth}
		\begin{tikzpicture}[/pgf/declare function={f=2*(x+3)*(x-5)/( (x+5)*(x-4));}]
			\begin{axis}[
					xmin=-10,xmax=10,
					ymin=-10,ymax=10,
					xtick={-8,-6,...,8},
					ytick={-5,5},
				]
				\addplot[asymptote,domain=-10:10]({-5},{x});
				\addplot[asymptote,domain=-10:10]({4},{x});
				\addplot[asymptote,domain=-10:10]({x},{2});
				\addplot[soldot] coordinates{(-3,0)(5,0)};
				\addplot[soldot] coordinates{(0,1.5)}node[axisnode,anchor=north west]{$\left( 0,\frac{3}{2} \right)$};
				\addplot[pccplot] expression[domain=-10:-5.56708]{f};
				\addplot[pccplot] expression[domain=-4.63511:3.81708]{f};
				\addplot[pccplot] expression[domain=4.13511:10]{f};
			\end{axis}
		\end{tikzpicture}
		\caption{}
		\label{rat:fig:sketchtwoasymptp2}
	\end{subfigure}%
	\caption{$y=\dfrac{2(x+3)(x-5)}{(x+5)(x-4)}$}
\end{figure}

The rational functions that we have considered so far have had simple
factors in the denominator; each function has behaved like $\frac{1}{x}$ 
around each of its vertical asymptotes. \Cref{rat:ex:2asympnozeros,rat:ex:2squaredasymp}
consider functions that have a repeated factor in the denominator.

%===================================
%   Author: Hughes
%   Date:   May 2012
%===================================
\begin{pccexample}\label{rat:ex:2asympnozeros}
	Use \crefrange{rat:step:first}{rat:step:last} to sketch a graph of the function $f$ 
	that has formula
	\[
		f(x)=\frac{100}{(x+5)(x-4)^2}
	\]
	\begin{pccsolution}
		\begin{steps}
			\item $f$ has vertical asymptotes at $-5$ and $4$. $f$ does not have 
			any holes. The curve of $f$ will have $3$ branches.
			\item $f$ does not have any zeros (since $100\ne 0$). The vertical intercept of $f$ 
			is $\left( 0,\frac{5}{4} \right)$.
			\item $f$ behaves like $\frac{1}{x}$ around $-5$ and behaves like $\frac{1}{x^2}$
			around $4$.
			\item The degree of the numerator of $f$ is $0$ and the degree of the 
			denominator of $f$ is $2$. $f$ has a horizontal asymptote with 
			equation $y=0$.
			\item We put the details we have obtained so far on \cref{rat:fig:2asympnozerosp1}. 
				
			The function $f$ is similar to the function $w$ that we considered in \cref{rat:ex:sketchtwoasymp}|
			it has two vertical asymptotes and $3$ branches, but in contrast to $w$ it does not have any zeros.
				
			We sketch $f$ in \cref{rat:fig:2asympnozerosp2}, using the middle branch as our guide
			because we have the most information about the function on the interval $(-5,4)$.
				
			Once we have drawn the middle branch, there is only one way to complete the graph 
			because of our observations about the behavior of $f$ around its vertical asymptotes (it behaves like $\frac{1}{x}$), 
			which we have done in \cref{rat:fig:2asympnozerosp2}.
				
			Note that we are not yet able to find the local minimum of $f$ algebraically on the interval $(-5,4)$, 
			so we make a reasonable guess as to where it is| we can be confident that it is above the horizontal axis
			since $f$ has no zeros. You may think that this is unsatisfactory, but once we have the tools of calculus, we will 
			be able to find local minimums more precisely.
		\end{steps}
	\end{pccsolution}
\end{pccexample}

\begin{figure}[!htbp]
	\begin{subfigure}{.45\textwidth}
		\begin{tikzpicture}
			\begin{axis}[
					xmin=-10,xmax=10,
					ymin=-10,ymax=10,
					xtick={-8,-6,...,8},
					ytick={-5,5},
				]
				\addplot[asymptote,domain=-10:10]({-5},{x});
				\addplot[asymptote,domain=-10:10]({4},{x});
				\addplot[asymptote,domain=-10:10]({x},{0});
				\addplot[soldot] coordinates{(0,1.25)}node[axisnode,anchor=south east]{$\left( 0,\frac{5}{4} \right)$};
			\end{axis}
		\end{tikzpicture}
		\caption{}
		\label{rat:fig:2asympnozerosp1}
	\end{subfigure}%
	\hfill
	\begin{subfigure}{.45\textwidth}
		\begin{tikzpicture}[/pgf/declare function={f=100/( (x+5)*(x-4)^2);}]
			\begin{axis}[
					xmin=-10,xmax=10,
					ymin=-10,ymax=10,
					xtick={-8,-6,...,8},
					ytick={-5,5},
				]
				\addplot[asymptote,domain=-10:10]({-5},{x});
				\addplot[asymptote,domain=-10:10]({4},{x});
				\addplot[asymptote,domain=-10:10]({x},{0});
				\addplot[soldot] coordinates{(0,1.25)}node[axisnode,anchor=south east]{$\left( 0,\frac{5}{4} \right)$};
				\addplot[pccplot] expression[domain=-10:-5.12022]{f};
				\addplot[pccplot] expression[domain=-4.87298:2.87298,samples=50]{f};
				\addplot[pccplot] expression[domain=5:10]{f};
			\end{axis}
		\end{tikzpicture}
		\caption{}
		\label{rat:fig:2asympnozerosp2}
	\end{subfigure}%
	\caption{$y=\dfrac{100}{(x+5)(x-4)^2}$}
\end{figure}

%===================================
%   Author: Hughes
%   Date:   May 2012
%===================================
\begin{pccexample}\label{rat:ex:2squaredasymp}
	Use \crefrange{rat:step:first}{rat:step:last} to sketch a graph of the function $g$ 
	that has formula
	\[
		g(x)=\frac{50(2-x)}{(x+3)^2(x-5)^2}
	\]
	\begin{pccsolution}
		\begin{steps}
			\item $g$ has vertical asymptotes at $-3$ and $5$. $g$ does 
			not have any holes. The curve of $g$ will have $3$ branches.
			\item $g$ has a simple zero at $2$. The vertical intercept of $g$ is
			$\left( 0,\frac{4}{9} \right)$.
			\item $g$ behaves like $\frac{1}{x^2}$ around both of its 
			vertical asymptotes.
			\item The degree of the numerator of $g$ is $1$ and the degree of the denominator
			of $g$ is $4$. Using \vref{rat:def:longrun}, we calculate that 
			the horizontal asymptote of $g$ has equation $y=0$.
			\item The details that we have found so far have been drawn in 
			\cref{rat:fig:2squaredasymp1}. The function $g$ is similar to the functions 
			we considered in \cref{rat:ex:sketchtwoasymp,rat:ex:2asympnozeros} because 
			it has $2$ vertical asymptotes and $3$ branches. 
				
			We sketch $g$ using the middle branch as our guide because we have the most information
			about $g$ on the interval $(-3,5)$. Note that there is no other way to draw this branch
			without introducing other zeros which $g$ does not have.
				
			Once we have drawn the middle branch, there is only one way to complete the graph 
			because of our observations about the behavior of $g$ around its vertical asymptotes| it 
			behaves like $\frac{1}{x^2}$.
				
		\end{steps}
	\end{pccsolution}
\end{pccexample}

\begin{figure}[!htbp]
	\begin{subfigure}{.45\textwidth}
		\begin{tikzpicture}
			\begin{axis}[
					xmin=-10,xmax=10,
					ymin=-10,ymax=10,
					xtick={-8,-6,...,8},
					ytick={-5,5},
				]
				\addplot[asymptote,domain=-10:10]({-3},{x});
				\addplot[asymptote,domain=-10:10]({5},{x});
				\addplot[asymptote,domain=-10:10]({x},{0});
				\addplot[soldot] coordinates{(2,0)(0,4/9)}node[axisnode,anchor=south west]{$\left( 0,\frac{4}{9} \right)$};
			\end{axis}
		\end{tikzpicture}
		\caption{}
		\label{rat:fig:2squaredasymp1}
	\end{subfigure}%
	\hfill
	\begin{subfigure}{.45\textwidth}
		\begin{tikzpicture}[/pgf/declare function={f=50*(2-x)/( (x+3)^2*(x-5)^2);}]
			\begin{axis}[
					xmin=-10,xmax=10,
					ymin=-10,ymax=10,
					xtick={-8,-6,...,8},
					ytick={-5,5},
				]
				\addplot[asymptote,domain=-10:10]({-3},{x});
				\addplot[asymptote,domain=-10:10]({5},{x});
				\addplot[asymptote,domain=-10:10]({x},{0});
				\addplot[soldot] coordinates{(2,0)(0,4/9)}node[axisnode,anchor=south west]{$\left( 0,\frac{4}{9} \right)$};
				\addplot[pccplot] expression[domain=-10:-3.61504]{f};
				\addplot[pccplot] expression[domain=-2.3657:4.52773]{f};
				\addplot[pccplot] expression[domain=5.49205:10]{f};
			\end{axis}
		\end{tikzpicture}
		\caption{}
		\label{rat:fig:2squaredasymp2}
	\end{subfigure}%
	\caption{$y=\dfrac{50(2-x)}{(x+3)^2(x-5)^2}$}
\end{figure}

Each of the rational functions that we have considered so far has had either 
a \emph{simple} zero, or no zeros at all. Remember from our work on polynomial 
functions, and particularly \vref{poly:def:multzero}, that a \emph{repeated} zero
corresponds to the curve of the function behaving differently at the zero
when compared to how the curve behaves at a simple zero. \Cref{rat:ex:doublezero} details a 
function that has a non-simple zero.

%===================================
%   Author: Hughes
%   Date:   June 2012
%===================================
\begin{pccexample}\label{rat:ex:doublezero}
	Use \crefrange{rat:step:first}{rat:step:last} to sketch a graph of the function $g$ 
	that has formula
	\[
		h(x)=\frac{(x-3)^2}{(x+4)(x-6)}
	\]
	\begin{pccsolution}
		\begin{steps}
			\item $h$ has vertical asymptotes at $-4$ and $6$. $h$ does 
			not have any holes. The curve of $h$ will have $3$ branches.
			\item $h$ has a zero at $3$ that has \emph{multiplicity $2$}. 
			The vertical intercept of $h$ is
			$\left( 0,-\frac{3}{8} \right)$.
			\item $h$ behaves like $\frac{1}{x}$ around both of its 
			vertical asymptotes.
			\item The degree of the numerator of $h$ is $2$ and the degree of the denominator
			of $h$ is $2$. Using \vref{rat:def:longrun}, we calculate that 
			the horizontal asymptote of $h$ has equation $y=1$.
			\item The details that we have found so far have been drawn in 
			\cref{rat:fig:doublezerop1}. The function $h$ is different 
			from the functions that we have considered in previous examples because 
			of the multiplicity of the zero at $3$.
				
			We sketch $h$ using the middle branch as our guide because we have the most information
			about $h$ on the interval $(-4,6)$. Note that there is no other way to draw this branch
			without introducing other zeros which $h$ does not have| also note how 
			the curve bounces off the horizontal axis at $3$.
				
			Once we have drawn the middle branch, there is only one way to complete the graph 
			because of our observations about the behavior of $h$ around its vertical asymptotes| it 
			behaves like $\frac{1}{x}$.
				
		\end{steps}
	\end{pccsolution}
\end{pccexample}

\begin{figure}[!htbp]
	\begin{subfigure}{.45\textwidth}
		\begin{tikzpicture}
			\begin{axis}[
					xmin=-10,xmax=10,
					ymin=-5,ymax=5,
					xtick={-8,-6,...,8},
					ytick={-3,3},
				]
				\addplot[asymptote,domain=-10:10]({-4},{x});
				\addplot[asymptote,domain=-10:10]({6},{x});
				\addplot[asymptote,domain=-10:10]({x},{1});
				\addplot[soldot] coordinates{(3,0)(0,-3/8)}node[axisnode,anchor=north west]{$\left( 0,-\frac{3}{8} \right)$};
			\end{axis}
		\end{tikzpicture}
		\caption{}
		\label{rat:fig:doublezerop1}
	\end{subfigure}%
	\hfill
	\begin{subfigure}{.45\textwidth}
		\begin{tikzpicture}[/pgf/declare function={f=(x-3)^2/((x+4)*(x-6));}]
			\begin{axis}[
					xmin=-10,xmax=10,
					ymin=-5,ymax=5,
					xtick={-8,-6,...,8},
					ytick={-3,3},
				]
				\addplot[asymptote,domain=-10:10]({-4},{x});
				\addplot[asymptote,domain=-10:10]({6},{x});
				\addplot[asymptote,domain=-10:10]({x},{1});
				\addplot[soldot] coordinates{(3,0)(0,-3/8)}node[axisnode,anchor=north west]{$\left( 0,-\frac{3}{8} \right)$};
				\addplot[pccplot] expression[domain=-10:-5.20088]{f};
				\addplot[pccplot] expression[domain=-3.16975:5.83642,samples=50]{f};
				\addplot[pccplot] expression[domain=6.20088:10]{f};
			\end{axis}
		\end{tikzpicture}
		\caption{}
		\label{rat:fig:doublezerop2}
	\end{subfigure}%
	\caption{$y=\dfrac{(x-3)^2}{(x+4)(x-6)}$}
\end{figure}
\begin{exercises}
%===================================
%   Author: Hughes
%   Date:   June 2012
%===================================
\begin{problem}[\Cref{rat:step:last}]\label{rat:prob:deduce}
\pccname{Katie} is working on graphing rational functions. She 
has been concentrating on functions $f$ that have formula
\begin{equation}\label{rat:eq:deducecurve}
	f(x)=\frac{a(x-b)}{x-c}
\end{equation}
Katie notes that functions with this type of formula have a zero
at $b$, and a vertical asymptote at $c$. Furthermore, these functions
behave like $\frac{1}{x}$ around their vertical asymptote, and the 
curve of each function will have $2$ branches.

Katie has been working with $3$ functions that have the form given 
in \cref{rat:eq:deducecurve}, and has followed \crefrange{rat:step:first}{rat:step:penultimate};
her results are shown in \cref{rat:fig:deducecurve}. There is just one
more thing to do to complete the graphs| follow \cref{rat:step:last}.
Help Katie finish each graph by deducing the curve of each function.
\begin{shortsolution}
	\Vref{rat:fig:deducecurve1}
		
	\begin{tikzpicture}[/pgf/declare function={f=3*(x+4)/(x+5);}]
		\begin{axis}[
				xmin=-10,xmax=10,
				ymin=-10,ymax=10,
				xtick={-8,-6,...,8},
			]
			\addplot[soldot] coordinates{(-4,0)(0,12/5)};
			\addplot[asymptote,domain=-10:10]({-5},{x});
			\addplot[asymptote,domain=-10:10]({x},{3});
			\addplot[pccplot] expression[domain=-10:-5.42857]{f};
			\addplot[pccplot] expression[domain=-4.76923:10,samples=50]{f};
		\end{axis}
	\end{tikzpicture}
		
	\Vref{rat:fig:deducecurve2}
		
	\begin{tikzpicture}[/pgf/declare function={f=-3*(x-2)/(x-4);}]
		\begin{axis}[
				xmin=-10,xmax=10,
				ymin=-10,ymax=10,
				xtick={-8,-6,...,8},
			]
			\addplot[soldot] coordinates{(2,0)(0,-3/2)};
			\addplot[asymptote,domain=-10:10]({4},{x});
			\addplot[asymptote,domain=-10:10]({x},{-3});
			\addplot[pccplot] expression[domain=-10:3.53846,samples=50]{f};
			\addplot[pccplot] expression[domain=4.85714:10]{f};
		\end{axis}
	\end{tikzpicture}
		
	\Vref{rat:fig:deducecurve4}
		
	\begin{tikzpicture}[/pgf/declare function={f=2*(x-6)/(x-4);}]
		\begin{axis}[
				xmin=-10,xmax=10,
				ymin=-10,ymax=10,
				xtick={-8,-6,...,8},
			]
			\addplot[soldot] coordinates{(6,0)(0,3)};
			\addplot[asymptote,domain=-10:10]({x},{2});
			\addplot[asymptote,domain=-10:10]({4},{x});
			\addplot[pccplot] expression[domain=-10:3.5,samples=50]{f};
			\addplot[pccplot] expression[domain=4.3333:10]{f};
		\end{axis}
	\end{tikzpicture}
\end{shortsolution}
\end{problem}

\begin{figure}[!htb]
	\begin{widepage}
	\setlength{\figurewidth}{0.3\textwidth}
	\begin{subfigure}{\figurewidth}
		\begin{tikzpicture}
			\begin{axis}[
					xmin=-10,xmax=10,
					ymin=-10,ymax=10,
					xtick={-8,-6,...,8},
				]
				\addplot[soldot] coordinates{(-4,0)(0,12/5)};
				\addplot[asymptote,domain=-10:10]({-5},{x});
				\addplot[asymptote,domain=-10:10]({x},{3});
			\end{axis}
		\end{tikzpicture}
		\caption{}
		\label{rat:fig:deducecurve1}
	\end{subfigure}%
	\hfill
	\begin{subfigure}{\figurewidth}
		\begin{tikzpicture}
			\begin{axis}[
					xmin=-10,xmax=10,
					ymin=-10,ymax=10,
					xtick={-8,-6,...,8},
				]
				\addplot[soldot] coordinates{(2,0)(0,-3/2)};
				\addplot[asymptote,domain=-10:10]({4},{x});
				\addplot[asymptote,domain=-10:10]({x},{-3});
			\end{axis}
		\end{tikzpicture}
		\caption{}
		\label{rat:fig:deducecurve2}
	\end{subfigure}%
	\hfill
	\begin{subfigure}{\figurewidth}
		\begin{tikzpicture}
			\begin{axis}[
					xmin=-10,xmax=10,
					ymin=-10,ymax=10,
					xtick={-8,-6,...,8},
				]
				\addplot[soldot] coordinates{(6,0)(0,3)};
				\addplot[asymptote,domain=-10:10]({x},{2});
				\addplot[asymptote,domain=-10:10]({4},{x});
			\end{axis}
		\end{tikzpicture}
		\caption{}
		\label{rat:fig:deducecurve4}
	\end{subfigure}
	\caption{Graphs for \cref{rat:prob:deduce}}
	\label{rat:fig:deducecurve}
	\end{widepage}
\end{figure}

%===================================
%   Author: Hughes
%   Date:   June 2012
%===================================
\begin{problem}[\Cref{rat:step:last} for  more complicated rational functions]\label{rat:prob:deducehard}
\pccname{David} is also working on graphing rational functions, and 
has been concentrating on functions $r$ that have formula
\[
	r(x)=\frac{a(x-b)(x-c)}{(x-d)(x-e)}
\]
David notices that functions with this type of formula have simple zeros 
at $b$ and $c$, and vertical asymptotes at $d$ and $e$. Furthermore, 
these functions behave like $\frac{1}{x}$ around both vertical asymptotes,
and the curve of the function will have $3$ branches. 

David has followed \crefrange{rat:step:first}{rat:step:penultimate} for 
$3$ separate functions, and drawn the results in \cref{rat:fig:deducehard}.
Help David finish each graph by deducing the curve of each function.
\begin{shortsolution}
	\Vref{rat:fig:deducehard1}
		
	\begin{tikzpicture}[/pgf/declare function={f=(x-6)*(x+3)/( (x-4)*(x+1));}]
		\begin{axis}[
				xmin=-10,xmax=10,
				ymin=-10,ymax=10,
				xtick={-8,-6,...,8},
			]
			\addplot[soldot] coordinates{(-3,0)(6,0)(0,9/2)};
			\addplot[asymptote,domain=-10:10]({-1},{x});
			\addplot[asymptote,domain=-10:10]({4},{x});
			\addplot[asymptote,domain=-10:10]({x},{2});
			\addplot[pccplot] expression[domain=-10:-1.24276]{f};
			\addplot[pccplot] expression[domain=-0.6666:3.66667]{f};
			\addplot[pccplot] expression[domain=4.24276:10]{f};
		\end{axis}
	\end{tikzpicture}
		
	\Vref{rat:fig:deducehard2}
		
	\begin{tikzpicture}[/pgf/declare function={f=3*(x-2)*(x+3)/( (x-6)*(x+5));}]
		\begin{axis}[
				xmin=-10,xmax=10,
				ymin=-10,ymax=10,
				xtick={-8,-6,...,8},
			]
			\addplot[soldot] coordinates{(-3,0)(2,0)(0,3/5)};
			\addplot[asymptote,domain=-10:10]({-5},{x});
			\addplot[asymptote,domain=-10:10]({6},{x});
			\addplot[asymptote,domain=-10:10]({x},{3});
			\addplot[pccplot] expression[domain=-10:-5.4861]{f};
			\addplot[pccplot] expression[domain=-4.68395:5.22241]{f};
			\addplot[pccplot] expression[domain=7.34324:10]{f};
		\end{axis}
	\end{tikzpicture}
		
	\Vref{rat:fig:deducehard3}
		
	\begin{tikzpicture}[/pgf/declare function={f=2*(x-7)*(x+3)/( (x+6)*(x-5));}]
		\begin{axis}[
				xmin=-10,xmax=10,
				ymin=-10,ymax=10,
				xtick={-8,-6,...,8},
			]
			\addplot[soldot] coordinates{(-3,0)(7,0)(0,1.4)};
			\addplot[asymptote,domain=-10:10]({-6},{x});
			\addplot[asymptote,domain=-10:10]({5},{x});
			\addplot[asymptote,domain=-10:10]({x},{2});
			\addplot[pccplot] expression[domain=-10:-6.91427]{f};
			\addplot[pccplot] expression[domain=-5.42252:4.66427]{f};
			\addplot[pccplot] expression[domain=5.25586:10]{f};
		\end{axis}
	\end{tikzpicture}
		
\end{shortsolution}
\end{problem}

\begin{figure}[!htb]
	\begin{widepage}
	\setlength{\figurewidth}{0.3\textwidth}
	\begin{subfigure}{\figurewidth}
		\begin{tikzpicture}
			\begin{axis}[
					xmin=-10,xmax=10,
					ymin=-10,ymax=10,
					xtick={-8,-6,...,8},
				]
				\addplot[soldot] coordinates{(-3,0)(6,0)(0,9/2)};
				\addplot[asymptote,domain=-10:10]({-1},{x});
				\addplot[asymptote,domain=-10:10]({4},{x});
				\addplot[asymptote,domain=-10:10]({x},{2});
			\end{axis}
		\end{tikzpicture}
		\caption{}
		\label{rat:fig:deducehard1}
	\end{subfigure}%
	\hfill
	\begin{subfigure}{\figurewidth}
		\begin{tikzpicture}
			\begin{axis}[
					xmin=-10,xmax=10,
					ymin=-10,ymax=10,
					xtick={-8,-6,...,8},
				]
				\addplot[soldot] coordinates{(-3,0)(2,0)(0,3/5)};
				\addplot[asymptote,domain=-10:10]({-5},{x});
				\addplot[asymptote,domain=-10:10]({6},{x});
				\addplot[asymptote,domain=-10:10]({x},{3});
			\end{axis}
		\end{tikzpicture}
		\caption{}
		\label{rat:fig:deducehard2}
	\end{subfigure}%
	\hfill
	\begin{subfigure}{\figurewidth}
		\begin{tikzpicture}
			\begin{axis}[
					xmin=-10,xmax=10,
					ymin=-10,ymax=10,
					xtick={-8,-6,...,8},
				]
				\addplot[soldot] coordinates{(-3,0)(7,0)(0,1.4)};
				\addplot[asymptote,domain=-10:10]({-6},{x});
				\addplot[asymptote,domain=-10:10]({5},{x});
				\addplot[asymptote,domain=-10:10]({x},{2});
			\end{axis}
		\end{tikzpicture}
		\caption{}
		\label{rat:fig:deducehard3}
	\end{subfigure}%
	\hfill
	\caption{Graphs for \cref{rat:prob:deducehard}}
	\label{rat:fig:deducehard}
	\end{widepage}
\end{figure}
%===================================
%   Author: Adams (Hughes)
%   Date:   March 2012
%===================================
\begin{problem}[\Crefrange{rat:step:first}{rat:step:last}]
Use \crefrange{rat:step:first}{rat:step:last} to sketch a graph of 
each of the following curves
\fixthis{need 2 more subproblems here}
\begin{multicols}{4}
	\begin{subproblem}
		$y=\dfrac{4}{x+2}$ 
		\begin{shortsolution}
			Vertical intercept: $(0,2)$; vertical asymptote: $x=-2$, horizontal asymptote: $y=0$.
						
			\begin{tikzpicture}
				\begin{axis}[
						framed,
						xmin=-5,xmax=5,
						ymin=-5,ymax=5,
						grid=both,
					]
					\addplot[pccplot] expression[domain=-5:-2.8]{4/(x+2)};
					\addplot[pccplot] expression[domain=-1.2:5]{4/(x+2)};
					\addplot[soldot]coordinates{(0,2)};
					\addplot[asymptote,domain=-5:5]({-2},{x});
					\addplot[asymptote,domain=-5:5]({x},{0});
				\end{axis}
			\end{tikzpicture}
		\end{shortsolution}
	\end{subproblem}
	\begin{subproblem}
		$y=\dfrac{2x-1}{x^2-9}$ 
		\begin{shortsolution}
			Vertical intercept:$\left( 0,\frac{1}{9} \right)$; 
			horizontal intercept: $\left( \frac{1}{2},0 \right)$;
			vertical asymptotes: $x=-3$, $x=3$, horizontal asymptote: $y=0$.
						
			\begin{tikzpicture}
				\begin{axis}[
						framed,
						xmin=-5,xmax=5,
						ymin=-5,ymax=5,
						grid=both,
					]
					\addplot[pccplot] expression[domain=-5:-3.23974]{(2*x-1)/(x^2-9)};
					\addplot[pccplot,samples=50] expression[domain=-2.77321:2.83974]{(2*x-1)/(x^2-9)};
					\addplot[pccplot] expression[domain=3.17321:5]{(2*x-1)/(x^2-9)};
					\addplot[soldot]coordinates{(0,1/9)(1/2,0)};
					\addplot[asymptote,domain=-5:5]({-3},{x});
					\addplot[asymptote,domain=-5:5]({3},{x});
					\addplot[asymptote,domain=-5:5]({x},{0});
				\end{axis}
			\end{tikzpicture}
		\end{shortsolution}
	\end{subproblem}
	\begin{subproblem}
		$y=\dfrac{x+3}{x-5}$ 
		\begin{shortsolution}
			Vertical intercept $\left( 0,-\frac{3}{5} \right)$; horizontal
			intercept: $(-3,0)$; vertical asymptote: $x=5$; horizontal asymptote: $y=1$.
						
			\begin{tikzpicture}
				\begin{axis}[
						framed,
						xmin=-10,xmax=10,
						ymin=-5,ymax=5,
						xtick={-8,-6,...,8},
						minor ytick={-3,-1,...,3},
						grid=both,
					]
					\addplot[pccplot] expression[domain=-10:3.666]{(x+3)/(x-5)};
					\addplot[pccplot] expression[domain=7:10]{(x+3)/(x-5)};
					\addplot[asymptote,domain=-5:5]({5},{x});
					\addplot[asymptote,domain=-10:10]({x},{1});
					\addplot[soldot]coordinates{(0,-3/5)(-3,0)};
				\end{axis}
			\end{tikzpicture}
		\end{shortsolution}
	\end{subproblem}
	\begin{subproblem}
		$y=\dfrac{2x+3}{3x-1}$ 
		\begin{shortsolution}
			Vertical intercept: $(0,-3)$; horizontal intercept: $\left( -\frac{3}{2},0 \right)$;
			vertical asymptote: $x=\frac{1}{3}$, horizontal asymptote: $y=\frac{2}{3}$.
						
			\begin{tikzpicture}[/pgf/declare function={f=(2*x+3)/(3*x-1);}]
				\begin{axis}[
						framed,
						xmin=-5,xmax=5,
						ymin=-5,ymax=5,
						grid=both,
					]
					\addplot[pccplot] expression[domain=-5:0.1176]{f};
					\addplot[pccplot] expression[domain=0.6153:5]{f};
					\addplot[asymptote,domain=-5:5]({1/3},{x});
					\addplot[asymptote,domain=-5:5]({x},{2/3});
					\addplot[soldot]coordinates{(0,-3)(-3/2,0)};
				\end{axis}
			\end{tikzpicture}
		\end{shortsolution}
	\end{subproblem}
	\begin{subproblem}
		$y=\dfrac{4-x^2}{x^2-9}$ 
		\begin{shortsolution}
			Vertical intercept: $\left( 0,-\frac{4}{9} \right)$;
			horizontal intercepts: $(2,0)$, $(-2,0)$; 
			vertical asymptotes: $x=-3$, $x=3$; horizontal asymptote: $y=-1$.
						
			\begin{tikzpicture}[/pgf/declare function={f=(4-x^2)/(x^2-9);}]
				\begin{axis}[
						framed,
						xmin=-5,xmax=5,
						ymin=-5,ymax=5,
						grid=both,
					]
					\addplot[pccplot] expression[domain=-5:-3.20156]{f};
					\addplot[pccplot,samples=50] expression[domain=-2.85774:2.85774]{f};
					\addplot[pccplot] expression[domain=3.20156:5]{f};
					\addplot[asymptote,domain=-5:5]({-3},{x});
					\addplot[asymptote,domain=-5:5]({3},{x});
					\addplot[asymptote,domain=-5:5]({x},{-1});
					\addplot[soldot] coordinates{(-2,0)(2,0)(0,-4/9)};
				\end{axis}
			\end{tikzpicture}
		\end{shortsolution}
	\end{subproblem}
	\begin{subproblem}
		$y=\dfrac{(4x+5)(3x-4)}{(2x+5)(x-5)}$ 
		\begin{shortsolution}
			Vertical intercept: $\left( 0,\frac{4}{5} \right)$; 
			horizontal intercepts: $\left( -\frac{5}{4},0 \right)$, $\left( \frac{4}{3},0 \right)$; 
			vertical asymptotes: $x=-\frac{5}{2}$, $x=5$; horizontal asymptote: $y=6$.
						
			\begin{tikzpicture}[/pgf/declare function={f=(4*x+5)*(3*x-4)/((2*x+5)*(x-5));}]
				\begin{axis}[
						framed,
						xmin=-10,xmax=10,
						ymin=-20,ymax=20,
						xtick={-8,-6,...,8},
						ytick={-10,0,...,10},
						minor ytick={-15,-5,...,15},
						grid=both,
					]
					\addplot[pccplot] expression[domain=-10:-2.73416]{f};
					\addplot[pccplot] expression[domain=-2.33689:4.2792]{f};
					\addplot[pccplot] expression[domain=6.26988:10]{f};
					\addplot[asymptote,domain=-20:20]({-5/2},{x});
					\addplot[asymptote,domain=-20:20]({5},{x});
					\addplot[asymptote,domain=-10:10]({x},{6});
					\addplot[soldot]coordinates{(0,4/5)(-5/4,0)(4/3,0)};
				\end{axis}
			\end{tikzpicture}
		\end{shortsolution}
	\end{subproblem}
\end{multicols}
\end{problem}
%===================================
%   Author: Hughes
%   Date:   March 2012
%===================================
\begin{problem}[Inverse functions]
Each of the rational functions $F$ and $G$ are invertible; the functions
have formulas
\[
	F(x)=\frac{2x+1}{x-3}, \qquad G(x)= \frac{1-4x}{x+3}
\]
\begin{subproblem}
	State the domain of each function. 
	\begin{shortsolution}
		\begin{itemize}
			\item The domain of $F$ is $(-\infty,3)\cup(3,\infty)$.
			\item The domain of $G$ is $(-\infty,-3)\cup(-3,\infty)$.
		\end{itemize}
	\end{shortsolution}
\end{subproblem}
\begin{subproblem}
	Find the inverse of each function, and state its domain.
	\begin{shortsolution}
		\begin{itemize}
			\item $F^{-1}(x)=\frac{3x+1}{x-2}$; the domain of $F^{-1}$ is $(-\infty,2)\cup(2,\infty)$.
			\item $G^{-1}(x)=\frac{3x+1}{x+4}$; the domain of $G^{-1}$ is $(-\infty,-4)\cup(-4,\infty)$.
		\end{itemize}
	\end{shortsolution}
\end{subproblem}
\begin{subproblem}
	Hence state the range of the original functions.
	\begin{shortsolution}
		\begin{itemize}
			\item The range of $F$ is the domain of $F^{-1}$, which is $(-\infty,2)\cup(2,\infty)$.
			\item The range of $G$ is the domain of $G^{-1}$, which is $(-\infty,-4)\cup(-4,\infty)$.
		\end{itemize}
	\end{shortsolution}
\end{subproblem}
\begin{subproblem}
	State the range of each inverse function. 
	\begin{shortsolution}
		\begin{itemize}
			\item The range of $F^{-1}$ is the domain of $F$, which is $(-\infty,3)\cup(3,\infty)$.
			\item The range of $G^{-1}$ is the domain of $G$, which is $(-\infty,-3)\cup(-3,\infty)$.
		\end{itemize}
	\end{shortsolution}
\end{subproblem}
\end{problem}
%===================================
%   Author: Hughes
%   Date:   March 2012
%===================================
\begin{problem}[Composition]
Let $r$ and $s$ be the rational functions that have formulas
\[
	r(x)=\frac{3}{x^2},\qquad s(x)=\frac{4-x}{x+5}
\]
Evaluate each of the following.
\begin{multicols}{3}
	\begin{subproblem}
		$(r\circ s)(0)$
		\begin{shortsolution}
			$\frac{75}{16}$ 
		\end{shortsolution}
	\end{subproblem}
	\begin{subproblem}
		$(s\circ r)(0)$ 
		\begin{shortsolution}
			$(s\circ r)(0)$ is undefined.
		\end{shortsolution}
	\end{subproblem}
	\begin{subproblem}
		$(r\circ s)(2)$ 
		\begin{shortsolution}
			$\frac{147}{4}$ 
		\end{shortsolution}
	\end{subproblem}
	\begin{subproblem}
		$(s\circ r)(3)$ 
		\begin{shortsolution}
			$192$ 
		\end{shortsolution}
	\end{subproblem}
	\begin{subproblem}
		$(s\circ r)(4)$ 
		\begin{shortsolution}
			$(s\circ r)(4)$ is undefined.
		\end{shortsolution}
	\end{subproblem}
	\begin{subproblem}
		$(s\circ r)(x)$ 
		\begin{shortsolution}
			$\dfrac{4x^2-3}{1+5x^2}$
		\end{shortsolution}
	\end{subproblem}
\end{multicols}
\end{problem}
%===================================
%   Author: Hughes
%   Date:   March 2012
%===================================
\begin{problem}[Piecewise rational functions]
The function $R$ has formula
\[
	R(x)=
	\begin{dcases}
		\frac{2}{x+3},    & x<-5     \\
		\frac{x-4}{x-10}, & x\geq -5 
	\end{dcases}
\]
Evaluate each of the following.
\begin{multicols}{4}
	\begin{subproblem}
		$R(-6)$    
		\begin{shortsolution}
			$-\frac{2}{3}$ 
		\end{shortsolution}
	\end{subproblem}
	\begin{subproblem}
		$R(-5)$ 
		\begin{shortsolution}
			$\frac{3}{5}$ 
		\end{shortsolution}
	\end{subproblem}
	\begin{subproblem}
		$R(-3)$ 
		\begin{shortsolution}
			$\frac{7}{13}$ 
		\end{shortsolution}
	\end{subproblem}
	\begin{subproblem}
		$R(5)$ 
		\begin{shortsolution}
			$-\frac{1}{5}$ 
		\end{shortsolution}
	\end{subproblem}
\end{multicols}
\begin{subproblem}
	What is the domain of $R$? 
	\begin{shortsolution}
		$(-\infty,10)\cup(10,\infty)$ 
	\end{shortsolution}
\end{subproblem}
\end{problem}
\end{exercises}

\section{Graphing rational functions (oblique asymptotes)}\label{rat:sec:oblique}
\begin{subproblem}
	$y=\dfrac{x^2+1}{x-4}$ 
	\begin{shortsolution}
		\begin{enumerate}
			\item $\left( 0,-\frac{1}{4} \right)$
			\item Vertical asymptote: $x=4$.
			\item A graph of the function is shown below
						
			\begin{tikzpicture}[/pgf/declare function={f=(x^2+1)/(x-4);}]
				\begin{axis}[
						framed,
						xmin=-20,xmax=20,
						ymin=-30,ymax=30,
						xtick={-10,10},
						minor xtick={-15,-5,...,15},
						minor ytick={-10,10},
						grid=both,
					]
					\addplot[pccplot,samples=50] expression[domain=-20:3.54724]{f};
					\addplot[pccplot,samples=50] expression[domain=4.80196:20]{f};
					\addplot[asymptote,domain=-30:30]({4},{x});
				\end{axis}
			\end{tikzpicture}
		\end{enumerate}
	\end{shortsolution}
\end{subproblem}
\begin{subproblem}
	$y=\dfrac{x^3(x+3)}{x-5}$ 
	\begin{shortsolution}
		\begin{enumerate}
			\item $(0,0)$, $(-3,0)$
			\item Vertical asymptote: $x=5$, horizontal asymptote: none.
			\item A graph of the function is shown below
						
			\begin{tikzpicture}[/pgf/declare function={f=x^3*(x+3)/(x-5);}]
				\begin{axis}[
						framed,
						xmin=-10,xmax=10,
						ymin=-500,ymax=2500,
						xtick={-8,-6,...,8},
						ytick={500,1000,1500,2000},
						grid=both,
					]
					\addplot[pccplot,samples=50] expression[domain=-10:4]{f};
					\addplot[pccplot] expression[domain=5.6068:9.777]{f};
					\addplot[asymptote,domain=-500:2500]({5},{x});
				\end{axis}
			\end{tikzpicture}
		\end{enumerate}
	\end{shortsolution}
\end{subproblem}

%\include{ideas}

%=======================
%   BEGIN SOLUTIONS
%=======================

% change the page geometry using \newgeometry
%\cleardoublepage
\clearpage
%\setbool{@twoside}{false} 
\fancyheadoffset[RE,RO]{2cm}
\fancyheadoffset[LE,LO]{1cm}
\renewcommand{\rightmark}{Solutions to Section \thesection}
\fancyhead[LO,RE]{\rightmark}
\newgeometry{left=4cm,right=4cm,showframe=true,
    marginratio=1:1,
    top=1.5cm,bottom=1.5cm,bindingoffset=0cm}

% finish the php file
\Writetofile{crossrefsWEB}{?>}

% close the solutions files
\Closesolutionfile{shortsolutions}
\Closesolutionfile{longsolutions}
%\Closesolutionfile{hints}
\Closesolutionfile{crossrefsWEB}

% when itemized lists are used in the solutions, they
% are actually at 2nd depth because the solution environment 
% uses an \itemize environment to get the indendation correct
\setlist[itemize,2]{label=\textbullet}

% SHORT solution to problem (show only odd, even, all)
% Note: this renewenvironment needs to go here
%       so that the answers package can still
%       display correctly to the page if needed
\newbool{oddproblemnumber}
\renewenvironment{shortSoln}[1]{%
    \exploregroups % needed to ignore {}
    % before the environment starts - this is a stretchable space
    \vskip 0.1cm plus 2cm minus 0.1cm%
    \fullexpandarg % need this line so that '.' are counted
    %
    % either problems, or subproblems, e.g: 3.1 or 3.1.4 respectively
    % determine which one by counting the '.'
    \StrCount{#1}{.}[\numberofdecimals]
    %
    % find the problem number by splitting the string
    \ifnumequal{\numberofdecimals}{0}%
    {%
        % problems, such as 4, 5, 6, ...
        \def\problemnumber{#1}%
    }%
    {}%
    \ifnumequal{\numberofdecimals}{1}%
    {%
        % subproblems, such as 4.3, 1.2, 10.5
        \StrBefore{#1}{.}[\problemnumber]%
    }%
    {}%
    \ifnumequal{\numberofdecimals}{2}%
    {%
        % subproblems such as 1.3.1, 1.2.4, 7.5.6
        % note that these aren't currently used, but maybe someday
        \StrBehind{#1}{.}[\newbit]%
        \StrBefore{\newbit}{.}[\problemnumber]%
     }%
    {}%
     %
     % determine if the problem number is odd or even
     % and depending on our choices above, display or not
     \ifnumodd{\problemnumber}%
     {%
         % set a boolean that says the problem number is odd (used later)
         \setbool{oddproblemnumber}{true}%
         % display or not 
         \ifbool{showoddsolns}%
         {%
            % if we want to show the odd problems
            \ifbool{coreproblemYesNo}%
            {%  Core problem
                \expandafter\itemize[label=\llap{$\bigstar$ }\bfseries \hyperlink{prob:#1:\thechapter:\thesection}{#1}.]\item%
            }%
            {%  NOT Core problem
                \expandafter\itemize[label=\bfseries \hyperlink{prob:#1:\thechapter:\thesection}{#1}.]\item%
            }%
         }%
         {%
            % otherwise don't show them!
            \expandafter\comment%          
         }%
     }%
     {%
         % even numbered problem, set the boolean (used later)
         \setbool{oddproblemnumber}{false}%
         \ifbool{showevensolns}%
         {%
            % if we want to show the even problems
            \ifbool{coreproblemYesNo}%
            {%  Core problem
                \expandafter\itemize[label=\llap{$\bigstar$ }\itshape \hyperlink{prob:#1:\thechapter:\thesection}{#1}.]\item%
            }%
            {%  NOT Core problem
                \expandafter\itemize[label=\itshape \hyperlink{prob:#1:\thechapter:\thesection}{#1}.]\item%
            }%
         }%
         {%
            % otherwise don't show them!
            \expandafter\comment%          
         }%
     }%
}%
{%
    % after the environment finishes
    \ifbool{oddproblemnumber}%
    {%
        % odd numbered problems 
         \ifbool{showoddsolns}%
         {%
            % if we want to show the odd problems
            % then the environment is finished
            \enditemize%
         }%
         {%
            % otherwise we need to finish the comment
            \expandafter\endcomment%
         }%
    }%
    {%
        % even numbered problems
         \ifbool{showevensolns}%
         {%
            % if we want to show the even problems
            % then the environment is finished
            \enditemize%
         }%
         {%
            % otherwise we need to finish the comment
            \expandafter\endcomment%
         }%
    }%
}

% LONG solution to problem (show only odd, even, all)
% Note: this renewenvironment needs to go here
%       so that the answers package can still
%       display correctly to the page if needed
\renewenvironment{longSoln}[1]{%
    \exploregroups % needed to ignore {}
    % before the environment starts - this is a stretchable space
    \vskip 0.1cm plus 2cm minus 0.1cm%
    \fullexpandarg % need this line so that '.' are counted
    %
    % either problems, or subproblems, e.g: 3.1 or 3.1.4 respectively
    % determine which one by counting the '.'
    \StrCount{#1}{.}[\numberofdecimals]
    %
    % find the problem number by splitting the string
    \ifnumequal{\numberofdecimals}{0}%
    {%
        % problems, such as 4, 5, 6, ...
        \def\problemnumber{#1}%
    }%
    {}%
    \ifnumequal{\numberofdecimals}{1}%
    {%
        % problems, such as 4.3, 1.2, 10.5
        \StrBefore{#1}{.}[\problemnumber]%
    }%
    {}%
    \ifnumequal{\numberofdecimals}{2}%
    {%
        % subproblems such as 1.3.1, 1.2.4, 7.5.6
        \StrBehind{#1}{.}[\newbit]%
        \StrBefore{\newbit}{.}[\problemnumber]%
     }%
    {}%
     %
     % determine if the problem number is odd or even
     % and depending on our choices above, display or not
     \ifnumodd{\problemnumber}%
     {%
         % set a boolean that says the problem number is odd (used later)
         \setbool{oddproblemnumber}{true}%
         % display or not 
         \ifbool{showoddsolns}%
         {%
            % if we want to show the odd problems
            {\bfseries \hyperlink{prob:#1:\thechapter:\thesection}{#1}.}%
         }%
         {%
            % otherwise don't show them!
            \expandafter\comment%          
         }%
     }%
     {%
         % even numbered problem, set the boolean (used later)
         \setbool{oddproblemnumber}{false}%
         \ifbool{showevensolns}%
         {%
            % if we want to show the even problems
            {\itshape \hyperlink{prob:#1:\thechapter:\thesection}{#1}.}%
         }%
         {%
            % otherwise don't show them!
            \expandafter\comment%          
         }%
     }%
}%
{%
    % after the environment finishes
    \ifbool{oddproblemnumber}%
    {%
        % odd numbered problems 
         \ifbool{showoddsolns}%
         {%
            % if we want to show the odd problems
            % then the environment is finished
         }%
         {%
            % otherwise we need to finish the comment
            \expandafter\endcomment%
         }%
    }%
    {%
        % even numbered problems
         \ifbool{showevensolns}%
         {%
            % if we want to show the even problems
            % then the environment is finished
         }%
         {%
            % otherwise we need to finish the comment
            \expandafter\endcomment%
         }%
    }%
}

% renew tikzpicture environment to make it use valign=t
% on every one, which fixes vertical alignment of tikzpicture
% with the solution label: http://tex.stackexchange.com/questions/30367/aligning-enumerate-labels-to-top-of-image
\BeforeBeginEnvironment{tikzpicture}{\begin{adjustbox}{valign=t}}
\AfterEndEnvironment{tikzpicture}{\end{adjustbox}}

% do the same for the tabular environment
\BeforeBeginEnvironment{tabular}{\begin{adjustbox}{valign=t}}
\AfterEndEnvironment{tabular}{\end{adjustbox}}

% set every picture in the solutions to have \solutionfigurewidth
\pgfplotsset{every axis/.append style={%
            width=\solutionfigurewidth}}

% input the SHORT solutions file
\IfFileExists{shortsolutions.tex}{\input{shortsolutions.tex}}{}

\clearpage
% input the LONG solutions file
%\IfFileExists{longsolutions.tex}{\input{longsolutions.tex}}{}

\clearpage
% input the HINTS file
%\IfFileExists{hints.tex}{\input{hints.tex}}{}
%=======================
%   END SOLUTIONS
%=======================

%=======================
%   INDEX
%=======================
\printindex

\end{document}
