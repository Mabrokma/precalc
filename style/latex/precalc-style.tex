% <!--********************************************************************
% Copyright 2013 Robert A. Beezer
% 
% This file is part of MathBook XML.
% 
% MathBook XML is free software: you can redistribute it and/or modify
% it under the terms of the GNU General Public License as published by
% the Free Software Foundation, either version 2 or version 3 of the
% License (at your option).
% 
% MathBook XML is distributed in the hope that it will be useful,
% but WITHOUT ANY WARRANTY; without even the implied warranty of
% MERCHANTABILITY or FITNESS FOR A PARTICULAR PURPOSE.  See the
% GNU General Public License for more details.
% 
% You should have received a copy of the GNU General Public License
% along with MathBook XML.  If not, see <http://www.gnu.org/licenses/>.
% *********************************************************************-->


%% *********************************************************************
%  MBX .tex output can be produced via xsltproc without specifying the stringparam "latex.style.extra", 
%  and default styling will be used. If you wish to use a style file like this one for further styling,
%  run xsltproc with --stringparam "latex.style.extra" "<path to this file or one like it>"
%
%  Tags in this file start with <*tagname>, and end with </tagname>, and should have some commentary on
%  what each one does. The contents of tags should be TeX commands, not just strings. For instance, if
%  the styling involves loading a package with options, you would use the entire \usepackage[]{}, not 
%  just spell out the package options. 
%
%  In MBX .tex output, the new command \mbxstylefile should point to this style file, or a similar file.
%  If it is an empty string, default styling will be used.
%
%  In MBX .tex output, the new command \mbxstyle appears multiple times in the preamble as \mbxstyle{#1}{#2}.
%  #2 is simply commands that will be used if a style.tex file like this one is not being used. #1 is a tagname
%  that appears in the style.tex file.
%
%  Once xcltproc is used to produce a .tex file using a style.tex file like this one, then a lot of 
%  customiztion of appearance can be done without further use of xsltproc. For instance, the style.tex file 
%  can be edited, and then the .tex file can go through pdflatex again, and changes will be reflected.
%
%  Or the author can manually edit the .tex MBX output and change the path to the style.tex file, to 
%  get a new styling. An author could have an array of style.tex files that they switch back and forth on.
%
%  Or the author can revert to default styling be redefining how \mbxstyle is defined. For convenience, a
%  \renewcommand line is provided that can be uncommented to achieve this.  Note that this doesn't work in 
%  the opposite direction. If xsltproc is run the first time without a style.tex file, then several important 
%  things are left out of the preamble that allow for use of a style.tex file in the first place.
%
%% *********************************************************************




%% styling of terminology
%% font commands to apply to vocabulary terms
%<*terminology>
\itshape{}
%</terminology>

%% styling of acronyms
%% font commands to apply to acronyms
%<*terminology>
\bfseries{}
%</terminology>

%% sisetup
%% siunitx sisetup arguments
%<*sisetup>
\sisetup{per-mode=reciprocal}
%</sisetup>



%% theoremstyle for theorems
%% Either use \theoremstyle{<existing style>} or define a new style and then use it.
%% An existing style would be an amsthm pre-defined style, or a style defined earlier
%% in the MBX .tex output's preamble. While this style sheet should maintain an order
%% that is consistent with that order, it is not a guarantee.
%% \surroundwithmdframed may be used to decorate the environment. See its documentation.
%<*theoremstyle-theorem>
\theoremstyle{plain}
%</theoremstyle-theorem>

%% theoremstyle for proofs
%% The proof environment is typically simple: The word "Proof", followed by a period, perhaps 
%% in italics or something like that, and then at the end the qed symbol. You may want to change
%% the fontstyle on the word "Proof:". You may want to change the qed symbol. You may wan to 
%% encase it all in a box using \surroundwithmframed. All of that can be done here.
%<*theoremstyle-proof>
\expandafter\let\expandafter\oldproof\csname\string\proof\endcsname
\let\oldendproof\endproof
\renewenvironment{proof}[1][\proofname]{%
  \oldproof[\scshape #1]%
}{\oldendproof}
\renewcommand{\qedsymbol}{\checkmark}
%</theoremstyle-proof>

%% theoremstyle for corollaries
%<*theoremstyle-corollary>
\theoremstyle{plain}
%</theoremstyle-corollary>

%% theoremstyle for lemmas
%<*theoremstyle-lemma>
\theoremstyle{plain}
%</theoremstyle-lemma>

%% theoremstyle for propositionss
%<*theoremstyle-proposition>
\theoremstyle{plain}
%</theoremstyle-proposition>

%% theoremstyle for claims
%<*theoremstyle-claim>
\theoremstyle{plain}
%</theoremstyle-claim>

%% theoremstyle for facts
%<*theoremstyle-fact>
\theoremstyle{plain}
%</theoremstyle-fact>

%% theoremstyle for conjectures
%<*theoremstyle-conjecture>
\theoremstyle{plain}
%</theoremstyle-conjecture>

%% theoremstyle for axioms
%<*theoremstyle-axiom>
\theoremstyle{plain}
%</theoremstyle-axiom>

%% theoremstyle for principles
%<*theoremstyle-principle>
\theoremstyle{plain}
%</theoremstyle-principle>

%% theoremstyle for definitions
%<*theoremstyle-definition>
\newtheoremstyle{definitionstyle}% name of the style to be used
  {3pt}% measure of space to leave above the theorem. E.g.: 3pt
  {3pt}% measure of space to leave below the theorem. E.g.: 3pt
  {\itshape}% name of font to use in the body of the theorem
  {0pt}% measure of space to indent
  {\bfseries}% name of head font
  {.}% punctuation between head and body
  { }% space after theorem head; " " = normal interword space
  {\thmname{#1}\thmnumber{ #2}:\thmnote{ #3}}% Manually specify head
\theoremstyle{definitionstyle}
\surroundwithmdframed[outerlinewidth=3,
                      innerlinewidth=2,
                      linecolor=gray,
                      backgroundcolor=yellow!40,
                      innerlinecolor=blue!50,
                      outerlinecolor=red!50!,
                      innertopmargin=1em,
                      splittopskip=\topskip,
                      skipabove=\baselineskip,
                      ]{definition}
%</theoremstyle-definition>

%% theoremstyle for examples
%<*theoremstyle-example>
\theoremstyle{definition}
\surroundwithmdframed[outerlinewidth=3,
                      linecolor=black,
                      ]{example}
%</theoremstyle-example>

%% theoremstyle for exercises
%<*theoremstyle-exercise>
\theoremstyle{definition}
%</theoremstyle-exercise>

%% theoremstyle for remarks
%<*theoremstyle-remark>
\theoremstyle{definition}
%</theoremstyle-remark>


%% theoremstyle for standout, an unnumbered environment with special commentary (original "pccspecialcomment")
%<*theoremstyle-standout>
\newtheoremstyle{standoutstyle}% name of the style to be used
  {3pt}% measure of space to leave above the theorem. E.g.: 3pt
  {3pt}% measure of space to leave below the theorem. E.g.: 3pt
  {}% name of font to use in the body of the theorem
  {0pt}% measure of space to indent
  {\bfseries}% name of head font
  {.}% punctuation between head and body
  { }% space after theorem head; " " = normal interword space
  {\thmnote{#3}}% Manually specify head
\theoremstyle{standoutstyle}
\surroundwithmdframed[outerlinewidth=3,
                      innerlinewidth=2,
                      linecolor=gray,
                      backgroundcolor=blue!20,
                      innerlinecolor=blue!50,
                      outerlinecolor=red!50!,
                      innertopmargin=1em,
                      splittopskip=\topskip,
                      skipabove=\baselineskip,
                      ]{standout}
%</theoremstyle-standout>


%% environment to surround for try-it-yourself exercises
%<*try-it-yourself-begin>
 \par%
 \rule{\textwidth}{2pt}
 \vskip-16pt
 \rule{\textwidth}{0.5pt}
 \vskip-5pt%
 {\large\textbf{$\bigstar$ try it yourself $\bigstar$}}%
%</try-it-yourself-begin>
%<*try-it-yourself-end>
 \hfill
 \itshape{make sure you try it!}
 \vskip-9pt
 \rule{\textwidth}{0.5pt}
 \vskip-15pt
 \rule{\textwidth}{2pt}
%</try-it-yourself-end>


%% environment to surround for outcomes
%<*outcomes-begin>
 \itshape%
 \begin{minipage}{0.6\linewidth}%
 % define a new environment local to outcomes- the outcomelist
 % environment will not work outside of outcomes!
 \newenvironment{outcomelist}{\begin{itemize}[itemsep=0pt, topsep=\parskip, partopsep=\parskip]}{\end{itemize}}%
 {Section Themes, Concepts, Issues, Competencies, and Skills:}%
%</outcomes-begin>
%<*outcomes-end>
 \par%
 \end{minipage}%
 \par%
%</outcomes-end>



