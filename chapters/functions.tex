%%                                    %%
%% Generated from MathBook XML source %%
%%    on 2014-10-27T21:36:00-07:00    %%
%%                                    %%
\documentclass[10pt,]{article}
%
% Load geometry package to allow page margin adjustments
\usepackage{geometry}
\geometry{letterpaper,total={5.0in,9.0in}}
%% Custom entries to preamble, early

%% Page layout adjustment
\geometry{}
%%
%% Symbols, align environment, bracket-matrix
\usepackage{amsmath}
%% allow more columns to a matrix
%% can make this even bigger by overiding with preamble addition
\setcounter{MaxMatrixCols}{30}
\usepackage{amssymb}
%%
%% XML, MathJax Conflict Macros
%% Two nonstandard macros that MathJax supports automatically
%% so we always define them in order to allow their use and
%% maintain source level compatibility
%% This avoids using two XML entities in source mathematics
\newcommand{\lt}{<}
\newcommand{\gt}{>}
%%
%% Semantic Macros
%% To preserve meaning in a LaTeX file
%% Only defined here if required in this document
%% Environments with amsthm package
\usepackage{amsthm}
% Theorem-like enviroments, italicized statement, proof, etc
% Numbering: X.Y numbering scheme
%   i.e. Corollary 4.3 is third item in Chapter 4 of a book
%   i.e. Lemma 5.6 is sixth item in Section 5 of an article
\theoremstyle{plain}
% Only variants actually used in document appear here
% Numbering: all theorem-like numbered consecutively
%   i.e. Corollary 4.3 follows Theorem 4.2
% definition-like, normal text
\theoremstyle{definition}
\newtheorem{definition}{Definition}
\newtheorem{example}{Example}
\newtheorem{exercise}{Exercise}
%% Raster graphics inclusion, wrapped figures in paragraphs
\usepackage{graphicx}
%% Colors for Sage boxes and author tools (red hilites)
\usepackage[usenames,dvipsnames,svgnames,table]{xcolor}
%% Hyperlinking in PDFs, all links solid and blue
\usepackage[pdftex]{hyperref}
\hypersetup{colorlinks=true,linkcolor=blue,citecolor=blue,filecolor=blue,urlcolor=blue}
\hypersetup{pdftitle={Functions}}
%%
%% Custom entries to preamble, late

%% Convenience macros

	\newcommand\num{}
	\newcommand\sq{\operatorname{sqrt}(}
	%\newcommand\sq[1][]{%
        %\operatorname{sqrt}%
        %\ifstrequal{#1}{(}{% if we want ()
        %    (\phantom{x})%
        %}%
        %{% else
        %    \ifstrempty{#1}{% if #1 is empty then do nothing
        %    }%
        %    {% otherwise put #1 in ()
        %        (#1)%
        %        }%
        %}%
        %}
        
%% Title page information for article
\title{Functions}
\author{Alex Jordan\\
Division of Mathematics and Industrial Technology\\
Portland Community College Sylvania\newline  Portland, Oregon, USA\\
\href{mailto:alex.jordan@pcc.edu}{\nolinkurl{alex.jordan@pcc.edu}}
}
\date{October 27, 2014}
\begin{document}
%
\maketitle
%
\thispagestyle{empty}
%
Functions\typeout{************************************************}
\typeout{Section 1 The Basics of Function Vocabulary}
\typeout{************************************************}
%
\section{The Basics of Function Vocabulary}\label{section-1}
%
Most of us are familiar with the $\sqrt{\phantom{x}}$ symbol.
		This symbols is used to turn numbers into their square roots. Sometimes it's
		simple to do this on paper or in our heads, and sometimes it helps a lot to
		have a calculator. We can see some calculations in 
%
\par The $\sqrt{\phantom{x}}$ symbol signifies a \emph{process}; it's a way for us to
		turn numbers into other numbers. This idea of having a process for turning numbers into other
		numbers is fundamental to the science and mathematics that uses college-level algebra.
%
\par This definition is so broad that you probably use functions all the time.
%
\par The process of using $\sqrt{\phantom{x}}$ to change numbers might feel more ``mathematical''
		than these examples. Let's continue thinking about $\sqrt{\phantom{x}}$ for now, since
		it's a formula-like symbol that we are familiar with. One concern with  $\sqrt{\phantom{x}}$
		is that although we live in the modern age of computers, this symbol is not found on most
		keyboards. And yet computers still tend to be capable of producing square roots. Computer
		technicians write $\sq[(]$ when they want to compute a square root, as we see in .
%
\par  The parentheses in $\sq[(]$ are very important. To see why, try to put yourself in the
		``mind'' of a computer, and look closely at $\sq\num{16}$. The computer will recognize $\sq$
		and know that it needs to compute a square root. But sometimes computers have myopic vision and 
		they might not see the entire number $\num{16}$. A computer might think that it needs to compute 
		$\sq\num{1}$ and then append a ``6'' to the end, which would produce a final result of $\num{16}$. 
		This is probably not what was intended. And so the purpose of the parentheses in $\sq[\num{16}]$ is 
		to denote exactly what number needs to be operated on.
%
\end{document}
