\typeout{************************************************}
\typeout{Chapter 1 Functions}
\typeout{************************************************}
%
\chapter{Functions}\label{}
%
\minitoc
%
\typeout{************************************************}
\typeout{Section 1.1 The Basics of Function Vocabulary}
\typeout{************************************************}
%
\section{The Basics of Function Vocabulary}\label{}
%
\begin{outcomes}
\begin{outcomelist}
\item You will understand the definition of a function.%
\item You will be able to use standard notation for functions
	                correctly, and recognize when notation has been used incorrectly.%
\item You will recognize some everyday examples of functions.%
\end{outcomelist}
\end{outcomes}
Most of us are familiar with the $\sqrt{\phantom{x}}$ symbol.
		This symbols is used to turn numbers into their square roots. Sometimes it's
		simple to do this on paper or in our heads, and sometimes it helps a lot to
		have a calculator. We can see some calculations in \cref{fun-tab-squareroots}.
%
\begin{margintable}\centering
\captionof{table}{Values of $\sqrt{x}$}
\label{fun-tab-squareroots}
			
			\begin{tabular}{r@{}c@{}l}
\beforeheading 
\afterheading 
$\sqrt{\num{9}}$&${}={}$&\num{3}\\\normalline
$\sqrt{\num{1/4}}$&${}={}$&\num{1/2}\\\normalline
$\sqrt{\num{2}}$&${}\approx{}$&\num{1.41}\ldots\\\lastline
\end{tabular}

		\end{margintable}
%
\par The $\sqrt{\phantom{x}}$ symbol represents a \emph{process}; it's a way for us to
		turn numbers into other numbers. This idea of finding some numbers based on other numbers is
		fundamental to science and mathematics that use college-level algebra. 
%
\begin{definition}[Function]\label{}
A function is a process for turning numbers into (potentially) different numbers. 
			It's important that any input consistently produces the same output.\end{definition}
%
\par This definition is so broad that you probably use functions all the time.
%
\begin{example}\label{}
Think about each of these examples. How do they fit the defintion of a function?
%
\begin{itemize}
\item If you use the year a person was born to determine how old they are, you are using a function.%
\item If you look up the Kelly Blue Book value of a Mazda Proteg\'{ e} based on how old it is, 
				you are using a function.%
\item If you use the the amount of money that you have on you to determine how many beers you could buy for 
				your friends at the bar, you are using a function.%
\end{itemize}
%
\end{example}
%
\par The process of using $\sqrt{\phantom{x}}$ to change numbers might feel more ``mathematical''
		than these examples. Let's continue thinking about $\sqrt{\phantom{x}}$ for now, since
		it's a formula-like symbol that we are familiar with. Even though we live in the age of computers,
		this symbol is not found on most
		keyboards. This doesn't stop people from using computers to calculate square roots though. Computer
		technicians write $\sq[(]$ when they want to compute a square root, as we see in \cref{fun-tab-sqrts}.
%
\begin{margintable}\centering
\captionof{table}{Values of $\sq[x]$}
\label{fun-tab-sqrts}
                        
                        \begin{tabular}{r@{}c@{}l}
\beforeheading 
\afterheading 
$\sq[\num{9}]$&${}={}$&\num{3}\\\normalline
$\sq\left(\num{1/4}\right)$&${}={}$&\num{1/2}\\\normalline
$\sq[\num{2}]$&${}\approx{}$&\num{1.41}\ldots\\\lastline
\end{tabular}

                \end{margintable}
%
\par The parentheses in $\sq[(]$ are very important. To see why, try to put yourself in the
		``mind'' of a computer, and look closely at $\sq\num{16}$. The computer will recognize $\sq$
		and know that it needs to compute a square root. But computers are very picky with how they interpret input, and 
		they might not see the entire number $\num{16}$. A computer might read as far as $\sq\num{1}$ and think that it needs to compute this.
		That would leave $1$ with a ``6'' character still hanging. The final result would be $\num{16}$, and we know we inteded the  
		result to be $4$. And so the purpose of the parentheses in $\sq[\num{16}]$ is 
		to denote exactly what number needs to be operated on.
%
\par This use of $\sq[(]$ serves as a model for the standard notation that is used worldwide to
        	write down most functions. By having a standard notation for communicating about functions,
        	people in China, Venezuela, Senegal, and the United States can all communicate mathematics
        	with each other more easily.
%
\par Functions have their own names. We've seen a function named $\sq$, but any name you can
        	imagine is allowable. In the sciences, it is common to name functions with whole words,
        	like $\operatorname{weight}$ or $\operatorname{health\_index}$. In mathematics, we often
        	abbreviate such function names to $w$ or $h$. And of course, since the word ``function''
        	itself starts with ``f'', we will often name a function $f$.
%
\par It's crucial to continue reminding ourselves that functions are \emph{processes} for
        	changing numbers; they are not numbers themselves. And that means that we have a potential
        	for confusion that we need to stay aware of. In some contexts, the symbol $t$ might
        	represent a variable---a number that is represented by a letter. For example, it might represent 
		how much time has passed since something started. But in other contexts, $t$
        	might represent a function---a process for changing numbers into other numbers. For example, if you 
		have a year in mind, $t$ might be the function that tells you how many tornadoes there were that year. 
		So $t$ would be a process for turning years into numbers of tornadoes. By
        	staying conscious of the context of an investigation, we avoid confusion.
%
\par Next we need to discuss how we go about using a function's name.
%
