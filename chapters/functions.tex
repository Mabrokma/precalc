\typeout{************************************************}
\typeout{Chapter 1 Functions}
\typeout{************************************************}
%
\chapter{Functions}\label{}
%
\minitoc
%
\typeout{************************************************}
\typeout{Section 1.1 The Basics of Function Vocabulary}
\typeout{************************************************}
%
\section{The Basics of Function Vocabulary}\label{}
%
\begin{outcomes}
\begin{outcomelist}
\item You will have an understanding of the definition of a function.%
\item You will be able to use standard notation concerning functions
	                correctly, and recognize when notation has been used incorrectly.%
\item You will recognize some real examples of functions in your life.%
\end{outcomelist}
\end{outcomes}
Most of us are familiar with the $\sqrt{\phantom{x}}$ symbol.
		This symbols is used to turn numbers into their square roots. Sometimes it's
		simple to do this on paper or in our heads, and sometimes it helps a lot to
		have a calculator. We can see some calculations in \cref{fun-tab-squareroots}
%
\begin{margintable}\centering
\captionof{table}{Values of $\sqrt{x}$}
\label{fun-tab-squareroots}
			
			\begin{tabular}{r@{}c@{}l}
\beforeheading 
\afterheading 
$\sqrt{\num{9}}$&${}={}$&\num{3}\\\normalline
$\sqrt{\num{1/4}}$&${}={}$&\num{1/2}\\\normalline
$\sqrt{\num{2}}$&${}\approx{}$&\num{1.41}\ldots\\\lastline
\end{tabular}

		\end{margintable}
%
\par The $\sqrt{\phantom{x}}$ symbol signifies a \emph{process}; it's a way for us to
		turn numbers into other numbers. This idea of having a process for turning numbers into other
		numbers is fundamental to the science and mathematics that uses college-level algebra.
%
\begin{definition}[Function]\label{}
A function is a process for turning numbers into (potentially) different numbers. 
			It's important that any input consistently produces the same output.\end{definition}
%
\par This definition is so broad that you probably use functions all the time.
%
\begin{example}\label{}
Think about each of these examples, where some process is used for turning one number into another.
%
\begin{itemize}
\item If you use a person's birth year to determine how old they are, you are using a function.%
\item If you look up the Kelly Blue Book value of a Mazda Proteg\'{ e} based on how old it is, you are using a function.%
\item If you use the the amount of money that you have available to determine how much
        you wish to spend on a birthday gift for your friend, you are using a function.%
\end{itemize}
%
\end{example}
%
\par The process of using $\sqrt{\phantom{x}}$ to change numbers might feel more ``mathematical''
		than these examples. Let's continue thinking about $\sqrt{\phantom{x}}$ for now, since
		it's a formula-like symbol that we are familiar with. One concern with  $\sqrt{\phantom{x}}$
		is that although we live in the modern age of computers, this symbol is not found on most
		keyboards. And yet computers still tend to be capable of producing square roots. Computer
		technicians write $\sq[(]$ when they want to compute a square root, as we see in \cref{fun-tab-sqrts}.
%
\begin{margintable}\centering
\captionof{table}{Values of $\sq[x]$}
\label{fun-tab-sqrts}
                        
                        \begin{tabular}{r@{}c@{}l}
\beforeheading 
\afterheading 
$\sq[\num{9}]$&${}={}$&\num{3}\\\normalline
$\sq\left(\num{1/4}\right)$&${}={}$&\num{1/2}\\\normalline
$\sq[\num{2}]$&${}\approx{}$&\num{1.41}\ldots\\\lastline
\end{tabular}

                \end{margintable}
%
\par  The parentheses in $\sq[(]$ are very important. To see why, try to put yourself in the
		``mind'' of a computer, and look closely at $\sq\num{16}$. The computer will recognize $\sq$
		and know that it needs to compute a square root. But sometimes computers have myopic vision and 
		they might not see the entire number $\num{16}$. A computer might think that it needs to compute 
		$\sq\num{1}$ and then append a ``6'' to the end, which would produce a final result of $\num{16}$. 
		This is probably not what was intended. And so the purpose of the parentheses in $\sq[\num{16}]$ is 
		to denote exactly what number needs to be operated on.
%
