\chapter{Functions}
\minitoc
\section{The Basics of Function Vocabulary}
\begin{outcomes}
	\begin{outcomelist}
		\item You will have an understanding of the definition of a function. 
		\item You will be able to use standard notation concerning functions 
		correctly, and recognize when notation has been used incorrectly.
		\item You will recognize some real examples of functions in your life.
	\end{outcomelist}
\end{outcomes}
%
Most of us are familiar with the $\sqrt{\phantom{x}}$ symbol. 
This symbols is used to turn numbers into their square roots. Sometimes it's 
simple to do this on paper or in our heads, and sometimes it helps a lot to 
have a calculator. We can see some calculations in \cref{fun:tab:squareroots}.
%
\begin{margintable}
	\centering
	\captionof{table}{Values of $\sqrt{x}$} \label{fun:tab:squareroots}
	\begin{tabular}{r@{}c@{}l}
		\beforeheading
		\afterheading
		$\sqrt{\num{9}}$   & ${}={}$       & $\num{3}$    \\\normalline
		$\sqrt{\num{1/4}}$ & ${}={}$       & $\num{1/2}$  \\\normalline
		$\sqrt{\num{2}}$   & ${}\approx{}$ & $\num{1.41}$ \\\lastline
	\end{tabular}
\end{margintable}
%
The $\sqrt{\phantom{x}}$ symbol signifies a \emph{process}; it's a way for us to 
turn numbers into other numbers. This idea of having a process for turning numbers into other 
numbers is fundamental to the science and mathematics that uses college-level algebra. 
%
\begin{pccdefinition}[Function]
A function is a process for turning numbers into (potentially) different numbers.
\end{pccdefinition}
%
This definition is so broad that you probably use functions all the time. 

%===================================
%   Author: Jordan
%   Date:   Sep/Oct 2012
%===================================
\begin{pccexample}
Think about each of these examples, where some process is used for turning one number into another. 
\begin{itemize}
	\item If you use a person's birth year to determine how old they are, you are using a function.
	\item If you look up the Kelly Blue Book value of a Mazda Proteg\'{e} based on how old it is, you are using a function.  
	\item If you use the the amount of money that you have available to determine how much 
	you wish to spend on a birthday gift for your friend, you are using a function. 
\end{itemize}
\end{pccexample}

The process of using $\sqrt{\phantom{x}}$ to change numbers might feel more ``mathematical'' 
than these examples. Let's continue thinking about $\sqrt{\phantom{x}}$ for now, since 
it's a formula-like symbol that we are familiar with. One concern with  $\sqrt{\phantom{x}}$ 
is that although we live in the modern age of computers, this symbol is not found on most 
keyboards. And yet computers still tend to be capable of producing square roots. Computer 
technicians write $\sq[(]$ when they want to compute a square root, as we see in \cref{fun:tab:sqrts}.
%
\begin{margintable}
	\centering
	\captionof{table}{Values of $\sqrt{x}$} \label{fun:tab:sqrts}
	\begin{tabular}{r@{}c@{}l}
		\beforeheading
		\afterheading
		$\sq[\num{9}]$              & ${}={}$       & $\num{3}$    \\\normalline
		$\sq\left(\num{1/4}\right)$ & ${}={}$       & $\num{1/2}$  \\\normalline
		$\sq[\num{2}]$              & ${}\approx{}$ & $\num{1.41}$ \\\lastline
	\end{tabular}\end{margintable}
	%
	The parentheses in $\sq[(]$ are very important. To see why, try to put yourself in the 
	``mind'' of a computer, and look closely at 
	\[
		\sq\num{16}
	\]
	The computer will recognize $\sq$ and know that it needs to compute a square root. 
	But sometimes computers have myopic vision and they might not see the entire number \num{16}. A 
	computer might think that it needs to compute $\sq\num{1}$ and then append a ``6'' to the end, which 
	would produce a final result of \num{16}. This is probably not what was intended. And so 
	the purpose of the parentheses in $\sq[\num{16}]$ is to denote exactly what number needs to be operated on. 
	
	This use of $\sq[(]$ serves as a model for the standard notation that is used worldwide to 
	write down most functions. By having a standard notation for communicating about functions, 
	people in China, Venezuela, Senegal, and the United States can all communicate mathematics 
	with each other more easily. 
	
	Functions have their own names. We've seen a function named $\sq$, but any name you can 
	imagine is allowable. In the sciences, it is common to name functions with whole words, 
	like $\operatorname{weight}$ or $\operatorname{health\_index}$. In mathematics, we often
	abbreviate such function names to $w$ or $h$. And of course, since the word ``function''
	itself starts with ``f'', we will often name a function $f$. 
	
	It's crucial to continue reminding ourselves that functions are \emph{processes} for
	changing numbers; they are not numbers themselves. And that means that we have a potential
	for confusion that we need to stay aware of. In some contexts, the symbol $t$ might
	represent a variable - a number that is represented by a letter. But in other contexts, $t$
	might represent a function - a process for changing numbers into other numbers. By
	staying conscious of the context of an investigation, we avoid confusion.
	
	Next we need to discuss how we go about using a function's name.
	%
	\begin{pccspecialcomment}[Function notation]
		The standard notation for referring to functions involves giving the function itself a name, and then writing
		\[
			\begin{array}{cc}
				\text{name}\\
				\text{of}\\
				\text{function}
			\end{array}
			\left(
			\begin{array}{cc}
				\\
				\text{input}\\
				\\						
			\end{array}\right)
		\]
	\end{pccspecialcomment}
	%
	%===================================
	%   Author: Jordan
	%   Date:   Sep/Oct 2012
	%===================================
	\begin{pccexample}
	$f(\num{13})$ is pronounced ``f of 13''. The word ``of'' is very important, 
	because it reminds us that $f$ is a process and we are about to apply that
	process to the input value \num{13}. So $f$ is the function, \num{13} is the
	input, and $f(\num{13})$ is the output we'd get from using \num{13} as input.
	
	$f(x)$ is pronounced ``f of x''. This is just like the previous example,
	except that the input is not any specific number. The value of $x$ could be
	\num{13} or any other number. Whatever $x$'s value, $f(x)$ means the corresponding
	output from the function $f$.
	
	$\operatorname{BudgetDeficit}(2009)$ is pronounced ``BudgetDeficit of 2009''.
	This is probably about a function that takes a year as input, and gives that
	year's federal budget deficit as output. The process here of changing a year
	into a dollar amount might not involve any mathematical formula, but rather
	looking up information from the Congressional Budget Office's website.
	
	$\operatorname{Celsius}(F)$ is pronounced ``Celsius of F''. This is probably
	about a function that takes a Fahrenheit temperature as input and gives the
	corresponding Celsius temperature as output. Maybe a formula is used to do this; 
	maybe a chart or some other tool is used to do this. Here, $\operatorname{Celsius}$
	is the function, $F$ is the input variable, and $\operatorname{Celsius}(F)$ is the output from the function.
	\end{pccexample}
	%
	\begin{pccspecialcomment}[Function Notation (continued)] 
		While a function has a name like $f$, and the input to that function often
		has a variable name like $x$, the expression $f(x)$ represents the output of
		the function. To be clear, $f(x)$ is \emph{not} a function. Rather, $f$ is a
		function, and $f(x)$ its output when the number $x$ was used as input.
	\end{pccspecialcomment}
	%
	As mentioned earlier, we need to remain conscious of the context of any symbol we
	are using. It's possible for $f$ to represent a function (a process), but it's also
	possible for $f$ to represent a variable (a number). Similarly, parentheses might
	indicate the input of a function, or they might indicate that two numbers need to
	be multiplied. It's up to our judgment to interpret mathematical expressions in the
	right context. Consider the expression $a(b)$. This could easily mean the output of
	a function $a$ with input $b$. It could also mean that two numbers $a$ and $b$ need
	to be multiplied. It all depends on the context in which these symbols are being used.
	%
	\begin{doyouunderstand}
		\begin{problem}
		Describe your own example of a function that has real context to it. You will
		need some kind of input variable, like ``number of years since 2000'' or
		``weight of the passengers in my car''. You will need a process for using that
		number to bring about a different kind of number. The process does not need to
		involve a formula; a verbal description would be great, as would a formula.
		
		Give your function a name. Write the symbol(s) that you would use to represent
		input. Write the symbol(s) that you would use to represent output.
		\begin{shortsolution}
			Answers will vary.
		\end{shortsolution}
		\end{problem}
	\end{doyouunderstand}
	%
	Sometimes it's helpful to think of a function as a machine. 
	\fixthis{Fancy tikz picture of an input-output machine}
	This illustrates how complicated functions are. A number is just that - a number.
	But a function has the capacity to take in all kinds of different numbers into
	it's hopper (feeding tray) and do different things to each of them.
	
	\subsection{Tables and Graphs}
	Since functions are potentially so complicated, we seek out new ways to understand
	them better. Two basic tools for understanding a function better are tables and graphs.
	%
	%===================================
	%   Author: Jordan
	%   Date:   Sep/Oct 2012
	%===================================
	\begin{pccexample} \label{fun:ex:BudgetDeficit}
	Consider the function $\operatorname{BudgetDeficit}$, that takes in a year as
	its input and outputs the US federal budget deficit for that year. For example,
	the Congressional Budget Office's website tells us that
	$\operatorname{BudgetDeficit}(2009)$ is $\SI{1.41}[\$]{\trillion}$. If we'd like
	to understand this function better, we might make a table of all the inputs and
	outputs we can find. Using the CBO's website\footnote{\href{http://data.bls.gov/timeseries/LNS14000000}{Congressional Budget Office}}, we put together \cref{fun:tab:BudgetDeficit}.
	
	\begin{table}[!htb]
		\centering
		\caption{}
		\label{fun:tab:BudgetDeficit}
		\begin{tabular}{S[table-format=4.0]S[table-format=1.2]}
			\beforeheading
			\heading{input}      & \heading{output}                                               \\
			\heading{$x$ (year)} & \heading{$\operatorname{BudgetDeficit}(x)$ (\si{\$\trillion})} \\
			\afterheading
			2007                 & 0.16                                                           \\\normalline
			2008                 & 0.46                                                           \\\normalline
			2009                 & 1.41                                                           \\\normalline
			2010                 & 1.29                                                           \\\normalline		
			2011                 & 1.30                                                           \\\lastline
		\end{tabular}
	\end{table}
	How is this table helpful? There are things about the function that we can see now by looking at the numbers in this table.
	\begin{itemize}
		\item We can see that the budget deficit has grown by quite a bit over the entire five-year period.
		\item We can see that there was a particularly large jump in 2008.
		\item We can see that the deficit reduced by a little bit between 2009 and 2010, and then remained stable. 
	\end{itemize}
	These observations serve to help us understand the function $\operatorname{BudgetDeficit}$ a little better.
	\end{pccexample} 
	%
	%===================================
	%   Author: Jordan
	%   Date:   Sep/Oct 2012
	%===================================
	\begin{pccexample} \label{fun:ex:sqrttable}
	Let's return to our example of the function $\sq$. Tabulating some inputs and outputs reveals \cref{fun:tab:sqrtexample}.
	
	\begin{table}[!htb]
		\centering
		\caption{}
		\label{fun:tab:sqrtexample}
		\begin{tabular}{S[table-format=1.0]c}
			\beforeheading
			\heading{input} & \heading{output}    \\
			\heading{$x$}   & \heading{$\sq[x]$}  \\
			\afterheading
			0               & \num{0}             \\\normalline
			1               & \num{1}             \\\normalline
			2               & $\approx\num{1.41}$ \\\normalline
			3               & $\approx\num{1.73}$ \\\normalline		
			4               & \num{2}             \\\lastline
		\end{tabular}
	\end{table}
	How is this table helpful? Here are some observations that we can make now.
	\begin{itemize}
		\item We can see that when input numbers increase, so do output numbers.
		\item We can see even though outputs are increasing, they increase by less and less with each step forward in $x$.
	\end{itemize}
	These observations help us understand $\sq$ a little better. For instance, 
	based on these observations which do you think is larger: the difference
	between $\sq[23]$ and $\sq[24]$, or the difference between $\sq[85]$ and $\sq[86]$?
	\end{pccexample} 
	%
	Another powerful tool for understanding functions better is a graph. 
	Given a function $f$, one way to make its graph is to take a table of input and 
	output values, and read each row as the coordinates of a point in the $xy$-plane.
	%===================================
	%   Author: Jordan
	%   Date:   Sep/Oct 2012
	%===================================
	\begin{pccexample}
	Returning to the function $\operatorname{BudgetDeficit}$ that we
	studied in \cref{fun:ex:BudgetDeficit}, in order to make a graph of this function
	we view \cref{fun:tab:BudgetDeficit} as a list of points with $x$ and $y$ coordinates,
	as in \cref{fun:tab:BudgetDeficitCoords}. We then plot these points on a set of
	coordinate axes, as in \cref{fun:fig:BudgetDeficit}. The points have been connected
	with a curve so that we can see the overall pattern given by the progression of points.
	Since there was not any actual data for inputs in between any two years, the curve is
	dashed. That is, this curve is dashed because it just represents someone's best guess
	as to how to connect the plotted points. Only the plotted points themselves are precise.
	\begin{table}[!htb]
		\begin{minipage}{.4\textwidth}
			\centering
			\captionof{table}{}
			\label{fun:tab:BudgetDeficitCoords}
			\begin{tabular}{c}
				\beforeheading
				\heading{(input, output)}\\
				\heading{$(x,\operatorname{BudgetDeficit}(x))$} \\
				\afterheading
				$(2007, 0.16)$                   \\ \normalline 
				$(2008, 0.46)$                   \\ \normalline 
				$(2009, 1.41)$                   \\ \normalline 
				$(2010, 1.30)$                   \\ \normalline 
				$(2011, 1.29)$                   \\ \lastline
			\end{tabular}
		\end{minipage}%
		\begin{minipage}{.6\textwidth}
			\centering
			\begin{tikzpicture}
				\begin{axis}[
					   xmin=2006,xmax=2012,
					   ymin=0,ymax=2,
					   xlabel={$x$ (year)},
					   ylabel={$y$ (\$ trillion)},
					   height=0.5\textwidth,
					   grid=major,
					   xtick={2007,2008,...,2012},
					   xticklabels={2007,2008,...,2011},
					   minor ytick={0,0.2,...,2},
					   ytick={0,1,...,2},
					   axis x discontinuity=crunch,
					   nodes near coords,
					   ]
					\addplot+[soldot,smooth,dashed,-] coordinates { 
					(2007,0.16)
					(2008,0.46)
					(2009,1.41)
					(2010,1.30)
					(2011,1.29)};					
				\end{axis}
			\end{tikzpicture}		
			\captionof{figure}{$y=\operatorname{BudgetDeficit}(x)$}
			\label{fun:fig:BudgetDeficit}
		\end{minipage}%
	\end{table}
	
	How has this graph helped us to understand the function better? All of the
	observations that we made in \cref{fun:ex:BudgetDeficit} are perhaps even
	more clear now. For instance, the spike in the deficit between 2008 and 2009
	is now visually apparent. Seeking an explanation for this spike, we recall
	that there was a financial crisis in late 2008. Revenue from income taxes
	dropped at the same time that federal money was spent to prevent further losses.
	\end{pccexample}
	%
	%===================================
	%   Author: Jordan
	%   Date:   Sep/Oct 2012
	%===================================
	\begin{pccexample}
	Let's now construct a graph for $\sq$. Tabulating inputs and outputs gives the
	points in \cref{fun:tab:sqrtCoords}, which in turn gives us the graph in \cref{fun:fig:sqrt}.
	\begin{table}[!htb]
		\begin{minipage}{.4\textwidth}
			\centering
			\captionof{table}{}
			\label{fun:tab:sqrtCoords}
			\begin{tabular}{c}
				\beforeheading
				\heading{(input, output)}\\
				\heading{$(x,\sq(x))$} \\
				\afterheading           
				$(0, 0)$                  \\ \normalline 
				$(1, 1)$                  \\ \normalline
				$\approx(2, 1.41)$        \\ \normalline          
				$\approx(3, 1.73)$        \\ \normalline            
				$(4, 2)$                  \\ \lastline
			\end{tabular}
		\end{minipage}%
		\begin{minipage}{.6\textwidth}
			\centering
			\begin{tikzpicture}
				\begin{axis}[
					   xmin=-1,xmax=5,
					   ymin=0,ymax=3,
					   height=0.5\textwidth,
					   grid=major,
					   xtick={-1,0,...,5},
					   xticklabels={-1,0,...,5},
					   ytick={0,1,...,3},
					   ]
					\addplot+[->,dashed]expression[domain=0:4.5,samples=50] {sqrt(x)};
					  \addplot[soldot]coordinates{ (0,0) (1,1) (2,1.414) (3,1.732) (4,2)};				
				\end{axis}
			\end{tikzpicture}		
			\captionof{figure}{$y=\operatorname{BudgetDeficit}(x)$}
			\label{fun:fig:sqrt}
		\end{minipage}%
	\end{table}
	Just as in the previous example, we've plotted points where we have concrete
	coordinates, and then we have made our best attempt to connect those points
	with a curve. Unlike the previous example, here we believe that points could
	continue to be computed and plotted indefinitely to the right, and so we
	have added an arrowhead to the graph.
	
	What has this graph done to improve our understanding of $\sq$?  As inputs
	($x$-values) increase, the outputs ($y$-values) increase too, although not at
	the same rate. In fact we can see that our graph is steep on its left, and less
	steep as we move to the right. This confirms our earlier observation in
	\cref{fun:ex:sqrttable} that outputs increase by smaller and smaller amounts
	as the input increases. 
	\end{pccexample}
	%
	\begin{pccspecialcomment}[The graph of a function]
		Given a function $f$, when we refer to a \emph{graph of $f$} we are \emph{not}
		referring to an entire picture, like \cref{fun:fig:sqrt}. A graph of $f$ is only
		\emph{part} of that picture - the curve and the points that it connects.
		Everything else: axes, tick marks, the grid, labels, and the surrounding white
		space is just useful decoration, so that we can read the graph more easily.
		
		It is also common to refer to the graph of $f$ as the graph of the \emph{equation}
		$y=f(x)$. However we should never refer to ``the graph of $f(x)$''. That would
		indicate a fundamental misunderstanding of our notation. We have decided that $f(x)$
		is the output for a certain input $x$. That means that $f(x)$ is just a number; a
		relatively uninteresting thing compared to $f$ the function, and not worthy of
		any two-dimensional picture.
	\end{pccspecialcomment}
	
	While it is important to be able to make a graph of a function $f$, we also need to be capable of looking at a graph and reading it well. A graph of $f$ provides us with helpful specific information about $f$; it tells us what $f$ does to its input values. When we were making graphs, we plotted points of the form 
	\[
		(\text{input},\text{output})
	\]
	Now given a graph of $f$, we interpret coordinates in the same way.
	
	\begin{figure}[!htb]
		\centering
		\begin{tikzpicture}
			\begin{axis}[
                    framed,
				   xmin=-1,xmax=5,
				   ymin=-1,ymax=4,
				   width=0.3\textwidth,
				   grid=major,
				   xtick={1,...,4},
				   ytick={1,...,3},
				   ]
				  \addplot+[smooth] coordinates{ (-1,0.5) (0,0.5) (1,2) (2,3) (3,3) (4,2) (5,1.5)};	
				  \addplot+[dashed,->] coordinates{ (-2,0) (1,0) (1,2) };			
			\end{axis}
		\end{tikzpicture}		
		\caption{$y=f(x)$}
		\label{fun:fig:readgraph}
	\end{figure}%
	
	In \cref{fun:fig:readgraph} we have a graph of a function $f$. If we wish to find $f(1)$, we recognize that \num{1} is being used as an input. So we would want to find a point of the form $(1,\phantom{y})$. Seeking out $x$-coordinate \num{1} in \cref{fun:fig:readgraph}, we find that the only such point is $(1,2)$. Therefore the output for \num{1} is \num{2}; in other words $f(1)=2$.
	
	\begin{doyouunderstand}
		\begin{problem}
		Use the graph of $f$ in \cref{fun:fig:readgraph} to find $f(0)$, $f(3)$, and $f(4)$.
		\begin{shortsolution}
			$f(0)=0.5$, $f(3)=3$, and $f(4)=2$
		\end{shortsolution}
		\end{problem}
	\end{doyouunderstand}
	
	%===================================
	%   Author: Jordan
	%   Date:   Sep/Oct 2012
	%===================================
	\begin{pccexample}\label{fun:ex:unemployment}
	Suppose that $u$ is the unemployment function of time. That is, $u(t)$ is the unemployment rate in the United States in year $t$. The graph of the equation $y=u(t)$ is given in \cref{fun:fig:unemployment}\footnote{\href{http://data.bls.gov/timeseries/LNS14000000}{US Bureau of Labor Statistics} }.
	
	\begin{figure}[!htb]
		\centering
		\begin{tikzpicture}
			\begin{axis}[
				   xmin=2001,xmax=2012,
				   ymin=0,ymax=10,
				   xlabel={$t$},
				   ylabel={$y$ (\si{\percent})},
				   width=0.8\textwidth,
				   height=0.25\textwidth,
				   grid=major,
				   axis line style=->,
				   xtick={2002,2003,...,2012},
				   ytick={0,2,...,10},
				     axis x discontinuity=crunch,
                     x tick label style={/pgf/number format/1000 sep=},
				   	 ]
				  \addplot+[smooth] coordinates{ (2002,5.78) (2003,5.99) (2004,5.54) (2005,5.08) (2006,4.61) (2007,4.62) 
				                                 (2008,5.80) (2009,9.28) (2010,9.63) (2011,8.95) (2012,8.17)};				
			\end{axis}
		\end{tikzpicture}		
		\caption{Unemployment in the United States}
		\label{fun:fig:unemployment}
	\end{figure}%
	
	What was the unemployment in 2002? It is a straightforward matter to use \cref{fun:fig:unemployment} to find that unemployment was about \SI{6}{\percent} in 2002. Asking this question is exactly the same thing as asking to find $u(2002)$. That is, we have one question that can either be asked in a everyday-English way or which can be asked in a terse, mathematical notation-heavy way:
	\begin{align*}
		\text{``What was unemployment in 2002?''} &   & \text{``Find $u(2002)$.''} 
	\end{align*}
	If we use the table to establish that $u(2009)\approx9.25$, then we should be prepared to translate that into everyday-English using the context of the function: In 2009, unemployment in the U.S.\ was about \SI{9.25}{\percent}.
	
	If we ask the question ``when was unemployment at \SI{5}{\percent}'', we can read the graph and see that there were two such times: early 2005 and mid-2007. But there is again a more mathematical notation-heavy way to ask this question. Namely, since we are being told that the output of $u$ is \num{5}, we are being asked to solve the equation $u(t)=5$. So the following communicate the same thing:
	\begin{align*}
		\text{``When was unemployment at \SI{5}{\percent}?''} &   & \text{``Solve the equation $u(t)=5$.''} 
	\end{align*}
And our answer to this question is:
	\begin{align*}
		\text{``Unemployment was at \SI{5}{\percent} in early 2005 and mid-2007''} &   & \text{``$t\approx2005$ or $t\approx2007.5$''} 
	\end{align*}
	
	\end{pccexample}
	
	\begin{doyouunderstand}
		Use the graph of $u$ in \cref{fun:fig:unemployment} to respond to the following.
		\begin{problem}
		Find $u(2011)$ and interpret it.
		\begin{shortsolution}
			$u(2011)\approx9$; In 2011 the US unemployment rate was about \SI{9}{\percent}.
		\end{shortsolution}
		\end{problem}
		\begin{problem}
		Solve the equation $u(t)=6$ and interpret your solution(s).
		\begin{shortsolution}
			$t\approx2003$ or $t\approx2008$; The points at which unemployment was \SI{6}{\percent} were in early 2003 and early 2008.
		\end{shortsolution}
		\end{problem}
	\end{doyouunderstand}
	
	
	
	\subsection{Translating Between Four Descriptions of the Same Function}
	
	We have noted that functions are complicated, and that we are seeking out ways to make them easier to understand. It's common to encounter a problem involving a function and not know how to proceed toward a solution to that problem. As it happens, most functions have at least four standard ways to think about them. If we become capable of translating between these four perspectives, perhaps we will find that one of them makes a given  problem easy to solve.
	
	\begin{figure}[!htb]
		\centering
   \begin{tikzpicture}[function/.style={draw, fill=blue!20, text width=5em,
      text centered, minimum height=2.5em,drop shadow},
      functionname/.style={draw, fill=blue!20, circle,text width=6em,
      text centered, minimum height=2.5em,drop shadow}]
 
      % Draw diagram elements
      \node (f) [functionname]  {A function, $f$};
      \node (verbal) [function,above=of f]  {Verbal Description};
      \node (tabular) [function,left=of f]  {Table of Inputs and Outputs};
      \node (graphical)[function,right=of f] {Graph};
      \node (formular)[function,below=of f] {Formula};
  
      % Draw arrows between elements
      \draw[<->] (f) -- (formular) ;
      \draw[<->] (f) -- (verbal);
      \draw[<->] (f) -- (graphical);
      \draw[<->] (f) -- (tabular);
      \draw[<->] (verbal.west) to[out=180,in=90](tabular.north) ;
      \draw[<->] (verbal.east) to[out=0,in=90](graphical.north) ;
      \draw[<->] (tabular.south) to[out=-90,in=180](formular.west) ;
      \draw[<->] (graphical.south)to[out=-90,in=0](formular.east);

      % Draw background
      \begin{pgfonlayer}{background}
            % Left-top corner of the background rectangle
            \path (tabular.west |- verbal.north)+(-0.5,0.5) node (a) {};
            % Right-bottom corner of the background rectanle
            \path (graphical.east |- formular.south)+(+0.5,-0.5) node (c) {};
            % Draw the background
            \path[fill=yellow!20,rounded corners, draw=black!50, dashed]
                  (a) rectangle (c);
      \end{pgfonlayer}
   \end{tikzpicture}
		\caption{Translating Between Perspectives}
\label{fun:fig:ruleoffour}
	\end{figure}

	The four modes for working with a given function are: 
	\begin{multicols}{2}
		\begin{itemize}
			\item a verbal description
			\item a table of inputs and outputs
			\item a graph of the function
			\item a formula for the function
		\end{itemize}
	\end{multicols}
    This has been visualized in \cref{fun:fig:ruleoffour}.
	

	%===================================
	%   Author: Jordan
	%   Date:   Sep/Oct 2012
	%===================================
	\begin{pccexample}
	Consider a function $f$ that squares its input and then adds \num{1}. Translate this verbal description of $f$ into a table, a graph, and a formula.
	
	\begin{pccsolution}
	To make a table for $f$, we'll have to select some input $x$-values. These choices are left entirely up to us, so we might as well choose small, easy-to-work-with values. However we shouldn't shy away from negative input values. Given the verbal description, we should be able to compute a column of output values. 
    \Cref{fun:tab:ruleoffourexample} is one possible table that we might end up with.
	
	Once we have a table for $f$, it's a simple matter to make a graph for $f$ as in \cref{fun:fig:ruleoffourexample}, using the table to plot points. 
	
	\begin{table}[!htb]
		\begin{minipage}{.4\textwidth}
			\centering
			\captionof{table}{}
			\label{fun:tab:ruleoffourexample}
			\begin{tabular}{S[table-format=1.0]r}
				\beforeheading
				\heading{$x$} & \heading{$f(x)$}               \\ 
				\afterheading
				-2            & $(\num{-2})^2+\num{1}=\num{5}$ \\\normalline
				-1            & $(\num{-1})^2+\num{1}=\num{2}$ \\\normalline
				0             & $\num{0}^2+\num{1}=\num{1}$    \\\normalline
				1             & $\num{1}^2+\num{1}=\num{2}$    \\\normalline
				2             & \num{5}                        \\\normalline
				3             & \num{10}                       \\\normalline
				4             & \num{17}                       \\\lastline
			\end{tabular}
		\end{minipage}%
		\begin{minipage}{.6\textwidth}
			\centering
			\begin{tikzpicture}
				\begin{axis}[
                        framed,
					   xmin=-3,xmax=5,
					   ymin=-5,ymax=20,
					   height=0.618\textwidth,
					   grid=major,
					   xtick={-2,...,4},
					   ytick={5,10,...,15},
					   minor ytick={0,1,...,20},
					   ]
					\addplot+[dashed]expression[domain=-2.3:4.3,samples=50] {x^2+1};
					  \addplot[soldot]coordinates{ (-2,5) (-1,2) (0,1) (1,2) (2,5) (3,10) (4,17)};				
				\end{axis}
			\end{tikzpicture}		
            \captionof{figure}{$y=f(x)$}
			\label{fun:fig:ruleoffourexample}
		\end{minipage}%
	\end{table}
	
Lastly, we must find a formula for $f$. This means we need to write a mathematical expression that says the same thing about $f$ as the verbal description, the table, and the graph. For this example, we can focus on the verbal description. Since $f$ takes its input, squares it, and adds \num{1}, then we have that 
\[
f(x)=x^2+1
\]

\end{pccsolution}
\end{pccexample}

%===================================
%   Author: Jordan
%   Date:   Sep/Oct 2012
%===================================
\begin{pccexample}
Let $F$ be the function that takes a Celsius temperature as input and outputs the corresponding Fahrenheit temperature. Translate this verbal description of $F$ into a table, a graph, and a formula.

\begin{pccsolution}
To make a table for $F$, we will need to rely on what we know about Celsius and Fahrenheit temperatures. It is a fact that the freezing temperature of water at sea level is \SI{0}{\celsius}, which equals \SI{32}{\Fahrenheit}. Also, boiling temperature of water at sea level is \SI{100}{\celsius}, which is the same as \SI{212}{\Fahrenheit}. One more piece of information we might have is that standard human body temperature is \SI{37}{\celsius}, or \SI{98.6}{\Fahrenheit}. All of this is compiled in \cref{fun:tab:fahrenheit}. Note that we tabulated inputs and outputs by working with the context of the function, not with any computations.

Once a table is established, making a graph by plotting points is a simple matter, as in \cref{fun:fig:fahrenheit}. The three plotted points seem to be in a straight line, so we think it is reasonable to connect them in that way.

\begin{table}[!htb]
	\begin{minipage}{.4\textwidth}
		\centering
		\captionof{table}{}
		\label{fun:tab:fahrenheit}
		\begin{tabular}{S[table-format=3.0]S[table-format=3.1]}
			\beforeheading
			\heading{$C$} & \heading{$F(C)$} \\ 
			\afterheading
			0             & 32               \\\normalline
			37            & 98.6             \\\normalline
			100           & 212              \\\lastline
		\end{tabular}
	\end{minipage}%
	\begin{minipage}{.6\textwidth}
		\centering
		\begin{tikzpicture}
			\begin{axis}[
                    framed,
				   xmin=-20,xmax=120,
				   ymin=-50,ymax=250,
				   height=0.618\textwidth,
				   xtick={20,40,...,100},
				   ytick={50,100,...,200},
				   grid=major,
				   ]
				\addplot+[dashed]expression[domain=-10:110,samples=50] {9/5*x+32};
				  \addplot[soldot, nodes near coords]coordinates{ (0,32) (37,98.6) (100,212)};				
			\end{axis}
		\end{tikzpicture}		\captionof{figure}{$y=F(x)$}
		\label{fun:fig:fahrenheit}
	\end{minipage}%
\end{table}

To find a formula for $F$, the verbal definition is not of much direct help. But $F$'s graph does seem to be a straight line. And linear equations are familiar to us. This line has a $y$-intercept at $(0,32)$ and a slope we can calculate: $\frac{212-32}{100-0}=\frac{180}{100}=\frac{9}{5}$. So the equation of this line is $y=\frac{9}{5}C+32$. On the other hand, the equation of this graph is $y=F(C)$, since it is a graph of the function $F$. So evidently, 
\[
F(C)=\frac{9}{5}C+32
\]
\end{pccsolution}
\end{pccexample}

%=================================== 
%   Author: Hughes 
%   Date:   October 2012 
%===================================
\begin{pccexample}[Referencing a function]
Label each of the following snippets as correct use of vocabulary ({\color{green}\checkmark}) or  incorrect use ({\color{red}$\times$});
if the usage is incorrect, give a brief reason why.
\begin{enumerate}
	\item The function $f$ is shown in Figure \ldots\label{fineusage1}
	\item The function $f(x)$ is shown in Figure \ldots\label{badusage1}
	\item Consider the function $g$ that has formula $g(x)=x^2+3$.\label{fineusage2}
	\item Consider the function $g(x)=x^2+1$ \ldots\label{badusage2}
\end{enumerate}
\begin{pccsolution}
 \Cref{fineusage1,fineusage2} are fine usage of vocabulary ({\color{green}\checkmark}). \Cref{badusage1} is not ({\color{red}$\times$}), since $f(x)$ is the value of $f$ at $x$; it is not a function. Also \cref{badusage2} is bad usage ({\color{red}$\times$}) because $g$ is the function; $g(x)=x^2+1$ is the \emph{formula} for the function $g$.
\end{pccsolution}

We will be using the correct language throughout this document; pay close
attention to it and try your best to use it in all of your mathematical 
work, both verbal and written.
\end{pccexample}

\begin{exercises}
%===================================
%   Author: Hughes
%   Date:   October 2012
%===================================
\begin{problem}[Slope and vertical intercept]
In each of the following problems you are given a formula for a linear function. 
State the slope of each function, and state the vertical intercept 
of each function as an ordered pair.
\begin{multicols}{4}
	\begin{subproblem}
		$f(x)=3x-1$ 
		\begin{shortsolution}
			Slope is $3$, vertical intercept is $(0,-1)$. 
		\end{shortsolution}
	\end{subproblem}
	\begin{subproblem}
		$g(x)=5-\frac{x}{2}$ 
		\begin{shortsolution}
			Slope is $-\frac{1}{2}$, vertical intercept is $(0,5)$. 
		\end{shortsolution}
	\end{subproblem}
	\begin{subproblem}
		$h(x)=\pi-10x$ 
		\begin{shortsolution}
			Slope is $-10$, vertical intercept is $(0,\pi)$. 
		\end{shortsolution}
	\end{subproblem}
	\begin{subproblem}
		$k(x)=mx+b$ 
		\begin{shortsolution}
			Slope is $m$, vertical intercept is $(0,b)$. 
		\end{shortsolution}
	\end{subproblem}
\end{multicols}
\end{problem}

%===================================
%   Author: Hughes
%   Date:   October 2012
%===================================
\begin{problem}[Linear or quadratic]
Decide which of the following formulas correspond to a linear function, 
and which correspond to a quadratic function. 
\begin{multicols}{4}
	\begin{subproblem}
		$f(x)=2x+1$ 
		\begin{shortsolution}
			Linear.
		\end{shortsolution}
	\end{subproblem}
	\begin{subproblem}
		$g(t)=t^2+2$ 
		\begin{shortsolution}
			Quadratic.
		\end{shortsolution}
	\end{subproblem}
	\begin{subproblem}
		$h(s)=9-5s$ 
		\begin{shortsolution}
			Linear. 
		\end{shortsolution}
	\end{subproblem}
	\begin{subproblem}
		$j(u)=4^2-\frac{u}{5}$ 
		\begin{shortsolution}
			Linear. 
		\end{shortsolution}
	\end{subproblem}
	\begin{subproblem}
		$\alpha(c)=c^2+2c+4$ 
		\begin{shortsolution}
			Quadratic. 
		\end{shortsolution}
	\end{subproblem}
	\begin{subproblem}
		$\beta(m)=4$ 
		\begin{shortsolution}
			Linear. 
		\end{shortsolution}
	\end{subproblem}
	\begin{subproblem}
		$\gamma(h)=\frac{4}{3}-h^2$ 
		\begin{shortsolution}
			Quadratic. 
		\end{shortsolution}
	\end{subproblem}
	\begin{subproblem}
		$\delta(z)=z$ 
		\begin{shortsolution}
			Linear. 
		\end{shortsolution}
	\end{subproblem}
\end{multicols}
\end{problem}
%===================================
%   Author: Hughes
%   Date:   October 2012
%===================================
\begin{problem}[Vertex of a quadratic function]
Each of the following formulas corresponds to a quadratic function. 
State the vertex and range of each function (the domain of 
each function is $(-\infty,\infty)$).
\begin{multicols}{4}
	\begin{subproblem}
		$f(x)=(x-3)^2+4$ 
		\begin{shortsolution}
			Vertex: $(3,4)$, range: $[4,\infty)$.
		\end{shortsolution}
	\end{subproblem}
	\begin{subproblem}
		$g(x)=4-(x-3)^2$ 
		\begin{shortsolution}
			Vertex: $(3,4)$, range: $(-\infty,4]$. 
		\end{shortsolution}
	\end{subproblem}
	\begin{subproblem}
		$h(x)=2(x-5)^2$ 
		\begin{shortsolution}
			Vertex: $(5,0)$, range: $[0,\infty)$.
		\end{shortsolution}
	\end{subproblem}
	\begin{subproblem}
		$j(x)=5(3x-4)^2+7$ 
		\begin{shortsolution}
			Vertex: $\left( \frac{4}{3},7 \right)$, range: $[7,\infty)$. 
		\end{shortsolution}
	\end{subproblem}
	\begin{subproblem}
		$F(x)=x^2+1$ 
		\begin{shortsolution}
			Vertex: $\left( -\frac{1}{2},\frac{5}{4} \right)$, range: $\left[ \frac{5}{4},\infty \right)$
		\end{shortsolution}
	\end{subproblem}
	\begin{subproblem}
		$G(x)=3x^2+5x$ 
		\begin{shortsolution}
			Vertex: $\left( -\frac{5}{6},-\frac{25}{12} \right)$, range: $\left[ -\frac{25}{12},\infty\right)$ 
		\end{shortsolution}
	\end{subproblem}
	\begin{subproblem}
		$H(x)=4-\frac{1}{2}x^2$ 
		\begin{shortsolution}
			Vertex: $(0,4)$, range: $(-\infty,4]$. 
		\end{shortsolution}
	\end{subproblem}
	\begin{subproblem}
		$J(x)=\frac{1}{3}-\frac{x}{2}-\frac{x^2}{5}$ 
		\begin{shortsolution}
			Vertex: $\left( -\frac{5}{4},\frac{1}{3} \right)$, range: $\left( -\infty,\frac{1}{3} \right]$.
		\end{shortsolution}
	\end{subproblem}
\end{multicols}
\end{problem}
  
\end{exercises}


\section{Domain and Range}

A function is a process for turning input values into output values. Occasionally a function $f$ will have input values for which the process breaks down.

%===================================
%   Author: Jordan
%   Date:   Sep/Oct 2012
%===================================
\begin{pccexample}\label{fun:ex:portlandpop}
Let $P$ be the population of Portland as a function of the year. According to Google\footnote{\href{http://www.google.com/publicdata/explore?ds=kf7tgg1uo9ude_&met_y=population&idim=place:4159000&dl=en&hl=en&q=population+of+portland}{http://www.google.com/publicdata/explore?ds=kf7tgg1uo9ude\_\&met\_y=population\&idim=place:4159000 \&dl=en\&hl=en\&q=population+of+portland}} we can say that: 
\begin{align*}
	P(2011) & = \num{593820} & P(1990) & = \num{487849} 
\end{align*}
But what if we asked to find $P(1600)$? The question doesn't really make sense anymore. While there were indigenous peoples living in the area then, the city of Portland was not incorporated until 1851. We say that $P(1600)$ is \emph{undefined}.
\end{pccexample}

%===================================
%   Author: Jordan
%   Date:   Sep/Oct 2012
%===================================
\begin{pccexample}\label{fun:ex:MassWeight}
If $m$ is a person's mass in \si{\kilo\gram}, let $w(m)$ be their weight in \si{\pound}. There is an approximate formula for $w$: 
\[
w(m)\approx 2.2m
\] 
From this formula we can find: 
\begin{align*}
	w(50) & \approx \num{110} & w(80) & \approx \num{176} 
\end{align*}
which tells us that a \SI[number-unit-product=-]{50}{\kilo\gram} person weighs \SI{110}{\pound}, and an \SI[number-unit-product=-]{80}{\kilo\gram} person weighs \SI{176}{\pound}.

What if we asked for $w(-100)$? In the context of this example, we would be asking for the weight of a person whose mass is \SI{-100}{\kilo\gram}. This is clearly nonsense. That means that $w(-100)$ is \emph{undefined}. Note that the \emph{context} of the example is telling us that $w(-100)$ is undefined even though the formula alone might suggest that $w(-100)=-220$.
\end{pccexample}

%===================================
%   Author: Jordan
%   Date:   Sep/Oct 2012
%===================================
\begin{pccexample}\label{fun:ex:contextfreedomain}
Let $g$ have the formula 
\[
g(x)=\frac{x}{x-7}
\] 
For most $x$-values, $g(x)$ is perfectly computable:
\begin{align*}
	g(2) & = -\frac{2}{5} & g(14) & = \num{2} 
\end{align*}
But if we try to compute $g(7)$, we run into an issue of arithmetic. 
\begin{align*}
	g(7) & = \frac{7}{7-7} \\
	     & = \frac{7}{0}   
\end{align*}
The expression $\frac{7}{0}$ is \emph{undefined}. There is no number that this could equal.
\end{pccexample}
\begin{doyouunderstand}
	\begin{problem}
	Find an input for the function $f$ that would cause an undefined output, where $f(x)=\dfrac{x+2}{x+8}$.
	\begin{shortsolution}
		\num{-8} would be a bad input value; it would lead to division by \num{0}
	\end{shortsolution}
	\end{problem}
\end{doyouunderstand}
\begin{pccdefinition}[Domain]
The \emph{domain} of a function $f$ is the collection of all of its valid input values.
\end{pccdefinition}
%===================================
%   Author: Jordan
%   Date:   Sep/Oct 2012
%===================================
\begin{pccexample}
Referring to the functions from \crefrange{fun:ex:portlandpop}{fun:ex:contextfreedomain},
\begin{itemize}
	\item The domain of $P$ is all years starting from 1851 and later. It would be reasonable to say the the domain is actually all years from 1851 up to the current year, since we cannot guarantee that Portland will still exist into the future.
	\item The domain of $w$ is all positive real numbers. It is nonsensical to have a person with negative mass or even one with zero mass. While there is some lower bound for the smallest mass a person could have, and also an upper bound for the largest mass a person could have, these boundaries are gray. We can say for sure that nonpositive numbers should never be used as input for $w$.
	\item The domain of $g$ is all real numbers except $7$. This is the only number that causes a breakdown in $g$'s formula. 
\end{itemize}
\end{pccexample}
\begin{doyouunderstand}
	\begin{problem}\label{fun:prob:sqrtdomain}
	What is the domain of the function $\sq$?
	\begin{shortsolution}
		Since $\sq$ can only be applied to nonnegative numbers, the domain is the set of all nonnegative numbers. That is, the domain is all numbers greater than or equal to \num{0}.
	\end{shortsolution}
	\end{problem}
\end{doyouunderstand}
\subsection{Interval, Set, and Set-Builder Notation}
Communicating the domain of a function can be wordy. In mathematics, we can communicate the same information using concise notation that is accepted for use almost worldwide. \Cref{fun:tab:IntervalNotation} contains example functions from this section, their domains, and illustrates \emph{interval notation} for these domains.
\begin{table}[!htb]
	\begin{widepage}
	\centering
	\caption{Domains from \cref{fun:ex:MassWeight}, \cref{fun:ex:contextfreedomain}, and \cref{fun:prob:sqrtdomain}}
	\label{fun:tab:IntervalNotation}
	\begin{tabular}{p{1.5in}p{1.5in}cc}
\beforeheading
\heading{function} & \heading{verbal}   & \heading{illustration of}     & \heading{interval notation}   \\
                   & \heading{domain}   & \heading{domain}              & \heading{for domain}          \\
\afterheading
$w$ from \cref{fun:ex:MassWeight}, computing mass from weight 
                   & all real numbers greater than \num{0}           
                   & \begin{adjustbox}{valign=m}{\begin{tikzpicture} 
\begin{axis}[
   xmin=-10,xmax=10,
   ymin=-1,ymax=1,
 axis y line=none,
   xlabel={$m$},
   width=2in,
   height=2cm,
   grid=none,
   xtick={-10,0,10},
 xticklabels = {$-\infty$,$0$,$\infty$},
 tick style={draw=none},
   ]
  \addplot+[->,line width=3pt]coordinates{ (0,0) (10,0) };				
  \addplot[holdot]coordinates{ (0,0) };				
\end{axis}
\end{tikzpicture}}\end{adjustbox}
                   & $(0,\infty)$                                    \\
$g$ from \cref{fun:ex:contextfreedomain}, with formula $g(x)=\frac{x}{x-7}$
                   & all real numbers except \num{7}                 
                   & \begin{adjustbox}{valign=m}{\begin{tikzpicture} 
\begin{axis}[
   xmin=-10,xmax=10,
   ymin=-1,ymax=1,
 axis y line=none,
   xlabel={$x$},
   width=2in,
   height=2cm,
   grid=none,
   xtick={-10,7,10},
 xticklabels = {$-\infty$,$7$,$\infty$},
 tick style={draw=none},
   ]
  \addplot+[line width=3pt]coordinates{ (-10,0) (10,0) };				
  \addplot[holdot]coordinates{ (7,0) };				
\end{axis}
\end{tikzpicture}}\end{adjustbox}
                   & $(-\infty,7)\cup(7,\infty)$                     \\
$\sq$ from \cref{fun:prob:sqrtdomain}
                   & all numbers greater than or equal to \num{0}    
                   & \begin{adjustbox}{valign=m}{\begin{tikzpicture} 
\begin{axis}[
   xmin=-10,xmax=10,
   ymin=-1,ymax=1,
 axis y line=none,
   xlabel={$x$},
   width=2in,
   height=2cm,
   grid=none,
   xtick={-10,0,10},
 xticklabels = {$-\infty$,$0$,$\infty$},
 tick style={draw=none},
   ]
  \addplot+[line width=3pt,->]coordinates{ (0,0) (10,0) };				
  \addplot[soldot]coordinates{ (0,0) };				
\end{axis}
\end{tikzpicture}}\end{adjustbox}
                   & $[0,\infty)$                                    
\end{tabular}
\end{widepage}
\end{table}

Interval notation comes in many forms. Each of the expressions $(a,b)$, $(a,b]$, $[a,b)$, and $[a,b]$ are examples of simple intervals. The notation is communicating that we wish to consider all real numbers between $a$ and $b$. If a round parenthesis is used, then that number itself should be excluded from consideration. If a square bracket is used, then that number itself should be included under consideration. The `$a$' might be the symbol $-\infty$, and the `$b$' might be the symbol $\infty$. If these symbols are used, then there is no lowermost  or uppermost bound to the interval. Lastly, two or more simple intervals can be joined together with the union symbol $\cup$. \Cref{fun:tab:intervals} gives more examples of interval notation in use.
\begin{table}[!htb]
	\centering
	\caption{Interval Notation}
	\label{fun:tab:intervals}
	\begin{tabular}{cc}
		\beforeheading
		\heading{picture of} & \heading{interval}                   \\
		\heading{interval}   & \heading{notation}                   \\
		\afterheading
		\begin{adjustbox}{valign=m}
          \begin{tikzpicture}
		\begin{axis}[
		   xmin=-10,xmax=10,
		   ymin=-1,ymax=1,
		 axis y line=none,
		   width=2in,
		   height=2cm,
		   grid=none,
		   xtick={-10,-2,3,10},
		 xticklabels = {$-\infty$,$-2$,$3$,$\infty$},
		 tick style={draw=none},
		   ]
		  \addplot+[-,line width=3pt]coordinates{ (-2,0) (3,0) };				
		  \addplot[holdot]coordinates{ (-2,0) };				
		  \addplot[soldot]coordinates{ (3,0) };
		\end{axis}
		\end{tikzpicture}
      \end{adjustbox}
		                     & $(-2,3]$                             \\
		  \begin{adjustbox}{valign=m}
            \begin{tikzpicture}
		\begin{axis}[
		   xmin=-10,xmax=10,
		   ymin=-1,ymax=1,
		 axis y line=none,
		   width=2in,
		   height=2cm,
		   grid=none,
		   xtick={-10,-5,1,3,5,10},
		 xticklabels = {$-\infty$,$-5$,$1$,$3$,$5$,$\infty$},
		 tick style={draw=none},
		   ]
		  \addplot+[line width=3pt,-]coordinates{ (-5,0) (1,0) };				
		  \addplot+[line width=3pt,-]coordinates{ (3,0) (5,0) };	
		  \addplot[holdot]coordinates{ (3,0) (5,0)};				
		  \addplot[soldot]coordinates{ (-5,0) (1,0)};
		  \end{axis}
		\end{tikzpicture}
      \end{adjustbox}
		                     & $[-5,1]\cup(3,5)$                    \\
		\begin{adjustbox}{valign=m}
          \begin{tikzpicture}
		\begin{axis}[
		   xmin=-10,xmax=10,
		   ymin=-1,ymax=1,
		 axis y line=none,
		   width=2in,
		   height=2cm,
		   grid=none,
		   xtick={-10,2,5,10},
		 xticklabels = {$-\infty$,$2$,$5$,$\infty$},
		 tick style={draw=none},
		   ]
		  \addplot+[line width=3pt]coordinates{ (-10,0) (10,0) };				
		  \addplot[holdot]coordinates{ (2,0) (5,0) };				
		\end{axis}
		\end{tikzpicture}
      \end{adjustbox}
		                     & $(-\infty,2)\cup(2,5)\cup(5,\infty)$ \\
		\begin{adjustbox}{valign=m}
          \begin{tikzpicture}
		\begin{axis}[
		   xmin=-10,xmax=10,
		   ymin=-1,ymax=1,
		 axis y line=none,
		   xlabel={$x$},
		   width=2in,
		   height=2cm,
		   grid=none,
		   xtick={-10,10},
		 xticklabels = {$-\infty$,$\infty$},
		 tick style={draw=none},
		   ]
		  \addplot+[line width=3pt]coordinates{ (-10,0) (10,0) };								
		\end{axis}
		\end{tikzpicture}
      \end{adjustbox}
		                     & $(-\infty,\infty)$                   
	\end{tabular}
\end{table}

Sometimes we will consider collections of only a small, finite number of numbers. In those cases, we use \emph{set notation}. With set notation, we have a list of numbers in mind, and we simply list all of those numbers. Curly braces are standard for encasing the list. \Cref{fun:tab:sets} illustrates set notation in use.
\begin{table}[!htb]
	\centering
	\caption{Set Notation}
	\label{fun:tab:sets}
	\begin{tabular}{cc}
		\beforeheading
		\heading{picture of} & \heading{set}                  \\
		\heading{set}   & \heading{notation}             \\
		\afterheading
		\begin{adjustbox}{valign=m}
          \begin{tikzpicture}
		\begin{axis}[
		   xmin=-10,xmax=10,
		   ymin=-1,ymax=1,
		 axis y line=none,
		   width=2in,
		   height=2cm,
		   grid=none,
		   xtick={-10,-2,3,10},
		 xticklabels = {$-\infty$,$-2$,$3$,$\infty$},
		 tick style={draw=none},
		   ]						
		  \addplot[soldot]coordinates{ (-2,0) (3,0) };
		\end{axis}
		\end{tikzpicture}
      \end{adjustbox}
		                     & $\setbuilder{-2,3}$            \\
		  \begin{adjustbox}{valign=m}
            \begin{tikzpicture}
		\begin{axis}[
		   xmin=-10,xmax=10,
		   ymin=-1,ymax=1,
		 axis y line=none,
		   width=2in,
		   height=2cm,
		   grid=none,
		   xtick={-10,-5,1,3,5,10},
		 xticklabels = {$-\infty$,$-5$,$1$,$3$,$5$,$\infty$},
		 tick style={draw=none},
		   ]					
		  \addplot[soldot]coordinates{ (-5,0) (1,0) (3,0) (5,0) };
		  \end{axis}
		\end{tikzpicture}
      \end{adjustbox}
		                     & $\setbuilder{-5,1,3,5}$        \\
		\begin{adjustbox}{valign=m}
          \begin{tikzpicture}
		\begin{axis}[
		   xmin=-10,xmax=10,
		   ymin=-1,ymax=1,
		 axis y line=none,
		   width=2in,
		   height=2cm,
		   grid=none,
		   xtick={-10,2,5,10},
		 xticklabels = {$-\infty$,$2$,$5$,$\infty$},
		 tick style={draw=none},
		   ]
		  \addplot+[line width=3pt,->]coordinates{ (5,0) (10,0) };				
		  \addplot[soldot]coordinates{ (2,0) };				
		  \addplot[holdot]coordinates{ (5,0) };
		\end{axis}
		\end{tikzpicture}
      \end{adjustbox}
		                     & $\setbuilder{2}\cup(5,\infty)$ 
	\end{tabular}
\end{table}

While most collections of numbers that we will encounter can be described using a combination of interval notation and set notation, there is another commonly used notation that all students of college algebra should be exposed to: \emph{set-builder notation}. Set-builder notation also uses curly braces. Set-builder notation provides a template for what a number that is under consideration might look like, and then it gives you restrictions on how to use that template. A very basic example of set-builder notation is 
\[
\setbuilder{x}[x\geq3]
\]
Verbally, this is `the set of all $x$ such that $x$ is greater than or equal to 3'. \Cref{fun:tab:setbuilder} gives more examples of set-builder notation in use.

\begin{table}[!htb]
	\centering
	\caption{Set-Builder Notation}
	\label{fun:tab:setbuilder}
	\begin{tabular}{cc}
		\beforeheading
		\heading{picture of} & \heading{set-builder}                         \\
		\heading{set}   & \heading{notation}                            \\
		\afterheading
		\begin{adjustbox}{valign=m}
          \begin{tikzpicture}
		\begin{axis}[
		   xmin=-10,xmax=10,
		   ymin=-1,ymax=1,
		 axis y line=none,
		   width=2in,
		   height=2cm,
		   grid=none,
		   		   xtick={-10,-2,3,10},
		 xticklabels = {$-\infty$,$-2$,$3$,$\infty$},
		 tick style={draw=none},
		   ]
		  \addplot+[-,line width=3pt]coordinates{ (-2,0) (3,0) };				
		  \addplot[holdot]coordinates{ (-2,0) };				
		  \addplot[soldot]coordinates{ (3,0) };
		\end{axis}
		\end{tikzpicture}
      \end{adjustbox}
		                     & $\setbuilder{x}[-2<x\text{ and }x\leq3]$      \\
		  \begin{adjustbox}{valign=m}
            \begin{tikzpicture}
		\begin{axis}[
		   xmin=-10,xmax=10,
		   ymin=-1,ymax=1,
		 axis y line=none,
		   width=2in,
		   height=2cm,
		   grid=none,
		   		   xtick={-10,1,3,10},
		 xticklabels = {$-\infty$,$1$,$3$,$\infty$},
		 tick style={draw=none},
		   ]
		  \addplot+[line width=3pt,<-]coordinates{ (-10,0) (1,0) };				
		  \addplot+[line width=3pt,->]coordinates{ (3,0) (10,0) };	
		  \addplot[holdot]coordinates{ (1,0) (3,0)};				
		
		  \end{axis}
		\end{tikzpicture}
      \end{adjustbox}
		                     & $\setbuilder{x}[x<1\text{ or }x>3]$           \\
		\begin{adjustbox}{valign=m}
          \begin{tikzpicture}
		\begin{axis}[
		   xmin=-10,xmax=10,
		   ymin=-1,ymax=1,
		 axis y line=none,
		   width=2in,
		   height=2cm,
		   grid=none,
		   		   xtick={-10,-4,4,10},
		 xticklabels = {$-\infty$,$4$,$4$,$\infty$},
		 tick style={draw=none},
		   ]
		  \addplot+[line width=3pt,-]coordinates{ (-4,0) (4,0) };				
		  \addplot[soldot]coordinates{ (-4,0) (4,0) };				
		\end{axis}
		\end{tikzpicture}
      \end{adjustbox}
		                     & $\setbuilder{x}[x^2\leq16]$                   \\
		\begin{adjustbox}{valign=m}
          \begin{tikzpicture}
		\begin{axis}[
		   xmin=-10,xmax=10,
		   ymin=-1,ymax=1,
		 axis y line=none,
		   width=2in,
		   height=2cm,
		   grid=none,
		   		   xtick={-10,0,10},
		 xticklabels = {$-\infty$,$0$,$\infty$},
		 tick style={draw=none},
		   ]
		  \addplot+[line width=3pt,->]coordinates{ (0,0) (10,0) };				
		  \addplot[holdot]coordinates{ (0,0) };				
		\end{axis}
		\end{tikzpicture}
      \end{adjustbox}
		                     & $\setbuilder{x^2}[x\text{ is a real number}]$ 
	\end{tabular}
\end{table}

The domain of a function is the collection of its possible inputs; there is a similar notion for \emph{output}.
\begin{pccdefinition}[Range]
The \emph{range} of a function $f$ is the collection of all of its possible output values.
\end{pccdefinition}
%===================================
%   Author: Jordan
%   Date:   Sep/Oct 2012
%===================================
\begin{pccexample}\label{fun:ex:range}
Let $f$ be the function defined by the formula $f(x)=x^2$. Finding $f$'s domain is particularly basic. Any number anywhere can be squared to produce an output, so $f$ has domain $(-\infty,\infty)$. What is the \emph{range} of $f$?
\begin{pccsolution}
We would like to describe the collection of possible numbers that $f$ can give as outputs. 
First we will use a graphical approach. \Cref{fun:fig:rangeexample} displays a graph of $f$, and the visualization that reveals $f$'s range.
\begin{figure}[!htb]
	\begin{widepage}
	\centering
    \begin{subfigure}{.3\textwidth}
	\begin{tikzpicture}
		\begin{axis}[
                framed,
			   xmin=-3,xmax=3,
			   ymin=-1,ymax=5,
			   grid=major,
			   xtick={-2,...,2},
			   ytick={1,...,4},
			   ]
			  \addplot+[domain=-2.2:2.2]{x^2};				
		\end{axis}
	\end{tikzpicture}
    \caption{}
    \label{fun:fig:rangeexamplesetup}
    \end{subfigure}%
    \hfill
    \begin{subfigure}{.3\textwidth}
	\begin{tikzpicture}
		\begin{axis}[
                framed,
			   xmin=-3,xmax=3,
			   ymin=-1,ymax=5,
			   grid=major,
			   xtick={-2,...,2},
			   ytick={1,...,4},
			   ]
			  \addplot expression[domain=-2.2:2.2]{x^2};			
			\addplot[->,pccplot,line width=2pt]coordinates{ (-0.707,0.499849) (0,0.499849) };				
			\addplot[->,pccplot,line width=2pt]coordinates{ (1,1) (0,1) };				
			\addplot[->,pccplot,line width=2pt]coordinates{ (-1.225,1.500625) (0,1.500625) };
			\addplot[->,pccplot,line width=2pt]coordinates{ (1.414,1.999396) (0,1.999396) };
			\addplot[->,pccplot,line width=2pt]coordinates{ (-1.581,2.499561) (0,2.499561) };
			\addplot[->,pccplot,line width=2pt]coordinates{ (1.732,2.999824) (0,2.999824) };
			\addplot[->,pccplot,line width=2pt]coordinates{ (-1.871,3.500641) (0,3.500641) };
			\addplot[->,pccplot,line width=2pt]coordinates{ (2,4) (0,4) };
			\addplot[->,pccplot,line width=2pt]coordinates{ (-2.121,4.498641) (0,4.498641) };
		\end{axis}
	\end{tikzpicture}
    \caption{}
	\label{fun:fig:slide}
    \end{subfigure}%
    \hfill
    \begin{subfigure}{.3\textwidth}
	\begin{tikzpicture}
		\begin{axis}[
                framed,
			   xmin=-3,xmax=3,
			   ymin=-1,ymax=5,
			   grid=major,
			   xtick={-2,...,2},
			   ytick={1,...,4},
			   ]
			  \addplot expression[domain=-2.2:2.2]{x^2};				
			\addplot [pccplot,line width=3pt,->]coordinates{ (0,0) (0,5) };				
			  \addplot[soldot]coordinates{ (0,0) };		
		\end{axis}
	\end{tikzpicture}	
    \caption{}
    \label{fun:fig:rangeexamplerange}
  \end{subfigure}
	\caption{$y=f(x)$, where $f(x)=x^2$}
	\label{fun:fig:rangeexample}
  \end{widepage}
\end{figure}

Output values are the $y$-coordinates in a graph. If we `slide the ink' across to the $y$-axis (\cref{fun:fig:slide}) 
to emphasize what the $y$-values in the graph are, we have $y$-values that start from $0$ and continue upward forever. Therefore the range is $[0,\infty)$ 
(see \cref{fun:fig:rangeexamplerange}).
\end{pccsolution}

\begin{pccsolution}
Here is an alternative solution. Occasionally it is possible to find the range directly , without the help of a graph. In the case of this function, we understand that the outputs must be nonnegative, since any real number squared is not negative. We also understand that any nonnegative output $y$ that you could imagine (\num{0}, \num{0.01}, \num{244}, \ldots) is a possible output if we feed $f$ the right input, namely $\sqrt{y}$. So the domain of $f$ is $[0,\infty)$.
\end{pccsolution}
\end{pccexample}

\begin{pccspecialcomment}[Finding range from a formula]
	\Cref{fun:ex:range} shows us that it is sometimes possible to compute a range without the aid of a graph. However until students learn some topics that will be covered later in this text and in a calculus course, it will often be difficult to do so. Therefore when you are asked to find the range of a function based on its formula, your first approach should be a graphical one.
\end{pccspecialcomment}

%===================================
%   Author: Jordan
%   Date:   Sep/Oct 2012
%===================================
\begin{pccexample}
Given the function $g$ graphed in \cref{fun:fig:rangeexample2}, find the domain and range of $g$.
\begin{figure}[!htb]
    \begin{minipage}{.45\textwidth}
	\centering
	\begin{tikzpicture}
		\begin{axis}[
                framed,
			   xmin=-2,xmax=2,
			   ymin=-0.5,ymax=3.5,
			   grid=major,
			   xtick={-1,...,1},
			   ytick={0,1,...,3},
			   ]
			  \addplot expression[domain=-2:2,samples=50]{3*3^(-x^2)};				
		\end{axis}
	\end{tikzpicture}
	\caption{$y=g(x)$}
	\label{fun:fig:rangeexample2}
    \end{minipage}%
    \hfill
    \begin{minipage}{.45\textwidth}
	\centering
	\begin{tikzpicture}[/pgf/declare function={f=1/(x-2);}]
		\begin{axis}[
                framed,
			   xmin=-3,xmax=7,
			   ymin=-3,ymax=3,
			   grid=major,
			   xtick={-2,...,6},
			   ytick={-2,...,2},
			   ]
			  \addplot expression[domain=-3:1.66]{f};				
			  \addplot expression[domain=2.33:7,pccplot]{f};			
		\end{axis}
	\end{tikzpicture}
	\caption{$y=h(x)$}
	\label{fun:fig:rangeexample3}
    \end{minipage}%
\end{figure}

\begin{pccsolution}
To find the domain, we can visualize all of the $x$-values that are valid inputs for this function, by `sliding the ink down onto the $x$-axis. The arrows indicate that whatever pattern we see in the graph continues off to the left and right. Here, we see that the arms of the graph are tapering down to the $x$-axis and extending left and right forever. Every $x$ value is covered, so the domain is $(\infty,\infty)$.

If we visualize the possible outputs by `sliding the ink' sideways onto the $y$-axis, we find that outputs as high as \num{3} are possible (including \num{3} itself). The outputs appear to be very close to \num{0} when $x$ is large, but they aren't quite equal to \num{0}. So the range is $(0,3]$.  
\end{pccsolution}
\end{pccexample}

\begin{doyouunderstand}
	\begin{problem}
	Find the domain and range of the function $h$ graphed in \cref{fun:fig:rangeexample3}.
	\begin{shortsolution}
		The domain is $(-\infty,2)\cup(2,\infty)$ and the range is $(\infty,0)\cup(0,\infty)$.
	\end{shortsolution}
	\end{problem}
\end{doyouunderstand}

The examples of finding domain and range so far have all involved either a verbal description of a function, a formula for that function, or a graph of that function. Recall that there is a fourth persepctive on functions: the table. In the case of a table, we have very limited information about the function's inputs and outputs. If the table is all that we have, then there are a handful of input values listed in the table for which we know outputs. For any other input, the output is undefined. 

\begin{margintable}
	\centering
	\captionof{table}{The function $k$}
	\label{fun:tab:rangetable}
	\begin{tabular}{SS}
		\beforeheading
		\heading{$x$} & \heading{$k(x)$} \\
		\afterheading
		3             & 4                \\ \normalline
		8             & 5                \\ \normalline
		10            & 5                \\\lastline
	\end{tabular}
\end{margintable}
\begin{marginfigure}
	\centering
	\begin{tikzpicture}
		\begin{axis}[
                framed,
			   xmin=0,xmax=12,
			   ymin=0,ymax=6,
			   width=.7\textwidth,
			   height=0.42\textwidth,
			   grid=major,
			   axis line style=->,
			   xtick={0,1,...,11},
			   ytick={0,1,...,5},
			   ]
			\addplot [soldot] coordinates { 
			(3,4)
			(8,5)
			(10,5)};		
		\end{axis}
	\end{tikzpicture}
	\captionof{figure}{$y=k(x)$}
	\label{fun:fig:rangetable}
\end{marginfigure}

%===================================
%   Author: Jordan
%   Date:   Sep/Oct 2012
%===================================
\begin{pccexample}
Consider the function $k$ given in \cref{fun:tab:rangetable}. What is the domain and range of $k$? 
\begin{pccsolution}
All that we know about $k$ is that $k(3)=4$, $k(8)=5$, and $k(10)=5$. Without any other information such as a formula for $k$ or a context for $k$ that tells us its verbal description, we must assume that its domain is $\setbuilder{3,8,10}$; these are the only valid input for $k$. Similarly, $k$'s range is $\setbuilder{4,5}$.
\end{pccsolution}
Note that we have used set notation, not interval notation, since the answers here were lists of $x$-values and not intervals. Also note that we could graph the information that we have regarding $k$, as in \cref{fun:fig:rangetable}, and the visualization of `sliding ink' to determine domain and range still works.
\end{pccexample}

\begin{exercises}
%===================================
%   Author: Hughes
%   Date:   October 2012
%===================================
\begin{problem}[Domain of radical functions]
Find the domain of each of the functions associated with the following formulas.
\begin{multicols}{4}
	\begin{subproblem}
		$f(x)=\sqrt{x}$
		\begin{shortsolution}
			$[0,\infty)$  
		\end{shortsolution}
	\end{subproblem}
	\begin{subproblem}
		$g(x)=\sqrt{x+10}$  
		\begin{shortsolution}
			$[-10,\infty)$  
		\end{shortsolution}
	\end{subproblem}
	\begin{subproblem}
		$h(x)=\sqrt[3]{5x}$  
		\begin{shortsolution}
			$(-\infty,\infty)$  
		\end{shortsolution}
	\end{subproblem}
	\begin{subproblem}
		$j(x)=\sqrt[4]{5x+2}$  
		\begin{shortsolution}
			$\left[ -\frac{2}{5},\infty\right)$  
		\end{shortsolution}
	\end{subproblem}
	\begin{subproblem}
		$k(x)=\sqrt[7]{3-x}$  
		\begin{shortsolution}
			$(-\infty,\infty)$  
		\end{shortsolution}
	\end{subproblem}
	\begin{subproblem}
		$l(x)=\sqrt[6]{2-x}$  
		\begin{shortsolution}
			$(-\infty,2]$  
		\end{shortsolution}
	\end{subproblem}
	\begin{subproblem}
		$m(x)=4-\sqrt[9]{x^2}$  
		\begin{shortsolution}
			$(-\infty,\infty)$  
		\end{shortsolution}
	\end{subproblem}
	\begin{subproblem}
		$n(x)=2-\sqrt[8]{x^2+1}$  
		\begin{shortsolution}
			$(-\infty,\infty)$  
		\end{shortsolution}
	\end{subproblem}
\end{multicols}
\end{problem}
\end{exercises}


\section{Increasing, decreasing, concave up/down }

Many functions that are worth studying use time as the input variable. This is quite convenient when you have a graph of those functions. As we read the graph and follow the curve left-to-right, we can imagine time moving forward. From this perspective, in \cref{fun:fig:increasing} we see a function whose outputs grow as time passes, while in \cref{fun:fig:decreasing} the outputs decrease. We would like to take this understanding and declare that some functions are \emph{increasing} functions, while others are \emph{decreasing} functions.

\begin{figure}[!htb]
    \begin{minipage}{.45\textwidth}
	\centering
	\begin{tikzpicture}
		\begin{axis}[
                framed,
			   xmin=-2,xmax=10,
			   ymin=-2,ymax=10,
			   grid=major,
			   xtick={-2,...,10},
			   ytick={-2,...,10},
			   ]
			  \addplot expression[domain=-2:10]{3+0.5*x};				
		\end{axis}
	\end{tikzpicture}
	\caption{$y=f(x)$}
	\label{fun:fig:increasing}
    \end{minipage}%
    \hfill
    \begin{minipage}{.45\textwidth}
	\centering
	\begin{tikzpicture}[/pgf/declare function={f=8*0.9^x;}]
		\begin{axis}[
                framed,
			   xmin=-2,xmax=10,
			   ymin=-2,ymax=10,
			   grid=major,
			   xtick={-2,...,10},
			   ytick={-2,...,10},
			   ]
			  \addplot expression[domain=-2:10]{f};							
		\end{axis}
	\end{tikzpicture}
	\caption{$y=g(x)$}
	\label{fun:fig:decreasing}
    \end{minipage}%
\end{figure}

Given a graph of some function $f$, it is usually apparent whether or not we would want to call the function increasing, decreasing, or neither. But if we left it that, then these new vocabulary terms wouldn't be very helpful. We would not be able to use them to prove anything of consequence, because their definitions would be vague. For this reason more formal definitions have been developed.

\begin{pccdefinition}[Increasing and Decreasing]\label{fun:def:incdec}

A function $f$ is an \emph{increasing} function if whenever $b>a$ with both $b$ and $a$ in $f$'s domain, the nature of $f$ implies that $f(b)>f(a)$.
%I've intentionally avoided more common shorter versions of this definition. I'm concerned about students who read "whenever $b>a$ with both $b$ and $a$ in $f$'s domain, then $f(b)>f(a)$" as some kind of guarantee that f(b) *will* be larger than f(a), rather than a condition for using the new vocabulary term. Other attempts at tackling this misconception welcomed!

Similarly, $f$ is a \emph{decreasing} function if whenever $b>a$ with both $b$ and $a$ in $f$'s domain, the nature of $f$ implies that $f(b)<f(a)$.

\end{pccdefinition}

This definition is consistent with our graphical intuition for what ``increasing'' and ``decreasing'' should mean. For instance in \cref{fun:fig:decreasing}, you can choose any two numbers that you like on the input-axis and label the larger number $b$ and the smaller one $a$. Once you do this, see that $g(b)<g(a)$. This confirmation of cause ($b>a$) and effect ($g(b)<g(a)$) on your part makes $g$ meet the definition of ``decreasing'' in \cref{fun:def:incdec}. But the defintion can be used in other nongraphical ways as the \cref{fun:ex:incformula,fun:ex:neithertable} show.

\begin{pccexample}\label{fun:ex:incformula}
Suppose we have a function $h$ given by $h(x)=3x+1$. Is $h$ an increasing function, a decreasing function, or neither? It is important to learn to answer a question like this according to the formal definition that we have introduced.

\begin{pccsolution}
Suppose that $b$ and $a$ are two numbers with $b>a$. Both of these numbers are in $h$'s domain, since $h$'s domain is $(-\infty,\infty)$. We must decide if the nature of $h$ guarantees that $f(b)>f(a)$, that $f(b)<f(a)$, or guarantess no such thing.

Well,
\begin{align*}
				&&b 	&>a\\
\implies&&3b	&>3a\\
\implies&&3b+1&>3a+1\\
\implies&&h(b)&>h(a) 
\end{align*}
So we have confirmed that $h$ is increasing.
\end{pccsolution}
\end{pccexample}
%
\begin{pccexample}\label{fun:ex:neithertable}
\begin{margintable}
	\centering
	\captionof{table}{$k$} \label{fun:tab:neithertable}
	\begin{tabular}{S[table-format=1.0]S[table-format=1.0]}
		\beforeheading
		\heading{$x$} & \heading{$k(x)$}\\
		\afterheading
		2   & 3    \\\normalline
		3 	& 8  \\\normalline
		4   & 7 \\\lastline
	\end{tabular}
\end{margintable}
Let $k$ be the function given in \cref{fun:tab:neithertable}. We only know outputs of $k$ for inputs in $\setbuilder{2,3,4}$. We can see inputs $3>2$ with outputs $k(3)>k(2)$. This is evidence that $k$ might be increasing. But then se see inputs $4>3$ and $k(4)<k(3)$, which is evidence that $k$ might be decreasing. So the only conclusion we can make is that $k$ is neither increasing nor decreasing.
\end{pccexample}

It's not satisfying to look at the graph of function like $f$ in \cref{fun:fig:incAndDec} and simply state that it is neither increasing nor decreasing. Part of that graph shows increasing behavior, and part of it shows decreasing behavior. We'd like to be able to specify this. Since $f$ seems to be decreasing above the interval $[0,2]$ and increasing above the interval $[2,\infty)$, we are motivated to introduce more definitions.
\begin{figure}[!htb]
	\begin{widepage}
	\centering
    \begin{subfigure}{.3\textwidth}
	\begin{tikzpicture}
		\begin{axis}[
                framed,
			   xmin=-2,xmax=5,
			   ymin=0,ymax=5,
			   width=.7\textwidth,
			   height=0.5\textwidth,
			   grid=major,
			   axis line style=->,
			   xtick={-2,...,5},
			   ytick={0,...,5},
			   ]
			\addplot+[->]expression[domain=0:4.5] {0.5*(x-2)^2+1};
			\addplot [soldot] coordinates { 
			(0,3)
			};		
		\end{axis}
	\end{tikzpicture}
    \caption{$y=f(x)$}
    \label{fun:fig:incAndDec}
    \end{subfigure}%
    \hfill
    \begin{subfigure}{.3\textwidth}
	\begin{tikzpicture}
		\begin{axis}[
                framed,
			   xmin=-2,xmax=5,
			   ymin=0,ymax=5,
			   width=.7\textwidth,
			   height=0.5\textwidth,
			   grid=major,
			   axis line style=->,
			   xtick={-2,...,5},
			   ytick={0,...,5},
			   ]
			\addplot+[->]expression[domain=0:4.5] {0.5*(x-2)^2+1};
			\addplot [soldot] coordinates { 
			(0,3)
			};		
			\addplot[->,pccplot,line width=2pt]coordinates{ (2,1) (2,0) };				
			\addplot[->,pccplot,line width=2pt]coordinates{ (2.5,1.125) (2.5,0) };				
			\addplot[->,pccplot,line width=2pt]coordinates{ (3,1.5) (3,0) };
			\addplot[->,pccplot,line width=2pt]coordinates{ (3.5,2.125) (3.5,0) };
			\addplot[->,pccplot,line width=2pt]coordinates{ (4,3) (4,0) };
			\addplot+[->,line width=3pt]coordinates{ (2,0) (5,0) };
			\addplot [soldot] coordinates { 
			(2,0)
			};
		\end{axis}
	\end{tikzpicture}
    \caption{$f$ is increasing on\ldots}
	\label{fun:fig:intervalOfInc}
    \end{subfigure}%
    \hfill
    \begin{subfigure}{.3\textwidth}
	\begin{tikzpicture}
		\begin{axis}[
                framed,
			   xmin=-2,xmax=5,
			   ymin=0,ymax=5,
			   width=.7\textwidth,
			   height=0.5\textwidth,
			   grid=major,
			   axis line style=->,
			   xtick={-2,...,5},
			   ytick={0,...,5},
			   ]
			\addplot+[->]expression[domain=0:4.5] {0.5*(x-2)^2+1};
			\addplot [soldot] coordinates { 
			(0,3)
			};		
			\addplot[->,pccplot,line width=2pt]coordinates{ (2,1) (2,0) };				
			\addplot[->,pccplot,line width=2pt]coordinates{ (1.5,1.125) (1.5,0) };				
			\addplot[->,pccplot,line width=2pt]coordinates{ (1,1.5) (1,0) };
			\addplot[->,pccplot,line width=2pt]coordinates{ (0.5,2.125) (0.5,0) };
			\addplot[->,pccplot,line width=2pt]coordinates{ (0,3) (0,0) };
			\addplot+[-,line width=3pt]coordinates{ (2,0) (0,0) };
			\addplot [soldot] coordinates { 
			(2,0)
			};
			\addplot [soldot] coordinates { 
			(0,0)
			};
		\end{axis}
	\end{tikzpicture}
    \caption{$f$ is decreasing on\ldots}
	\label{fun:fig:intervalOfInc}
    \end{subfigure}
    \caption{Incresing and decreasing behavior}
  \end{widepage}
\end{figure}

\begin{pccdefinition}[Increasing and Decreasing on an Interval]
Given a function $f$, we say that $f$ is \emph{increasing on the interval $I$} if whenever $b>a$ with both $b$ and $a$ in $I$, the nature of $f$ implies that $f(b)>f(a)$. Here, $I$ could be an interval of any form: $(p,q)$, $(p,q]$, etc.

Similarly,  $f$ is \emph{decreasing on the interval $I$} if whenever $b>a$ with both $b$ and $a$ in $I$, the nature of $f$ implies that $f(b)<f(a)$. 
\end{pccdefinition}

\begin{doyouunderstand}
	\begin{problem}
	Find all intervals on which the function $f$ in \cref{fun:fig:increasingIntervals} is increasing.
		\begin{figure}[!htb]
		\centering
		\begin{tikzpicture}
			\begin{axis}[
				   xmin=-5,xmax=15,
				   ymin=-5,ymax=10,
				   xlabel={$x$},
				   ylabel={$y$},
				   width=0.8\textwidth,
				   height=0.25\textwidth,
				   grid=major,
				   axis line style=->,
				   xtick={-5,...,15},
				   ytick={-5,...,10},
				   ]
				  \addplot+[smooth] coordinates{ (-4,-2) (0,6) (1,7.2) (2,7.7) (3,8) (4,7) (6, 3) (7,2) (8,2.5) (14,6)};				
			\end{axis}
		\end{tikzpicture}		
		\caption{$y=f(x)$}
		\label{fun:fig:increasingIntervals}
	\end{figure}%	
	\begin{shortsolution}
		$f$ is increasng on $(-\infty,3]$ and on $[7,\infty)$. We could also say that $f$ is increasing on $(-\infty,3]\cup[7,\infty)$
	\end{shortsolution}
	\end{problem}
\end{doyouunderstand}
	
	
\subsection{Concavity}
  %===================================
%   Author: Simonds, Jordan
%   Date:   Feb 2011, Mar 2013
%===================================
In earlier math classes you spent a lot of time talking about linear functions.  
One defining trait of a linear function is that its rate of change is constant; 
we call this constant rate of change the slope of the function.  For example, the 
slope of the function $f$, where $f(x)=3x+2$, is three.  This tells us (among other things) 
that every time the value of $x$ increases by $1$, the value of $f(x)$  increases by $3$.  
This is reflected in the values shown in in \cref{fun:tab:threexplustwo}.
\begin{margintable}
	\centering
	\captionof{table}{$f(x)=3x+2$} \label{fun:tab:threexplustwo}
	\begin{tabular}{S[table-format=1.0]S[table-format=2.0]}
		\beforeheading
		\heading{$x$} & \heading{$f(x)$} \\\afterheading
		1             & 5                \\\normalline
		2             & 8                \\\normalline
		3             & 11               \\\normalline
		4             & 14               \\\normalline
		5             & 17               \\\lastline
	\end{tabular}
\end{margintable}

On the other hand, the function $g$ defined by $g(x)=x^2$ does not change at a constant rate.  
If we look at how the function behaves over the positive integers (see \cref{fun:tab:xsquared}), we 
clearly see that as the value of $x$ continually increases by $1$, the value of the function
increases at a faster and faster rate; another way to express this is to say that $g$ is 
\emph{concave up}. This vocabulary is motivated by the graph of $g$, which has a concavity above it. 
\begin{margintable}
	\centering
	\captionof{table}{$g(x)=x^2$} \label{fun:tab:xsquared}
	\begin{tabular}{S[table-format=1.0]S[table-format=2.0]}
		\beforeheading
		\heading{$x$} & \heading{$g(x)$} \\\afterheading
		1             & 1                \\\normalline
		2             & 4                \\\normalline
		3             & 9                \\\normalline
		4             & 16               \\\normalline
		5             & 25               \\\lastline
	\end{tabular}
 \end{margintable}
 
 \begin{marginfigure}
	\centering
	\captionof{figure}{$y=x^2$} \label{fun:fig:xsquared}
		\begin{tikzpicture}[/pgf/declare function={f=x^2;}]
		\begin{axis}[
                framed,
			   xmin=-2,xmax=6,
			   ymin=-5,ymax=30,
			   grid=major,
			   xtick={-2,...,6},
			   ytick={-5,0,...,30},
			   ]
			  \addplot+expression[domain=-2:5.4]{f};
			  \addplot [pccplot,line width=0.5pt,-, nodes near coords]coordinates{ 
					(1,1)
					(2,4)
					(3,9)
					(4,16)
					(5,25)};								
		\end{axis}
	\end{tikzpicture}
	\label{fun:fig:xsquared}
\end{marginfigure}



\begin{pccdefinition}[Concavity]
A function $f$ is \emph{concave up} on an interval $I$ if every way of taking two numbers $a$ and $b$ from $I$, locating $(a, f(a))$ and $(b, f(b))$ on the graph of $f$, and connecting them with a straight line segment yields a line segment that is above the graph of $f$, touching the graph of $f$ only at the segment's endpoints.

Here, $I$ may be an interval of any type: $(p, q)$, $(p, q]$, etc.

A function $f$ is \emph{concave down} on an interval $I$ if such line segments are below the graph of $f$.
\end{pccdefinition}
%===================================
%   Author: Hughes, Jordan
%   Date:   July 2011, March 2013
%===================================
\begin{pccexample}
\Cref{fun:fig:concaveup,fun:fig:concavedown} demonstrate some functions and their concavities.
\begin{figure}[!htb]
	\setlength{\figurewidth}{0.2\textwidth}
	\begin{minipage}{.5\textwidth}
		\centering
		\begin{tikzpicture}
			\begin{axis}[
			   framed,
			   width=\figurewidth,
			   xmin=-5,xmax=5,
			   ymin=-1,ymax=5,
			   xtick={-6},
			   ytick={-6},
			   ]
			   \addplot expression[domain=-4.5:2.2]{2^x};
			   \addplot [pccplot,line width=0.5pt,-]coordinates{ 
					(-3,0.125)
					(0,1)
					};
			   \addplot [pccplot,line width=0.5pt,-]coordinates{ 
					(-2,0.25)
					(1,2)
					};
			   \addplot [pccplot,line width=0.5pt,-]coordinates{ 
					(-1,0.5)
					(2,4)
					};	
			\end{axis}
		\end{tikzpicture}
		\begin{tikzpicture}
			\begin{axis}[
			   framed,
			   width=\figurewidth,
			   xmin=-5,xmax=5,
			   ymin=-1,ymax=5,
			   xtick={-6},
			   ytick={-6},
			   ]
			   \addplot expression[domain=-4.5:4.5]{x^2/8+1};
			   \addplot [pccplot,line width=0.5pt,-]coordinates{ 
					(-4,3)
					(-1,1.125)
					};
			   \addplot [pccplot,line width=0.5pt,-]coordinates{ 
					(-3,2.125)
					(1,1.125)
					};
			   \addplot [pccplot,line width=0.5pt,-]coordinates{ 
					(0,1)
					(3,2.125)
					};	
			\end{axis}
		\end{tikzpicture}
		\captionof{figure}{Concave up on $(\infty,\infty)$}
		\label{fun:fig:concaveup}
	\end{minipage}%
	\begin{minipage}{.5\textwidth}
		\centering
		\begin{tikzpicture}
			\begin{axis}[
			   framed,
			   width=2\figurewidth,
			   xmin=-5,xmax=5,
			   ymin=-1,ymax=5,
			   xtick={-1,2},
			   ytick={-6},
			   ]
			   \addplot expression[domain=-3.8:4.3]{0.5*(x^4/12-x^3/6-x^2)+x+2};
			   \addplot [pccplot,line width=0.5pt,-]coordinates{ 
					(-1,0.625)
					(1,2.458)
					};
			   \addplot [pccplot,line width=0.5pt,-]coordinates{ 
					(-0.5,1.388)
					(1.5,2.305)
					};
			   \addplot [pccplot,line width=0.5pt,-]coordinates{ 
					(0,2)
					(2,2)
					};				\end{axis}
		\end{tikzpicture}
		\captionof{figure}{Concave down on $[-1,2]$}
		\label{fun:fig:concavedown}
	\end{minipage}%
\end{figure}
\end{pccexample}

%===================================
%   Author: Hughes
%   Date:   August 2011
%===================================
\begin{pccexample}
Graph each of the functions defined 
by the following formulas on the interval $(-5,5)$, using either a table of values or technology; 
state if each function is concave up or concave down on $(-\infty,\infty)$.
\begin{align*}
	f(x)=\frac{1}{4}(x-2)^4 &   & g(x)=3+2x-x^2 &   & k(x)=5-2^x 
\end{align*}
\begin{pccsolution}
We graph the functions $f$, $g$, and $h$ in \crefrange{exp:fig:concavef}{exp:fig:concavek}. We observe that
\begin{itemize}
	\item $f$ is concave up on $(-\infty,\infty)$;
	\item $g$ is concave down on $(-\infty,\infty)$;
	\item $k$ is concave down on $(-\infty,\infty)$.
\end{itemize}
\end{pccsolution}

\begin{figure}[!htb]
	\mbox{}
	\hfill
	\begin{minipage}{.25\textwidth}
		\begin{tikzpicture}
			\begin{axis}[
			   framed,
			   xmin=-5,xmax=5,
			   ymin=-1,ymax=5,
			   xtick={-4,-2,...,4},
			   ytick={-4,-2,...,4},
			   grid=both,
			   ]
			   \addplot expression[domain=-0.1:4.1]{0.25*(x-2)^4};
			\end{axis}
		\end{tikzpicture}
		\caption{$f$}
		\label{exp:fig:concavef}
	\end{minipage}%
	\hfill
	\begin{minipage}{.25\textwidth}
		\begin{tikzpicture}
			\begin{axis}[
			   framed,
			   xmin=-5,xmax=5,
			   ymin=-1,ymax=5,
			   xtick={-4,-2,...,4},
			   ytick={-4,-2,...,4},
			   grid=both,
			   ]
			   \addplot expression[domain=-1.2:3.2]{3+2*x-x^2};
			\end{axis}
		\end{tikzpicture}
		\caption{$g$}
		\label{exp:fig:concaveg}
	\end{minipage}%
	\hfill
	\begin{minipage}{.25\textwidth}
		\begin{tikzpicture}
			\begin{axis}[
			   framed,
			   xmin=-5,xmax=5,
			   ymin=-1,ymax=5,
			   xtick={-4,-2,...,4},
			   ytick={-4,-2,...,4},
			   grid=both,
			   ]
			   \addplot expression[domain=-2.2:2.5]{5-2^x};
			\end{axis}
		\end{tikzpicture}
		\caption{$k$}
		\label{exp:fig:concavek}
	\end{minipage}%
	\hfill
	\mbox{}
\end{figure}
\end{pccexample}

%===================================
%   Author: Hughes
%   Date:   Feb 2012
%===================================
\begin{doyouunderstand}
	\begin{problem}
Graph each of the functions defined 
by the following formulas on the interval $(-5,5)$, using either a table of values or technology; 
	state if each function is concave up or concave down on $(-\infty,\infty)$.
	\begin{align*}
		f(x)=-x^2 &   & g(x)=0.1x^4+0.5x^2-3x+2 &   & h(x)=x^3 
	\end{align*}
	\begin{shortsolution}
		$f$ is concave down on $(-\infty,\infty)$.
		
		\begin{tikzpicture}
			\begin{axis}[
			   framed,
			   xmin=-5,xmax=5,
			   ymin=-10,ymax=10,
			   xtick={-4,-3,...,4},
			   ytick={-8,-6,...,8},
			   grid=major,
			   ]
			   \addplot expression[domain=-3.1:3.1]{-x^2};
			\end{axis}
		\end{tikzpicture}
		
		$g$ is concave up on $(-\infty,\infty)$.
		
		\begin{tikzpicture}
			\begin{axis}[
			   framed,
			   xmin=-5,xmax=5,
			   ymin=-10,ymax=10,
			   xtick={-4,-3,...,4},
			   ytick={-8,-6,...,8},
			   grid=major,
			   ]
			   \addplot expression[domain=-1.7:3.3]{0.1*x^4+0.5*x^2-3*x+2};
			\end{axis}
		\end{tikzpicture}
		
		$k$ is neither concave up on $(-\infty,\infty)$ nor concave down on on $(-\infty,\infty)$. It is concave up on on $[0,\infty)$ and concave down on $(-\infty,0]$
		
		\begin{tikzpicture}
			\begin{axis}[
			   framed,
			   xmin=-5,xmax=5,
			   ymin=-10,ymax=10,
			   xtick={-4,-3,...,4},
			   ytick={-8,-6,...,8},
			   grid=major,
			   ]
			   \addplot expression[domain=-2.1:2.1]{x^3};
			\end{axis}
		\end{tikzpicture}
	\end{shortsolution}
	\end{problem}
\end{doyouunderstand}


We introduced this section on concavity by discussing the rate of change of the functions in  \cref{fun:tab:threexplustwo,fun:tab:xsquared}, but then defined concavity as a geometric property of a function's graph. There is a connection. Let's take a look at the functions in \cref{fun:fig:concUpIncreasing,fun:fig:concUpDecreasing,fun:fig:concUpIncDec} which are concave up on $(-\infty,\infty)$, reading each graph left-to-right.

\begin{figure}[!htb]
	\begin{widepage}
	\centering
    \begin{subfigure}{.3\textwidth}
	\begin{tikzpicture}
		\begin{axis}[
                framed,
			   xmin=-2,xmax=5,
			   ymin=0,ymax=5,
			   width=.7\textwidth,
			   height=0.5\textwidth,
			   grid=major,
			   axis line style=->,
			   xtick={-2,...,5},
			   ytick={0,...,5},
			   ]
			\addplot+[]expression[domain=-1.5:5] {1.3^x};		
		\end{axis}
	\end{tikzpicture}
    \caption{Concave up and increasing}
    \label{fun:fig:concUpIncreasing}
    \end{subfigure}%
    \hfill
    \begin{subfigure}{.3\textwidth}
	\begin{tikzpicture}
		\begin{axis}[
                framed,
			   xmin=-2,xmax=5,
			   ymin=0,ymax=5,
			   width=.7\textwidth,
			   height=0.5\textwidth,
			   grid=major,
			   axis line style=->,
			   xtick={-2,...,5},
			   ytick={0,...,5},
			   ]
			\addplot+[]expression[domain=-1.5:5] {0.7^x+1};
		\end{axis}
	\end{tikzpicture}
    \caption{Concave up and decreasing}
	\label{fun:fig:concUpDecreasing}
    \end{subfigure}%
    \hfill
    \begin{subfigure}{.3\textwidth}
	\begin{tikzpicture}
		\begin{axis}[
                framed,
			   xmin=-2,xmax=5,
			   ymin=0,ymax=5,
			   width=.7\textwidth,
			   height=0.5\textwidth,
			   grid=major,
			   axis line style=->,
			   xtick={-2,...,5},
			   ytick={0,...,5},
			   ]
			\addplot+[]expression[domain=-1.5:5] {0.3*(x-2)^2+1};
		\end{axis}
	\end{tikzpicture}
    \caption{Concave up}
	\label{fun:fig:concUpIncDec}
    \end{subfigure}
  \caption{Concave up functions}
  \end{widepage}
\end{figure}

In \cref{fun:fig:concUpIncreasing}, the graph is increasing everywhere. At the beginning it is increasing slowly, and as we move to the right, it increases with a higher and higher rate of change.

In \cref{fun:fig:concUpDecreasing}, the graph is decreasing everywhere. At the beginning it is decreasing quickly, and as we move to the right, it decreases with a smaller and smaller rate of change.

In \cref{fun:fig:concUpIncDec}, the graph is decreasing at first, and then increases. At the beginning it is decreasing quickly, and as we move toward the low point, it decreases with a smaller and smaller rate of change. Beyond the low point, it is increasing slowly at first, and then increases with a larger and larger rate of change. 

These three situations capture what it could mean for a funciton to be concave up.

\begin{pccspecialcomment}[Concavity and rates of change]
A function $f$ is concave up on an interval $I$ if any of the following statements is true (see \cref{fun:fig:concUpIncreasing,fun:fig:concUpDecreasing,fun:fig:concUpIncDec}):
\begin{itemize}
	\item $f$ increases at a faster and faster rate; 
	\item $f$ decreases at a slower and slower rate;
	\item $f$ transitions from decreasing at a slower and slower rate to increasing at a faster and faster rate.
\end{itemize}

Here, $I$ may be an interval of any type: $(p, q)$, $(p, q]$, etc.

A function $f$ is concave down on an interval $I$ if any of the following statements is true:
\begin{itemize}
	\item $f$ decreases at a faster and faster rate;
	\item $f$ increases at a slower and slower rate;
	\item $f$ transitions from increasing at a slower and slower rate to decreasing at a faster and faster rate.
\end{itemize}
\end{pccspecialcomment}

If you are comfortable with negative numbers, then there is an even simpler way to summarize this. In \cref{concUpIncreasing}, the rate of change begins small and positive and gradually becomes larger. In \cref{concUpDecreasing}, the rate of change begins large and negative and gradually becomes a smaller negative number; that is the rate of change becomes larger on a number line. In \cref{concUpIncDec}, the rate of change begins negative and gradually moves higher on a number line until it is positive.

\begin{pccspecialcomment}[Concavity and rates of change again]
A function $f$ is concave up on an interval $I$ if the rate of change becomes larger and larger in a number-line sense. 

Here, $I$ may be an interval of any type: $(p, q)$, $(p, q]$, etc.

A function $f$ is concave up on an interval $I$ if the rate of change becomes smaller and smaller in a number-line sense.\end{pccspecialcomment}

\begin{exercises}
%===================================
%   Author: Hughes
%   Date:   October 2012
%===================================
\begin{problem}[Intervals of increase, decrease and concavity]\label{fun:prob:incdec}
\Cref{fun:fig:incdec} shows the graphs of four functions $p$, $q$, $r$, and $s$.

\begin{figure}[!htb]
	\begin{widepage}
	\setlength{\figurewidth}{.23\textwidth}
	\centering
	\begin{subfigure}{\figurewidth}
		\begin{tikzpicture}
			\begin{axis}[
				            framed,
				            xmin=-6,xmax=8,ymin=-10,ymax=10,
				            xtick={-4,-2,...,6},
				            ytick={-8,-4,4,8},
				            minor ytick={-6,-2,...,6},
				            grid=both,
				            ]
				            \addplot expression[domain=-4.818:6.081,samples=50]{x^3/6-x^2/4-3*x};
			\end{axis}
		\end{tikzpicture}
		\caption{$y=p(x)$}
		\label{fun:fig:incdec1}
	\end{subfigure}
	\hfill
	\begin{subfigure}{\figurewidth}
		\begin{tikzpicture}
			\begin{axis}[
				            framed,
				            xmin=-10,xmax=10,ymin=-10,ymax=10,
				            xtick={-8,-6,...,8},
				            ytick={-8,-6,...,8},
				            grid=major,
				            ]
				            \addplot expression[domain=-6.08:4.967,samples=50]{x^4/20+x^3/15-6/5*x^2+1};
			\end{axis}
		\end{tikzpicture}
		\caption{$y=q(x)$}
		\label{fun:fig:incdec2}
	\end{subfigure}
	\hfill
	\begin{subfigure}{\figurewidth}
		\begin{tikzpicture}
			\begin{axis}[
				            framed,
				            xmin=-10,xmax=10,
				            ymin=-10,ymax=10,
				            xtick={-8,-6,...,8},
				            ytick={-8,-6,...,8},
				            grid=major,
				            ]
				            \addplot expression[domain=-5.28:4.68,samples=50]{-x^5/50-x^4/40+2*x^3/5+6};
			\end{axis}
		\end{tikzpicture}
		\caption{$y=r(x)$}
		\label{fun:fig:incdec3}
	\end{subfigure}
	\hfill
	\begin{subfigure}{\figurewidth}
		\begin{tikzpicture}
			\begin{axis}[
				            framed,
				            xmin=-10,xmax=10,ymin=-10,ymax=10,
				            xtick={-8,-4,4,8},
				            ytick={-8,-4,4,8},
				            minor xtick={-6,-2,...,6},
				            minor ytick={-6,-2,...,6},
				            grid=both,
				            ]
				            \addplot expression[domain=-9.77:8.866,samples=50]{-x^6/6000-x^5/2500+67*x^4/4000+17/750*x^3-42/125*x^2};
			\end{axis}
		\end{tikzpicture}
		\caption{$y=s(x)$}
		\label{fun:fig:incdec4}
	\end{subfigure}
	\caption{Graphs for \cref{fun:prob:incdec}.}
	\label{fun:fig:incdec}
	\end{widepage}
\end{figure}

\begin{subproblem}
	Approximate the zeros of each function.
	\begin{shortsolution}
		\begin{itemize}
			\item $p$ has zeros at about $-3.8$, $0$, and $5$.
			\item $q$ has zeros at about $-5.9$, $-1$, $1$, and $4$.
			\item $r$ has zeros at about $-5$, $-2.9$, and $4.1$.
			\item $s$ has zeros at about $-9$, $-6$, $4.2$, $8.1$, and $0$.
		\end{itemize}
	\end{shortsolution}
\end{subproblem}
\begin{subproblem}
	Approximate the local maximums and minimums of each of the functions.
	\begin{shortsolution}
		\begin{itemize}
			\item $p$ has a local maximum of approximately $3.9$ at $-2$, and a local minimum of approximately $-6.5$ at $3$.
			\item $q$ has a local minimum of approximately $-10$ at $-4$, and $-4$ at $3$; $q$ has a local maximum of approximately $1$ at $0$.
			\item $r$ has a local minimum of approximately $-5.5$ at $-4$, and a local maximum of approximately $10$ at $3$.
			\item $s$ has a local maximum of approximately $5$ at $-8$, $0$ at $0$, and $5$ at  $7$; $s$ has local minimums 
			of approximately $-3$ at $-4$, and $-1$ at $3$.
		\end{itemize}
	\end{shortsolution}
\end{subproblem}
\begin{subproblem}
	Approximate the global maximums and minimums of each of the functions.
	\begin{shortsolution}
		\begin{itemize}
			\item $p$ does not have a global maximum, nor a global minimum.
			\item $q$ has a global minimum of approximately $-10$; it does not have a global maximum.
			\item $r$ does not have a global maximum, nor a global minimum.
			\item $s$ has a global maximum of approximately $5$; it does not have a global minimum.
		\end{itemize}
	\end{shortsolution}
\end{subproblem}
\begin{subproblem}
	Approximate the intervals on which each function is increasing and decreasing.
	\begin{shortsolution}
		\begin{itemize}
			\item $p$ is increasing on $(-\infty,-2)\cup (3,\infty)$, and decreasing on $(-2,3)$.
			\item $q$ is increasing on $(-4,0)\cup (3,\infty)$, and decreasing on $(-\infty,-4)\cup (0,3)$.
			\item $r$ is increasing on $(-4,3)$, and decreasing on $(-\infty,-4)\cup (3,\infty)$.
			\item $s$ is increasing on $(-\infty,-8)\cup (-4,0)\cup (3,5)$, and decreasing on $(-8,-4)\cup (0,3)\cup (5,\infty)$.
		\end{itemize}
	\end{shortsolution}
\end{subproblem}
\begin{subproblem}
	Approximate the intervals on which each function is concave up and concave down.
	\begin{shortsolution}
		\begin{itemize}
			\item $p$ is concave up on  $(1,\infty)$, and concave down on  $(-\infty,1)$.
			\item $q$ is concave up on $(-\infty,-1)\cup (1,\infty)$, and concave down on $(-1,1)$.
			\item $r$ is concave up on $(-\infty,-3)\cup (0,2)$, and concave down on $(-3,0)\cup (2,\infty)$.
			\item $s$ is concave up on $(-6,-2)\cup (2,5)$, and concave down on $(-\infty,-6)\cup (-2,2)\cup (5,\infty)$.
		\end{itemize}
	\end{shortsolution}
\end{subproblem}
\end{problem}

%===================================
%   Author: Hughes
%   Date:   October 2012
%===================================
\begin{problem}[Given properties, sketch a function]
In each of the following problems, sketch a function that 
has the given properties.
\begin{subproblem}
	Increasing and concave up.  
	\begin{shortsolution}
		\begin{tikzpicture}
			\begin{axis}[
				   framed,
				   xmin=-5,xmax=5,
				   ymin=-1,ymax=5,
				   xtick={-6},
				   ytick={-6},
				   ]
				   \addplot expression[domain=-4.5:2.2]{2^x};
			\end{axis}
		\end{tikzpicture}
	\end{shortsolution}
\end{subproblem}
\begin{subproblem}
	Decreasing and concave up.
	\begin{shortsolution}
		\begin{tikzpicture}
			\begin{axis}[
				   framed,
				   xmin=-5,xmax=5,
				   ymin=-1,ymax=5,
				   xtick={-6},
				   ytick={-6},
				   ]
				   \addplot expression[domain=-2.2:4.5]{(0.5)^x};
			\end{axis}
		\end{tikzpicture}
	\end{shortsolution}
\end{subproblem}
\begin{subproblem}
	Decreasing and concave down.  
	\begin{shortsolution}
		\begin{tikzpicture}
			\begin{axis}[
				   framed,
				   xmin=-5,xmax=5,
				   ymin=-1,ymax=5,
				   xtick={-6},
				   ytick={-6},
				   ]
				   \addplot expression[domain=-4.5:2.2]{-1*2^x+4};
			\end{axis}
		\end{tikzpicture}
	\end{shortsolution}
\end{subproblem}
\begin{subproblem}
	Increasing and concave down.
	\begin{shortsolution}
		\begin{tikzpicture}
			\begin{axis}[
				   framed,
				   xmin=-5,xmax=5,
				   ymin=-1,ymax=5,
				   xtick={-6},
				   ytick={-6},
				   ]
				   \addplot expression[domain=-2.2:4.5]{-1*(0.5)^x+4};
			\end{axis}
		\end{tikzpicture}
	\end{shortsolution}
\end{subproblem}
\end{problem}

%===================================
%   Author: Hughes
%   Date:   October 2012
%===================================
\begin{problem}[Increasing or decreasing from a table]\label{fun:prob:incdecnumerically}
\Crefrange{fun:tab:incdecf}{fun:tab:incdecj} show values of functions $f$, 
$g$, $h$, and $j$. Decide if each function is increasing or decreasing.
\begin{shortsolution}
	\begin{enumerate}
		\item $f$ is decreasing.
		\item $g$ is constant-- neither increasing nor decreasing.
		\item $h$ is increasing.
		\item $j$ is increasing.
	\end{enumerate}
\end{shortsolution}

\begin{table}[!htb]
	\centering
	\begin{widepage}
	\caption{Tables for \cref{fun:prob:incdecnumerically}}
	\begin{subtable}{.2\textwidth}
		\centering
		\caption{$y=f(x)$}
		\label{fun:tab:incdecf}
		% y=-\sqrt{x}
		\begin{tabular}{S[table-format=1.0]S[table-format=2.0]}
			\beforeheading
			\heading{$x$} & \heading{$y$} \\            
			\afterheading
			0           & 0           \\\normalline 
			1           & -1          \\\normalline  
			2           & -4          \\\normalline   
			3           & -9          \\\normalline  
			4           & -16         \\\normalline   
			5           & -25         \\\normalline   
			6           & -36         \\\normalline  
			7           & -49         \\\normalline  
			8           & -64         \\\lastline    
		\end{tabular}
	\end{subtable}
	\hfill
	\begin{subtable}{.2\textwidth}
		\centering
		\caption{$y=g(x)$}
		\label{fun:tab:incdecg}
		% y=\pi
		\begin{tabular}{S[table-format=2.0]S[table-format=1.0]}
			\beforeheading
			\heading{$x$} & \heading{$y$} \\ \afterheading 
			-10         & \pi         \\\normalline       
			-6          & \pi         \\\normalline        
			-2          & \pi         \\\normalline         
			2           & \pi         \\\normalline        
			6           & \pi         \\\normalline         
			10          & \pi         \\\normalline         
			14          & \pi         \\\normalline        
			18          & \pi         \\\normalline        
			22          & \pi         \\\lastline          
		\end{tabular}
	\end{subtable}
	\hfill
	\begin{subtable}{.2\textwidth}
		\centering
		\caption{$y=h(x)$}
		\label{fun:tab:incdech}
		% y = 2*x+1
		\begin{tabular}{S[table-format=1.0]S[table-format=1.0]}
			\beforeheading
			\heading{$x$} & \heading{$y$} \\ \afterheading 
			-4          & -7          \\\normalline       
			-3          & -5          \\\normalline         
			-2          & -3          \\\normalline       
			-1          & -1          \\\normalline         
			0           & 1           \\\normalline         
			1           & 3           \\\normalline         
			2           & 5           \\\normalline       
			3           & 7           \\\normalline         
			4           & 9           \\\lastline          
		\end{tabular}
	\end{subtable}
	\hfill
	\begin{subtable}{.2\textwidth}
		\centering
		\caption{$y=j(x)$}
		\label{fun:tab:incdecj}
		% y=2^x
		\begin{tabular}{S[table-format=1.0]S[table-format=2.0]}
			\beforeheading
			\heading{$x$} & \heading{$y$}  \\ \afterheading 
			-4          & \num{1/16} \\\normalline       
			-3          & \num{1/8}  \\\normalline         
			-2          & \num{1/4}  \\\normalline       
			-1          & \num{1/2}  \\\normalline         
			0           & 1            \\\normalline         
			1           & 2            \\\normalline         
			2           & 4            \\\normalline       
			3           & 8            \\\normalline         
			4           & 16           \\\lastline          
		\end{tabular}
	\end{subtable}
	\end{widepage}
\end{table}
\end{problem}

%===================================
%   Author: Hughes
%   Date:   October 2012
%===================================
\begin{problem}[Determine concavity from a table]
Decide if each of the functions depicted in \Crefrange{fun:tab:incdecf}{fun:tab:incdecj} 
are concave up or concave down. You might like to plot the values 
from each table to help you visualize the concavity.
\begin{shortsolution}
	\begin{enumerate}
		\item $f$ is concave up.
		\item $g$ is neither concave up nor concave down.
		\item $h$ is neither concave up nor concave down.
		\item $j$ is concave up.
	\end{enumerate}
\end{shortsolution}
\end{problem}

%===================================
%   Author: Hughes
%   Date:   October 2012
%===================================
\begin{problem}[Counter examples]
The following statements are \emph{all false}. Provide the formula 
for a function that contradicts each statement-- note that there are 
many different functions that will work and as we progress through 
the later chapters, you will be able to provide more interesting 
examples.
\begin{subproblem}
	All increasing functions are concave up.  
	\begin{shortsolution}
		$f(x)=2x$ or $g(x)=\sqrt{x}$; many other choices are available.
	\end{shortsolution}
\end{subproblem}
\begin{subproblem}
	All increasing functions are concave down.  
	\begin{shortsolution}
		$g(x)=2x$; many other choices are available.  
	\end{shortsolution}
\end{subproblem}
\begin{subproblem}
	All decreasing functions are concave up.  
	\begin{shortsolution}
		$h(x)=-3x$; many other choices are available.
	\end{shortsolution}
\end{subproblem}
\begin{subproblem}
	All decreasing functions are concave down.  
	\begin{shortsolution}
		$j(x)=-5x$; many other choices are available.  
	\end{shortsolution}
\end{subproblem}
\begin{subproblem}
	Quadratic functions always increase. 
	\begin{shortsolution}
		Consider $k$ that has formula $k(x)=x^2$ which decreases on $(-\infty,0)$.  
	\end{shortsolution}
\end{subproblem}
\begin{subproblem}
	Quadratic functions always decrease.  
	\begin{shortsolution}
		Consider $l$ that has formula $l(x)=x^2$ which increases on $(0,\infty\infty\infty\infty\infty\infty\infty\infty)$.  
	\end{shortsolution}
\end{subproblem}
\end{problem}
\end{exercises}

%===================================
%   Author: Jordan
%   Date:   March 2013
%===================================
\section{Simplification Issues}

Throughout your algebra course (and beyond), you are going to encounter the algebra simplification steps that we are about to discuss. Algebra simplification is a skill - like cooking noodles or painting a wall. It's not always the most exciting thing in the world, but it does serve a greater purpose. Also like cooking noodles or painting a wall, it's not usually difficult. And yet there are common avoidable mistakes that people make. With practice from this section, you'll have experience to prevent yourself from overcooking the noodles or ruining your paintbrush.

Let's start by reminding ourselves one more time what the meaning of our function notation is. When we write $f(x)$, we have a process $f$ that is working its business on an input value $x$. Whatever is inside those parentheses is the input to the function.

\begin{pccexample}\label{fun:ex:fminusx}
If $f(x)=x^2+3x-4$, find and simplify a formula for $f(-x)$.
\begin{pccsolution}
Those parentheses encase ``$-x$'', so we are meant to treat ``$-x$'' as the input. The rule that we have been given for $f$ is 
\[f(x)=x^2+3x-4\]
But the $x$'s that are in this formula are just place holders. What $f$ does to a number can just as well be communicated with
\[f(\phantom{x})=(\phantom{x})^2+3(\phantom{x})-4\]
So now that we are meant to treat ``$-x$'' as the input, we will insert ``$-x$'' into those slots, after which we can do more familiar algebraic simplification:
\begin{align*}
f(-x)&=(-x)^2+3(-x)-4\\
&=x^2-3x-4
\end{align*}
\end{pccsolution}
\end{pccexample}

\begin{pccexample}\label{fun:ex:fthreex}
If $f(x)=2x^2+8$, find and simplify a formula for $f(3x)$.
\begin{pccsolution}
Those parentheses encase ``$3x$'', so we are meant to treat ``$3x$'' as the input. 
\begin{align*}
f(\phantom{x})&=2(\phantom{x})^2+8\\
f(3x)&=2(3x)^2+8\\
&=2(9x^2)+8\\
&=18x^2+8
\end{align*}
\end{pccsolution}
\end{pccexample}

\begin{pccexample}\label{fun:ex:fxminusfour}
If $f(x)=x^2-3x$, find and simplify a formula for $f(x-4)$.
\begin{pccsolution}
This kind of example is often challenging for college algebra students. But let's focus on those parentheses one more time. They encase ``$x-4$'', so we are meant to treat ``$x-4$'' as the input. 
\begin{align*}
f(\phantom{x})&=(\phantom{x})^2-3(\phantom{x})\\
f(x-4)&=(x-4)^2-3(x-4)\\
&=x^2-8x+16-3x+12\\
&=x^2-11x+28
\end{align*}
\end{pccsolution}
\end{pccexample}

The tasks that are shown in \cref{fun:ex:fminusx,fun:ex:fthreex,fun:ex:fxminusfour} are the kind of task that will make it easier to understand interesting and useful material in later chapters and sections, particularly in \cref{fun:sec:transformations}. This skill is also essential for getting off the ground in a calculus course, which might be in your future.

\begin{pccexample}
Consider the function $f$ given by $f(x)=\frac{1}{4}x^2+x+2$, graphed in \cref{fun:fig:difquotient}. Let's introduce a value on the $x$-axis and call that value $a$. \Cref{fun:fig:difquotient2} illustrates a possible location for $a$, but we do not wish to specify any particular number for $a$. Directly above $a$, we have the point $(a,f(a))$ on the graph of $f$. Let's imagine stepping \num{3} units forward on the $x$-axis\footnote{There is nothing special about \num{3}; we are just choosing a number for the example.}. What $x$-value would we now be at?


\begin{figure}[!htb]
  \begin{minipage}{.45\textwidth}
  			\centering
			\begin{tikzpicture}
				\begin{axis}[
                        framed,
 					   xmin=-4,xmax=2,
					   ymin=-1,ymax=6,
					   grid=major,
					   xtick={-4,...,2},
					   ytick={-1,...,6},
					   ]
					\addplot+[]expression[domain=-4:2] {0.25*x^2+x+2};				
				\end{axis}
			\end{tikzpicture}		
            \caption{figure}{$y=f(x)$}
			\label{fun:fig:difquotient}
\end{minipage}\hfill
  \begin{minipage}{.45\textwidth}
  			\centering
			\begin{tikzpicture}
				\begin{axis}[
                        framed,
 					   xmin=-4,xmax=2,
					   ymin=-1,ymax=6,
					   grid=major,
					   xtick={-4,...,2},
					   ytick={-1,...,6},
					   ]
					\addplot+[]expression[domain=-4:2] {0.25*x^2+x+2};				
					\addplot+[soldot] coordinates { 
					(-2.7,0)
					(-2.7,1.1225)
					(0.3,0)
					(0.3,2.3225)
					};					\end{axis}
			\end{tikzpicture}		
            \caption{figure}{$y=f(x)$}
			\label{fun:fig:difquotient2}
\end{minipage}
\end{figure}
   
\fixthis{Put label nodes at a, a+3, (a,f(a)), and (a+3,f(a+3))}

After stepping \num{3} units forward, the new $x$-value would be $a+3$. Since points on the graph are all of the form $(\text{input},\text{output})$, then above this on the graph is the point $(a+3,f(a+3))$. This is also marked in \cref{fun:fig:difquotient2}.

We are interested in the slope of the line that connects these two points on $f$'s graph. Our task is to find and simplify an expression for that slope.

\begin{pccsolution} As we recall, the slope of a line can be computed by measuring the rise and run between two points on that line, and taking their ratio. We see a run of \num{3} between these two points. What is the rise between them? The right point has $y$-value $f(a+3)$ and the left point has $y$-value $f(a)$. So the rise is $f(a+3)-f(a)$. And that means the slope of the line is given by
\[\frac{f(a+3)-f(a)}{3}\]
To simplify this, we will again pay careful attention to the meaning of those parentheses. In general,
\[f(\phantom{x})=\frac{1}{4}(\phantom{x})^2+(\phantom{x})+2\]
So
\begin{align*}
\frac{{\color{blue}f(a+3)}-{\color{magenta}f(a)}}{3}
&=\frac{{\color{blue}\frac{1}{4}(a+3)^2+(a+3)+2}-{\color{magenta}\big(\frac{1}{4}(a)^2+(a)+2\big)}}{3}\\
&=\frac{\frac{1}{4}(a^2+6a+9)+a+5-\frac{1}{4}a^2-a-2}{3}\\
&=\frac{\frac{1}{4}a^2+\frac32a+\frac94+a+5-\frac{1}{4}a^2-a-2}{3}\\
&=\frac{\frac32a+\frac{21}4}{3}\\
&=\frac12a+\frac{7}4
\end{align*}

To clarify what we just computed: wherever we place the $x$-value $a$, the slope of the segment that connects the graph to a point \num{3} units further to the right will always be $\frac12a+\frac{7}4$.
\end{pccsolution}
\end{pccexample}





\section{Composition}


\begin{figure}[!htb]
  \centering
    \begin{tikzpicture}
		% set up nodes
		\path node (x) at (0,0){$x$}
		      node[right=3cm of x] (f){$f(x)$}
		      node[right=3cm of f] (gf){$g(f(x))$};
		% connect them
		\path[very thick,->] (x) edge[blue,bend left=25] node[pos=0.5,anchor=south](fname){$f$} (f)
		                     (f) edge[red,bend left=25] node[pos=0.5,anchor=south](gname){$g$} (gf)
		                     (x) edge[purple,bend right=25] node[pos=0.5,anchor=north](goffname){$g\circ f$} (gf);
        % draw background
		\begin{pgfonlayer}{background}
        \node[fill=yellow!20,fit=(x)(gf)(fname) (goffname),rounded corners, draw=black!50, dashed]{};
		\end{pgfonlayer}
	\end{tikzpicture}
    \caption{}
  \end{figure}
%===================================
%   Author: Hughes/Fresh/Barkin
%   Date:   November 2012
%===================================
\begin{pccexample}[Coupons]\label{fun:ex:coupons}
\pccname{Jon} and \pccname{Kevin} are shopping for a pair of jeans; 
they have a coupon for \$5 off a pair of jeans. When they arrive at 
the store, they find that all jeans have an extra \SI{25}{\percent} 
marked off the price. They find a pair of jeans for \$55; they 
would like to save as much money as possible, but can't agree on a 
good stratedgy:
\begin{enumerate}
  \item Jon wants to use the coupon first, and then apply the \SI{25}{\percent} off.
  \item Kevin wants to apply the \SI{25}{\percent} off first, and then 
    use the coupon.
\end{enumerate}
Help them resolve their dispute.
\begin{pccsolution}
\begin{enumerate}
  \item If they use the coupon first, then the total cost of the jeans is
    calculated using
    \[
        (55-5)\cdot (0.75)=37.5
    \]
    The jeans cost \$37.50 using Jon's method.
  \item If they deduct \SI{25}{\percent} first, then the cost of the jeans
    is calculated using
    \[
        (0.75)\cdot 55 -5 = 36.25
    \]
    The jeans cost \$36.25 using Kevin's method.
\end{enumerate}
If Jon and Kevin want to minimize cost, they should take \SI{25}{\percent}
off the price first, and then use the coupon.
\end{pccsolution}
\end{pccexample}

%===================================
%   Author: Hughes/Fresh/Barkin
%   Date:   November 2012
%===================================
\begin{pccexample}[Coupons continued]
  Jon and Kevin (from \cref{fun:ex:coupons}) are still thinking about jeans, 
  and decide to try and generalize their findings to jeans that cost $x$ dollars. 

  They let $f$ be the function that represents the cost of the jeans after
  using the \$5 coupon, and $g$ be the function that represents the cost of 
  the jeas after applying the \SI{25}{\percent} discount. They write 
  the following formulas for $f(x)$ and $g(x)$
  \begin{align*}
    f(x)&= x-5\\
    g(x)&=0.75x
  \end{align*}
  Jon suggests using a function $r$ to represent the cost of the jeans 
  when using the coupon first and then applying the \SI{25}{\percent} off;  
  Kevin suggests using a function $s$ to represent the cost of the jeans
  when applying the \SI{25}{\percent} first, and then the coupon.
  They find the following formulas for $r(x)$ and $s(x)$
  \begin{align*}
    r(x)&= (g\circ f)(x)    & s(x)&=(f\circ g)(x)\\
        &= g(x-5)           &      & = f(0.75x)\\
        & = 0.75(x-5)       &      & = 0.75x-5 
  \end{align*}
  They decide to test their formulas by evaluating $r(55)$ and $s(55)$ 
  as follows
  \begin{align*}
    r(55)& = 0.75(55-5) &   s(55)&=0.75(55)-5\\
        &=37.5          &        &=36.25
  \end{align*}
  Both calculations agree with what they found in \cref{fun:ex:coupons}, 
  as expected.
\end{pccexample}


%===================================
%   Author: Hughes
%   Date:   October 2012
%===================================
\begin{pccexample}[Composition]
  Let $f$ and $g$ be the functions that have formulas
  \[
        f(x)=x^2, \qquad g(x)=2x+1
  \]
  We can \emph{compose} the functions to form new functions $f\circ g$
  and $g\circ f$ as follows
  \begin{align*}
    (f\circ g)(x)& = f(g(x))    &  (g\circ f)(x)&=g(f(x))\\
    &=(2x+1)^2   &  &=2x^2+1
  \end{align*}
\end{pccexample}

%===================================
%   Author: Hughes
%   Date:   October 2012
%===================================
\begin{pccexample}
\pccname{Isla} is considering the functions $f$ and $g$ that have formulas
\[
	f(x)=\sqrt{x-3}, \qquad g(x)= \frac{3}{x}
\]
Help Isla find the domain of the following functions:
\begin{enumerate}
	\item $g\circ f$
	\item $f\circ g$
\end{enumerate}
\begin{pccsolution}
\begin{enumerate}
	\item The function $g\circ f$ has formula
	\[
		(g\circ f)(x) = \frac{3}{\sqrt{x-3}}
	\]
	The domain of $g\circ f$ is $(3,\infty)$.
	\item Isla composes $f$ and $g$ to form the function $f\circ g$ that has formula
	\[
		(f\circ g)(x) = \sqrt{\frac{3}{x}-3}
	\]
	and says that the domain of $f\circ g$ is $(-\infty,1)$ because
	\begin{align*}
		\frac{3}{x}-3 \geq 0 & \Rightarrow \frac{3}{x}\geq 3 \\
		                     & \Rightarrow 1\geq x           
	\end{align*}
	\pccname{Oscar} stops by and says that since $0$ is supposedly in the domain 
	of $f\circ g$, he should be able to compute $(f\circ g)(0)$, but immediately 
	runs into trouble since $g(0)$ is undefined.
	
	\begin{marginfigure}
		\centering
		\begin{tikzpicture}
			\begin{axis}[
				axis on top,
				   framed,
				   xmin=-5,xmax=5,
				   ymin=-5,ymax=5,
				   xtick={-4,-2,...,4},
				   ytick={-4,-2,...,4},
				   ]
				   \draw[fill=blue!30](axis cs:0,0)--(axis cs:0,5)--(axis cs: 1,5)--(axis cs:1,0);
				\addplot[pccplot] expression[domain=-5:-0.6,samples=50]{3/x};
				   \addplot[pccplot] expression[domain=0.6:5,samples=50]{3/x}node[axisnode,pos=0.7,anchor=south west]{$y=\frac{3}{x}$};
				   \addplot[pccplot] expression[domain=-5:5]{3}node[axisnode,pos=0.9,anchor=south]{$y=3$};			
			\end{axis}
		\end{tikzpicture}
		\captionof{figure}{}
		\label{fun:fig:compdomain}
	\end{marginfigure}
	Oscar retraces Isla's steps and remembers that 
	\[
		\frac{3}{x}\geq 3\Rightarrow 1\geq x
	\]
	but \emph{only when} $x>0$; when $x<0$ it switches the inequality symbol
	and implies that $x\geq 1$ which is clearly a contradiction. Oscar visualizes 
	this algebra in \cref{fun:fig:compdomain}; in particular, the shaded region 
	highlights the interval on which $\frac{3}{x}\geq 3$.
	
	Isla and Oscar therefore conclude that the domain of $f\circ g$ is actually 
	$(0,1]$.
\end{enumerate}
\end{pccsolution}
\end{pccexample}


  \begin{exercises}
%===================================
%   Author: Hughes
%   Date:   November 2012
%===================================
\begin{problem}[Composition using tables]\label{fun:prob:compnumerically}
	\Crefrange{fun:tab:compf}{fun:tab:compj} show values of the functions
	$F$, $G$, $H$, and $J$. Use these tables to find the following values; if 
	the values are defined, state so.
  \begin{multicols}{4}
  \begin{subproblem}
$(F\circ G)(8)$	  
	\begin{shortsolution}
	$0$	
	\end{shortsolution}
  \end{subproblem}
  \begin{subproblem}
$(G\circ H)(-3)$	  
	\begin{shortsolution}
	$5$	
	\end{shortsolution}
  \end{subproblem}
  \begin{subproblem}
$(H\circ J)(11)$	  
	\begin{shortsolution}
	$4$	
	\end{shortsolution}
  \end{subproblem}
  \begin{subproblem}
$(J\circ F)(7)$	  
	\begin{shortsolution}
	$13$	
	\end{shortsolution}
  \end{subproblem}
  \begin{subproblem}
$(G\circ F)(5)$	  
	\begin{shortsolution}
	$3$	
	\end{shortsolution}
  \end{subproblem}
  \begin{subproblem}
$(H\circ G)(7)$	  
	\begin{shortsolution}
	$13$	
	\end{shortsolution}
  \end{subproblem}
  \begin{subproblem}
$(J\circ H)(4)$	  
	\begin{shortsolution}
	$8$	
	\end{shortsolution}
  \end{subproblem}
  \begin{subproblem}
$(F\circ J)(17)$	  
	\begin{shortsolution}
	$6$	
	\end{shortsolution}
  \end{subproblem}
  \end{multicols}
\end{problem}

\begin{table}[!htb]
	\centering
	\begin{widepage}
	\caption{Tables for \cref{fun:prob:compnumerically}}
	\begin{subtable}{.2\textwidth}
		\centering
		\caption{$y=F(x)$}
		\label{fun:tab:compf}
		\begin{tabular}{S[table-format=1.0]S[table-format=1.0]}
			\beforeheading
			\heading{$x$} & \heading{$y$} \\            
			\afterheading
			0           & 0           \\\normalline 
			1           & 1          \\\normalline  
			2           & 2          \\\normalline   
			3           & 3          \\\normalline  
			4           & 4         \\\normalline   
			5           & 5         \\\normalline   
			6           & 6         \\\normalline  
			7           & 7         \\\normalline  
			8           & 8         \\\lastline    
		\end{tabular}
	\end{subtable}
	\hfill
	\begin{subtable}{.2\textwidth}
		\centering
		\caption{$y=G(x)$}
		\label{fun:tab:compg}
		\begin{tabular}{S[table-format=1.0]S[table-format=1.0]}
			\beforeheading
			\heading{$x$} & \heading{$y$} \\ \afterheading 
			0           & 8           \\\normalline 
			1           & 7          \\\normalline  
			2           & 6          \\\normalline   
			3           & 5          \\\normalline  
			4           & 4         \\\normalline   
			5           & 3         \\\normalline   
			6           & 2         \\\normalline  
			7           & 1         \\\normalline  
			8           & 0         \\\lastline    
		\end{tabular}
	\end{subtable}
	\hfill
	\begin{subtable}{.2\textwidth}
		\centering
		\caption{$y=H(x)$}
		\label{fun:tab:comph}
		\begin{tabular}{S[table-format=1.0]S[table-format=2.0]}
			\beforeheading
			\heading{$x$} & \heading{$y$} \\ \afterheading 
			-4           & 2           \\\normalline 
			-3           & 3          \\\normalline  
			-2           & 5          \\\normalline   
			-1           & 7          \\\normalline  
			0           & 11         \\\normalline   
			1           & 13         \\\normalline   
			2           & 17         \\\normalline  
			3           & 19         \\\normalline  
			4           & 23         \\\lastline    
			\end{tabular}
	\end{subtable}
	\hfill
	\begin{subtable}{.2\textwidth}
		\centering
		\caption{$y=J(x)$}
		\label{fun:tab:compj}
		\begin{tabular}{S[table-format=2.0]S[table-format=1.0]}
			\beforeheading
			\heading{$x$} & \heading{$y$}  \\ \afterheading 
			2           & 0           \\\normalline 
			3           & 1          \\\normalline  
			5           & 2          \\\normalline   
			7           & 3          \\\normalline  
			11           & 4         \\\normalline   
			13           & 5         \\\normalline   
			17           & 6         \\\normalline  
			19           & 7         \\\normalline  
			23           & 8         \\\lastline    
			\end{tabular}
	\end{subtable}
	\end{widepage}
\end{table}

%===================================
%   Author: Hughes
%   Date:   October 2012
%===================================
\begin{problem}[Composition using graphs]
  The functions $p$, $q$, $r$, and $s$ are shown in \cref{fun:fig:comp}. Use the  
  graphs to evaluate each of the following-- if the value is undefined, then state 
  so.
  \begin{multicols}{3}
  \begin{subproblem}
	$(q\circ p)(8)$  
	\begin{shortsolution}
	$2$	
	\end{shortsolution}
  \end{subproblem}
  \begin{subproblem}
$(p\circ q)(0)$	  
	\begin{shortsolution}
	$0$	
	\end{shortsolution}
  \end{subproblem}
  \begin{subproblem}
	 $(q\circ r)(0)$ 
	\begin{shortsolution}
	$2$	
	\end{shortsolution}
  \end{subproblem}
  \begin{subproblem}
$(p\circ r)(-2)$	  
	\begin{shortsolution}
	$-4$	
	\end{shortsolution}
  \end{subproblem}
  \begin{subproblem}
$(q\circ r)(6)$	  
	\begin{shortsolution}
	Undefined; in fact, the function $q\circ r$ is not defined for any values of $x$.
	\end{shortsolution}
  \end{subproblem}
  \begin{subproblem}
$(q\circ s)(6)$	  
	\begin{shortsolution}
	$2$	
	\end{shortsolution}
  \end{subproblem}
  \end{multicols}
\end{problem}

\begin{figure}[!htb]
	\begin{widepage}
	\setlength{\figurewidth}{.23\textwidth}
	\centering
	\begin{subfigure}{\figurewidth}
		\begin{tikzpicture}
			\begin{axis}[
			   framed,
			   xmin=-10,xmax=10,
			   ymin=-5,ymax=5,
			   xtick={-8,-6,...,8},
			   ytick={-4,-2,...,4},
               minor xtick={-9,-7,...,9},
               minor ytick={-3,-1,...,3},
               grid=major,
			   ]
               \addplot expression[domain=-10:10,samples=100]{2*abs(x)^(1/3)*x/abs(x)};
			\end{axis}
		\end{tikzpicture}
		\caption{$y=p(x)$}
		\label{fun:fig:comp1}
	\end{subfigure}
	\hfill
	\begin{subfigure}{\figurewidth}
		\begin{tikzpicture}
			\begin{axis}[
			   framed,
			   xmin=-2,xmax=10,
			   ymin=-2,ymax=5,
			   xtick={2,4,...,8},
			   ytick={-1,...,4},
               minor xtick={-9,-7,...,9},
               grid=both,
			   ]
               \addplot+[->] expression[domain=0:10,samples=100]{sqrt(x)};
	       \addplot[soldot]coordinates{(0,0)};
			\end{axis}
		\end{tikzpicture}
\caption{$y=q(x)$}
		\label{fun:fig:comp2}
	\end{subfigure}
	\hfill
	\begin{subfigure}{\figurewidth}
		\begin{tikzpicture}
			\begin{axis}[
			   framed,
			   xmin=-10,xmax=10,
			   ymin=-10,ymax=2,
			   xtick={-8,-6,...,8},
			   ytick={-8,-6,...,-2},
               minor xtick={-9,-7,...,9},
               minor ytick={-9,-7,...,1},
               grid=major,
			   ]
               \addplot expression[domain=-2.2:8.2,samples=100]{-.3*(x-3)^2-1};
			\end{axis}
		\end{tikzpicture}
\caption{$y=r(x)$}
		\label{fun:fig:comp3}
	\end{subfigure}
	\hfill
	\begin{subfigure}{\figurewidth}
		\begin{tikzpicture}
			\begin{axis}[
			   framed,
			   xmin=-10,xmax=10,
			   ymin=-10,ymax=10,
			   xtick={-8,-6,...,8},
			   ytick={-8,-6,...,8},
               minor xtick={-9,-7,...,9},
               minor ytick={-9,-7,...,9},
               grid=major,
			   ]
               \addplot expression[domain=-10:10]{4};
			\end{axis}
		\end{tikzpicture}
\caption{$y=s(x)$}
		\label{fun:fig:comp4}
	\end{subfigure}
	\caption{}
	\label{fun:fig:comp}
	\end{widepage}
\end{figure}

  \end{exercises}
  \section{Transformations}\label{fun:sec:transformations}
%===================================
%   Author: Vollet/Hughes
%   Date:   November 2012
%===================================
\begin{pccexample}
  \Cref{fun:fig:transformorig} shows a function $f$; \cref{fun:fig:transformtransformed} 
  shows a function $g$ that is a horizontal transformation of $f$. The formula 
  for $g(x)$ is
  \begin{equation}\label{fun:eq:transform}
        g(x)=f(2x-4)
  \end{equation}
    Find the sequence of transformations that transforms $f(x)$ into $g(x)$.

  \begin{figure}[!htb]
    \begin{widepage}
    \begin{subfigure}{.45\textwidth}
\begin{tikzpicture}
  \begin{axis}[
    framed,
    xmin=-2,xmax=7,
    ymin=-2,ymax=6,
    xtick={2,4,6},
    ytick={2,4},
    grid=major,
    visualization depends on={\thisrow{x}\as \myxcoord},
    visualization depends on={\thisrow{y}\as \myycoord},
    nodes near coords={(\pgfmathprintnumber\myxcoord,\pgfmathprintnumber\myycoord)},
            every node near coord/.append style={
                anchor=east},
    ]
    \addplot+[soldot,sharp plot,-] table{
    x y 
    -1 1  
    0 0     
    2 1    
    3 0    
    5 5     
    }; 
  \end{axis}
\end{tikzpicture}
\caption{$y=f(x)$}
\label{fun:fig:transformorig}
    \end{subfigure}%
    \hfill
    \begin{subfigure}{.45\textwidth}
\begin{tikzpicture}
  \begin{axis}[
    framed,
    xmin=-2,xmax=7,
    ymin=-2,ymax=6,
    xtick={2,4,6},
    ytick={2,4},
    grid=major,
    xmin=-2,xmax=7,
    ymin=-2,ymax=6,
    visualization depends on={\thisrow{x}\as \myxcoord},
    visualization depends on={\thisrow{y}\as \myycoord},
    nodes near coords={(\pgfmathprintnumber\myxcoord,\pgfmathprintnumber\myycoord)},
            every node near coord/.append style={
                anchor=east},
    ]
    \addplot+[soldot,sharp plot,-] table{
    x y 
    1.5 1  
    2 0
    2.5 1     
    3.5 0    
    4.5 5     
    }; 
  \end{axis}
\end{tikzpicture}
\caption{$y=g(x)$}
\label{fun:fig:transformtransformed}
    \end{subfigure}%
    \caption{A function $f$ and a transformation of $f$}
    \end{widepage}
  \end{figure}

  \begin{pccsolution}
    Let's consider how we can get the new $x$-values, $x_n$, for $g(x)$
    in terms of the old $x$-values, $x_o$, from $f(x)$.

    We set the argument of $g(x)$ equal to an \emph{old} $x$-value, and then 
    solve for the \emph{new} value
    \begin{align*}
      2x_n-4&=x_0           && \text{original function}\\
      2x_n&=x_0+4           && \text{shift to the \emph{right} by $4$ units}\\
      x_n&=\frac{x_0+4}{2}  && \text{compress by a factor of $\frac{1}{2}$ towards the $y$-axis}
    \end{align*}
    You might prefer to look at this using ordered pairs; \cref{fun:tab:transform} shows 
    how the steps above can be applied to the ordered pairs of the 
    original function $f$ depicted in \cref{fun:fig:transformorig}.
    
    \begin{table}[!htb]
      \caption{Transforming $f(x)$ into $g(x)$ numerically}
      \label{fun:tab:transform}
      \begin{subtable}{.5\textwidth}
      \centering
      \caption{Shift \emph{right} $4$}
      \label{fun:tab:horizshift}
      \begin{tabular}{ll}
        \beforeheading
        \heading{$f(x)$}   &   \heading{New}   \\
        \afterheading   
        $(-1,1)$    &   $(-1+4,1)=(3,1)$    \\\normalline
        $(0,0)$ &  $(0+4,0)=(4,0)$  \\\normalline
        $(2,1)$ &   $(6,1)$ \\\normalline
        $(3,0)$ &   $(7,0)$ \\\normalline
        $(5,5)$ &   $(9,5)$\\\lastline
      \end{tabular}
      \end{subtable}%
      \begin{subtable}{.5\textwidth}
      \centering
      \caption{Compress by factor of $\frac{1}{2}$}
      \label{fun:tab:compress}
      \begin{tabular}{ll}
        \beforeheading
        \heading{Result from \cref{fun:tab:horizshift}}   &   \heading{$g(x)$}   \\
        \afterheading   
        $(3,1)$ &   $(3\div 2,1)=(1.5,2)$\\\normalline
        $(4,0)$ & $(4\div 2,0)=(2,0)$   \\\normalline
        $(6,1)$ & $(3,1)$   \\\normalline
        $(7,0)$ & $(3.5,0)$ \\\normalline
        $(9,5)$ & $(4.5,0)$\\\lastline
      \end{tabular}
      \end{subtable}%
    \end{table}

    The original formula for $g(x)$ in \cref{fun:eq:transform} can 
    be expressed in a slightly different way
    \begin{align*}
      g(x)&=f(2x-4)\\
        &=f(2(x-2))
    \end{align*}
    It seems that this formula for $g(x)$ will lead us to make 
    different transformations from $f(x)$ to $g(x)$; let's see if 
    we can replicate our previous result.

    We set the argument of $g(x)$ equal to an \emph{old} $x$-value, and then 
    solve for the \emph{new} value
    \begin{align*}
      2(x_n-2)&=x_o && \text{the original function}\\
      x_n-2 & = \frac{x_0}{2}   && \text{compress by a factor of $\frac{1}{2}$ towards the $y$-axis}\\
      x_n & = \frac{x_0}{2}+2   && \text{shift to the \emph{right} by $2$ units}
    \end{align*}
    If you prefer an approach using ordered pairs, then study \cref{fun:tab:alternative}.
\begin{table}[!htb]
      \caption{Transforming $f(x)$ into $g(x)$ (alternative approach)}
      \label{fun:tab:alternative}
      \begin{subtable}{.4\textwidth}
      \centering
      \caption{Compress by a factor of $\frac{1}{2}$}
      \label{fun:tab:alternative1}
      \begin{tabular}{ll}
        \beforeheading
        \heading{$f(x)$}   &   \heading{New}   \\
        \afterheading   
        $(-1,1)$    &   $(-1\div 2,1)=(-0.5,1)$    \\\normalline
        $(0,0)$ &  $(0\div 0,0)=(0,0)$  \\\normalline
        $(2,1)$ &   $(1,1)$ \\\normalline
        $(3,0)$ &   $(1.5,0)$ \\\normalline
        $(5,5)$ &   $(2.5,5)$\\\lastline
      \end{tabular}
      \end{subtable}%
      \begin{subtable}{.6\textwidth}
      \centering
      \caption{Shift right by $2$ units}
      \begin{tabular}{ll}
        \beforeheading
        \heading{Result from \cref{fun:tab:alternative1}}   &   \heading{$g(x)$}   \\
        \afterheading   
        $(-0.5,1)$ &   $(-0.5+2,0)=(1.5,0)$\\\normalline
        $(0,0)$ & $(0+2,0)=(2,0)$   \\\normalline
        $(1,1)$ & $(3,1)$   \\\normalline
        $(1.5,0)$ & $(3.5,0)$ \\\normalline
        $(2.5,5)$ & $(4.5,5)$\\\lastline
      \end{tabular}
      \end{subtable}%
    \end{table}

    Notice that in the end, both sets of transformations yeild the same
    formula for $g(x)$. 
  \end{pccsolution}
  
  
\end{pccexample}
